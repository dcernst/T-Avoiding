\chapter{T-Avoiding Elements in Types $\widetilde{A}_n, A_n, D_n, F_n$, and $I_2(m)$}\label{chap:TandTavoid}

%%%%%%%%%%%%%%%%%%%%%%%%%%
In this Chapter we classify the non-trivial T-avoiding elements of Coxeter systems in types $\widetilde{A}_n, A_n, D_n, F_n$ and $I_2(m)$.


\section{Types $\widetilde{A}_n$ and $A_n$}\label{sec:tavoidA}
We start with T-avoiding elements in Coxeter systems of type $\widetilde{A}_n$ and $A_n$. We first focus on non-trivial T-avoiding elements in $W(\widetilde{A}_n)$.

\begin{proposition}
 If $n \geq 2$ and $n$ is odd, then there are no non-trivial T-avoiding elements in $W(\widetilde{A}_n)$. Otherwise, if $n \geq 2$ and $n$ is even then $W(\widetilde{A}_n)$ contains non-trivial T-avoiding elements.
\begin{proof}
	This is~\cite[Proposition~3.1.2]{Fan1999} after a translation of terminology.\qedhere
\end{proof}
\end{proposition}

%Now we will classify the non-trivial T-avoiding elements in $W(A_n)$. Recall that $W(A_n)$ is a star reducible Coxeter group which by our observations above implies that if $W(A_n)$ has non-trivial T-Avoiding elements, these elements will be not FC. In~\cite{Fan1999}, the non-trivial T-avoiding elements were classified. The classification is seen in the following theorem.

%We will now proceed into the classification of T-avoiding elements in $W(\widetilde{A}_n)$. Similar to $W(A_n)$, $W(\widetilde{A}_n)$ will also have trivial T-avoiding elements since $W(\widetilde{A}_n)$ contains elements that are products of commuting generators. We will now proceed to classifying non-trivial T-avoiding elements in $W(\widetilde{A_n})$. The following classification is a result from~\cite{Fan1999}.

The classification seen in~\cite{Fan1999} did not specifically classify the non-trivial T-avoiding elements for type $\widetilde{A}_n$ for $n$ odd. Since $W(\widetilde{A}_n)$ for $n$ odd is not star reducible, the non-trivial T-avoiding elements could be FC. The following is our conjecture regarding what the non-trivial T-avoiding elements are in $W(\widetilde{A}_n)$ for $n$ odd.
\begin{conjecture}
	The only non-trivial T-avoiding elements in $W(\widetilde{A}_n)$ for $n$ odd are of the form $w=(s_0s_2 \cdots s_{n-2}s_ns_1s_3 \cdots s_{n-3}s_{n-1})^k$  for $k \in \mathbb{Z}^+$. 
\end{conjecture} 

Recall that $W(\widetilde{A}_n)$, for $n$ even, is not a star reducible Coxeter group. Hence it makes sense that the T-avoiding elements in $W(\widetilde{A}_n)$, for $n$ even, can be FC. Further, as $W(A_n)$ is a parabolic subgroup of $W(\widetilde{A}_n)$ and $W(A_n)$ is a star reducible Coxeter group, any FC non-trivial T-avoiding elements must have full support. First notice, that $w=(s_0s_2 \cdots s_{n-2}s_ns_1s_3 \cdots s_{n-3}s_{n-1})^k$ is reduced, FC, and has full support. In addition, $w$ is in fact T-avoiding. As stated in the conjecture we believe that these are the only non-trivial T-avoiding elements. However, it is not immediately obvious that there are not any non-FC non-trivial T-avoiding elements. Classifying these non-trivial T-avoiding elements remains an open problem. We now proceed with the classification of T-avoiding elements in Coxeter groups of type $A_n$. 

\begin{corollary}
There are no non-trivially T-avoiding elements in $W(A_n)$. 
\begin{proof}
Notice that the Coxeter graph of type $A_n$ can be obtained from the Coxeter graph of type $\widetilde{A}_k$, for $k > n$. This is done by removing the appropriate number of vertices and edges from the Coxeter graph of type $\widetilde{A}_k$. Since $W(\widetilde{A}_k)$ for $k$ even has no non-trivial T-avoiding elements this forces $W(A_n)$ to not have non-trivial T-avoiding elements. Thus $W(A_n)$ does not have any non-trivial T-avoiding elements.  
%Since $W(A_n)$ is a parabolic subgroup of $W(\widetilde{A}_n)$ this is a consequence of~\cite[Proposition 3.1.2.]{Fan1999}. Specifically, we can obtain the Coxeter graph of type $A_n$ from the Coxeter graph of type $\widetilde{A}_n$ for $n$ even by removing the appropriate number of vertices and edges. From this we can see that if $W(A_n)$ was to have non-trivial T-avoiding elements, this would imply that $W(\widetilde{A}_n)$ for $n$ even would also have non-trivial T-avoiding elements as well. Thus $W(A_n)$ can not have bad elements.
\end{proof}
\end{corollary}


%%%%%%%%%%%%%%%%

\section{Type $D_n$}

In this section summarize the classification of the T-avoiding elements in Coxeter systems of type $D_n$. Recall that $W(D_n)$ is a star reducible Coxeter group and as a result any potential non-trivial T-avoiding elements are not FC.

\begin{proposition}
 The are non-trivial T-avoiding elements in $W(D_n)$ for $n \geq 4$.
\begin{proof}
	This is a consequence of~\cite[Section 2.2]{Gern2013a}. 
\end{proof}
\end{proposition}

We now will classify these elements as seen in~\cite{Gern2013a}. Before we do so we state interval notation useful to the classification from~\cite[Definition 2.3.1]{Gern2013a}. For $2 \leq i \leq j$ denote the element $s_{i}s_{i+1} \cdots s_{j-1}s_j$ by $[i,j]$. For $i \geq 3$, denote $s_1s_3s_4\cdots s_i$ by $[1,i]$ and for $j \geq 2$ denote $s_1s_2s_3 \cdots s_j$ by $[0,j]$. If $0 \leq j <i$ and $i \geq 2$ define $[j,i]=[i,j]^{-1}$. Finally, for $i \leq -3$ and $j \geq 3$ denote $s_1s_{i-1}s_{i-2} \cdots s_4s_3s_2s_3s_4 \cdots s_j$ by $[-i,j]$. The following determines the classification for T-avoiding elements in $W(D_n)$. 

\begin{proposition}
	Let $w \in W(B_n)$ be non-trivially T-avoiding. 
%	Then $w$ has the signed permutation notation 
%	\[ w_m= \begin{cases}
% 			[(-1)^{\frac{m}{2}},\underline{m},3, \underline{m-2}, 5, \ldots, \underline{4}, m-1, \underline{2}, m+1, m+2, \ldots, n] & \text{if, $m$ is even} \\
% 			[(-1)^{\frac{m-1}{2}}, \underline{m-1}, 3, \underline{m-3}, 5, \ldots, \underline{4}, m-2, \underline{2}, m, m+1, \ldots n] & \text{if, $m$ is odd}	
% \end{cases}
%\]
Then $w=w_nu$ reduced for some $m \leq n$, where $u$ is a product of commuting generators such that $\supp(u) \subseteq{s_{m+2}, s_{m+3}, s_{m+4}, \ldots s_n}$ and 
\[ w_n=
\begin{cases}
	[2,0][4,0] \cdots [n-2,0][n,0][n-k,n-2k] \cdots [n-1,n-2][n,n] \\  %\text{if, $n$ is even}
	[2,0][4,0] \cdots [m-2,0][m,0][m-k,m-2k] \cdots [m-1,m-2][m,m]  %\text{if, $n$ is odd}
\end{cases}
\] where the first case above corresponds to $n$ even, the second case corresponds to $n$ odd, $m=n-1$ and 
\[
k= 
\begin{cases}
\frac{n}{2}-2 & \text{if, $n$ is even}\\
\frac{n-1}{2}-2 & \text{if, $n$ is odd.}
\end{cases}
\]
\begin{proof}
	This is~\cite[Lemmas 2.2.18 and 2.3.4]{Gern2013a}. Although it is not immediately obvious, $w_n$ is reduced and not FC.
\end{proof}
\end{proposition}

In Figure~\ref{fig:Dtavoid}, we see two different elements that are T-avoiding in $W(D_5)$. Notice that the blocks that are highlighted in \textcolor{rred}{red} alternate, this prevents the \textcolor{teal}{teal} highlighted braid from forcing its way to the top or the bottom of the heap. Due to the fork in the graph we must make slight alterations to heaps for $W(D_n)$. Specifically we allow $s_0$ and $s_1$ to occupy the same horizontal placement. 

\begin{figure}[h!]
\begin{tabular}{m{7cm} m{7cm}}
\begin{subfigure}{0.5\textwidth} \centering
\begin{tikzpicture}[scale=0.5]
	\heapblock{1}{10}{}{white}
	\heapblock{1}{12}{}{white}
	\heapblock{1}{0}{}{white}
	\heapblock{1}{8}{1}{rred}
	\heapblock{3}{8}{3}{teal}
	\heapblock{2}{6}{2}{teal}
	\heapblock{1}{4}{0}{rred}
	\heapblock{3}{4}{3}{teal}
\end{tikzpicture}	
\caption{}
\end{subfigure} &

\begin{subfigure}{0.5\textwidth} \centering
\begin{tikzpicture}[scale=0.5]
	\heapblock{1}{12}{0}{rred}
	\heapblock{3}{12}{3}{purple}
	\heapblock{5}{12}{5}{purple}
	\heapblock{2}{10}{2}{purple}
	\heapblock{4}{10}{4}{purple}
	\heapblock{1}{8}{1}{rred}
	\heapblock{3}{8}{3}{teal}
	\heapblock{2}{6}{2}{teal}
	\heapblock{1}{4}{0}{rred}
	\heapblock{3}{4}{3}{teal}
	\heapblock{2}{2}{2}{purple}
	\heapblock{4}{2}{4}{purple}
	\heapblock{1}{0}{1}{rred}
	\heapblock{3}{0}{3}{purple}
	\heapblock{5}{0}{5}{purple}
\end{tikzpicture}
\caption{}	
\end{subfigure}
\end{tabular}
\caption{Visual representation of non-trivial T-avoiding elements in $W(D_5)$.}\label{fig:Dtavoid}
\end{figure}

%%%%%%%%%%%%%%%

\section{Type $F_n$}

In this section we classify what is known regarding the T-avoiding elements in the Coxeter systems of type $F_n$ for $n \geq 4$. Note that all of the following results are unpublished. %Recall that $W(F_n)$ is a star reducible Coxeter group so any non-trivial T-avoiding element in $W(F_n)$ will not be FC.

We start with the Coxeter system of type $F_5$.  Recall that $W(F_5)$ is a star reducible Coxeter group so any non-trivial T-avoiding elements will not be FC. Before we begin the classification we introduce the notion of a specific element in $W(F_5)$ called a \emph{bowtie}, which is given by the heap in Figure~\ref{fig:singbowtie}. Note that in Figure~\ref{fig:singbowtie4}, the \textcolor{orange}{orange} blocks correspond to the elements that have bond strength 4. It turns out that the expression determined by this heap is in fact reduced. Looking at the heap in Figure~\ref{fig:singbowtiebraid}, we have highlighted a braid in \textcolor{teal}{teal}. We can obtain a ``stack of bowties" by removing the top most layer of the given heap for the bowtie and adding a new single bowtie to the stack. Doing this repeatedly results in the heap seen in Figure~\ref{fig:stackobowties}. Similar to a single bowtie, the expression that corresponds to a stack of bowties is reduced and not FC. These heaps are referenced in the following unpublished theorem by Cross, Ernst, Hills-Kimball, and Quaranta in 2012, which classifies the T-avoiding elements in the Coxeter systems of type $F_5$.

\begin{figure*}[h!]
\begin{tabular}{m{7cm} m{7cm}}
\begin{subfigure}{0.5\textwidth} \centering
\begin{tikzpicture}[scale=0.5]
	\heapblock{1}{10}{1}{purple}
	\heapblock{3}{10}{3}{orange}
	\heapblock{5}{10}{5}{purple}
	\heapblock{2}{8}{2}{orange}
	\heapblock{4}{8}{4}{purple}
	\heapblock{3}{6}{3}{orange}
	\heapblock{2}{4}{2}{orange}
	\heapblock{4}{4}{4}{purple}
	\heapblock{1}{2}{1}{purple}
	\heapblock{3}{2}{3}{orange}
	\heapblock{5}{2}{5}{purple}
\end{tikzpicture}
\caption{}\label{fig:singbowtie4}
\end{subfigure}&

\begin{subfigure}{0.5\textwidth}\centering
\begin{tikzpicture}[scale=0.5]
	\heapblock{1}{10}{1}{purple}
	\heapblock{3}{10}{3}{purple}
	\heapblock{5}{10}{5}{purple}
	\heapblock{2}{8}{2}{purple}
	\heapblock{4}{8}{4}{teal}
	\heapblock{3}{6}{3}{teal}
	\heapblock{2}{4}{2}{purple}
	\heapblock{4}{4}{4}{teal}
	\heapblock{1}{2}{1}{purple}
	\heapblock{3}{2}{3}{purple}
	\heapblock{5}{2}{5}{purple}
\end{tikzpicture}
\caption{}\label{fig:singbowtiebraid}
\end{subfigure}
\end{tabular}
\caption{Heap of a single bowtie in $W(F_5)$.}\label{fig:singbowtie}	
\end{figure*}

\begin{figure*}[h!] \centering
\begin{tikzpicture}[scale=0.5]
\heapblock{1}{10}{1}{purple}
	\heapblock{3}{10}{3}{orange}
	\heapblock{5}{10}{5}{purple}
	\heapblock{2}{8}{2}{orange}
	\heapblock{4}{8}{4}{purple}
	\heapblock{3}{6}{3}{orange}
	\heapblock{2}{4}{2}{orange}
	\heapblock{4}{4}{4}{purple}
	\heapblock{1}{2}{1}{purple}
	\heapblock{3}{2}{3}{orange}
	\heapblock{5}{2}{5}{purple}
	\heapblock{2}{0}{2}{orange}
	\heapblock{4}{0}{4}{purple}
	\heapblock{3}{-2}{3}{orange}
	\heapblock{2}{-4}{2}{orange}
	\heapblock{4}{-4}{4}{purple}
	\heapblock{1}{-6}{1}{purple}
	\heapblock{3}{-6}{3}{orange}
	\heapblock{5}{-6}{5}{purple}
	
	\node[] at (3,-8){$\vdots$};
	
	\heapblock{1}{-10}{1}{purple}
	\heapblock{3}{-10}{3}{orange}
	\heapblock{5}{-10}{5}{purple}
	\heapblock{2}{-12}{2}{orange}
	\heapblock{4}{-12}{4}{purple}
	\heapblock{3}{-14}{3}{orange}
	\heapblock{2}{-16}{2}{orange}
	\heapblock{4}{-16}{4}{purple}
	\heapblock{1}{-18}{1}{purple}
	\heapblock{3}{-18}{3}{orange}
	\heapblock{5}{-18}{5}{purple}
\end{tikzpicture}
\caption{Heap of a stack of bowties in $W(F_5)$.}\label{fig:stackobowties}
\end{figure*}

\begin{proposition}
The only non-trivial T-avoiding elements in $W(F_5)$ are stacks of bowties. \qed	
\end{proposition}


As a result of the classification in type $F_5$, Cross et al. were also able to classify the T-avoiding elements in $W(F_4)$. 

\begin{corollary}
There are no non-trivial T-avoiding elements in the Coxeter system of type $F_4$. \qed	
\begin{proof}
	Since there are no non-trivial T-avoiding elements in $W(F_5)$ that do not have full support, we know that there are not any non-trivial T-avoiding elements in $W(F_4)$. Because if there were non-trivial T-avoiding elements they would also be non-trivially T-avoiding in $W(F_5)$.
\end{proof}
\end{corollary}

Cross et al. conjectured that in Coxeter systems of type $F_n$ for $n \geq 5$, an element is non-trivially T-avoiding if and only if it is a stack of bowties multiplied by a product of commuting generators. In 2013, Gilbertson and Ernst worked with this conjecture and quickly found it to be false. The heap seen in Figure~\ref{fig:f6bat} corresponds to a non-trivial T-avoiding element in Type $F_6$ that is not a bowtie. It turns out that like the bowties discussed above these elements can also be stacked to create an infinite number of non-trivial T-avoiding elements. In addition, as $n$ gets large there are a number of modifications that can be made that result in additional non-trivially T-avoiding elements. From this we conjecture that the classification of T-avoiding elements in Coxeter systems of type $F_n$ for $n \geq 6$ gets complicated very quickly. Classifying T-avoiding elements in $W(F_n)$ for $n \geq 6$ remains an open problem. 

\begin{figure}[h!]\centering
\begin{tikzpicture}[scale=0.5]
	\heapblock{2}{10}{2}{orange}
	\heapblock{4}{10}{4}{purple}
	\heapblock{6}{10}{6}{purple}
	\heapblock{3}{8}{3}{orange}
	\heapblock{5}{8}{5}{purple}
	\heapblock{2}{6}{2}{orange}
	\heapblock{4}{6}{4}{purple}
	\heapblock{1}{4}{1}{purple}
	\heapblock{3}{4}{3}{orange}
	\heapblock{2}{2}{2}{orange}
	\heapblock{1}{0}{1}{purple}
	\heapblock{3}{0}{3}{orange}
	\heapblock{2}{-2}{2}{orange}
	\heapblock{4}{-2}{4}{purple}
	\heapblock{3}{-4}{3}{orange}
	\heapblock{5}{-4}{5}{purple}
	\heapblock{2}{-6}{2}{orange}
	\heapblock{4}{-6}{4}{purple}
	\heapblock{6}{-6}{6}{purple}
\end{tikzpicture}
\caption{Heap of a non-trivial T-avoiding element in $W(F_6)$}\label{fig:f6bat}
\end{figure}

%%%%%%%%%%%%%%%%

\section{Type $I_2(m)$}\label{sec:tavoidI}

We next classify the T-avoiding elements in Coxeter systems of type $I_2(m)$. Note that in Coxeter systems of type $I_2(m)$, the only products of commuting generators have length 1. Although the following is a quick result, we believe that it does not already appear in the literature.
\begin{theorem}
There are no non-trivial T-avoiding elements in Coxeter systems of type $I_2(m)$.
\begin{proof}
	The graph for the Coxeter system of $I_2(m)$ appears in Figure~\ref{fig:I}. Note that the graph consists of two vertices, namely, $s_1$ and $s_2$, and a single edge with weight $m$. Also, recall that $W(I_2(m))$ is a star reducible Coxeter group. This implies that any non-trivial T-avoiding elements in $W(I_2(m))$ must not be FC, as all of the FC elements have Property T or are trivially T-avoiding. The only non-FC element in $W(I_2(m))$ is the element of length $m$ that has exactly two reduced expressions consisting of alternating products of $s_1$ and $s_2$. Cleary, this element begins and ends with a product of noncommuting generators. Thus, this element has Property T. Hence $W(I_2(m))$ has no non-trivial T-avoiding elements. 
\end{proof}	
\end{theorem}
 



