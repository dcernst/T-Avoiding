\chapter{T-Avoiding Elements in Type $\C_n$}\label{chap:Cn}



%%%%%%%%%%%%%%%%%%%%%%%%%%
\section{Classification of T-Avoiding Elements in Type $\C_n$}\label{sec:TAC}

In this section we will classify the T-avoiding elements in Coxeter systems of type $\C_n$, a new result. Since $W(A_n)$ and $W(B_n)$ are parabolic subgroups of $W(\C_n)$ and these groups have no $\tII$ elements,  any $\tII$ elements of $W(\C_n)$ must have full support. We will first show that there are no $\tII$ elements that are not FC in $W(\C_n)$.

Before we begin the proof we must first define the notion of a pushed-down representation of a heap. First recall that there are potentially many ways to draw the lattice point representation of a heap, each differing by the amount of vertical space between blocks. We wish to fix one such representation. Let $\w$ be a reduced expression for $w \in W(\C_n)$. We construct the \emph{pushed-down representation of a heap} of $H(\w)$ by first giving all blocks fully exposed to the bottom the same vertical position, and then all blocks are as low as possible in the heap. Loosely speaking the heap has been constructed by placing all blocks in the lowest possible vertical position of the heap. Notice that we can now label the rows in the heap from bottom to top where the bottom-most row is row 1 and proceeding naturally upward from there. 

We now define the height of a braid. Given the presence of a braid in the heap of an reduced expression $\w$, we say that the \emph{height of the braid} is the row number in which the uppermost block involved in the braid is located in the pushed-down representation. It is important to note that in the pushed-down representation a braid may not appear in consecutive rows. That is, some of the blocks may be lower in the heap and the braid may not be immediately apparent. 

\begin{example}
Let $\w=s_0s_1s_3s_2s_1s_0s_1s_3$ be a reduced expression for $w \in W(C_3)$. The pushed-down representation of a heap of $\w$ is given in Figure~\ref{fig:pusheddownheap}. The height of the braid $s_1s_2s_1$ which is highlighted in \textcolor{teal}{teal} in Figure~\ref{fig:pusheddownheap} is 5 since the upper block for $s_1$ is located in the fifth row of the pushed-down representation heap. Notice that the block for $s_3$ can slide up higher in the heap. If we were to slide the block for $s_3$ up until it hits the block for $s_2$ we would obtain the braid $s_3s_2s_3$ in the heap. In this case, the height for the braid $s_3s_2s_3$ is also 5.   

\begin{figure}[h!] \centering
\begin{tikzpicture}[scale=0.45]
	\heapblock{0}{10}{0}{purple}
	\heapblock{3}{8}{3}{purple}
	\heapblock{1}{8}{1}{teal}
	\heapblock{2}{6}{2}{teal}
	\heapblock{1}{4}{1}{teal}
	\heapblock{0}{2}{0}{purple}
	\heapblock{1}{0}	{1}{purple}
	\heapblock{3}{0}{3}{purple}
%	\heapblock{0}{0}{0}{purple}
%	\heapblock{2}{0}{2}{purple}
%	\heapblock{4}{0}{4}{purple}
\end{tikzpicture}	
\caption{Pushed-down representation of a heap}\label{fig:pusheddownheap}
\end{figure}

	
\end{example}


\begin{theorem}\label{thm:TavoidC}
There are no $\tII$ elements in $W(\C_n)\setminus \FC(\C_n)$. 	
\begin{proof}
	We proceed by contradiction. Let $w \in W(\C_n)\setminus\FC(\C_n)$ such that $w$ has full support and $w$ is T-avoiding. Consider all possible pushed-down representations for heaps of $w$. Choose a representation  that has a minimal height braid among all braids appearing in all heaps for $w$ and let $k$ represent that minimum height. There may be a tie, in which case choose your favorite. Without loss of generality we will call the generators involved in the braid $s_i,s_{i+1}$ where the bond strength is case specific and will be given in the following cases. In the following cases whenever we refer to a block being in a specific row, we are considering the pushed-down representation of the heap. However, in order to consider the braids that we are looking for we need to allow some flexibility when referring to the absolute vertical position of a given block. In the following cases, whenever we refer to a subheap of $w$ they are assumed to be convex.
	
	Case (1): Suppose the lowest braid occurs in the bottom-most row where $k=3$ (respectively, $k=4$) if $m(s_i,s_{i+1})=3$ (respectively, $m(s_i,s_{i+1})=4$). In this case, we are assuming that the braid is located in consecutive rows with the upper-most block in the above specified row and the lowest block in the heap located in the bottom-most row of the heap. Without loss of generality assume $s_{i+1}$ is in the bottom-most row of the heap. Clearly, the block for $s_{i-1}$ must be in the bottom-most row of the heap as well, otherwise $w$ has Property T which is a contradiction to the original choice of $w$. From this restricting our view to the subheap of $w$ that contains the braid we are considering  we see that the heap of $w$ has the following form
	\begin{center}
	\begin{tikzpicture}[scale=0.45]
		\heapblock{1}{0}{s_{i-1}}{purple}
		\heapblock{3}{0}{s_{i+1}}{purple}
		\heapblock{2}{2}{s_i}{purple}
		\heapblock{3}{4}{s_{i+1}}{purple}
		\sheapblock{2}{6}{s_i}{purple}
	\end{tikzpicture}	
	\end{center}
 	where the striped heap block represents the fourth block in the braid if $m(s_i,s_{i+1})=4$. Applying the braid move we get the subheap seen here
 	\begin{center}
 	\begin{tikzpicture}[scale=0.45]
 		\heapblock{1}{0}{s_{i-1}}{purple}
 		\heapblock{2}{2}{s_i}{purple}
 		\heapblock{3}{4}{s_{i+1}}{purple}
 		\heapblock{2}{6}{s_i}{purple}
 		\sheapblock{3}{8}{s_{i+1}}{purple}
 	\end{tikzpicture}
 	\end{center}
	which clearly has Property T since $s_{i-1}$ is now in the 1st row of the pushed-down representation. This is a contradiction to the way in which we chose $w$. T For the rest of the cases we will assume that $k \geq 4$.
	
	Case (2): Suppose the braid has height $k$ and assume the braid does not contain $s_0, s_1, s_{n-1}$ or $s_n$. Without loss of generality assume $s_i$ is in the $k$th row of the heap and if necessary we have brought the blocks for $s_{i-1}$ and $s_i$ up next to $s_i$ in row $k$. We now consider what can be in the $(k-3)$th row of the heap in two cases. 
	
	Subcase (a): Assume that the block for $s_{i-1}$ is in the heap in the $(k-3)$th row and we allow for the block for $s_{i+1}$ to be in the same row as well, but it does not necessarily have to be. In the following pictures the block for $s_{i+1}$ will be represented in a purple striped block to indicate that it could be present but it does not have to be. The following is the subheap that we are considering
	\begin{center}
	\begin{tikzpicture}[scale=0.45]
		\heapblock{2}{0}{s_{i-1}}{purple}
		\sheapblock{4}{0}{s_{i+1}}{purple}
		\heapblock{1}{2}{s_{i-2}}{purple}
		\heapblock{3}{2}{s_i}{teal}
		\heapblock{2}{4}{s_{i-1}}{teal}
		\heapblock{3}{6}{s_i}{teal}
	\end{tikzpicture}
	\end{center}
	where we have highlighted the braid in \textcolor{teal}{teal}. Notice that the block for $s_{i-2}$ is present in the $(k-2)$th row. If it was not there, $w$ would have had the braid $s_{i-1}s_{i}s_{i-1}$ since $m(s_{i-1},s_i)=3$ and we would have had a heap with a lower braid to choose. Applying the braid move to the heap we get the following subheap
	\begin{center}
	\begin{tikzpicture}[scale=0.45]
		\heapblock{2}{0}{s_{i-1}}{rred}
		\sheapblock{4}{0}{s_{i+1}}{purple}
		\heapblock{1}{2}{s_{i-2}}{rred}
		\heapblock{2}{4}{s_{i-1}}{rred}
		\heapblock{3}{6}{s_i}{teal}
		\heapblock{2}{8}{s_{i-1}}{teal}
	\end{tikzpicture}
	\end{center}
	which has a new braid in it. This braid which we have highlighted in \textcolor{rred}{red} for emphasis has height $k-1$. In applying the braid move we have obtained a subheap which has a braid that has a lower height than our original choice. This is a contradiction to the way in which we chose our heap.
	
	Subcase (b): Assume that the block for $s_{i+1}$ is in the $(k-3)$th row of the heap and the block for $s_{i-1}$ does not appear in the $(k-3)$th row and $s_{k-2}$ does not appear in the $(k-2)$th. The following is the subheap we are considering
	\begin{center}
	\begin{tikzpicture}[scale=0.45]
		\heapblock{3}{0}{s_{i+1}}{purple}
		\dheapblock{1}{0}{}{black}
		\dheapblock{0}{2}{}{black}
		\heapblock{2}{2}{s_i}{teal}
		\heapblock{1}{4}{s_{i-1}}{teal}
		\heapblock{2}{6}{s_i}{teal}
	\end{tikzpicture}
	\end{center}
 	where the dotted square represents that no block may occupy this space and the braid is highlighted in \textcolor{teal}{teal}. Applying the braid move in the subheap we obtain the following heap
 	\begin{center}
 	\begin{tikzpicture}[scale=0.45]
 		\heapblock{2}{0}{s_{i+1}}{purple}
 		\heapblock{0}{0}{s_{i-1}}{teal}
 		\heapblock{1}{2}{s_i}{teal}
 		\heapblock{0}{4}{s_{i-1}}{teal}
 	\end{tikzpicture}
 	\end{center}
	where the braid now occurs in the $(k-1)$th row. This contradicts the way in which we chose the heap of $w$. From this we gather that the braid must contain $s_0,s_1, s_{n-1},$ or $s_n$.
	
	Case 3: Suppose the braid is located in the $k$th row and assume the braid contains $s_2$ or $s_{n-2}$. Without loss of generality we assume that the braid contains $s_2$ as the other argument is symmetric to the one presented here, and assume $s_2$ is in the $k$th row. Notice that if the braid contains $s_2s_3s_2$, we are in Case (2), as a result we assume our braid is not of the form $s_2s_3s_2$. Assume that if necessary the blocks $s_1$ and $s_2$ have been brought up next to $s_2$ in row $k$. We now consider what can be in the $(k-3)$th and $(k-4)$th rows of the heap in two cases. 
	
	Subcase (a): Assume the block for $s_1$ is in row $k-3$, and $s_0$ is in row $k-4$ but $s_3$ is not in the $(k-3)$th row and $s_2$ is not in the $(k-4)$th row. Then the subheap we are considering is as follows
	\begin{center}
	\begin{tikzpicture}[scale=0.45]
		\heapblock{0}{0}{0}{purple}
		\dheapblock{2}{0}{}{black}
		\heapblock{1}{2}{1}{purple}
		\dheapblock{3}{2}{}{black}
		\heapblock{0}{4}{0}{purple}
		\heapblock{2}{4}{2}{teal}
		\heapblock{1}{6}{1}{teal}
		\heapblock{2}{8}{2}{teal}	
	\end{tikzpicture}
	\end{center}
	where the braid we are considering in the heap is highlighted in \textcolor{teal}{teal}. Notice that the block for $s_0$ is in the $(k-2)$th row. It must be here, as if it was not $w$ would not be reduced. Applying the braid move we get the following heap
	\begin{center}
	\begin{tikzpicture}[scale=0.45]
		\heapblock{0}{0}{0}{rred}
		\dheapblock{2}{0}{}{black}
		\heapblock{1}{2}{1}{rred}
		\dheapblock{3}{2}{}{black}
		\heapblock{0}{4}{0}{rred}
		\heapblock{1}{6}{1}{rred}
		\heapblock{2}{8}{2}{teal}
		\heapblock{1}{10}{1}{teal}
	\end{tikzpicture}
	\end{center}
	where a new braid has appeared highlighted in \textcolor{rred}{red}. Notice that the height of this new braid is $k-1$. This braid is lower in the heap than our original braid. This is a contradiction to the original choice of heap.
	
	Subcase (b): Assume the block for $s_0$ is in the $(k-2)$th row, and $s_1$ and $s_3$ are in the $(k-3)$th row. Then the subheap we are considering is 
	\begin{center}
	\begin{tikzpicture}[scale=0.45]
		\heapblock{1}{0}{1}{purple}
		\heapblock{3}{0}{3}{purple}
		\heapblock{2}{2}{2}{teal}
		\heapblock{0}{2}{0}{purple}
		\heapblock{1}{4}{1}{teal}
		\heapblock{2}{6}{2}{teal}
	\end{tikzpicture}	
	\end{center}
	where the braid is highlighted in \textcolor{teal}{teal}. Applying the braid move we get the following subheap
	\begin{center}
	\begin{tikzpicture}[scale=0.45]
		\heapblock{1}{0}{1}{purple}
		\heapblock{3}{0}{3}{purple}
		\heapblock{0}{2}{0}{purple}
		\heapblock{1}{4}{1}{teal}
		\heapblock{2}{6}{2}{teal}
		\heapblock{1}{8}{1}{teal}
	\end{tikzpicture}
	\end{center}
	where no new braids have appeared, and in fact the original braid is higher now with height $k+1$. Notice however, that the $(k-3)$th row will not be the bottom row of our heap because then $w$ would contain Property T a contradiction to our assumption. With this in mind we consider the $(k-4)$th row. Notice that $s_0$ will not appear in the $(k-4)$th row as we would have a lower braid. This implies that the $(k-4)$th row contains at least one of $s_2$ or $s_4$. With this in mind we represent this with the following subheap
	\begin{center}
	\begin{tikzpicture}[scale=0.45]
		\sheapblock{2}{0}{2}{rred}
		\sheapblock{4}{0}{4}{rred}
		\heapblock{1}{2}{1}{purple}
		\heapblock{3}{2}{3}{purple}
		\heapblock{0}{4}{0}{purple}
		\heapblock{1}{6}{1}{teal}
		\heapblock{2}{8}{2}{teal}
		\heapblock{1}{10}{1}{teal}
	\end{tikzpicture}	
	\end{center}
	where we have highlighted the additions in \textcolor{rred}{red} and striped blocks. Here these striped blocks represent that at least one of the blocks appear but possibly both. Notice that if this new row is row 1, then $w$ would have Property T. Again, this implies that this will not be the bottom row of the heap. Repeating this process again we see that $s_1$ will not be in row $k-5$ since this would create a lower braid, thus we must have at least one of $s_3$ or $s_5$ in the $(k-5)$th row. We represent this with the following subheap
	\begin{center}
	\begin{tikzpicture}[scale=0.45]
		\sheapblock{3}{0}{3}{rred}
		\sheapblock{5}{0}{5}{rred}
		\sheapblock{2}{2}{2}{rred}
		\sheapblock{4}{2}{4}{rred}
		\heapblock{1}{4}{1}{purple}
		\heapblock{3}{4}{3}{purple}
		\heapblock{0}{6}{0}{purple}
		\heapblock{1}{8}{1}{teal}
		\heapblock{2}{10}{2}{teal}
		\heapblock{1}{12}{1}{teal}
	\end{tikzpicture}	
	\end{center}
	where again we have highlighted the additions in \textcolor{rred}{red} and striped blocks. Again, if the $(k-5)$th row is the first row, then  $w$ would have Property T. This implies that the $(k-5)$th row is not the bottom-most row in our heap. Iterating this process we obtain the subheap as follows
	\begin{center}
	\begin{tikzpicture}[scale=0.45] 
		\heapblock{1}{12}{1}{teal}
		\heapblock{2}{10}{2}{teal}
		\heapblock{1}{8}{1}{teal}
		\heapblock{0}{6}{0}{purple}
		\heapblock{1}{4}{1}{purple}
		\heapblock{3}{4}{3}{purple}
		\sheapblock{2}{2}{2}{rred}
		\sheapblock{3}{0}{3}{rred}
		\sheapblock{4}{2}{4}{rred}
		\sheapblock{5}{0}{5}{rred}
		\sheapblock{4}{-2}{4}{rred}
		
		\node[] at (7,-2){$\ddots$};
		\node[] at (7,-4){$\ddots$};
		
		\sheapblock{9}{-3}{n-3}{rred}
		\sheapblock{11}{-3}{n-1}{rred}
		\sheapblock{10}{-5}{n-2}{rred}
		\sheapblock{12}{-5}{n}{rred}
	\end{tikzpicture}
	\end{center}
	where again we see that if the row containing the blocks for $s_{n-2}$ and $s_n$ corresponds to row 1, then $w$ would have Property T. From this we obtain the following subheap
	\begin{center}
	\begin{tikzpicture}[scale=0.45] 
		\heapblock{1}{12}{1}{teal}
		\heapblock{2}{10}{2}{teal}
		\heapblock{1}{8}{1}{teal}
		\heapblock{0}{6}{0}{purple}
		\heapblock{1}{4}{1}{purple}
		\heapblock{3}{4}{3}{purple}
		\sheapblock{2}{2}{2}{rred}
		\sheapblock{3}{0}{3}{rred}
		\sheapblock{4}{2}{4}{rred}
		\sheapblock{5}{0}{5}{rred}
		\sheapblock{4}{-2}{4}{rred}
		
		\node[] at (7,-2){$\ddots$};
		\node[] at (7,-4){$\ddots$};
		
		\sheapblock{9}{-3}{n-3}{rred}
		\sheapblock{11}{-3}{n-1}{rred}
		\sheapblock{10}{-5}{n-2}{rred}
		\sheapblock{12}{-5}{n}{rred}
		\sheapblock{11}{-7}{n-1}{rred}
		\sheapblock{9}{-7}{n-3}{rred}
		\sheapblock{10}{-9}{n-2}{rred}
		
		\node[] at (7, -9){$\iddots$};
		\node[] at (7,-11){$\iddots$};
		
		\sheapblock{3}{-10}{3}{rred}
		\sheapblock{1}{-10}{1}{rred}
		\sheapblock{2}{-12}{2}{rred}
		\sheapblock{0}{-12}{0}{rred}
		\sheapblock{4}{-12}{4}{rred}
	\end{tikzpicture}
	\end{center}
	where the striped \textcolor{rred}{red} blocks correspond to a portion of an FC element. Originally we said that the white inner triangle of this heap must be empty, however this contradicts~\cite[Lemma 3.3]{Ernst2012c} which says that elements that have the triangle of white space must be filled completely in order to maintain that the heap is FC. That is, in the blank space between the \textcolor{purple}{purple} blocks $s_0, s_1$ and the striped \textcolor{red}{red} blocks $s_1,s_3$ in the 2nd row every block will actually be present. This leads to the heap seen here
	\begin{center}
	\begin{tikzpicture}[scale=0.45] 
		\heapblock{1}{12}{1}{teal}
		\heapblock{2}{10}{2}{teal}
		\heapblock{1}{8}{1}{orange}
		\heapblock{0}{6}{0}{orange}
		\heapblock{1}{4}{1}{orange}
		\heapblock{3}{4}{3}{purple}
		\heapblock{0}{2}{0}{orange}
		\sheapblock{2}{2}{2}{rred}
		\sheapblock{3}{0}{3}{rred}
		\sheapblock{4}{2}{4}{rred}
		\sheapblock{5}{0}{5}{rred}
		\sheapblock{4}{-2}{4}{rred}
		
		\node[] at (7,-2){$\ddots$};
		\node[] at (7,-4){$\ddots$};
		
		\sheapblock{9}{-3}{n-3}{rred}
		\sheapblock{11}{-3}{n-1}{rred}
		\sheapblock{10}{-5}{n-2}{rred}
		\sheapblock{12}{-5}{n}{rred}
		\sheapblock{11}{-7}{n-1}{rred}
		\sheapblock{9}{-7}{n-3}{rred}
		\sheapblock{0}{-5}{}{rred}
		\sheapblock{2}{-5}{}{rred}
	
		\node[] at (7, -7){$\iddots$};
		\node[] at (7,-9){$\iddots$};
		\node[] at (5,-5){$\cdots$};
		\node[] at (0,-1.5){$\vdots$};
		\node[] at (0,-7){$\vdots$};
		
		\sheapblock{3}{-9}{3}{rred}
		\sheapblock{1}{-9}{1}{rred}
		\sheapblock{2}{-11}{2}{rred}
		\sheapblock{0}{-11}{0}{rred}
		\sheapblock{4}{-11}{4}{rred}
	\end{tikzpicture}
	\end{center}
	where we see that this has led to $s_0$ being located in the $(k-4)$th row. Notice that a new braid has now appeared in the heap. We have highlighted this in \textcolor{orange}{orange}. This new braid has height $k-1$. This is a contradiction as the braid appears lower than the original braid we chose. 
		
	Case (4): Suppose the braid is located in the $k$th row and assume the braid contains $s_1$ or $s_{n-1}$. Without loss of generality we assume that the braid contains $s_1$, as the the other argument is symmetric to the one presented here and assume $s_1$ is in the $k$th row. Assume that, if necessary, the blocks that complete the braid have been brought up next to $s_1$ in the $k$th row. We now consider what can happen in the $(k-3)$th row and $(k-4)$th row in two cases. Notice that if the braid is  $s_1s_2s_1$, then we are in Case 1, so assume the braid consists of $s_0$ and $s_1$. 
	
	Subcase (a): Assume the braid involves $s_1$ and $s_0$ and the block for $s_2$ is located in the $(k-3)$th row. Then the subheap we are considering follows here
	\begin{center}
	\begin{tikzpicture}[scale=0.45]
		\heapblock{0}{0}{0}{orange}
		\heapblock{2}{0}{2}{purple}
		\heapblock{1}{2}{1}{orange}
		\heapblock{0}{4}{0}{orange}
		\heapblock{1}{6}{1}{orange}
	\end{tikzpicture}	
	\end{center}
	where the braid we are considering is highlighted in \textcolor{orange}{orange}. Applying the braid move we get the following subheap
	\begin{center}
	\begin{tikzpicture}[scale=0.45]
		\heapblock{2}{0}{2}{purple}
		\heapblock{1}{2}{1}{orange}
		\heapblock{0}{4}{0}{orange}
		\heapblock{1}{6}{1}{orange}
		\heapblock{0}{8}{0}{orange}
	\end{tikzpicture}
	\end{center}
	in which we see that the original braid is now located higher in the heap with height $k+1$. By Case (1), we know that the original heap we started with implies that $s_0$ and $s_2$ are located above row 1. This implies that the subheap has more rows underneath to fill in. 
	
	Subcase (i): We first consider if the block for $s_1$ is located in row $k-4$ and $s_3$ is allowed but not required to be there. This leads to the following heap
	\begin{center}
	\begin{tikzpicture}[scale=0.45]
		\sheapblock{3}{0}{3}{purple}
		\heapblock{1}{0}{1}{rred}
		\dheapblock{3}{0}{}{black}
		\heapblock{2}{2}{2}{rred}
		\heapblock{1}{4}{1}{rred}
		\heapblock{0}{6}{0}{orange}
		\heapblock{1}{8}{1}{orange}
		\heapblock{0}{10}{0}{orange}
	\end{tikzpicture}	
	\end{center}
	where a new braid appears  which we have highlighted in \textcolor{rred}{red}. This new braid has height $k-2$ which is lower than the height of the original braid that we chose. This is a contradiction to the way in which we chose $w$. 
	
	Subcase (ii): Now we consider if the block for $s_3$ is in the $(k-4)$th row and $s_1$ is not. This leads to the subheap seen here 
	\begin{center}
	\begin{tikzpicture}[scale=0.45]
		\dheapblock{1}{0}{}{black}
		\heapblock{3}{0}{3}{rred}
		\heapblock{2}{2}{2}{purple}
		\heapblock{1}{4}{1}{orange}
		\heapblock{0}{6}{0}{orange}
		\heapblock{1}{8}{1}{orange}
		\heapblock{0}{10}{0}{orange}
	\end{tikzpicture}
	\end{center}
	where again there are no new braids present. However, the bottom row of the subheap is not the bottom row of the heap of $w$ since otherwise $w$ would have Property T. Using this notion we extend our heap to look like
	\begin{center}
	\begin{tikzpicture}[scale=0.45]
		\dheapblock{2}{0}{}{black}
		\heapblock{4}{0}{4}{rred}
		\dheapblock{1}{2}{}{black}
		\heapblock{3}{2}{3}{rred}
		\heapblock{2}{4}{2}{purple}
		\heapblock{1}{6}{1}{orange}
		\heapblock{0}{8}{0}{orange}
		\heapblock{1}{10}{1}{orange}
		\heapblock{0}{12}{0}{orange}
	\end{tikzpicture}
	\end{center} 
	where again we see no new braids. Again, we know that the bottom row of the subheap is not the bottom row of the heap of $w$ since otherwise $w$ would have Property T. Iterating this process we obtain a heap that looks like
	\begin{center}
	\begin{tikzpicture}[scale=0.45]
		\heapblock{0}{12}{0}{orange}
		\heapblock{1}{10}{1}{orange}
		\heapblock{0}{8}{0}{orange}
		\heapblock{1}{6}{1}{orange}
		\heapblock{2}{4}{2}{purple}
		\dheapblock{1}{2}{}{black}
		\heapblock{3}{2}{3}{rred}
		\dheapblock{2}{0}{}{black}
		\heapblock{4}{0}{4}{rred}
		
		\node[] at (6,-1.5){$\ddots$};
		\node[] at (4,-1.5){$\ddots$};
		
		\heapblock{8}{-3.5}{n-1}{rred}
		\dheapblock{6}{-3.5}{}{black}
		\heapblock{9}{-5.5}{n}{rred}
		\dheapblock{7}{-5.5}{}{black}
		\heapblock{8}{-7.5}{n-1}{rred}
		\dheapblock{6}{-7.5}{}{black}
		
		\node[] at (6,-9){$\iddots$};
		\node[] at (4,-9){$\iddots$};
	\end{tikzpicture}	
	\end{center}
	where the zig-zag pattern continues. That is, the \textcolor{rred}{red} row of blocks continues without blocks above or below these blocks. Notice that if the zig-zag was to end before reaching $s_0$, the heap would have Property T, which is a contradiction to the way in which we chose $w$. Suppose that the zig-zag continues on after reaching $s_0$. Then reverting back to the original configuration of the subheap we are able to drop the block for $s_0$ down and create a lower braid. This is a contradiction to the way in which we chose the heap. 
	
	Case (5): Suppose the braid is located in the $k$th row and assume the braid contains $s_0$ or $s_n$. Without loss of generality we assume that the braid contains $s_0$ as the other argument is symmetric to the one presented here, and assume $s_0$ is in row $k$. We now consider what is in row $k-4$. Notice that $s_0$ can not be in the $(k-4)$th row as $w$ would not be reduced. This implies that $s_2$ is in the $(k-4)$th row and we get the subheap below.
	\begin{center}
	\begin{tikzpicture}[scale=0.45]
		\heapblock{2}{0}{2}{purple}
		\dheapblock{0}{0}{}{black}
		\heapblock{1}{2}{1}{orange}
		\heapblock{0}{4}{0}{orange}
		\heapblock{1}{6}{1}{orange}
		\heapblock{0}{8}{0}{orange}
	\end{tikzpicture}	
	\end{center}
	where we have highlighted the braid in \textcolor{orange}{orange}. Applying the braid move we get the following subheap
	
	\begin{center}
	\begin{tikzpicture}[scale=0.45]
		\heapblock{0}{0}{0}{orange}
		\heapblock{2}{0}{2}{purple}
		\heapblock{1}{2}{1}{orange}
		\heapblock{0}{4}{0}{orange}
		\heapblock{1}{6}{1}{orange}
	\end{tikzpicture}	
	\end{center}
	which does not have any new braids. However, notice that the height of the braid is now $k-1$. This is a contradiction to our original assumption that the heap we started with contains the lowest braid.
	
	Therefore, it follow that $W(\C_n)$ does not contain any not FC $\tII$ elements. 
\end{proof}
\end{theorem}

We will now classify the $\tII$ elements in $W(\C_n)$. We first classify $\tII$ elements in $W(\C_n)$ for $n$ odd and then proceed to the classification for $n$ even.

\begin{theorem}
	If $n$ is odd, then there are no $\tII$ elements in the Coxeter system of type $\C_n$.
	\begin{proof}
		Consider the Coxeter system of type $\C_n$. By Theorem~\ref{thm:TavoidC} we know that $W(\C_n)$ contains no $\tII$ elements that are not FC. Recall $W(\C_n)$ is a star reducible Coxeter group, which implies that $W(\C_n)$ contains no $\tII$ elements that are FC. Thus as $W(\C_n)$ has no $\tII$ elements that are FC and no $\tII$ elements that are not FC, $W(\C_n)$ has no $\tII$ elements.
	\end{proof}
\end{theorem}

We next will classify the $\tII$ elements in the Coxeter system of type $\C_n$ for $n$ even. Recall that $W(\C_n)$ for $n$ even is not a star reducible Coxeter group. In Theorem~\ref{thm:TavoidC} we showed that $W(\C_n)$ does not have $\tII$ elements that are not FC. This leaves us with only the FC elements to check.

\begin{theorem}
	If $n$ is even, then the only $\tII$ elements in $W(\C_n)$ are sandwich stacks.
	\begin{proof}
		Let $w \in W(\C_n)$. By Theorem~\ref{thm:TavoidC}, we know that $w$ is an FC element. Further, we can restrict our search down to the subset of non-cancellable elements that are not star reducible. Specifically we can consider the non-cancellable elements that do not contain Property T. In Remark~\ref{rem:noncancel} we stated the classification of the only $\tII$ elements with full support. Recall these to be sandwich stacks. Thus the only $\tII$ elements in $W(\C_n)$ for $n$ odd are sandwich stacks.
	\end{proof}
\end{theorem}


%%%%%%%%%%%%%%%%%%%%
\section{Future Work}\label{sec:open}
In Sections~\ref{sec:tavoidA}--\ref{sec:tavoidI}, we relayed the known results involving T-avoiding elements in types $\widetilde{A}_n, A_n, D_n, F_4, F_5$, and proved results involving T-avoiding elements in type $I_2(m)$. It remains to be shown that the conjecture in Section~\ref{sec:tavoidA} regarding the classification of the $\tII$ elements in type $\widetilde{A}_n$ holds. The classification of $\tII$ elements in Coxeter systems of type $F_n$ for $n \geq 6$ still remains open.

We also mentioned several other Coxeter systems in Figures~\ref{fig:fincoxgraphs} and~\ref{fig:infincoxgraphs}. The classification of $\tII$ elements in the Coxeter systems of type $E_n$ remains an open problem. However, we do know that these groups have $\tII$ elements as $W(D_n)$ (which has $\tII$ elements) is a parabolic subgroup of $W(E_n)$. The classification of $\tII$ elements in the Coxeter systems of type $H_n$ is also an open problem. 

A majority of the irreducible affine Coxeter systems  currently do not have a classification of the $\tII$ elements. Specifically, Coxeter systems of type $\widetilde{B}_n, \widetilde{D}_n, \widetilde{E}_6, \widetilde{E}_7, \widetilde{E}_8$, and $\widetilde{G}_4$ do not have a classification. Future work could include classifying the $\tII$ elements of the Coxeter systems mentioned above.