\chapter{Property T and T-Avoiding}

\section{Property T and T-Avoiding Elements}\label{Tavoid}

As mentioned in Section~\ref{Star} Green classified all star reducible Coxeter groups. In~\cite{Green2006a}, Green utilizes the following theorem to help classify the star reducible Coxeter groups. 
\begin{theorem}[Green,~\cite{Green2006a}]
	Let $W$ be a star reducible Coxeter group, and let $w \in W$. Then one of the following possibilities occurs for some Coxeter generators s,t, u with $m(s,t) \neq 2$, $m(t,u) \neq 2$, and $m(s,u)=2$:
	\begin{enumerate}
	\item $w$ is a product of commuting generators;\label{it:triv}
	\item $w$ has a reduced expression beginning with $st$;\label{it:proptend}
	\item $w$ has a reduced expression ending in $ts$;\label{it:proptbeg}
	\item $w$ has a reduced expression beginning with $sut$.\label{it:tavoid}	\qed
	\end{enumerate}
\end{theorem}

In the following discussion we will give name to elements that exhibit the property in Item~\ref{it:triv}, as well as name elements that exhibit the properties seen in Items~\ref{it:proptend} and~\ref{it:proptbeg}, and name elements that exhibit the property seen in Item~\ref{it:tavoid}. We will then focus on discussing the irreducible finite and affine Coxeter groups that have elements with properties seen in Items~\ref{it:triv} and~\ref{it:tavoid}.

We first begin by defining the notion of Property T. Let $(W,S)$ be a Coxeter system of type $\Gamma$ and let $w \in W$. We say that $w$ has \emph{Property T} if and only if there exists a reduced product for $w$ such that $w=stu$ or $w=uts$ where $m(s,t)\geq 3$. That is, $w$ has Property T if there exists a reduced expression for $w$ that begins or ends with a product of non-commuting generators. It should be noted that by the symmetry of the definition $w$ has Property T if and only if $w^{-1}$ has Property T.

\begin{example}\label{ex:prop-T}
Let $w \in W(A_5)$ with reduced expression $\overline{w}=s_1s_4s_2s_3s_5$. At first glance it may appear that $w$ does not have Property T, since both $s_1$ and $s_4$ commute as well as $s_3$ and $s_5$. However, note that applying a commutation to $s_4s_2$ results in $\overline{w}_1=s_1s_2s_4s_3s_5$. Hence $w$ has Property T, since $m(s_1,s_2)=3$ and there is a reduced expression for $w$ that begins with $s_1s_2$.	
\end{example}

\begin{example}\label{ex:tavoid}
Let $w \in W(A_5)$ with reduced expression $\overline{w}=s_1s_3s_5$. It turns out that since $w$ is a product of commuting generators there is no reduced expression for $w$ that begins or ends with a pair of non-commuting generators. This implies that $w$ does not have Property T.	
\end{example}

As with star reducible elements it may be helpful to visualize Property T through heaps. Figure~\ref{fig:heapw/T} provides a representation of an element in $W(\Gamma)$ with Property T and Figure~\ref{fig:heapnoT} provides a representation of an element without Property T. Notice that if we were to remove the block for $s_1$ in the bottom row of Figure~\ref{fig:heapw/T}, the heap would become one less row in height and we would have a new bottom row in the heap. However, in Figure~\ref{fig:heapnoT}, we are not able to remove able to remove any bricks and have a new brick come to the top or bottom row as the heap is just one row. From this we can gather that when we are using heaps to visualize whether or not an element of $W(\Gamma)$ has Property T we must observe either of the following things. The first of these is that the heap decreased in height. That is, there is one less row than the original heap. The second thing that we could observe is that when we remove a block from the heap a new element, that was originally trapped in the second (respectively, second to last) row is now able to move into the first (respectively, last) row of the heap. This implies that an element that does not have Property-T, has every element in the second and second to last row of the heap blocked by two blocks in the first and last rows of the heap, as this would imply that we could not remove a block and have a new element come to the first or last row as it would still be blocked by the other element that remained.

\begin{figure}[h!]
\begin{tabular}{m{7cm} m{7cm}}
\begin{subfigure}{0.5\textwidth} \centering
\begin{tikzpicture}[scale=0.5]
\heapblock{5}{6}{5}{purple}
\heapblock{3}{6}{3}{purple}
\heapblock{2}{4}{2}{orange}
\heapblock{4}{4}{4}{purple}
\heapblock{1}{2}{1}{orange}
\end{tikzpicture}
\caption{Heap of an element with Property T} \label{fig:heapw/T}	
\end{subfigure}&

\begin{subfigure}{0.5\textwidth} \centering
\begin{tikzpicture}[scale=0.5]
\heapblock{1}{6}{1}{purple}
\heapblock{3}{6}{3}{purple}
\heapblock{5}{6}{5}{purple}
\end{tikzpicture}
\caption{Heap of a T-Avoiding element}\label{fig:heapnoT}
\end{subfigure}
\end{tabular}
\caption{Heaps of an element with Property T and a T-Avoiding element}
\end{figure}

An element $w \in W(\Gamma)$ is called \emph{T-avoiding} if $w$ and $w^{-1}$ do not have Property T. We will call an element that is a product of commuting generators \emph{trivially T-avoiding}. The reason behind this comes in the following theorem. If $w$ is T-avoiding and not a product of commuting generators, we will say that $w$ is \emph{non-trivially T-avoiding.} 

\begin{theorem}
Let $(W,S)$ be a Coxeter system of type $\Gamma$ and let $w \in W(\Gamma)$ such that $w$ is a product of commuting generators. Then $w$ is T-avoiding. \qed	
\end{theorem}

Visually this is seen in Example~\ref{fig:heapnoT} elements in $W(\Gamma)$ that are products of commuting generators are always going to be one row in the heap. This implies that we are not able to remove generators and have elements come into rows that they were previously in as the heap is only one row and there can be no lateral movement when we remove bricks. 

\begin{example}
Let $w \in W(A_5)$ and let $\overline{w}=s_1s_3s_5$. Then by Example~\ref{ex:tavoid}, we know that $w$ is T-avoiding and since $w$ is a product of commuting generators, $w$ is trivially T-avoiding.	
\end{example}

\begin{example}
Let $w \in W(\widetilde{C}_4)$ with reduced expression $\overline{w}=s_0s_2s_4s_1s_3s_0s_2s_4$. It turns out that $w$ is FC and non-trivially T-avoiding. The heap for $w$ is seen if Figure~\ref{fig:sandwich1}. Notice that no matter which block we remove from the top row of $w$ no new element can be pushed into the topmost row. The same applies to the bottom row. Hence there is not a single block that can be removed from the top that allows a new element to come into the top row and no single block that can be removed from the bottom row to allow a new element to come into the bottom row. Thus, $w$ is non-trivially T-avoiding. 
\begin{figure}[h!]
\centering
\begin{tikzpicture}[scale=0.5]
\heapblock{0}{6}{0}{purple}
\heapblock{2}{6}{2}{purple}
\heapblock{4}{6}{2}{purple}
\heapblock{1}{4}{1}{purple}
\heapblock{3}{4}{3}{purple}
\heapblock{0}{2}{0}{purple}
\heapblock{2}{2}{2}{purple}
\heapblock{4}{2}{4}{purple}
\end{tikzpicture}
\caption{Heap of a non-trivially T-Avoiding element in $W(\widetilde{C}_4)$.}\label{fig:sandwich1}	
\end{figure}
\end{example}

Referring back to Green's classification of what elements in star reducible Coxeter groups look like, we see that Item~\ref{it:triv} corresponds to the element $w$ being trivially T-avoiding, Items~\ref{it:proptend} and~\ref{it:proptbeg} refer to the element $w$ having Property T at the beginning and end respectively, and Item~\ref{it:tavoid} corresponds to the element $w$ being T-avoiding. Notice that this implies that some Coxeter groups will have elements that are non-trivially T-avoiding while others may not. For example, as will be seen in the following sections, the Coxeter groups of type $A_n$ and $B_n$ have no non-trivial T-avoiding elements, while the Coxeter group of type $D_n$ does have non-trivial T-avoiding elements. One interesting thing to note about star reducible Coxeter groups is that if they contain non-trivially T-avoiding elements, these elements will not be FC. This is because FC elements in star reducible Coxeter groups are star reducible and star reducible elements in the way in which we have defined them are analogous to the element having Property-T.

One thing to notice here is that all Coxeter groups have trivial T-avoiding elements as they all contain products of commuting generators. The more interesting non-trivial T-avoiding elements do not appear in all Coxeter groups. The remainder of this thesis discusses what is currently know regarding T-avoiding elements in the irreducible finite Coxeter groups and the irreducible affine Coxeter groups, as well as it classifies the T-avoiding elements in Coxeter groups of types $B_n$ and $\widetilde{C}_n$. In the next few sections we will summarize what is known about the T-avoiding elements in Coxeter groups of types $\widetilde{A}_n$, $A_n$, $D_n$, $F_n$ and $I_2(m)$.


%%%%%%%%%%%%%%%%%%%%%%%%%%

\section{T-Avoiding Elements in Types $\widetilde{A}_n$ and $A_n$}
We start by classifying the T-avoiding elements in Coxeter groups of type $\widetilde{A}_n$ and $A_n$. Specifically we will classify the T-avoiding elements in $W(A_n)$ and then move to $W(\widetilde{A}_n)$. Clearly $W(A_n)$ contains trivial T-avoiding elements and $W(A_n)$ contains products of commuting generators. Now we will classify the non-trivial T-avoiding elements in $W(A_n)$. Recall that $W(A_n)$ is a star reducible Coxeter group which by our observations above implies that if $W(A_n)$ has non-trivial T-Avoiding elements, these elements will be not FC. In~\cite{Fan1999}, the non-trivial T-avoiding elements were classified. The classification is seen in the following theorem.

\begin{theorem}
Then there are no non-trivially T-avoiding elements in $W(A_n)$. 
\begin{proof}
This is a consequence of~\cite[Proposition 3.1.2.]{Fan1999}.	 
\end{proof}
\end{theorem}

We will now proceed into the classification of T-avoiding elements in $W(\widetilde{A}_n)$. Similar to $W(A_n)$, $W(\widetilde{A}_n)$ will also have trivial T-avoiding elements since $W(\widetilde{A}_n)$ contains elements that are products of commuting generators. We will now proceed to classifying non-trivial T-avoiding elements in $W(\widetilde{A_n})$. The following classification is a result from~\cite{Fan1999}.

\begin{theorem}
 If $n \geq 2$ and $n$ is even, then there are no non-trivial T-avoiding elements in $W(\widetilde{A}_n)$. Otherwise, if $n \geq 2$ and $n$ is even then $W(\widetilde{A}_n)$ contains non-trivial T-avoiding elements.
\begin{proof}
	This is~\cite[Proposition~3.1.2.]{Fan1999}.
\end{proof}
\end{theorem}

This implies that in the Coxeter group of type $\widetilde{A}_n$, when $n$ is even there are no non-trivial T-avoiding elements, while when $n$ is odd, there are non-trivial T-avoiding elements. One interesting observation to go along with this is $W(\widetilde{A}_n)$ is star reducible for $n$ even but not star reducible for $n$ odd. The classification seen in~\cite{Fan1999} did not specifically classify the non-trivial T-avoiding elements for type $\widetilde{A}_n$ for $n$ odd. Before we state our conjecture as to what the non-trivial T-Avoiding elements are in $W(\widetilde{A}_n)$ we will discuss shortly what the heaps look like in $W(\widetilde{A}_n)$. Because the Coxeter graph for type $\widetilde{A}_n$ is not a straight line Coxeter graph, the heaps can no longer be 2 dimensional and now must take on a 3 dimensional representation. The nature of the Coxeter graph of type $\widetilde{A}_n$ leads us to represent heaps in the sense of a turret on a castle where the blocks are stacked in a circular manner. We now continue with our conjectured classification. Because type $A_n$ does not contain any non-trivial T-avoiding elements we know that the non-trivial T-avoiding elements in $W(\widetilde{A}_n)$ for $n$ odd must have full support. We conjecture that the only non-trivial T-avoiding elements are castle turrets that have no missing blocks in the walls. However, this remains an open problem.

%%%%%%%%%%%%%%%%

\section{T-Avoiding Elements in Type $D_n$}

In this section we will classify the T-avoiding elements in the Coxeter group of type $D_n$. We begin with trivial T-avoiding elements. Note that like $W(A_n)$, $W(D_n)$ has elements that are products of commuting generators. Hence, $W(D_n)$ contains trivial T-avoiding elements. We will now classify the non-trivial T-avoiding elements of Coxeter groups of type $D_n$. Recall that $W(D_n)$ is a star reducible Coxeter group and as a result of this any non-trivial T-avoiding element will not be fully commutative.

\begin{theorem}
Let $W=W(D_n)$. Then $W$ contains non-trivial T-avoiding elements.
\begin{proof}
	This is a consequence of~\cite[Section 2.2]{Gern2013a}.
\end{proof}
\end{theorem}

In addition, to showing that there are non-trivial T-avoiding elements in type $D_n$ Gern also classified the non-trivial T-avoiding elements as well. The following is his classification translated into heaps. \textcolor{red}{Once we figure out the heap include it here.} For the full details regarding his classification see~\cite{Gern2013a}. Note that in his classification Gern  refers to non-trivially T-avoiding elements as ''bad."

%%%%%%%%%%%%%%%

\section{T-Avoiding Elements in Type $F_n$}

In this section we classify what is known regarding the non-trivial T-avoiding elements in the Coxeter groups of type $F_n$ for $n \geq 4$. Note that all of the following results are unpublished. First as with all other types we have seen so far, $F_n$ has trivial T-avoiding elements since the groups contain products of commuting generators. For the rest of this section we will focus on the non-trivial T-avoiding elements in $F_n$. Recall that $F$ is a star reducible Coxeter group so any non-trivial T-avoiding element in $F_n$ will not be FC.

We start with the Coxeter group of type $F_5$. In 2012, Cross, Ernst, Hills-Kimball, and Quaranta classified all non-trivial T-avoiding elements in the following theorem.

\begin{theorem}
An element in $F_5$ is non-trivially T-avoiding if and only if it is a stack of bowties. \qed	
\end{theorem}

The above theorem references a stack of bowties. This refers to what the heap of the given non-trivial T-avoiding element looks like. We first restrict our attention to a single bowtie, this is seen in Figures~\ref{}. Note that in this figure, the orange blocks correspond to the elements that have bond strength 4 (Figure~\ref{}). Since the element is not FC we also wished to highlight the braid which is seen in the center of the element as colored in orange in Figure~\ref{}. In stacking the single bowties together, we get the stack of bowties referenced in the theorem which is seen in Figure~\ref{}. As a result of the classification in $F_5$, Cormier et. al. were also able to classify the non-trivial T-avoiding elements in $F_4$. The following is their classification.

\begin{corollary}
There are no non-trivial T-avoiding elements in $F_4$. \qed	
\end{corollary}

As a result of their work, Cross et. al. conjectured that in Coxeter groups of type $F_n$ for $n \geq 5$, an element is non-trivially T-avoiding if and only if it is a stack of bowties multiplied by a product of commuting generators. In 2013, Gilbertson and Ernst worked with this conjecture and quickly found out that it was incorrect. The heap seen in Figure~\ref{} corresponds to a non-trivial T-avoiding element in $F_6$. It turns out that like the bowties discussed above these elements can also be stacked to create and infinite number of non-trivial T-avoiding elements as well. In addition, as $n$ gets large there are a number of things that can be altered in the group element that result in the element being non-trivially T-avoiding. This leads us to believe that there are potentially even more non-trivially T-avoiding elements in $F_n$ for $n \geq 6$. From this we can see that the classification of T-avoiding elements in $F_n$ for $n \geq 6$ gets complicated very quickly. Hence, classifying T-avoiding elements in $F_n$ for $n \geq 5$ remains an open problem.

%%%%%%%%%%%%%%%%

\section{T-Avoiding Elements in Type $I_2(m)$}

We next will classify the T-avoiding elements in Coxeter groups of type $I_2(m)$. Differing from the preceding classifications, $W(I_2(m))$ has exactly 2 trivial T-avoiding elements as $W(I_2(m))$ contains only 2 generators and as a result the only products of commuting generators are single generators. We now classify the non-trivial T-avoiding elements in $W(I_2(m))$. Although the following is a quick result, we believe that the result does not already appear in literature.
\begin{theorem}
The Coxeter group $W(I_2(m))$ has no non-trivial T-avoiding elements.
\begin{proof}
	The graph for the Coxeter group of type $I_2(m)$ appears in Figure~\ref{fig:I}. Note that the graph consists of two 2 vertices, $s_1,s_2$, and an edge with weight $m(s,t)$. First recall, that $W(I_2(m))$ is a star reducible Coxeter group. This implies that any non-trivial T-avoiding elements in $W(I_2(m))$ must not be FC. As all of the FC elements have Property T. We now consider the case when $w \in W(I_2(m))$ is not FC. Since $s_1$ and $s_2$ are involutions this implies that the elements in $W$ must alternate between $s_1$ and $s_2$. Since $w$ is not FC this implies that a reduced expression for $w$ is $\overline{w}=\underbrace{s_1s_2 \cdots s_1}_{m(s_1,s_2)}$. By definition $m(s_1,s_2)>3$ and hence all elements in $W(I_2(m))$ that are not products of commuting generators have Property T, as they clearly have a reduced expression that begins with a product of non-commuting generators. Hence $W(I_2(m))$ has no non-trivial T-avoiding elements. 
\end{proof}	
\end{theorem}
 



