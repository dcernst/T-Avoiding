\chapter{Star Operations and Property T}


\section{Star Operations}\label{sec:star}

The notion of a star operation was originally introduced by Kazhdan and Lusztig in~\cite{Kazhdan1979} for simply-laced Coxeter systems (i.e., $m(s,t) \leq 3$ for all $s,t \in S$), and was later generalized to all Coxeter systems in~\cite{Lusztig1985}. If $I=\{s,t\}$ is a pair of non-commuting generators of a Coxeter group $W$, then $I$ induces four partially defined maps from $W$ to itself, known as \emph{star operations}. A star operation, when it is defined, increases or decreases the length of an element to which it is applied by 1. For our purposes it is enough to only define the star operations that decrease the length of an element by 1, and as a result we will not develop the notion in full generality.

Let $(W,S)$ be a Coxeter system of type $\Gamma$ and let $I=\{s,t\}\subseteq S$ be a pair of generators with $m(s,t) \geq 3$. Let $w \in W(\Gamma)$ such that $s \in \mathcal{L}(w)$. We define $w$ to be \emph{left star reducible by $s$ with respect to $t$} if there exists $t \in \mathcal{L}(sw)$. We analogously define $w$ to be \emph{right star reducible by $s$ with respect to $t$}. Observe that $w$ is left (respectively, right) star reducible if and only if $w=stu$ (respectively, $w=uts$), where the product on the right hand side of the equation is reduced and $m(s,t) \geq 3$. We say that $w$ is \emph{star reducible} if it is either left or right star reducible.

\begin{example}\label{ex:starred}
Let $\w=s_0s_1s_0s_2$ be a reduced expression for $w \in W(B_4)$. We see that $w$ is left star reducible by $s_0$ with respect to $s_1$ to $s_1s_0s_2$ since $m(s_0,s_1)=4$ and $s_0 \in \mathcal{L}(w)$ while $s_1 \in \mathcal{L}(s_0w)$. Notice that $w$ if FC. We see that $ws_2=s_0s_1s_0$ and $ws_0=s_0s_1s_2$. Note that in both instances $s_1 \notin \RD(ws_2)$ and $s_1 \notin \LD(ws_0)$. Because of this $w$ is not right star reducible. 
\end{example}

It may be helpful to visualize star reductions in terms of heaps. Let $(W,S)$ be a Coxeter system of type $\Gamma$ and let $I=\{s,t\}\subseteq S$ be a pair of generators with $m(s,t) \geq 3$. Suppose $w$ is left star reducible by $s$ with respect to $t$. Then there exists a heap for $w$ where the block for $s$ is fully exposed to the top. Removing the block for $s$ off of the top allows for $t$ to now be fully exposed to the top. Similarly if $w$ is right star reducible by $s$ with respect to $t$, then there exists a heap for $w$ where the block for $s$ is fully exposed to the bottom. Removing the block for $s$ off of the bottom allows for $t$ to now be fully exposed to the bottom. Conversely, if a heap for $w \in W(\Gamma)$ has this property, then $w$ is star reducible. In Figure~\ref{fig:heapwithT} we see the heap representation of an element that is left star reducible, where the dotted brick represents that there can not be an element there. Notice that flipping the heap upside in Figure~\ref{fig:heapwithT} will result in a heap being right star reducible. It is important to note that if the group element we are evaluating for Property T is not FC, then we must consider all heap representations for the element before concluding that an element does not have Property T. 

\begin{figure*}[h!]
\begin{tabular}{m{7cm} m{7cm}}
\begin{subfigure}{0.5\textwidth} \centering
\begin{tikzpicture}[scale=0.5]
	\dheapblock{2}{2}{}{black}
	\heapblock{0}{2}{s}{purple}
	\heapblock{1}{0}{t}{purple}
\end{tikzpicture}
\caption{}
\end{subfigure} &

\begin{subfigure}{0.5\textwidth} \centering
\begin{tikzpicture}[scale=0.5]
	\dheapblock{3}{2}{}{black}
	\heapblock{1}{2}{s}{purple}
	\heapblock{2}{0}{t}{purple}
\end{tikzpicture}
\caption{}	
\end{subfigure}
\end{tabular}
\caption{A visual representation of an element with Property T at the top.}\label{fig:heapwithT}
\end{figure*} 

The following example utilizes heaps to show that an element is star reducible.

\begin{example}\label{ex:starredheap}
Let $\w=s_0s_1s_0s_2$ be a reduced expression for $w \in W(B_4)$. Note that $w$ is FC. By Example~\ref{ex:starred} we know that $w$ is left star reducible by $s_0$ with respect to $s_1$. In Figure~\ref{fig:heapy}, we see the heap for $w$. Notice that the block for $s_0$ is fully exposed to the top of the heap. Removing the block for $s_0$ gives the heap in Figure~\ref{fig:multiplied}. Notice that the block for $s_1$ is now fully exposed to the top of the heap. However, notice that the block for $s_0$ and $s_2$ are fully exposed to the bottom. In removing either of these we are unable to fully expose $s_1$ to the bottom. Thus we can see that $w$ is not right star reducible.  
\end{example}


%It may be helpful to visualize star reductions in terms of heaps. Figure~\ref{fig:heapy} represents $H(\w)$. Note that we can see $s_0$ is in the left descent set of $w$ since $s_0$ is in the top row of the heap. Furthermore, multiplying on the left by $s_0$ we get the heap in Figure~\ref{fig:multiplied}. Again, since $s_1$ is in the top row of the heap, $s_1 \in \mathcal{L}(s_1w)$. In Figure~\ref{fig:heapy} we also see that $s_3$ is in the right descent set of $w$ since $s_3$ is in the bottom row of the heap. Multiplying on the left by $s_3$ we can see that $s_2$ would be in the bottom level of the heap so $s_2 \in \mathcal{R}(ws_3)$. From this we can interpret visually an element $w \in W(\Gamma)$ is right star reducible (respectively, left star reducible) if there exists a heap that we can pull a block off the bottom row of the heap (respectively, top of the heap) and a new block that wasn't previously in the bottom row (respectively, top row) is now in the bottom row (respectively, top row) of the heap. That is, we can systematically dismantle the heap for a given element by pulling blocks off of the bottom row (respectively, top row) of the heap and have the heap decrease in height, meaning there are fewer rows of the heap than there were in the original, or a block that was previously trapped in the second (respectively, second to last) row of the heap is now free to be in either the first or second (respectively, second to last or last) row. If the heap has the same number of rows as the original and if now blocks that were in the second or second to last row of the heap can now be in the first or last row of the heap when we attempt pulling a single block off the top and a single block off the bottom and we try all possible combination of single blocks in the top row and bottom row, then the heap is not star reducible.

\begin{figure}[h!]
\begin{tabular}{m{7cm} m{7cm}}
\begin{subfigure}{0.5\textwidth} \centering
\begin{tikzpicture}[scale=0.5]
\heapblock{2}{2}{2}{purple}
\heapblock{0}{2}{0}{purple}
\heapblock{1}{4}{1}{purple}
\heapblock{0}{6}{0}{purple}
\end{tikzpicture}
\caption{} \label{fig:heapy}
\end{subfigure} &

\begin{subfigure}{0.5\textwidth} \centering
\begin{tikzpicture}[scale=0.5]
\heapblock{2}{2}{2}{purple}
\heapblock{0}{2}{0}{purple}
\heapblock{0}{6}{}{white}
\heapblock{1}{4}{1}{purple}
\end{tikzpicture}
\caption{} \label{fig:multiplied}
\end{subfigure}
\end{tabular}
\caption{Visualization of Example~\ref{ex:starred}.}
\label{fig:starred}
\end{figure}

Notice that if $w$ is not FC then we are not be able to say that $w$ is not star reducible as there could be a different heap for $w$ in which we are able to fully expose an element that was previously blocked.

\begin{example}
Let $\w_1=s_3s_1s_2s_1s_0s_1s_3s_0s_2s_4$ be a reduced expression for $w \in W(\C_3)$. The heap for $\w$ is given in Figure~\ref{fig:starrednfc1} where we have highlighted the braid in \textcolor{teal}{teal}. Notice that this heap appears to not be star reducible as if we were to remove the brick for $s_1$ or $s_3$ we would still not fully expose $s_2$ to the top of the heap. The same goes for fully exposing bricks in the bottom of the heap. However, when we perform the braid operation resulting in the heap seen in Figure~\ref{fig:starrednfc2} it is now obvious that the element is star reducible. Thus when considering a non-FC element for star reducibility via the heap, it is very important to consider all heaps for that element.

\begin{figure}[h!]
\begin{tabular}{m{7cm} m{7cm}}
\begin{subfigure}{0.5\textwidth} \centering
\begin{tikzpicture}[scale=0.5]
	\heapblock{1}{10}{1}{teal}
	\heapblock{3}{10}{3}{purple}
	\heapblock{2}{8}{2}{teal}
	\heapblock{1}{6}{1}{teal}
	\heapblock{0}{4}{0}{purple}
	\heapblock{1}{2}{1}{purple}
	\heapblock{3}{2}{3}{purple}
	\heapblock{0}{0}{0}{purple}
	\heapblock{2}{0}{2}{purple}
	\heapblock{4}{0}{2}{purple}	
\end{tikzpicture}
\caption{}\label{fig:starrednfc1}	
\end{subfigure}&

\begin{subfigure}{0.5\textwidth} \centering
\begin{tikzpicture}[scale=0.5]
	\heapblock{2}{10}{2}{teal}
	\heapblock{3}{12}{3}{purple}
	\heapblock{1}{8}{1}{teal}
	\heapblock{2}{6}{2}{teal}
	\heapblock{0}{6}{0}{purple}
	\heapblock{1}{4}{1}{purple}
	\heapblock{3}{4}{3}{purple}
	\heapblock{0}{2}{0}{purple}
	\heapblock{2}{2}{2}{purple}
	\heapblock{4}{2}{2}{purple}	
\end{tikzpicture}
\caption{}\label{fig:starrednfc2}	
\end{subfigure}
\end{tabular}
\caption{Visualization of Example 2.1.3}\label{fig:starrednfc}
\end{figure}
\end{example}


Using the notion of star reduction we are now able to introduce the concept of a star reducible Coxeter group. We say that a Coxeter group $W(\Gamma)$, or its Coxeter graph $\Gamma$, is \emph{star reducible} if every element of $\FC(\Gamma)$ is star reducible to a product of commuting generators. That is, $W(\Gamma)$ is star reducible if when we apply star reductions repeatedly to $w \in \FC(\Gamma)$, eventually we obtain a product of commuting generators. Notice that in the previous definition we are restricting the elements that must be star reducible to a product of commuting generators to just the FC elements in $W(\Gamma)$. Visually a star reducible Coxeter group can be thought of in the following way, given a heap in $\FC(\Gamma)$, we are able to systematically remove a fully exposed block from the top or bottom of the heap and have a block that was previously not fully exposed become fully exposed until we are left with a heap that is one row in height. 

In~\cite{Green2006a}, Green classified all star reducible Coxeter groups.

%we are able to pull the top or bottom most block off of the heap and have the heap decrease in height (the number of rows that it consists of) or have a new block come into the top or bottom most rows. We can perform this pulling of blocks off the top and bottom of the heap systematically until the heap consists of one row, corresponding to an element that is a product of commuting generators. For example in Figure~\ref{fig:heapy}, we were able to apply a star reduction to remove the topmost block corresponding to the generator $s_0$ and obtain the new heap seen in Figure~\ref{fig:multiplied}. We see that the heap in Figure~\ref{fig:multiplied} has one less row than Figure~\ref{fig:heapy}. We could do the same with the block on bottom of Figure~\ref{fig:multiplied} corresponding to $s_3$ and have the heap again decrease in the number of rows it has. Performing one more star reduction in pulling the brick off the top of Figure~\ref{fig:multiplied} which corresponds to $s_1$ in the reduced expression would leave us with a heap that is only 1 row which is a product of commuting generators. Thus Figure~\ref{fig:heapy} can be star reduced to a product of commuting generators.  In~\cite{Green2006a}, Green classified all star reducible Coxeter groups. 
%The Coxeter groups $W(A_n)$, $W(B_n)$ and $W(\widetilde{C}_n)$ are star reducible. However, $W(A_n)$ and $W(B_n)$ don't have non-trivial T-avoiding elements, while $W(\widetilde{C}_n)$ in one parity does have non-trivial T-avoiding elements.
\begin{theorem}[Green,~\cite{Green2006a}]
	Let $(W,S)$ be a Coxeter system of type $\Gamma$. Then $(W,S)$ is star reducible if and only if each component of $\Gamma$ is either a complete graph with labels $m(s,t)\geq 3$, or is one of the following types: type $A_n$ $(n \geq 1)$, type $B_n$ $(n \geq 2)$, type $D_n$ $(n \geq 4)$, type $F_n$ $(n \geq 4)$, type $H_n$ $(n \geq 2)$, type $I_2(m)$ $(m \geq 3)$, type $\widetilde{A}_{n-1}$ $(n \geq 3 \textrm{ and } n \textrm{ odd })$, type $\widetilde{C}_{n-1}$ $(n\geq 4 \textrm{ and } n \textrm{ even })$, type $\widetilde{E}_6$, or type $\widetilde{F}_5$. \qed
\end{theorem}    

%%%%%%%%%%%%%%%%%

\section{Property T}\label{Tavoid}

As previously mentioned Green classified all star reducible Coxeter groups. In~\cite{Green2006a}, Green utilizes the following theorem to help classify the star reducible Coxeter groups. 
\begin{theorem}[Green,~\cite{Green2006a}]\label{thm:starred}
	Let $(W,S)$ be a star reducible Coxeter system of type $\Gamma$, and let $w \in W$. Then one of the following possibilities occurs for some Coxeter generators s,t, u with $m(s,t) \neq 2$, $m(t,v) \neq 2$, and $m(s,u)=2$:
	\begin{enumerate}[leftmargin=2cm]
	\item $w$ is a product of commuting generators;\label{it:triv}
	\item $w$ has a reduced product $w=stu$;\label{it:proptend}
	\item $w$ has a reduced product $w=uts$;\label{it:proptbeg}
	\item $w$ has a reduced product $w=svtu$.\label{it:tavoid}	\qed
	\end{enumerate}
\end{theorem}

In the following discussion we will give name to elements that exhibit the properties above. However, first notice that Items~\ref{it:proptend} and~\ref{it:proptbeg} refer to an element that is star reducible. While Item~\ref{it:tavoid} refers to an element that is not star reducible provided no reduced expression for the element exhibits Items~\ref{it:proptend} and~\ref{it:proptbeg}.

We first begin by defining the notion of Property T which is motivated by Items~\ref{it:proptend} and~\ref{it:proptbeg} above. Let $(W,S)$ be a Coxeter system of type $\Gamma$ and let $w \in W$. We say that $w$ has \emph{Property T} if and only if there exists a reduced product for $w$ such that $w=stu$ or $w=uts$ where $m(s,t)\geq 3$. That is, $w$ has Property T if there exists a reduced expression for $w$ that begins or ends with a product of non-commuting generators.

Recall the definition of star reducible from Section~\ref{sec:star}. Notice that if an element in a Coxeter group is star reducible, then it has Property T. The difference be between the two definitions is that when we call an element star reducible we are referring to being able to dismantle the heap in a specified manner. Whereas, when we refer to an element having Property T, we are referring to the element having a pair of non-commuting generators in the top or bottom of the heap. Since elements that are star reducible also have Property T we already know how to visualize Property T in terms of heaps. Recall that an element is star reducible if we can remove a fully exposed block from the top or bottom of a heap and have a new block become fully exposed. 

An element $w \in W(\Gamma)$ is called \emph{T-avoiding} if $w$ does not have Property T. This implies that a T-avoiding element is not star reducible.

\begin{theorem}\label{thm:trivTavoid}
Let $(W,S)$ be a Coxeter system of type $\Gamma$. If $w \in W(\Gamma)$ such that $w$ is a product of commuting generators, then $w$ is T-avoiding. \qed	
\end{theorem}

We will call an element that is a product of commuting generators \emph{trivially T-avoiding}. It is clear that a product of commuting generators is T-avoiding, which we state as a theorem. Visually a product of commuting generators is a one row heap (as in Figure~\ref{fig:heapnoT}), it is clear a one row heap will not portray the characteristic of Property T as seen in Figure~\ref{fig:heapwithT}. If $w$ is T-avoiding and not a product of commuting generators, we will say that $w$ is \emph{non-trivially T-avoiding.} 

%Visually this is seen in Example~\ref{fig:heapnoT} elements in $W(\Gamma)$ that are products of commuting generators are always going to be one row in the heap. This implies that we are not able to remove generators and have elements come into rows that they were previously in as the heap is only one row and there can be no lateral movement when we remove bricks. 

\begin{example}\label{ex:tavoid}
Let $w \in W(A_5)$ with reduced expression $\w=s_1s_3s_5$. It turns out that since $w$ is a product of commuting generators by Theorem~\ref{thm:trivTavoid} we know that $w$ is trivially T-avoiding.	
\end{example}

\begin{example}\label{ex:prop-T}
Let $w \in W(A_5)$ with reduced expression $\w_1=s_1s_4s_2s_3s_5$. At first glance it may appear that $w$ does not have Property T, since both $s_1$ and $s_4$ commute as well as $s_3$ and $s_5$. However, note that applying a commutation move $s_4s_2 \mapsto s_2s_4$ results in $\w_2=s_1s_2s_4s_3s_5$. Hence $w$ has Property T, since $m(s_1,s_2)=3$ and there is a reduced expression for $w$ that begins with $s_1s_2$.	
\end{example}



%As with star reducible elements it may be helpful to visualize Property T through heaps. Figure~\ref{fig:heapw/T} provides a representation of an element in $W(\Gamma)$ with Property T and Figure~\ref{fig:heapnoT} provides a representation of an element without Property T. Notice that if we were to remove the block for $s_1$ in the bottom row of Figure~\ref{fig:heapw/T}, the heap would become one less row in height and we would have a new bottom row in the heap. However, in Figure~\ref{fig:heapnoT}, we are not able to remove able to remove any bricks and have a new brick come to the top or bottom row as the heap is just one row. From this we can gather that when we are using heaps to visualize whether or not an element of $W(\Gamma)$ has Property T we must observe either of the following things. The first of these is that the heap decreased in height. That is, there is one less row than the original heap. The second thing that we could observe is that when we remove a block from the heap a new element, that was originally trapped in the second (respectively, second to last) row is now able to move into the first (respectively, last) row of the heap. This implies that an element that does not have Property-T, has every element in the second and second to last row of the heap blocked by two blocks in the first and last rows of the heap, as this would imply that we could not remove a block and have a new element come to the first or last row as it would still be blocked by the other element that remained.

\begin{example}
Let $\w_1=s_1s_4s_2s_3s_5$ be a reduced expression for $w_1 \in W(A_5)$ as seen in Example~\ref{ex:prop-T}	and let $\w_2=s_1s_3s_5$ be a reduced expression for $w_2 \in W(A_5)$. In Figure~\ref{fig:heapw/T} we see the heap for $w_1$. Note that we can see Property T in the bottom of the heap highlighted in \textcolor{orange}{orange}. In Figure~\ref{fig:heapnoT} we see the heap for $w_2$. Note that as the heap is only one row and $w_2$ is FC, it is clear that $w_2$ does not have Property T.
\begin{figure}[h!]
\begin{tabular}{m{7cm} m{7cm}}
\begin{subfigure}{0.5\textwidth} \centering
\begin{tikzpicture}[scale=0.5]
\heapblock{5}{6}{5}{purple}
\heapblock{3}{6}{3}{purple}
\heapblock{2}{4}{2}{orange}
\heapblock{4}{4}{4}{purple}
\heapblock{1}{2}{1}{orange}
\end{tikzpicture}
\caption{Heap of an element with Property T} \label{fig:heapw/T}	
\end{subfigure}&

\begin{subfigure}{0.5\textwidth} \centering
\begin{tikzpicture}[scale=0.5]
\heapblock{3}{4}{}{white}
\heapblock{3}{8}{}{white}
\heapblock{1}{6}{1}{purple}
\heapblock{3}{6}{3}{purple}
\heapblock{5}{6}{5}{purple}
\end{tikzpicture}
\caption{Heap of a T-Avoiding element}\label{fig:heapnoT}
\end{subfigure}
\end{tabular}
\caption{Heaps of an element with Property T and a T-Avoiding element}\label{fig:prptheaps}
\end{figure}
\end{example}




\begin{example}
Let $w \in W(\widetilde{C}_4)$ with reduced expression $\w=s_0s_2s_4s_1s_3s_0s_2s_4$. It turns out that $w$ is FC and non-trivially T-avoiding. The heap for $w$ is seen in Figure~\ref{fig:sandwich1}. Notice that no matter which block we remove that is fully exposed to the top of the heap no new element becomes fully exposed. The same applies to the bottom of the heap. Thus, $w$ is non-trivially T-avoiding. 
\begin{figure}[h!]
\centering
\begin{tikzpicture}[scale=0.5]
\heapblock{0}{6}{0}{purple}
\heapblock{2}{6}{2}{purple}
\heapblock{4}{6}{2}{purple}
\heapblock{1}{4}{1}{purple}
\heapblock{3}{4}{3}{purple}
\heapblock{0}{2}{0}{purple}
\heapblock{2}{2}{2}{purple}
\heapblock{4}{2}{4}{purple}
\end{tikzpicture}
\caption{Heap of a non-trivially T-Avoiding element in $W(\widetilde{C}_4)$.}\label{fig:sandwich1}	
\end{figure}
\end{example}

Referring back to Green's classification (Theorem~\ref{thm:starred}) of what elements in star reducible Coxeter groups look like, we see that Item~\ref{it:triv} corresponds to an element $w$ being trivially T-avoiding, Items~\ref{it:proptend} and~\ref{it:proptbeg} refer to the element $w$ having Property T at the beginning and end respectively, and Item~\ref{it:tavoid} refers to an element being non-trivially T-avoiding provided no reduced expression for the element exhibits Items~\ref{it:proptend} and~\ref{it:proptbeg}. In star reducible Coxeter groups, every FC element is star reducible to a product of commuting generators, which implies that no FC element can be non-trivially T-avoiding. For example, as will be seen in the following sections, the Coxeter groups of type $A_n$ and $B_n$ have no non-trivial T-avoiding elements, while the Coxeter group of type $D_n$ does have non-trivial T-avoiding elements. 



%We will now extend the definition of Property T to Coxeter groups. Let $(W,S)$ be a Coxeter group of type $\Gamma$. We say that $W(\Gamma)$ has \emph{Property T} if all elements in $W(\Gamma)$ that are not the products of commuting generators have Property T. It is clear that if a Coxeter group has Property T, then it also has Property S. It remains an open question whether or not a Coxeter group with Property S, also has Property T, but we conjecture this is true. If this is the case, then by a remark following the definition of Property S in~\cite{Green2007}, the classification of the T-avoiding elements for any connected, nonbranching Coxeter graph of finite or affine type, except the Coxeter system of type $\widetilde{F}_4$, would be complete. 

One thing to notice here is that all Coxeter groups have trivial T-avoiding elements as they all contain products of commuting generators. As a result of this we will avoid mentioning them in the following classification. The more interesting non-trivial T-avoiding elements do not appear in all Coxeter groups. Chapters~\ref{chap:TandTavoid} and~\ref{chap:BnandCn}  discusse what is currently known regarding T-avoiding elements in the irreducible finite Coxeter groups and the irreducible affine Coxeter groups. In Chapter~\ref{chap:TandTavoid} we will summarize what is known about the T-avoiding elements in Coxeter groups of types $\widetilde{A}_n$, $A_n$, $D_n$, $F_n$, and $I_2(m)$, and in Chapter~\ref{chap:BnandCn} we classify the T-avoiding elements in Coxeter groups of types $B_n$ and $\widetilde{C}_n$. 


%%%%%%%%%%%%%%


\section{Non-Cancellable Elements}\label{sec:noncancel}
 
We now introduce the concept of weak star reducible, which is related to the notion of cancellable in~\cite{Fan1997}. Let $(W,S)$ be a Coxeter system of type $\Gamma$ and let $I=\{s,t\} \subseteq S$ be a pair of noncommuting generators. If $w  \in \FC(\Gamma)$, then $w$ is \emph{left weak star reducible by $s$ with respect to $t$ to $sw$} if
\begin{enumerate}[leftmargin=2cm]
\item $w$ is left star reducible by $s$ with respect to $t$, and
\item $tw \notin \FC(W)$.	
\end{enumerate}
Notice that condition (2) implies that $l(tw)>l(w)$. Also note that we are restricting out definition of weak star reducible to the set of $\FC$ elements of $W(\Gamma)$. We analogously define \emph{right weak star reducible by $s$ with respect to $t$ to $ws$}. We say that $w$ is \emph{weak star reducible} if $w$ is either left or right weak star reducible. Otherwise, we say that $w$ is \emph{non-cancellable} or \emph{weak star irreducible}. Notice that from this we know that weak star reducible implies star reducible. However, star reducible does not imply weak star reducible.

\begin{example}\label{ex:noncancel}
 Let $\w=s_0s_1s_0s_2$ be a reduced expression for $w \in W(B_4)$ as in Example~\ref{ex:starred}. From Example~\ref{ex:starred} we know that $w$ is left star reducible. However, $tw=s_1s_0s_1s_0s_2$ which is not in $\FC(B_4)$. Thus, we see that $w$ is left weak star reducible by $s_0$ with respect to $s_1$ to $s_1s_0s_2$. In addition, Example~\ref{ex:starred} showed that $w$ is not right star reducible and hence $w$ is not right weak star reducible. However, since $w$ is left weak star reducible we know that $w$ is not non-cancellable.
\end{example}

Again it might be useful to visualize the concept of weak star reducible in terms of heaps. Recall that in Section~\ref{sec:star} we described what a star reduction looks like in a heap and what a star reducible heap looks like. Since the definition of weak star reducible includes that a heap is star reducible we again need to have those properties. In addition, for a heap to be weak star reducible when adding the block that becomes fully exposed when a block is removed from the heap must create a braid in  the heap forcing the new heap to not be FC. That is, one of the impermissible configurations seen in Section~\ref{sec:Heaps} will appear.

\begin{example}
	Let $\w=s_0s_1s_0s_2$ be a reduce expression for $w \in W(B_4)$ as in Example~\ref{ex:noncancel}. From Example~\ref{ex:noncancel}, we know that $w$ is left weak star reducible. Recall in Figure~\ref{fig:starred} the heap for $w$ was seen along with what it star reduced to. In Figure~\ref{fig:noncancel} we see that adding $s_1$ to the top of the heap creates a braid which is highlighted in \textcolor{orange}{orange}.
\end{example}
  

%Recall in Figure~\ref{fig:heapy} we have a representation for $w$ as described in Example~\ref{ex:noncancel}. In Section~\ref{sec:star}, we described how  In Figure~\ref{fig:noncancel} we can see that when we multiply $w$ by $s_1$ on the right we end up with a braid, which is  highlighted in orange. Since the heap in Figure~\ref{fig:noncancel} has the impermissible subheap seen in Figure~\ref{fig:C&B2}, $s_1w \notin \FC(B_4)$.  When using heaps to identify whether an element $w$ of $\FC(\Gamma)$ is Non-Cancellable or not there are two key properties that must be observed. The first of the properties is that the $H(w)$ must be able to be dismantled from the top or bottom so that the heap resulting from pulling the top block off or the bottom block off is one row shorter or allows for a new brick to be in the top most or bottom most row of the heap. The second property that must be observed is that given the block that appears in the top or bottom of the heap when the star operation is performed, adding another of that block to the top or bottom of the original heap respectively will result in an impermissible subheap appearing. If both of these properties occur, then the element that corresponds to the heap is not Non-Cancellable.

\begin{figure*}[h!] \centering
\begin{tikzpicture}[scale=0.4]
\heapblock{2}{0}{2}{purple}
\heapblock{0}{0}{0}{orange}
\heapblock{1}{2}{1}{orange}
\heapblock{0}{4}{0}{orange}
\heapblock{1}{6}{1}{orange}
\end{tikzpicture}
\caption{Visualization of a weak star reducible element of $\FC(B_4)$.} \label{fig:noncancel}
\end{figure*}

\begin{example}
Let $w \in \FC(B_4)$ and let $\w=s_0s_1$ be a reduced expression for $w$. Note that $w$ is left (respectively, right) star reducible by $s_0$ with respect to $s_1$ (respectively, by $s_1$ with respect to $s_0$). However, $s_1s_0s_1 \in \FC(B_4)$ (respectively, $s_0s_1s_0 \in \FC(B_4)$). Thus $w$ is non-cancellable. Visually the heap appears in Figure~\ref{fig:noncancelvisual}. Clearly when $s_0$ is added to the bottom of the heap, the new heap is still in $\FC(B_4)$ and the same can be said when $s_1$ is added to the top of the heap.
\end{example}

\begin{figure*}[h!] \centering
\begin{tikzpicture}[scale=0.4]
\heapblock{0}{2}{0}{purple}
\heapblock{1}{0}{1}{purple}	
\end{tikzpicture}	
\caption{Visualization of a non-cancellable element of $\FC(B_4)$.}\label{fig:noncancelvisual}
\end{figure*}

In~\cite{Ernst2010}, Ernst classified the non-cancellable elements in Coxeter groups of type $W(B_n)$ and $W(\C_n)$. We will state part of the classification here as it is important to the development of the non-trivial T-avoiding elements in $W(\C_n)$. To see the full classification see~\cite[Sections 4.2 and 5]{Ernst2010}. 

Before we state the classification we first define a specific group element called a Type II element in $W(\C_n)$. We will refer to this element as a single \emph{sandwich stack}, which is seen in Figure~\ref{fig:singsandstack}.

\begin{figure}[h!] \centering
\begin{tikzpicture}[scale=0.5]
	\heapblock{0}{4}{0}{purple}
	\heapblock{2}{4}{2}{purple}
	\node[] at (4,4){$\cdots$};
	\heapblock{6}{4}{n-2}{purple}
	\heapblock{8}{4}{n}{purple}
	\heapblock{1}{2}{1}{purple}
	\heapblock{3}{2}{3}{purple}
	\node[] at (5,2){$\cdots$};
	\heapblock{7}{2}{n-1}{purple}
	\heapblock{0}{0}{0}{purple}
	\heapblock{2}{0}{2}{purple}
	\node[] at (4,0){$\cdots$};
	\heapblock{6}{0}{n-2}{purple}
	\heapblock{8}{0}{n}{purple}
\end{tikzpicture}
\caption{A single sandwich stack in $W(\C_n)$.}\label{fig:singsandstack}
\end{figure}

We can stack single sandwich stacks together and get a ``stack of sandwich stacks." This is done by removing the top most layer of the heap and adding a new single bowtie to the stack. A stack of bowties is seen in Figure~\ref{fig:stacksandstack}. These heaps are referenced in the classification of non-trivial T-avoiding elements in $W(\C_n)$.

\begin{figure}[h!] \centering
\begin{tikzpicture}[scale=0.5]
	\heapblock{0}{4}{0}{purple}
	\heapblock{2}{4}{2}{purple}
	\node[] at (4,4){$\cdots$};
	\heapblock{6}{4}{n-2}{purple}
	\heapblock{8}{4}{n}{purple}
	\heapblock{1}{2}{1}{purple}
	\heapblock{3}{2}{3}{purple}
	\node[] at (5,2){$\cdots$};
	\heapblock{7}{2}{n-1}{purple}
	\heapblock{0}{0}{0}{purple}
	\heapblock{2}{0}{2}{purple}
	\node[] at (4,0){$\cdots$};
	\heapblock{6}{0}{n-2}{purple}
	\heapblock{8}{0}{n}{purple}
	\node[] at (5,-2){$\vdots$};
	\heapblock{1}{-4}{1}{purple}
	\heapblock{3}{-4}{3}{purple}
	\node[] at (5,-4){$\cdots$};
	\heapblock{7}{-4}{n-1}{purple}
	\heapblock{0}{-6}{0}{purple}
	\heapblock{2}{-6}{2}{purple}
	\node[] at (4,-6){$\cdots$};
	\heapblock{6}{-6}{n-2}{purple}
	\heapblock{8}{-6}{n}{purple}
\end{tikzpicture}	
\caption{A stack of sandwich stacks in $W(\C_n)$.}\label{fig:stacksandstack}
\end{figure}

In~\cite{Ernst2010}, Ernst classified the sandwich stack as a non-cancellable element. Ernst also classified two other types of non-cancellable elements. The first does not have full support in $W(\C_n)$ and the second clearly has Property T. Thus from now on we will only consider stacks of sandwich stacks.
%\begin{theorem}
%Let $w \in \FC(B_n)$. Then $w$ is non-cancellable if and only if $w$ is either a product of commuting generators, $s_0s_1u$, and $s_1s_0u$ where $u$ is a product of commuting generators with $s_1, s_2,s_3 \in \	supp(w)$. \textcolor{red}{Dana this isn't related to the thesis but to satisfy my curiosity why can't $s_3$ be in $supp(w)$} \qed
%\end{theorem}
%
%\begin{figure}[h!]
%\begin{tabular}{m{7cm} m{7cm}}
%\begin{subfigure}{0.5\textwidth} \centering
%\begin{tikzpicture}[scale=0.4]
%	\heapblock{0}{6}{0}{rred}
%	\heapblock{1}{4}{1}{rred}
%	\heapblock{3}{4}{}{purple}
%	\heapblock{5}{4}{}{purple}
%	
%	\node[] at (7,4){$\cdots$};
%	
%	\heapblock{9}{4}{}{purple}
%\end{tikzpicture}
%%\caption{}	
%\end{subfigure}&
%
%\begin{subfigure}{0.5\textwidth}\centering
%	\begin{tikzpicture}[scale=0.4]
%	\heapblock{1}{6}{1}{rred}
%	\heapblock{0}{4}{0}{rred}
%	\heapblock{2}{4}{}{purple}
%	\heapblock{4}{4}{}{purple}
%	
%	\node[] at (6,4){$\cdots$};
%	
%	\heapblock{8}{4}{}{purple}
%\end{tikzpicture}
%%\caption{}
%\end{subfigure}
%\end{tabular}
%\caption{Visualization of all non-cancellable elements in $W(B_n)$.}
%\end{figure}
%\end{theorem}

