\chapter{Property T and T-Avoiding}

\section{Property T and T-Avoiding Elements}\label{Tavoid}

\textcolor{red}{Introduction and Motivation for this section}

We first begin by defining the notion of Property T. Let $(W,S)$ be a Coxeter system of type $\Gamma$ and let $w \in W$ we say that $w$ has \emph{Property T} if and only if there exists a reduced expression $\overline{w}$ such that $\overline{w}=stu$ or $\overline{w}=uts$ where $m(s,t)\geq 3$. That is, $w$ has Property T if there exists a reduced expression for $w$ that begins with a product of non-commuting generators or ends with a product of non-commuting generators. It should be noted that by the symmetry of the definition if $w$ has Property T, then $w^{-1}$ has Property T.

\begin{example}\label{ex:prop-T}
Let $w \in W(A_5)$ and let $\overline{w}=s_1s_4s_2s_3s_5$. Note that applying a commutation to $s_4s_5$ results in $\overline{w}_1=s_1s_2s_4s_3s_5$. Hence $w$ has Property T, since $m(s_1,s_2)=3$ and there is a reduced expression for $w$ which starts with $s_1s_2$.	
\end{example}

\begin{example}\label{ex:tavoid}
Let $w \in W(A_5)$ and let $\overline{w}=s_1s_3s_5$. It turns out that since $w$ is a product of commuting generators there is no reduced expression for $w$ that begins or ends with a pair of non-commuting generators. This implies that $w$ does not have Property T.	
\end{example}

As with star reducible elements it may be helpful to visualize Property T through heaps. Figure~\ref{fig:heapw/T} provides a representation of $\overline{w}$ seen in Example~\ref{ex:prop-T} and Figure~\ref{fig:heapnoT} provides a representation of $\overline{w}$ seen in Example~\ref{ex:tavoid}. Notice that if we were to remove the block for $s_1$ in the bottom row of Figure~\ref{fig:heapw/T} , we would have a new bottom row. However, in Figure~\ref{fig:heapnoT}, we are not able to remove able to remove any bricks and have a new brick come to the top or bottom row as the heap is just one row.  Property T provides the ability for us get a new row in our heap when we remove the top most or bottom most row from a specific reduced expression for $w$.

\begin{figure}[h!]
\begin{tabular}{m{7cm} m{7cm}}
\begin{subfigure}{0.5\textwidth} \centering
\begin{tikzpicture}[scale=0.5]
\heapblock{5}{6}{5}{purple}
\heapblock{3}{6}{3}{purple}
\heapblock{2}{4}{2}{orange}
\heapblock{4}{4}{4}{purple}
\heapblock{1}{2}{1}{orange}
\end{tikzpicture}
\caption{Heap of an element with Property T} \label{fig:heapw/T}	
\end{subfigure}&

\begin{subfigure}{0.5\textwidth} \centering
\begin{tikzpicture}[scale=0.5]
\heapblock{1}{6}{1}{purple}
\heapblock{3}{6}{3}{purple}
\heapblock{5}{6}{5}{purple}
\end{tikzpicture}
\caption{Heap of a T-Avoiding element}\label{fig:heapnoT}
\end{subfigure}
\end{tabular}
\caption{Heaps of an element with Property T and a T-Avoiding element}
\end{figure}

An element $w \in W(\Gamma)$ is called \emph{T-avoiding} if $w$ and $w^{-1}$ do not have Property T. As seen in Example~\ref{ex:tavoid} an element $w \in W(\Gamma)$ will be T-avoiding if $w$ is a product of commuting generators. We will call an element that is a product of commuting generators \emph{trivially T-avoiding}. If $w$ is T-avoiding and not a product of commuting generators, we will say that $w$ is \emph{non-trivially T-avoiding.}

\begin{example}
Let $w \in W(A_5)$ and let $\overline{w}=s_1s_3s_5$. Then by Example~\ref{ex:tavoid}, we know that $w$ is T-avoiding. Furthermore, since $w$ is a product of commuting generators, $w$ is trivially T-avoiding.	
\end{example}

\begin{example}
Let $w \in W(\widetilde{C}_4)$ and let $\overline{w}=s_0s_2s_4s_1s_3s_0s_2s_4$. It turns out that $w$ is non-trivially T-avoiding. 	
\begin{figure}[h!]
\centering
\begin{tikzpicture}[scale=0.5]
\heapblock{0}{6}{0}{purple}
\heapblock{2}{6}{2}{purple}
\heapblock{4}{6}{2}{purple}
\heapblock{1}{4}{1}{purple}
\heapblock{3}{4}{3}{purple}
\heapblock{0}{2}{0}{purple}
\heapblock{2}{2}{2}{purple}
\heapblock{4}{2}{4}{purple}
\end{tikzpicture}
\caption{Heap of a non-trivially T-Avoiding element in $W(\widetilde{C}_4)$.}	
\end{figure}
\end{example}

One thing to notice here is that all Coxeter groups have trivial T-avoiding elements as they all contain products of commuting generators. The more interesting non-trivial T-avoiding elements don't appear in all Coxeter groups. The remainder of this thesis classifies what is known regarding non-trivial T-avoiding elements in the irreducible finite Coxeter groups and the irreducible affine Coxeter groups. 

As mentioned in Section~\ref{Star} Green classified all star reducible Coxeter groups. It is interesting to note that in a star reducible Coxeter group there are no non-trivially T-avoiding FC elements. In~\cite{Green2006a}, Green utilizes the following theorem to help classify the star reducible Coxeter groups. 
\begin{theorem}[Green,~\cite{Green2006a}]
	Let $W$ be a star reducible Coxeter group, and let $w \in W$. Then one of the following possibilities occurs for some Coxeter generators s,t, u with $m(s,t) \neq 2$, $m(t,u) \neq 2$, and $m(s,u)=2$:
	\begin{enumerate}
	\item $w$ is a product of commuting generators;\label{it:triv}
	\item $w$ has a reduced expression beginning with $st$;\label{it:proptend}
	\item $w$ has a reduced expression ending in $ts$;\label{it:proptbeg}
	\item $w$ has a reduced expression beginning with $sut$.\label{it:tavoid}	\qed
	\end{enumerate}
\end{theorem}

We see that Item~\ref{it:triv} corresponds to the element $w$ being trivially T-avoiding, Items~\ref{it:proptend} and~\ref{it:proptbeg} refer to the element $w$ having Property T at the beginning and end respectively, and Item~\ref{it:tavoid} corresponds to the element $w$ being T-avoiding. Notice that this implies that some Coxeter groups will have elements that are non-trivially T-avoiding while others may not. For example, as will be seen in the following sections, the Coxeter groups of type $A_n$ and $B_n$ have no non-trivial T-avoiding elements, while the Coxeter group of type $D_n$ does have non-trivial T-avoiding elements. The following sections and chapters classify T-avoiding elements in different Coxeter groups. 


%%%%%%%%%%%%%%%%%%%%%%%%%%

\section{T-Avoiding Elements in Type $\widetilde{A}_n$ and $A_n$}
We start by classifying the non-trivial T-avoiding elements in Coxeter groups of type $\widetilde{A}_n$ and $A_n$. 
\begin{theorem}
Let $W=W(\widetilde{A}_n)$.  If $n \geq 2$ and $n$ is even, then there are no non-trivial T-avoiding elements in $W$. Otherwise, if $n \geq 2$ and $n$ is even then $W$ contains non-trivial T-avoiding elements.
\begin{proof}
	This is~\cite[Proposition~3.1.2.]{Fan1999}.
\end{proof}
\end{theorem}

Although Green and Fan stated their results in terms of the more general type $\widetilde{A}_n$ we can state the result as a corollary in type $A_n$.
\begin{corollary}
Let $W=W(A_n)$. Then there are no non-trivially T-avoiding elements in $W$. 
\begin{proof}
This is a consequence of~\cite[Proposition 3.1.2.]{Fan1999}.	 That is,  we know that $W(A_n)$ is a parabolic subgroup of $W(\widetilde{A}_n)$. Since this is true if $W(A_n)$ were to have non-trivial T-avoiding elements, then both parities of $W(\widetilde{A}_n)$ would also have non-trivial T-avoiding elements which by above we know that the even parity of $W(\widetilde{A}_n)$ does not have non-trivial T-avoiding elements.
\end{proof}
\end{corollary}

This implies that in the Coxeter group of type $\widetilde{A}_n$, when $n$ is even there are no non-trivial T-avoiding elements, while when $n$ is odd, there are non-trivial T-avoiding elements. One interesting observation to go along with this is $W(\widetilde{A}_n)$ is star reducible for $n$ even but not star reducible for $n$ odd. \cite{Fan1999} did not specifically classify the non-trivial T-avoiding elements for type $\widetilde{A}_n$ for $n$ odd. Before we state our conjecture as to what the non-trivial T-Avoiding elements are in $W(\widetilde{A}_n)$ we will discuss shortly what the heaps look like. Because the Coxeter graph for type $\widetilde{A}_n$ is not a straight line Coxeter graph, the heaps can no longer be 2 dimensional and now must take on a 3 dimensional representation. The nature of the Coxeter graph of type $\widetilde{A}_n$ leads us to represent heaps in the sense of a turret on a castle where the bricks are stacked in a circular manner. We now continue with our conjectured classification. Because type $A_n$ does not contain any non-trivial T-avoiding elements we know that the non-trivial T-avoiding elements in $W(\widetilde{A}_n)$ for $n$ odd must have full support. We conjecture that the only non-trivial T-avoiding elements are castle turrets that have no missing blocks in the walls. However, this remains an open problem to be proved.

%%%%%%%%%%%%%%%%

\section{T-Avoiding Elements in Type $D_n$}

In this section we will classify the non-trivial T-avoiding elements in the Coxeter group of type $D_n$. Recall that $W(D_n)$ is a star reducible Coxeter group and as a result of this any non-trivial T-avoiding element will not be fully commutative.

\begin{theorem}
Let $W=W(D_n)$. Then $W$ contains non-trivial T-avoiding elements.
\begin{proof}
	This is a consequence of~\cite[Section 2.2]{Gern2013a}.
\end{proof}
\end{theorem}

In addition, to showing that there are non-trivial T-avoiding elements in type $D_n$ Gern also classified the non-trivial T-avoiding elements as well. The following is his classification translated into heaps. \textcolor{red}{Once we figure out the heap include it here.} For the full details regarding his classification see~\cite{Gern2013a}. Note that in his classification Gern  refers to non-trivially T-avoiding elements as ''bad."

%%%%%%%%%%%%%%%

\section{T-Avoiding Elements in Type $F_n$}

In this section we state what is known regarding the non-trivial T-avoiding elements in the Coxeter groups of type $F_n$ for $n \geq 4$. Note that all of the following results are unpublished. 

We start with the Coxeter group of type $F_5$. 


%%%%%%%%%%%%%%%%

\section{T-Avoiding Elements in Type $I_2(m)$}

We next will classify the non-trivial T-avoiding elements in Coxeter groups of type $I_2(m)$. Although it the following is a quick result, this is the first time that we believe it has been written.
\begin{theorem}
Let $W=W(I_2(m))$. Then $W$ has no non-trivial T-avoiding elements.
\begin{proof}
	Let $W=W(I_2(m))$. Recall that the Coxeter graph for Type $I_2(m)$ consists 2 vertices, $s,t$, and an edge with weight $m(s,t)$. Since $s$ and $t$ are involutions this implies that the elements in $W$ must alternate between $s$ and $t$. This implies that the fully commutative elements in $W$ are just $s$ and $t$. The rest of the elements in $W$ must be alternating products of $s$ and $t$. By definition $m(s,t)>3$ and hence all elements in $W$ that are not products of commuting generators have Property T. Hence $W$ has no non-trivial T-avoiding elements. 
\end{proof}	
\end{theorem}
 
As mentioned before the Coxeter group of type $I_2(m)$ is a star reducible Coxeter group. Similar to the Coxeter group of type $A_n$ and $\widetilde{A_n}$ for $n$ odd both star reducible Coxeter groups, $W(I_2(m))$ has no non-trivial T-avoiding elements. 


