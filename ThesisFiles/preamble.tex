% This is the preamble file. Add your own macros and newtheorem commands.

\title{A Study of T-Avoiding Elements of Coxeter Groups}

\author{Taryn Laird}
\def\discipline{Mathematics}
\def\date{May 2016}
\def\chair{Dana Ernst, Ph.D.}
\def\second{Michael Falk, Ph.D.}
\def\third{Nandor Sieben, Ph.D.}

\theoremstyle{definition}
\newtheorem{theorem}{Theorem}[section] 
\newtheorem{lemma}[theorem]{Lemma}
\newtheorem{proposition}[theorem]{Proposition}
\newtheorem{corollary}[theorem]{Corollary}
\newtheorem{conjecture}[theorem]{Conjecture}
\newtheorem{definition}[theorem]{Definition}
\newtheorem{example}[theorem]{Example}
\newtheorem{examples}[theorem]{Examples} 
\newtheorem{problem}[theorem]{Problem} 
\newtheorem{remark}[theorem]{Remark}

\renewcommand{\qed}{\hfill \mbox{$\Box$}}



%%%%%%%%%%%%%% Heap Code %%%%%%%%%%%%%%
\newcommand\heapblock[4]{\fill[fill=#4, fill opacity=0.35, draw=#4, line width=1.1pt, rounded corners,shift={(\xxaxis:#1)},shift={(\yyaxis:#2)}] (-1,-1) rectangle (1,1);\node at (#1,#2) {\footnotesize $#3$};}

\newcommand\dheapblock[4]{\draw[dotted, draw=#4, line width=1.1pt, rounded corners,shift={(\xxaxis:#1)},shift={(\yyaxis:#2)}] (-1,-1) rectangle (1,1);\node at (#1,#2) {\footnotesize $#3$};}

\newcommand\xxaxis{0}
\newcommand\yyaxis{90}

\definecolor{orange}{RGB}{255,102,0}
\definecolor{ggreen}{RGB}{0,153,0}
\definecolor{darkblue}{RGB}{0,0,255}
\definecolor{purple}{RGB}{153,51,255}
\definecolor{turq}{RGB}{72,209,204}
\definecolor{gray}{RGB}{220,220,220}
\definecolor{orange2}{RGB}{255,100,0}
\definecolor{purple2}{RGB}{159,51,250}
\definecolor{rred}{rgb}{0.9, 0.17, 0.31}

%%%%%%%%%%%%%%% Extra Things To make Typing Easier %%%%%%%%%%%%%%

\newcommand{\supp}{\mathrm{supp}}
\newcommand{\FC}{\mathrm{FC}}
\newcommand{\Sym}{\mathrm{Sym}}
\newcommand{\w}{\mathsf{w}}
\newcommand{\C}{\widetilde{C}}
\newcommand{\sgn}{\mathrm{sign}}
\newcommand{\RD}{\mathcal{R}}
\newcommand{\LD}{\mathcal{L}}

\usepackage{enumitem}
\setenumerate[1]{label={(\arabic*)}}
	\setenumerate{listparindent=\parindent}
	
\usepackage{mathdots}	
	
%%%%%%%%%%%%%%%%% Color Table %%%%%%%%%%%%%%%%%%
\usepackage[table]{colortbl}
