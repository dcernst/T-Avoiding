\begin{abstract}
\doublespacing

	Kazhdan--Lusztig polynomials arise in the context of Hecke algebras associated to Coxeter groups. The computation of these polynomials is very difficult for examples even of moderate rank. Motivated by a desire to understand the Kazhdan--Lusztig theory of the Hecke algebra of the underlying Coxeter group, R.M. Green classified the so-called star reducible Coxeter groups, which have the property that all fully commutative elements (in the sense of Stembridge) can be sequentially reduced via star operations to a product of commuting generators. It turns out that in some Coxeter groups there are elements, called T-avoiding elements, which cannot be systematically dismantled in this way. More specifically an element $w$ is called T-avoiding if $w$ does not have a reduced expression beginning or ending with a pair of non-commuting generators. Clearly, a product of commuting generators is trivially T-avoiding. However, sometimes there are more interesting non-trivially T-avoiding elements. A natural question is \emph{which Coxeter groups have non-trivially T-avoiding elements and which do not?} Computation of Kazhdan--Lusztig polynomials involving the non-trivially T-avoiding elements is difficult in general. However, having an understanding of the T-avoiding elements provides valuable information about possible obstructions to determining the Kazhdan--Lusztig polynomials. In this thesis we  begin by summarizing the previously known results regarding T-avoiding elements in certain Coxeter groups and then classify the T-avoiding elements in Coxeter groups of types $B_n$ and $\C_n$. 	
 
	
\end{abstract}

%A star reducible Coxeter group is one in which every fully commutative element can be reduced to a product of commuting generators through repeated star reductions. A complete list of these star re
