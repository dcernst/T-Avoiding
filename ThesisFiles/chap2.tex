\chapter{T-Avoiding Elements in Types $B_n$ and $\C_n$}

In this section we classify the T-Avoiding elements in Coxeter groups of type $B_n$ and type $\C_n$. We start by introducing necessary tools for type $B_n$ and finish with a proof of the classification in $B_n$. We then conclude with the classification of $\C_n$. Note that the proof for Coxeter systems of type $B_n$ closely follows the classification of T-avoiding elements of Type $D_n$ seen in~\cite{Gern2013a}.

\section{Tools for the Classification}

Recall from Example~\ref{ex:B} that we can represent each element $w \in W(B_n)$ as a member of the signed permutation group. As a result we can write $w \in W(B_n)$ using one-line notation 
\[ w=[w(1),w(2), \ldots, w(n-1), w(n)] \]
where we write a bar underneath a number in place of a negative sign in order to simplify notation. This is obtained from the Coxeter group in the following way. We identify $s_i \in S(B_n)$ via 
\[s_i=[1,2, \ldots i-1, i+1,i,i+2, \ldots, n-1,n] \] and we identify $s_0 \in S(B_n)$ via
\[s_0=[\underbar{1}, 2 \ldots, n].\] Further $w(-i)=-w(i)$ for $i \in \{1,2, \ldots, n\}$.

\begin{example}
Let $w \in W(B_6)$ with a given reduced expression $\w=s_0s_1s_3s_4s_5s_2$. Then we can write $w=[2, 4, \underbar{1}, 5, 6, 3]$. 
\end{example}

It will be useful to easily determine what what happens to the window notation of a given element $w \in W(B_n)$ when we multiply on the right or left by $s_i \in S(B_n)$. The following Proposition allows us to do this.

\begin{proposition}
	Let $w \in W(B_n)$ with corresponding signed permutation 
	\[w=[w(1),w(2), \ldots w(n)].\]
	Suppose $s_i \in S(B_n)$. If $i \geq 1$, then multiplying $w$ on the right by $s_i$ has the effect of interchanging $w(i)$ and $w(i+1)$. Multiplying on the left by $s_i$ has the effect of interchanging the entries whose absolute values are $i$ and $i+1$.
	
	If $i=0$, then multiplying $w$ on the right by $s_i$ has the effect of switching the sign of $w(1)$. Multiplying $w$ on the left by $s_i$ has the effect of switching the sign of the entry whose absolute value is $1$.
	\begin{proof}
	This follows from~\cite[Section 8.1 and A3.1]{Bjorner2005}.	
	\end{proof}
\end{proposition}

Given the one line notation for an element $w \in W(B_n)$ we can easily calculate the left and right descent sets of $w$. The following proposition explains how.

\begin{proposition}[Bj\"{o}rner,~\cite{Bjorner2005}]\label{prop:descent}
Let $w \in W(B_n)$. Then 
\[ \mathcal{R}(w)=\{s_i \in S: w(i) > w(i+1)\} \]
where w(0)=0 by definition.
\begin{proof}
	This is,~\cite{Bjorner2005} Proposition 8.1.2.
\end{proof}
\end{proposition}

We now will introduce the concept of signed pattern avoidance which will help with the classification of the T-avoiding elements in the Coxeter group of type $B_n$. This notion was first introduced in~\cite{Gern2013a}. Let $w \in W(B_n)$. We say that $w$ \emph{avoids the consecutive pattern} $abc$ if there is no $i \in \{1,2, \ldots, n-2\}$ such that $(w(i),~w(i+1),~w(i+2))$ is in the same relative order as $(a,b,c).$ We say that $w$ \emph{avoids the signed consecutive pattern} $abc$ if there is no $i \in \{1,2, \ldots, n-2\}$ such that $\left(|w(i)|, |w(i+1)|, |w(i+2)|\right)$ is in the same consecutive order as $\left(|a|, |b|, |c| \right)$ and such that $\sgn(w(i))=\sgn(a)$, $\sgn(w(i+1))=\sgn(b)$, and $\sgn(w(i+2))=\sgn(c)$.

\begin{example}
Let $w \in W(B_4)$ with signed permutation \[w=[\underbar{2},4, \underbar{1}, 3].\] We see that $w$ has the signed consecutive pattern $\underbar{2} 3 \underbar{1}$, since $(|w(1)|, |w(2)|, |w(3)|)$ are in the same relative order as $(|-2|, |3|, |-1|)$, and $\sgn(w(1))=\sgn(-2)$, $\sgn(w(2))=\sgn(3)$, and $\sgn(w(3))=\sgn(-1)$. However, $w$ avoids the signed consecutive pattern $1\underbar{2}3$.
\end{example}

\section{Classification of T-Avoiding Elements in Type $B_n$}\label{sec:TAB}

In this section we will classify the T-avoiding elements in Coxeter groups of type $B_n$. Specifically we will classify the non-trivial T-avoiding elements in $W(B_n)$. The following is our classification.

\begin{theorem}\label{thm:classificationofB}
There are no non-trivial T-avoiding elements in $W(B_n)$.	
\end{theorem}
 
In order to prove this we will use the notion of signed pattern avoidance seen above. Before we prove this theorem we first need some preparatory lemmas. 

\begin{lemma}\label{lem:sts}
Let $s,t \in S(B_n)$ such that $m(s,t)=3$, and $s_0 \notin \{s,t\}.$ Then $w$ has a reduced expression ending in $sts$ if and only if $w$ has the consecutive pattern $321$.
\begin{proof}
	Let $i \geq 1$, let $I=\{s_i,s_{i+1}\}$ and write $w=w^Iw_I$ as in 2.2.4 in~\cite{Bjorner2005}. Observe that if $w$ has a reduced expression ending in two non-commuting generators $s_i, s_{i+1}$ in some order then we have $w_I \in \{s_is_{i+1}, s_{i+1}s_i\}$.
	
	Suppose $w$ has the consecutive pattern $321$. Then there is some $i$ such that $w(i) > w(i+1) > w(i+1)$. By \ref{prop:descent} $s_i,s_{i+1} \in \mathcal{R}(w)$. By \textcolor{red}{Tyson's reference to simply laced coxeter group stuff 1.2.1} $w$ ends in $s_is_{i+1}s_{i+2}$.
	
	Conversely, suppose $w$ ends in $s_is_i+1s_i$. This implies that either $w_I=s_is_{i+1}$ or $w_I=s_{i+1}s_i$ which implies that $s_i,s_{i+1} \in \mathcal{R}(w)$. Since $s_i,s_{i+1} \in \mathcal{R}(w)$, we see that $w(i)>w(i+1)>w(i+2)$ by \ref{prop:descent}. Thus $w$ has the consecutive pattern 321.
	Therefore, $w$ has a reduced expression ending in $sts$ if and only if $w$ has the consecutive pattern $321$. 
\end{proof}	
\end{lemma}

\begin{corollary}\label{lem:endswithsts}
	Let $s,t \in S(B_n)$ such that $m(s,t)=3$, and $s_0 \notin\{s,t\}$. Then $w$ has a reduced expression beginning with $sts$ if and only if $w^{-1}$ has the consecutive pattern $321$.
	\begin{proof}
		Let $s,t \in S(B_n)$ such that $m(s,t)=3$, and $s_0 \notin\{s,t\}$. We know that $w$ has no reduced expressions beginning with $sts$ if and only if $w^{-1}$ has no reduced expression ending with $sts$ which by Theorem~\ref{lem:endswithsts} happens only if $w^{-1}$ avoids the consecutive pattern $321$.
	\end{proof}
\end{corollary}

\begin{lemma}\label{lem:ts}
Let $s,t \in S(B_n)$ such that $m(s,t)=3$, and $s_0 \notin \{s,t\}.$ Then $w$ has a reduced expression ending in $st$ if and only if $w$ has the consecutive pattern $231$.
\begin{proof}
	Let $i \geq 1$, let $I=\{s_i,s_{i+1}\}$ and write $w=w^Iw_I$ as in 2.2.4 in~\cite{Bjorner2005}. Observe that if $w$ has a reduced expression ending in two non-commuting generators $s_i, s_{i+1}$ in some order then we have $w_I \in \{s_is_{i+1}, s_{i+1}s_i\}$.
	
	Suppose that $w$ has the consecutive pattern $231$. Then there is some $i$ such that $w(i+1)>w(i)>w(i+2)$. By \ref{prop:descent} $s_{i+1} \in \mathcal{R}(w)$. Now multiplying on the right by $s_{i+1}$ we see that $ws_{i+1}(i+1)=w(i+2)$ and $ws_{i+1}(i)=w(i)$. We know that $w(i+2)<w(i)$, this implies that $s_i \in \mathcal{R}(ws_{i+1})$. This implies $w$ has a reduced expression that ends in $s_is_{i+1}$.
	
	 Conversely, suppose that $w$ has a reduced expression ending in $s_is_{i+1}$. Then $w(i+2)<w(i+1)$ and $w(i)<w(i+1)$. Since $s_i \in \mathcal{R}(ws_{i+1})$ we have $w(i+2)=ws_{i+1}(i+1)<ws_{i+1}(i)=w(i)$. Thus we have that $w(i+1) > w(i) > w(i+2)$. Hence $w$ has the consecutive pattern 231.
	Therefore, $w$ has a reduced expression ending in $st$ if and only if $w$ has the consecutive pattern $231$.
\end{proof}	
\end{lemma}

\begin{corollary}\label{lem:endswithst}
	Let $s,t \in S(B_n)$ such that $m(s,t)=3$, and $s_0 \notin\{s,t\}$. Then $w$ has a reduced expression beginning with $st$ if and only if $w^{-1}$ has the consecutive pattern $231$.
	\begin{proof}
		Let $s,t \in S(B_n)$ such that $m(s,t)=3$, and $s_0 \notin\{s,t\}$. We know that $w$ has no reduced expressions beginning with $st$ if and only if $w^{-1}$ has no reduced expression ending with $st$ which by Theorem~\ref{lem:endswithsts} happens only if $w^{-1}$ avoids the consecutive pattern $231$.
	\end{proof}
\end{corollary}

\begin{lemma}\label{lem:st}
Let $s,t \in S(B_n)$ such that $m(s,t)=3$, and $s_0 \notin \{s,t\}.$ Then $w$ has a reduced expression ending in $ts$ if and only if $w$ has the consecutive pattern $312$.
\begin{proof}
	Let $i \geq 1$, let $I=\{s_i,s_{i+1}\}$ and write $w=w^Iw_I$ as in 2.2.4 in~\cite{Bjorner2005}. Observe that if $w$ has a reduced expression ending in two non-commuting generators $s_i, s_{i+1}$ in some order then we have $w_I \in \{s_is_{i+1}, s_{i+1}s_i\}$.
	
	Suppose that $w$ has the consecutive pattern $312$.  Then there is some $i$ such that $w(i)>w(i+2)>w(i+1)$. By \ref{prop:descent} we see that $s_i \in \mathcal{R}(w)$. Multiplying on the right by $s_i$ we get $ws_i(i+1)=w(i)$ and $ws_i(i+2)=w(i+2)$. By above $w(i)>w(i+2)$, and by \ref{prop:descent} $s_{i+1} \in \mathcal{R}(ws_i)$. This implies that $w$ has a reduced expression ending in $s_{i+1}s_i$. 
	
	Conversely suppose $w$ ends in a reduced expression with $s_{i+1}s_i$ Then $w_I=s_{i+1}s_i$. We see that $w(i)>w(i+1)$ and $w(i+2)>w(i+1)$. Since $s_{i+1} \in \mathcal{R}(ws_i)$, we have $w(i+2)=ws_i(i+2)<ws_i(i+1)=w(i)$. From this we have $w(i)>w(i+2)$, so $w(i)>w(i+2)>w(i+1)$. Hence, $w$ has the consecutive pattern $312$. Therefore, $w$ has a reduced expression ending in $ts$ if and only if $w$ has the consecutive pattern $312$.
\end{proof}
\end{lemma}

\begin{corollary}\label{lem:endswithts}
	Let $s,t \in S(B_n)$ such that $m(s,t)=3$, and $s_0 \notin\{s,t\}$. Then $w$ has a reduced expression beginning with $ts$ if and only if $w^{-1}$ has the consecutive pattern $312$.
	\begin{proof}
		Let $s,t \in S(B_n)$ such that $m(s,t)=3$, and $s_0 \notin\{s,t\}$. We know that $w$ has no reduced expressions beginning with $ts$ if and only if $w^{-1}$ has no reduced expression ending with $ts$ which by Theorem~\ref{lem:endswithsts} happens only if $w^{-1}$ avoids the consecutive pattern $312.$
	\end{proof}
\end{corollary}

\begin{lemma}\label{lem:endswiths0}
Let $w \in W(B_n)$. Then $w$ has a reduced expression ending in $s_1s_0$ if and only if $w(0)>w(1)$ and $-w(1)>w(2)$.
\begin{proof}
	Suppose $w \in W(B_n)$ such that $w$ ends with $s_1s_0$. Then $s_0 \in \RD(w)$ and $s_1 \in \RD(ws_0)$. This implies that $ws_0(1)>ws_0(2)$ by~\ref{prop:descent}. We see that $ws_0(1)=w(-1)=-w(1)$ and $ws_0(2)=2$. Hence $-w(1)=ws_0(1)>ws_0(2)=w(2)$. Further, since $s_0 \in \RD(w)$, we see that $w(0)>w(1)$.
	
	Conversely, suppose $w \in W(B_n)$ such that $w(0)>w(1)$ and $-w(1)>w(2)$. Since $w(0)>w(1)$ so $s_0 \in \RD(w)$. Multiplying on the right by $s_0$ we see that $ws_0(1)=-w(1)$ and $ws_0(2)=w(2)$. Note that since $ws_0(1)=-w(1)>w(2)=ws_0(2)$, $s_1 \in \RD(ws_0)$. Thus $w$ ends with $s_1s_0$. Therefore, $w$ has a reduced expression ending in $s_1s_0$ if and only if $w(0)>w(1)$ and $-w(1)>w(2)$.
\end{proof}
\end{lemma}

\begin{corollary}\label{lem:beginswiths0}
	Let $w \in W(B_n)$. Then $w$ has a reduced expression beginning in $s_0s_1$ if and only if $w^{-1}(0)>w^{-1}(1)$ and $-w^{-1}(1)>w^{-1}(2)$.
	\begin{proof}
		Let $w \in W(B_n)$. We know that $w$ has no reduced expressions beginning in $s_0s_1$ if and only if $w^{-1}$ has no reduced expressions ending in $s_0s_1$. By Lemma~\ref{lem:endswiths0} we know that this occurs if and only if $w^{-1}(0)>w^{-1}(1)$ and $-w^{-1}(1)>w^{-1}(2)$.
	\end{proof}
\end{corollary}

\begin{lemma}\label{lem:endswiths_1}
Let $w \in W(B_n)$. Then $w$ has a reduced expression ending in $s_0s_1$ if and only if $w(0)>w(2)$ and $w(1)>w(2)$.
\begin{proof}
	Suppose $w \in W(B_n)$ such that $w$ ends with $s_0s_1$. Then $s_1 \in \RD(w)$ and $s_0 \in \RD(ws_1)$. Then $ws_1(0)>ws_1(1)$. We see that $ws_1(0)=0$ and $ws_1(1)=w(2)$. This implies that $0=ws_1(0)>ws_1(1)=2$. Further, since $s_1 \in \RD(w)$ this implies that $w(1) > w(2)$. Thus if $w$ ends with $s_0s_1$, then $w(1)>w(2)$ and $w(0)>w(2)$.
	
	Conversely, suppose $w \in W(B_n)$ such that $w(1)>w(2)$ and $w(0)>w(2)$. This implies that $s_1 \in \RD(W)$. Multiplying $w$ on the right by $s_1$ we see that $ws_1(0)=w(0)$ and $ws_1(1)=w(2)$. Note that since $ws_1(0)=w(0)>w(2)=ws_1(1)$, $s_0 \in \RD(ws_1)$. Thus $w$ ends with $s_0s_1$. Therefore, $w$ has a reduced expression ending in $s_0s_1$ if and only if $w(1)>w(2)$ and $w(0)>w(2)$.
\end{proof}	
\end{lemma}

\begin{corollary}\label{lem:beginswiths1}
	Let $w \in W(B_n)$. Then $w$ has a reduced expression beginning in $s_1s_0$ if and only if $w^{-1}(0)>w^{-1}(2)$ and $w^{-1}(1)>w^{-1}(2)$.
	\begin{proof}
		Let $w \in W(B_n)$. We know that $w$ has no reduced expressions beginning in $s_1s_0$ if and only if $w^{-1}$ has no reduced expressions ending in $s_1s_0$. By Lemma~\ref{lem:endswiths0} we know that this occurs if and only if $w^{-1}(0)>w^{-1}(2)$ and $w^{-1}(1)>w^{-1}(2)$.
	\end{proof}
\end{corollary}

\begin{lemma}\label{lem:prodofcommA}
Let $w \in W(B_n)$ such that each entry for $w$ in the one-line notation is positive and both $w$ and $w^{-1}$ avoid the consecutive patterns $321$, $231$, and $312$, then $w$ is a product of commuting generators.
\begin{proof}
	This is~\cite[Lemma 2.2.9]{Gern2013a}.
\end{proof}	
\end{lemma}

\begin{lemma}\label{lem:prodofCommB}
Let $w \in W(B_n)$ be trivially T-avoiding and let $i \in \{1,2, \ldots, n\}$. Then $w$ satisfies the following conditions:
\begin{enumerate}
\item $w(j) > \min(\{w(i-1), w(i)\})$ for all $j >i$;\label{it:trivT1}
\item $w(k) < \max(\{w(i-1), w(i)\})$ for all $k < i-1$;\label{it:trivT2}
\item if $w(i), w(i+1) > 0$, then $w(j)>0$ for all $j \geq i$;\label{it:trivT3}
\item if $w(i), w(i+1) < 0$, then $w(j)<0$ for all $j \leq i+1$.\label{it:trivT4}
\end{enumerate}
\begin{proof}
	Suppose there is some least $j>i$ such that $w(j) \leq \min(\{w(i-1), w(i)\})$. Note that $j>i$ so $j \neq i$, and $j \neq i-1$ so $w(j) < \min(\{w(i-1), w(i)\})$. Note that $j$ is the least so $w(j-2) \geq \min(\{w(i+1), w(i)\})>w(j)$. This implies that either $w(j-1)>w(j-2)>w(j)$ or $w(j-2)>w(j-1)>w(j)$, which implies $w$ has the consecutive pattern $231$ or $321$ which is a contradiction to $w$ being a non-trivial T-avoiding element by Lemmas~\ref{lem:sts} and~\ref{lem:st}. Thus proving~\ref{it:trivT1}.
	
	Suppose there exists a maximal $k<i-1$ such that $w \geq \max(\{w(i-1), w(i)\})$. Note that $k < i-1$ so $k \neq i$ and $k \neq i-1$. Then $w(k)> \max(\{w(i-1), w(i)\})$. Since $k$ is maximal then $w(k+1) \leq  \max(\{w(i-1), w(i)\})$ and $w(k+2) \leq \max(\{w(i-1), w(i)\})$. This implies that either $w(k+2)<w(k+1)<w(k)$ or $w(k+1)<w(k+2)<w(k)$, which implies $w$ has the consecutive patter $321$ or $312$ which is a contradiction to $w$ being a non-trivial T-avoiding element by Lemmas~\ref{lem:sts} and~\ref{lem:ts}. Thus proving~\ref{it:trivT2}.
	
	It is easy to see that assertion~\ref{it:trivT1} implies~\ref{it:trivT3} and assertion~\ref{it:trivT2} implies~\ref{it:trivT4}.
\end{proof}	
\end{lemma}

\begin{lemma}\label{lem:231}
Let $w \in W(B_n)$ such that $w$ has the consecutive pattern $\underbar{2}31$. Then $w$ has Property T.
\begin{proof}
	Let $w \in W(B_n)$ such that $w$ has the consecutive pattern $\underbar{2}31$.
	
	Case 1: Suppose $w$ has the one-line notation $w=[\underbar{2},3,1]$. This implies that $w=s_1s_0s_2$. Clearly, $w$ begins with a product of non-commuting generators. Thus $w$ has Property T.
	
	Case 2: Suppose that $w$ has the one-line notation $w=[\underbar{a},b,c, \ast, \ldots, \ast]$ where $\underbar{a},b,c$ correspond to the signed consecutive pattern $\underbar{2},3,1$. We now consider the signed consecutive pattern that can arise involving $b,c, \ast$. The following are the possibilities for the signed consecutive pattern that can arise: $31\pm2$, $32 \pm1$, or $21\pm3$. We know that $b,c$ must be positive since they are positive in $w$ and we also know that $b>c$ by the original signed consecutive pattern. Note that by Lemmas~\ref{lem:sts},~\ref{lem:ts}, and~\ref{lem:endswiths0} all of these patterns imply that $w$ ends or begins with a product of noncommuting generators. Thus $w$ has Property T.
	
	Case 3: Suppose that $w$ has the one-line notation $w=[\ast, \ldots, \ast, \underbar{a},b,c]$ where $\underbar{a},b,c$ correspond to the signed consecutive pattern $\underbar{2},3,1$. We now consider the signed consecutive pattern that can arise involving $\ast, \underbar{a}, b$. The following are the possibilities for the signed consecutive pattern that can arise: $\pm1 \underbar{2} 3$, $\pm 2 \underbar{1} 3$, and $\pm 3 \underbar{1} 2$. Note that by Lemmas~\ref{lem:ts},~\ref{lem:endswiths0}, and~\ref{lem:endswiths_1} all of these patterns implies that $w$ ends or begins with a product of noncommuting generators. Thus $w$ has Property T. 

	Therefore, if $w \in W(B_n)$ contains the consecutive pattern $\underbar{2}31$, then $w$ has Property T.
\end{proof}	
\end{lemma}

\begin{lemma}\label{lem:2312}
Let $w \in W(B_n)$ such that $w$ has the consecutive pattern $\underbar{2}3\underbar{1}$. Then $w$ has Property T.
\begin{proof}
	Let $w \in W(B_n)$ such that $w$ has the consecutive pattern $\underbar{2}3\underbar{1}$.
	
	Case 1: Suppose $w$ has the one-line notation $w=[\underbar{2},3,\underbar{1}]$. This implies that $w=s_0s_1s_0s_2$. Clearly, $w$ begins with a product of non-commuting generators. Thus $w$ has Property T.
	
	Case 2: Suppose that $w$ has the one-line notation $w=[\underbar{a},b,\underbar{c}, \ast, \ldots, \ast]$ where $\underbar{a},b,\underbar{c}$ correspond to the signed consecutive pattern $\underbar{2},3,\underbar{1}$. We now consider the signed consecutive pattern that can arise involving $b,\underbar{c}, \ast$. The following are the possibilities for the signed consecutive pattern that can arise: $3\underbar{1}\pm2$, $3\underbar{2} \pm1$, or $2\underbar{1}\pm3$. We know that $b$ must be positive since it is positive in $w$, $c$ must be negative since it is negative in $w$, and we also know that $|b|>|c|$ by the original signed consecutive pattern. Note that by Lemmas~\ref{lem:sts},~\ref{lem:ts}, and~\ref{lem:endswiths0} all of these patterns imply that $w$ ends or begins with a product of noncommuting generators. Thus $w$ has Property T.
	
	Case 3: Suppose that $w$ has the one-line notation $w=[\ast, \ldots, \ast, \underbar{a},b,\underbar{c}]$ where $\underbar{a},b,\underbar{c}$ correspond to the signed consecutive pattern $\underbar{2},3,\underbar{1}$. We now consider the signed consecutive pattern that can arise involving $\ast, \underbar{a}, b$. The following are the possibilities for the signed consecutive pattern that can arise: $\pm1 \underbar{2} 3$, $\pm 2 \underbar{1} 3$, and $\pm 3 \underbar{1} 2$. We know that $a$ must be negative, $b$ must be positive and $|a|<|b|$ by the original signed permutation. Note that by Lemmas~\ref{lem:ts},~\ref{lem:endswiths0}, and~\ref{lem:endswiths_1} all of these patterns implies that $w$ ends or begins with a product of noncommuting generators. Thus $w$ has Property T. 

	Therefore, if $w \in W(B_n)$ contains the consecutive pattern $\underbar{2}3\underbar{1}$, then $w$ has Property T.
\end{proof}	
\end{lemma}

\begin{lemma}\label{lem:123}
Let $w \in W(B_n)$ such that $w$ has the consecutive pattern $\underbar{1}23$. Then $w$ has Property T or is a trivial T-avoiding element.
\begin{proof}
	Let $w \in W(B_n)$ such that $w$ has the consecutive pattern $\underbar{1}23$.
	
	Case 1: Suppose $w$ has the one-line notation $w=[\underbar{1}23]$. This implies that $w=s_0$. Clearly, $w$ is a trivial T-avoiding element as it is a single generator.
	
	Case 2: Suppose that $w$ has the one-line notation $w=[\underbar{a},b,c, \ast, \ldots, \ast]$ where $\underbar{a},b,c$ correspond to the signed consecutive pattern $\underbar{1},2,3$. We now consider the signed consecutive pattern that can arise involving $b,c, \ast$. The following are the possibilities for the signed consecutive pattern that can arise: $23\pm1$, $13 \pm2$, or $12\pm3$. We know that $b,c$, and we also know that $|b|<|c|$ by the original signed consecutive pattern. Note that by Lemmas~\ref{lem:sts},~\ref{lem:ts}, and~\ref{lem:endswiths0} all of these patterns imply that $w$ ends or begins with a product of noncommuting generators. Thus $w$ has Property T.
	
	Case 3: Suppose that $w$ has the one-line notation $w=[\ast, \ldots, \ast, \underbar{a},b,c]$ where $\underbar{a},b,c$ correspond to the signed consecutive pattern $\underbar{2},3,1$. We now consider the signed consecutive pattern that can arise involving $\ast, \underbar{a}, b$. The following are the possibilities for the signed consecutive pattern that can arise: $\pm3 \underbar{1} 2$, $\pm 2 \underbar{1} 3$, and $\pm 1 \underbar{2} 3$. We know that $a$ must be negative, $b$ must be positive and $|a|<|b|$ by the original signed permutation. Note that by Lemmas~\ref{lem:ts},~\ref{lem:endswiths0}, and~\ref{lem:endswiths_1} all of these patterns implies that $w$ ends or begins with a product of noncommuting generators. Thus $w$ has Property T. 

	Therefore, if $w \in W(B_n)$ contains the consecutive pattern $\underbar{1}23$, then $w$ has Property T or is a trivial T-avoiding element.
\end{proof}	
\end{lemma}

\begin{lemma}\label{lem:132}
Let $w \in W(B_n)$ such that $w$ has the consecutive pattern $\underbar{1}32$. Then $w$ has Property T or is a trivial T-avoiding element.
\begin{proof}
	Let $w \in W(B_n)$ such that $w$ has the consecutive pattern $\underbar{1}32$.
	
	Case 1: Suppose $w$ has the one-line notation $w=[\underbar{1}32]$. This implies that $w=s_0s_2$. Clearly, $w$ is a trivial T-avoiding element as it is a single generator.
	
	Case 2: Suppose that $w$ has the one-line notation $w=[\underbar{a},b,c, \ast, \ldots, \ast]$ where $\underbar{a},b,c$ correspond to the signed consecutive pattern $\underbar{1},2,3$. We now consider the signed consecutive pattern that can arise involving $b,c, \ast$. The following are the possibilities for the signed consecutive pattern that can arise: $23\pm1$, $13 \pm2$, or $12\pm3$. We know that $b,c$, and we also know that $|b|<|c|$ by the original signed consecutive pattern. Note that by Lemmas~\ref{lem:sts},~\ref{lem:ts}, and~\ref{lem:endswiths0} all of these patterns imply that $w$ ends or begins with a product of noncommuting generators. Thus $w$ has Property T.
	
	Case 3: Suppose that $w$ has the one-line notation $w=[\ast, \ldots, \ast, \underbar{a},b,c]$ where $\underbar{a},b,c$ correspond to the signed consecutive pattern $\underbar{2},3,1$. We now consider the signed consecutive pattern that can arise involving $\ast, \underbar{a}, b$. The following are the possibilities for the signed consecutive pattern that can arise: $\pm3 \underbar{1} 2$, $\pm 2 \underbar{1} 3$, and $\pm 3 \underbar{2} 1$. We know that $a$ must be negative, $b$ must be positive and $|a|<|b|$ by the original signed permutation. Note that by Lemmas~\ref{lem:ts},~\ref{lem:endswiths0}, and~\ref{lem:endswiths_1} all of these patterns implies that $w$ ends or begins with a product of noncommuting generators. Thus $w$ has Property T. 

	Therefore, if $w \in W(B_n)$ contains the consecutive pattern $\underbar{1}23$, then $w$ has Property T or is a trivial T-avoiding element.
\end{proof}	
\end{lemma}

We can now prove Theorem~\ref{thm:classificationofB}.

\begin{proof}
Suppose that $w \in W(B_n)$ is a non-trivial T-avoiding element. There are $2^3 \cdot 3!$ possible choices of signed consecutive patterns for $w(1)w(2)w(3)$ where $w=[w(1), w(2), w(3), \ast, \ldots, \ast]$.

\begin{center}
\begin{tabular}{|l|l|l|l|l|l|l|l|}
\hline
\cellcolor{blue!30}$123$ & \cellcolor{orange2!40}$\underbar{1}23$ & \cellcolor{brown!50}$1\underbar{2}3$ & \cellcolor{red!25}$12\underbar{3}$ & \cellcolor{brown!50}$\underbar{12}3$ & \cellcolor{red!25}$\underbar{1}2\underbar{3}$ & \cellcolor{turq!40}$1\underbar{23}$ & \cellcolor{turq!40}$\underbar{123}$ \\
\hline
\cellcolor{blue!30}$132$ & \cellcolor{orange2!40}$\underbar{1}32$ & \cellcolor{brown!50}$1\underbar{3}2$ & \cellcolor{red!25}$13\underbar{2}$ & \cellcolor{brown!50}$\underbar{13}2$ & \cellcolor{red!25}$\underbar{1}3\underbar{2}$ & \cellcolor{ggreen!50}$1\underbar{32}$ & \cellcolor{ggreen!50}$\underbar{132}$ \\
\hline
\cellcolor{blue!30}$213$ & \cellcolor{yellow!50}$\underbar{2}13$ & \cellcolor{brown!50}$2\underbar{1}3$ & \cellcolor{turq!40}$21\underbar{3}$ & \cellcolor{yellow!50}$\underbar{21}3$ & \cellcolor{red!25}$\underbar{2}1\underbar{3}$ & \cellcolor{turq!40}$2\underbar{13}$ & \cellcolor{red!25}$\underbar{213}$ \\
\hline
\cellcolor{red!25}$231$ & \cellcolor{purple2!50}$\underbar{2}31$ & \cellcolor{ggreen!50}$2\underbar{3}1$ & \cellcolor{red!25}$23\underbar{1}$ & \cellcolor{brown!50}$\underbar{23}1$ & \cellcolor{purple2!50}$\underbar{2}3\underbar{1}$ & \cellcolor{ggreen!50}$2\underbar{31}$ & \cellcolor{brown!50}$\underbar{231}$ \\
\hline
\cellcolor{ggreen!50}$312$ & \cellcolor{yellow!50}$\underbar{3}12$ & \cellcolor{ggreen!50}$3\underbar{1}2$ &\cellcolor{turq!40}$31\underbar{2}$ & \cellcolor{yellow!50}$\underbar{31}2$ & \cellcolor{yellow!50}$\underbar{3}1\underbar{2}$ & \cellcolor{turq!40}$3\underbar{12}$ & \cellcolor{yellow!50}$\underbar{312}$ \\
\hline
\cellcolor{turq!40}$321$ & \cellcolor{yellow!50}$\underbar{3}21$ & \cellcolor{ggreen!50}$3\underbar{2}1$ & \cellcolor{turq!40}$32\underbar{1}$ & \cellcolor{yellow!50}$\underbar{32}1$ & \cellcolor{yellow!50}$\underbar{3}2\underbar{1}$ & \cellcolor{ggreen!50}$3\underbar{21}$ & \cellcolor{yellow!50}$\underbar{321}$\\
\hline
\end{tabular}
\end{center}

We can use Lemma~\ref{lem:sts} and Corollary~\ref{lem:endswithsts} to eliminate the signed consecutive patterns highlighted in \textcolor{turq}{turquoise}. We can use Lemma~\ref{lem:st} and Corollary~\ref{lem:endswithst} to eliminate the signed consecutive patterns highlighted in \textcolor{red}{red}. We can use Lemma~\ref{lem:ts} and Corollary~\ref{lem:endswithts} to eliminate the consecutive patterns highlighted in \textcolor{ggreen}{green}.  We can use Lemma~\ref{lem:endswiths0} and Corollary~\ref{lem:beginswiths0} to eliminate the signed consecutive patterns highlighted in \textcolor{yellow}{yellow}. We can use Lemma~\ref{lem:endswiths_1} and Corrollary~\ref{lem:beginswiths1} to eliminate signed consecutive patterns highlighted in \textcolor{brown}{brown}. We can use Lemma~\ref{lem:prodofcommA} to elminate the signed consecutive patterns highlighted in \textcolor{blue}{blue}. We can use Lemmas~\ref{lem:231} and~\ref{lem:2312} to eliminate signed consecutive patterns highlighted in \textcolor{purple}{purple}. Finally we can use Lemmas~\ref{lem:123} and~\ref{lem:132} to eliminate signed consecutive patterns highlighted in \textcolor{orange2}{orange}. Since all of the above patterns are eliminated as possibilities for $w(1)w(2)w(3)$ and there are no other signed consecutive patterns that are possible for these positions, $w$ can not be a non-trivial T-avoiding element in the Coxeter group of type B. Therefore, there are no non-trivial T-avoiding elements in $W(B_n)$.
\end{proof}

%%%%%%%%%%%%%%%%%%%%%%%%%%
\section{Classification of T-Avoiding Elements in $W(\C_n)$}

In this section we will classify the T-avoiding elements in Coxeter groups of type $\C_n$. Because, $W(A_n)$ and $W(B_n)$ are parabolic subgroups of $W(\C_n)$, this implies that if $W(\C_n)$ is to have any non-trivial T-avoiding elements they will have full support, because if they did not the problem is reduced to a cross product of $W(A_n)$ and $W(B_n)$ in some way. We will first show that there are no non-trivial T-avoiding elements that are not FC in $W(\C_n)$.

\begin{theorem}\label{thm:TavoidC}
There are no non-trivial T-avoiding elements in $W(\C_n)/\FC(\Gamma)$. 	
\begin{proof}
	Let $w$ be a reduced expression in $W(\C_n)$ such that $w$ has full support and $w$ does not have Property T. Consider all possible heaps for $w$ and choose a heap a bottom most braid. That is, choose a heap where the braid is as low as possible in the heap, which means the generators below the braid are FC.
	
	Case 1: Suppose the braid does not contain $0,1$ or$n-1,n$. Subcase a: Suppose $w$ has the fixed reduced product $w=u\textcolor{teal}{s_ks_{k-1}s_k}s_{k-1}s_{k-2}v$ or $w=u\textcolor{teal}{s_ks_{k-1}s_k}s_{k+1}s_{k-1}s_{k-2}v$ where $v$ is fully commutative, and the braid is highlighted in \textcolor{teal}{teal}. Applying the braid move we obtain the element $w=us_{k-1}s_k\textcolor{teal}{s_{k-1}s_{k-2}s_{k-1}}v$ or $w=us_{k-1}s_k\textcolor{teal}{s_{k-1}s_{k-2}s_{k-1}}s_{k+1}v$. Notice that the braid is now located next to $v$ having moved closer to the bottom in the heap. This is a contradiction to choosing a heap with the lowest braid. Therefore $w$ does not have the fixed reduced product $w=u\textcolor{teal}{s_ks_{k-1}s_k}s_{k-1}s_{k-2}v$ or $w=u\textcolor{teal}{s_ks_{k-1}s_k}s_{k+1}s_{k-1}s_{k-2}v$. Visually we see this in Figure~\ref{fig:Case1a}. Notice how there are two braids located in Figure~\ref{fig:case:a2}, the braid that starts in \textcolor{purple}{purple} and ends with \textcolor{teal}{teal} and the braid that is fully highlighted in \textcolor{teal}{teal}. 
		
	\begin{figure}[h!]
	\begin{tabular}{m{7cm} m{7cm}}
	\begin{subfigure}{0.5\textwidth} \centering
	\begin{tikzpicture}[scale=0.5]
		\heapblock{3}{-2}{}{white}
		\heapblock{4}{6}{k}{teal}
		\heapblock{3}{4}{k-1}{teal}
		\heapblock{4}{2}{k}{teal}
		\heapblock{2}{2}{k-2}{purple}
		\dheapblock{5}{0}{k+1}{black}
		\heapblock{3}{0}{k-1}{purple}
	\end{tikzpicture}
	\caption{}\label{fig:case:a1}
	\end{subfigure}&

	\begin{subfigure}{0.5\textwidth} \centering
	\begin{tikzpicture}[scale=0.5]
		\heapblock{3}{8}{k-1}{purple}
		\heapblock{4}{6}{k}{purple}
		\heapblock{3}{4}{k-1}{teal}
		\heapblock{2}{2}{k-2}{teal}
		\dheapblock{5}{0}{k+1}{black}
		\heapblock{3}{0}{k-1}{teal}
	\end{tikzpicture}
	\caption{}\label{fig:case:a2}
	\end{subfigure}	
	\end{tabular}
	\caption{Visual representation of the heap configuration discussed in Case 1a.}\label{fig:Case1a}
	\end{figure}
	
	Subcase b: Suppose $w$ has the fixed reduced product $w=u\textcolor{teal}{s_ks_{k-1}s_k}s_{k+1}v$ where $v$ is FC and does not contain $s_{k-2}$ and $s_{k-1}$ in the left descent set. Again we have highlighted the braid in \textcolor{teal}{teal}. Applying the braid move we obtain an new fixed product $w=u\textcolor{teal}{s_{k-1}s_ks_{k-1}}s_{k+1}v$. Notice that the braid is now able to be written next to $v$ whereas it previously was not. This again contradicts choosing a heap with the braid in the lowest location. Visually we see this in Figure~\ref{fig:Case1b}. Notice how in Figure~\ref{fig:caseb2} the braid is located next to the block for $s_{k+1}$ whereas in Figure~\ref{fig:caseb1} the braid is below the block for $s_{k+1}$. 
	
	\begin{figure}[h!]
	\begin{tabular}{m{7cm} m{7cm}}
	\begin{subfigure}{0.5\textwidth} \centering
	\begin{tikzpicture}[scale=0.5]
		\heapblock{4}{4}{k}{teal}
		\heapblock{3}{2}{k-1}{teal}
		\dheapblock{2}{0}{}{black}
		\heapblock{4}{0}{k}{teal}
		\dheapblock{3}{-2}{}{black}
		\heapblock{5}{-2}{k+1}{purple}
	\end{tikzpicture}
	\caption{}\label{fig:caseb1}
	\end{subfigure} &

	\begin{subfigure}{0.5\textwidth} \centering
	\begin{tikzpicture}[scale=0.5]
		\heapblock{3}{6}{}{white}
		\heapblock{3}{4}{k-1}{teal}
		\heapblock{4}{2}{k}{teal}
		\dheapblock{2}{2}{}{black}
		\heapblock{3}{0}{k-1}{teal}
		\heapblock{5}{0}{k+1}{purple}
	\end{tikzpicture}
	\caption{}\label{fig:caseb2}
	\end{subfigure}
	\end{tabular}
	\caption{Visual representation of the heap configuration discussed in Case 1b.}\label{fig:Case1b}
	\end{figure}

	Case 2: Suppose the braid contains $2$ or $n-2$. Without loss of generality we will take the braid to contain $2$ the other argument is symmetric to the one presented here. Notice that if the braid is of the form $s_2s_3s_2$ we are in the case above as a result we assume that the braid we refer to in the following subcases do not involve $s_2s_3s_2$ to start. Subcase a: Suppose $w$ has the fixed reduced product $w=u\textcolor{teal}{s_2s_1s_2}s_0s_1s_0v$, where $v$ is FC and does not contain $s_2$ in the left descent set. Again we highlight the braid for emphasis in \textcolor{teal}{teal}. Applying the braid move we obtain the reduced product $w=us_1s_2\textcolor{orange}{s_1s_0s_1s_0}v$. Notice that the braid is now able to touch $v$ as it wasn't before. This contradicts our original choice of heap and as a result we can not choose the reduced product $w=u\textcolor{teal}{s_2s_1s_2}s_0s_1s_0v$. Visually this is seen Figure~\ref{fig:Case2a}. Notice how there are two braids located in Figure~\ref{fig:case2b2}. The braid highlighted in orange did not appear in our original heap seen in Figure~\ref{fig:case2b1} and is lower in the heap than the original.
		\begin{figure}[h!]
	\begin{tabular}{m{7cm} m{7cm}}
	\begin{subfigure}{0.5\textwidth} \centering
	\begin{tikzpicture}[scale=0.5]
		\heapblock{0}{-2}{}{white}
		\heapblock{2}{8}{2}{teal}
		\heapblock{1}{6}{1}{teal}
		\heapblock{2}{4}{2}{teal}
		\heapblock{0}{4}{0}{purple}
		\heapblock{1}{2}{1}{purple}
		\heapblock{0}{0}{0}{purple}
		\dheapblock{2}{0}{}{black}
	\end{tikzpicture}
	\caption{}\label{fig:case2b1}
	\end{subfigure} &

	\begin{subfigure}{0.5\textwidth} \centering
	\begin{tikzpicture}[scale=0.5]
		\heapblock{1}{8}{1}{purple}
		\heapblock{2}{6}{2}{purple}
		\heapblock{1}{4}{1}{orange}
		\heapblock{0}{2}{0}{orange}
		\heapblock{1}{0}{1}{orange}
		\heapblock{0}{-2}{0}{orange}
		\dheapblock{2}{-2}{}{black}
	\end{tikzpicture}
	\caption{}\label{fig:case2b2}
	\end{subfigure}
	\end{tabular}
	\caption{Visual representation of the heap configuration discussed in Case 2a.}\label{fig:Case2a}
	\end{figure}
	
	Subcase b: Suppose $w$ has the fixed reduced product $w=u\textcolor{teal}{s_2s_1s_2}s_0s_1s_3s_2v$ where $v$ is FC. Again we have highlighted the braid in \textcolor{teal}{teal}. Applying the braid move we end up with the reduced product $w=u\textcolor{teal}{s_1s_2s_1}s_0s_1s_3s_2v$. Notice this time the braid does not force a higher braid. Visually this appears in Figure~\ref{fig:Case2b}. In Figures~\ref{fig:caseb2a} and~\ref{fig:caseb2b} we see that the braid actually moves higher in the heap.
	
	\begin{figure}[h!]
	\begin{tabular}{m{7cm} m{7cm}}
	\begin{subfigure}{0.5\textwidth} \centering
	\begin{tikzpicture}[scale=0.5]
		\heapblock{0}{12}{}{white}
		\heapblock{2}{10}{2}{teal}
		\heapblock{1}{8}{1}{teal}
		\heapblock{2}{6}{2}{teal}
		\heapblock{0}{6}{0}{purple}
		\heapblock{1}{4}{1}{purple}
		\heapblock{3}{4}{3}{purple}
	\end{tikzpicture}
	\caption{}\label{fig:caseb2a}
	\end{subfigure} &

	\begin{subfigure}{0.5\textwidth} \centering
	\begin{tikzpicture}[scale=0.5]
		\heapblock{1}{10}{1}{teal}
		\heapblock{2}{8}{2}{teal}
		\heapblock{1}{6}{1}{teal}
		\heapblock{0}{4}{0}{purple}
		\heapblock{1}{2}{1}{purple}
		\heapblock{3}{2}{3}{purple}
	\end{tikzpicture}
	\caption{}\label{fig:caseb2b}
	\end{subfigure}
	\end{tabular}
	\caption{Visual representation of the heap configuration discussed in Case 2b.}\label{fig:Case2b}
	\end{figure}

	Since we assumed that $w$ does not have Property T, we know that $u$ in the fixed reduced product that we have for $w$ is non-trivial. That is, it contains some generators. Given our original reduced fixed expression for $w$ we add a new row to our heap if we were to add $s_0$ to the new row we would have a braid appear higher in the heap so we will not add $s_0$. This forces us to add $s_2$ and we get the configuration seen here \begin{figure}[h!] \centering
\begin{tikzpicture}[scale=0.5] 
	\heapblock{2}{10}{2}{teal}
	\heapblock{1}{8}{1}{teal}
	\heapblock{2}{6}{2}{teal}
	\heapblock{0}{6}{0}{purple}
	\heapblock{1}{4}{1}{purple}
	\heapblock{3}{4}{3}{purple}
	\heapblock{2}{2}{2}{rred}
\end{tikzpicture}\label{fig:heap2}
\end{figure}\\
Again this can not be the top row in our heap so we must add another level in our heap. Notice that adding $s_1$ would create a braid in our heap so we will add $s_3$ however in doing so we will also need to add $s_4$. The resulting heap is seen here \begin{figure}[h!] \centering
\begin{tikzpicture}[scale=0.5] 
	\heapblock{2}{10}{2}{teal}
	\heapblock{1}{8}{1}{teal}
	\heapblock{2}{6}{2}{teal}
	\heapblock{0}{6}{0}{purple}
	\heapblock{1}{4}{1}{purple}
	\heapblock{3}{4}{3}{purple}
	\heapblock{2}{2}{2}{rred}
	\heapblock{3}{0}{3}{rred}
	\heapblock{4}{2}{4}{rred}
\end{tikzpicture}\label{fig:heap3}
\end{figure}\\

We once again have the same issue arise that this can not be the top level of our configuration as $w$ would clearly have Property T on the top. Iterating this process we create the heap seen here
\begin{figure}[h!] \centering
\begin{tikzpicture}[scale=0.5] 
	\heapblock{2}{10}{2}{teal}
	\heapblock{1}{8}{1}{teal}
	\heapblock{2}{6}{2}{teal}
	\heapblock{0}{6}{0}{purple}
	\heapblock{1}{4}{1}{purple}
	\heapblock{3}{4}{3}{purple}
	\heapblock{2}{2}{2}{rred}
	\heapblock{3}{0}{3}{rred}
	\heapblock{4}{2}{4}{rred}
	\heapblock{5}{0}{5}{rred}
	\heapblock{4}{-2}{4}{rred}
	
	\node[] at (7,-2){$\ddots$};
	\node[] at (7,-4){$\ddots$};
	
	\heapblock{9}{-3}{n-3}{rred}
	\heapblock{11}{-3}{n-1}{rred}
	\heapblock{10}{-5}{n-2}{rred}
	\heapblock{12}{-5}{n}{rred}
	
\end{tikzpicture}\label{fig:heap4}
\end{figure}

Notice that again if this were the bottom row of the heap we would have Property T. Thus our construction can not be done. With this in mind we add $s_{n-1}$ to the heap which will still have Property T. As a result we play this game again and create the heap seen here.\begin{figure}[h!] \centering
\begin{tikzpicture}[scale=0.5] 
	\heapblock{2}{10}{2}{teal}
	\heapblock{1}{8}{1}{teal}
	\heapblock{2}{6}{2}{teal}
	\heapblock{0}{6}{0}{purple}
	\heapblock{1}{4}{1}{purple}
	\heapblock{3}{4}{3}{purple}
	\heapblock{2}{2}{2}{rred}
	\heapblock{3}{0}{3}{rred}
	\heapblock{4}{2}{4}{rred}
	\heapblock{5}{0}{5}{rred}
	\heapblock{4}{-2}{4}{rred}
	
	\node[] at (7,-2){$\ddots$};
	\node[] at (7,-4){$\ddots$};
	
	\heapblock{9}{-3}{n-3}{rred}
	\heapblock{11}{-3}{n-1}{rred}
	\heapblock{10}{-5}{n-2}{rred}
	\heapblock{12}{-5}{n}{rred}
	\heapblock{11}{-7}{n-1}{rred}
	\heapblock{9}{-7}{n-3}{rred}
	\heapblock{10}{-9}{n-2}{rred}
	
	\node[] at (7, -9){$\iddots$};
	\node[] at (7,-11){$\iddots$};
	
	\heapblock{3}{-10}{3}{rred}
	\heapblock{1}{-10}{1}{rred}
	\heapblock{2}{-12}{2}{rred}
	\heapblock{0}{-12}{0}{rred}
	\heapblock{4}{-12}{4}{rred}
\end{tikzpicture}\label{fig:heap5}
\end{figure}
	
Recall that $v$ is FC by assumption, where $v$ is the \textcolor{rred}{red} in the above heap. In~\cite[Lemma 3.3]{Ernst2012c}	 Ernst classified that an FC element of this sort has the blank space in the middle filled in. This forces our heap to look like the one seen in Figure~\ref{fig:heap6} where all of the blocks in the middle of the red v are filled in. As a result of this we now have $s_0$ in our heap. After applying the braid move to the \textcolor{teal}{teal} braid in Figure~\ref{fig:heap6}. This leads to the heap seen in Figure~\ref{fig:heap7} where a new \textcolor{orange}{orange} braid appeared. This implies that for the fixed reduced product $w=u\textcolor{teal}{s_2s_1s_2}s_0s_1s_3s_2v$ there is a heap with a braid that is lower in the heap. A contradiction to the way in which we chose $w$. Thus $w$ can not have the reduced expression $w=u\textcolor{teal}{s_2s_1s_2}s_0s_1s_3s_2v$.

\begin{figure}[h!]
\begin{tabular}{m{9cm} m{9cm}}
\begin{subfigure}{0.5\textwidth} \centering
\begin{tikzpicture}[scale=0.45] 
	\heapblock{2}{12}{}{white}
	\heapblock{2}{10}{2}{teal}
	\heapblock{1}{8}{1}{teal}
	\heapblock{2}{6}{2}{teal}
	\heapblock{0}{6}{0}{purple}
	\heapblock{1}{4}{1}{purple}
	\heapblock{3}{4}{3}{purple}
	\heapblock{2}{2}{2}{rred}
	\heapblock{3}{0}{3}{rred}
	\heapblock{4}{2}{4}{rred}
	\heapblock{5}{0}{5}{rred}
	\heapblock{4}{-2}{4}{rred}
	
	\node[] at (7,-2){$\ddots$};
	\node[] at (7,-4){$\ddots$};
	
	\heapblock{9}{-3}{n-3}{rred}
	\heapblock{11}{-3}{n-1}{rred}
	\heapblock{10}{-5}{n-2}{rred}
	\heapblock{12}{-5}{n}{rred}
	\heapblock{11}{-7}{n-1}{rred}
	\heapblock{9}{-7}{n-3}{rred}
	%\heapblock{10}{-7}{n-2}{rred}
	
	\node[] at (7, -7){$\iddots$};
	\node[] at (7,-9){$\iddots$};
	\node[] at (0,-4){$\vdots$};
	
	\heapblock{3}{-8}{3}{rred}
	\heapblock{1}{-8}{1}{rred}
	\heapblock{2}{-10}{2}{rred}
	\heapblock{0}{-10}{0}{rred}
	\heapblock{4}{-10}{4}{rred}
\end{tikzpicture}
\caption{}\label{fig:heap6}
\end{subfigure}&

\begin{subfigure}{0.5\textwidth} \centering
\begin{tikzpicture}[scale=0.45] 
	\heapblock{1}{12}{1}{teal}
	\heapblock{2}{10}{2}{teal}
	\heapblock{1}{8}{1}{orange}
	\heapblock{0}{6}{0}{orange}
	\heapblock{1}{4}{1}{orange}
	\heapblock{3}{4}{3}{purple}
	\heapblock{0}{2}{0}{orange}
	\heapblock{2}{2}{2}{rred}
	\heapblock{3}{0}{3}{rred}
	\heapblock{4}{2}{4}{rred}
	\heapblock{5}{0}{5}{rred}
	\heapblock{4}{-2}{4}{rred}
	
	\node[] at (7,-2){$\ddots$};
	\node[] at (7,-4){$\ddots$};
	
	\heapblock{9}{-3}{n-3}{rred}
	\heapblock{11}{-3}{n-1}{rred}
	\heapblock{10}{-5}{n-2}{rred}
	\heapblock{12}{-5}{n}{rred}
	\heapblock{11}{-7}{n-1}{rred}
	\heapblock{9}{-7}{n-3}{rred}
	%\heapblock{10}{-7}{n-2}{rred}
	
	\node[] at (7, -7){$\iddots$};
	\node[] at (7,-9){$\iddots$};
	\node[] at (0,-4){$\vdots$};
	
	\heapblock{3}{-8}{3}{rred}
	\heapblock{1}{-8}{1}{rred}
	\heapblock{2}{-10}{2}{rred}
	\heapblock{0}{-10}{0}{rred}
	\heapblock{4}{-10}{4}{rred}
\end{tikzpicture}
\caption{}\label{fig:heap7}	
\end{subfigure}
\end{tabular}
\end{figure}


	
	Case 3: Suppose the braid contains $1$ or $n-1$. Without loss of generality we will assume the braid contains $1$ as the other argument is symmetric to the one presented here. Subcase a: Suppose $w$ has the reduced product $w=u\textcolor{orange}{s_0s_1s_0s_1}s_2v$ where $v$ is fully commutative and does not contain $s_0$ in the left descent set. Notice that the braid is highlighted in \textcolor{orange}{orange}. Applying the braid move leads to the reduced product $w=u\textcolor{orange}{s_1s_0s_1s_0}s_2v$. Notice that the braid is now able to be in the same level of the heap as $s_2$ whereas it previously was not. Visually this is seen in Figure~\ref{fig:Case3a}. Notice how the braid in Figure~\ref{fig:case3a2} is located next to the block for $s_2$ whereas in Figure~\ref{fig:case3a1} the braid is stuck above the block for $s_2$. This is a contradiction to picking the heap with the lowest braid. %Thus $w$ can not have the reduced product $w=u\textcolor{orange}{s_0s_1s_0s_1}s_2v$.
	
	\begin{figure}[h!]
	\begin{tabular}{m{7cm} m{7cm}}
	\begin{subfigure}{0.5\textwidth} \centering
	\begin{tikzpicture}[scale=0.4]
		\heapblock{0}{10}{0}{orange}
		\heapblock{1}{8}{1}{orange}
		\heapblock{0}{6}{0}{orange}
		\heapblock{1}{4}{1}{orange}
		\heapblock{2}{2}{2}{purple}
		\dheapblock{0}{2}{}{black}
	\end{tikzpicture}
	\caption{}\label{fig:case3a1}
	\end{subfigure} &

	\begin{subfigure}{0.5\textwidth} \centering
	\begin{tikzpicture}[scale=0.4]
		\heapblock{0}{12}{}{white}
		\heapblock{1}{10}{1}{orange}
		\heapblock{0}{8}{0}{orange}
		\heapblock{1}{6}{1}{orange}
		\heapblock{0}{4}{0}{orange}
		\heapblock{2}{4}{2}{purple}
	\end{tikzpicture}
	\caption{}\label{fig:case3a2}
	\end{subfigure}
	\end{tabular}
	\caption{Visual representation of the heap configuration discussed in Case 3a.}\label{fig:Case3a}
	\end{figure}
	
	Subcase b: Suppose $w$ has the reduced product $w=u\textcolor{teal}{s_1s_2s_1}s_0v$ where $v$ is fully commutative and does not contain $s_2$ in the left descent set. Notice that the braid is highlighted in \textcolor{teal}{teal}. Applying the braid move leads to the reduced product $w=u\textcolor{teal}{s_2s_1s_2}s_0v$. Notice that the braid is now able to be located in the same level of the heap as $s_0$ whereas it previously was not. Visually this is seen in Figure~\ref{fig:Case3b}. Notice how the braid in Figure~\ref{fig:case3b2} is located next to the block for $s_0$, but the braid in Figure~\ref{fig:case3b1} it is stuck above the block for $s_0$. This is a contradiction to the way in which we picked our heap. Thus $w$ can not have the reduced product $w=u\textcolor{teal}{s_1s_2s_1}s_0v$. 
	
	\begin{figure}[h!]
	\begin{tabular}{m{7cm} m{7cm}}
	\begin{subfigure}{0.5\textwidth} \centering
	\begin{tikzpicture}[scale=0.40]
		\heapblock{1}{8}{1}{teal}
		\heapblock{2}{6}{2}{teal}
		\heapblock{1}{4}{1}{teal}
		\heapblock{0}{2}{0}{purple}
		\dheapblock{2}{2}{}{black}
	\end{tikzpicture}
	\caption{}\label{fig:case3b1}
	\end{subfigure} &

	\begin{subfigure}{0.5\textwidth} \centering
	\begin{tikzpicture}[scale=0.4]
		\heapblock{0}{8}{}{white}
		\heapblock{2}{6}{2}{teal}
		\heapblock{1}{4}{1}{teal}
		\heapblock{2}{2}{2}{teal}
		\heapblock{0}{2}{0}{purple}
	\end{tikzpicture}
	\caption{}\label{fig:case3b2}
	\end{subfigure}
	\end{tabular}
	\caption{Visual representation of the heap configuration discussed in Case 3b.}\label{fig:Case3b}
	\end{figure}
	
	Case 4: Suppose the braid contains $0$ or $n$. Without loss of generality we will assume the braid contains $0$ as the other argument is symmetric to the one presented here. Suppose $w$ has the fixed reduced product $w=u\textcolor{orange}{s_1s_0s_1s_0}s_2s_1v$ where $v$ is an FC element. Notice that we have highlighted the braid in \textcolor{orange}{orange}.
		\begin{figure}[h!]
	\begin{tabular}{m{7cm} m{7cm}}
	\begin{subfigure}{0.5\textwidth} \centering
	\begin{tikzpicture}[scale=0.40]
		\heapblock{0}{0}{}{white}
		\heapblock{1}{10}{1}{orange}
		\heapblock{0}{8}{0}{orange}
		\heapblock{1}{6}{1}{orange}
		\heapblock{0}{4}{0}{orange}
		\heapblock{2}{4}{2}{purple}
		\heapblock{1}{2}{1}{purple}
	\end{tikzpicture}
	\caption{}\label{fig:case4a}
	\end{subfigure} &

	\begin{subfigure}{0.5\textwidth} \centering
	\begin{tikzpicture}[scale=0.40]
		\heapblock{0}{10}{0}{orange}
		\heapblock{1}{8}{1}{orange}
		\heapblock{0}{6}{0}{orange}
		\heapblock{1}{4}{1}{teal}
		\heapblock{2}{2}{2}{teal}
		\heapblock{1}{0}{1}{teal}
	\end{tikzpicture}
	\caption{}\label{fig:case4b}
	\end{subfigure}
	\end{tabular}
	\caption{Visual representation of the heap configuration discussed in Case 4.}\label{fig:Case4}
	\end{figure} Applying the braid move we obtain the fixed reduced product $w=us_0s_1s_0\textcolor{teal}{s_1s_2s_1}v$. In applying the braid the resulting reduced product now has the braid highlighted in \textcolor{teal}{teal}. Notice that this braid is located next to $v$ which is located lower in the heap than our original $w$. Visually we see this in Figure~\ref{fig:Case4}. We see in Figure~\ref{fig:case4b} the braid in \textcolor{teal}{teal} is located below the braid that stars in \textcolor{orange}{orange} and ends with the \textcolor{teal}{$s_1$}. It is clear that this braid is lower than the \textcolor{orange}{orange} braid seen in Figure~\ref{fig:case4a}. Thus $w$ can not have the reduced product $w=u\textcolor{orange}{s_1s_0s_1s_0}s_2s_1v$.
	

	
	From this we see there is no possible way to find a reduced expression in $W(\C_n)/\FC(\Gamma)$ with full support and does not have Property T. Thus there are no non-trivial T-avoiding elements in $W(\C_n)/\FC(\Gamma)$.
	\end{proof}
\end{theorem}
 
We have now shown that there are no non-trivial T-avoiding elements in $W(\C_n)/\FC(\Gamma)$. We now proceed in a parity argument. We first will classify non-trivial T-avoiding elements in $W(\C_n)$ for $n$ odd. First recall that $W(\C_n)$ for $n$ odd is a Star reducible Coxeter group. This implies that there are no non-trivial T-avoiding FC elements in $W(\C_n)$ for $n$ odd. This leads to the following Theorem.

\begin{theorem}
	There are no non-trivial T-avoiding elements in the Coxeter group of type $\C_n$ for $n$ odd.
	\begin{proof}
		Consider the Coxeter group of type $\C_n$. By Theorem~\ref{thm:TavoidC} we know that $W(\C_n)$ contains no non-trivial T-avoiding elements that are not FC. Also since $W(\C_n)$ is a star reducible Coxeter group, we know that $W(\C_n)$ contains no non-trivial T-avoiding elements that are FC. Thus $W(\C_n)$ has no non-trivial T-avoiding elements.
	\end{proof}
\end{theorem}

We next will classify the non-trivial T-avoiding elements in the Coxeter group of type $\C_n$ for $n$ even. Recall that $W(\C_n)$ for $n$ even is not a Star reducible Coxeter group. In Theorem~\ref{thm:TavoidC} we showed that $W(\C_n)$ does not have non-trivial T-avoiding elements that are not FC. This leaves us with only the FC elements to check.

\begin{theorem}
	There are non-trivial T-avoiding elements in the Coxeter group of type $\C_n$.
	\begin{proof}
		Let $\w=s_0s_2 \cdots s_{n-3}s_{n-1}(s_1s_3\cdots s_{n-2}s_0s_2\cdots s_{n-3}s_{n-1})^k$ be a reduced expression for $w \in W(\C_n)$.
	\end{proof}
\end{theorem}
