%%\section{String diagrams}\label{sec:strings}
%%    Following~\cite{billey07}, in Coxeter groups of type $A_n$, the heap representation can be combined with another model for permutations in which the entries from the one-line notation are represented by strings, called a \emph{string diagram}. We draw the strings from bottom to top so that they cross at each entry in the heap.
%%    The string diagram helps us to visualize the relationship between the one-line notation for a permutation and the corresponding heap.
%%    For some reduced expression $\w$ in $W(A_n)$, there exists a string diagram that is obtained from $H(\w)$ by adding strings that cross at each lattice point of $H(\w)$.
%%    
%%\begin{example} Let $w \in W(A_4)$ have reduced expression $\w = 12434$. Then one representation of the heap of $\w$ is
%%\begin{center} \begin{tabular}{m{2.75cm} m{0.5cm}}
%%\begin{tikzpicture}
%%    \sq{0.5}{2};  \node at (1,1.5)    {\footnotesize $4$};
%%    \sq{0}{1};    \node at (0.5,0.5)  {\footnotesize $3$};
%%    \sq{0.5}{0};  \node at (1,-0.5)   {\footnotesize $4$};
%%    \sq{-0.5}{2}; \node at (0,1.5)    {\footnotesize $2$};
%%    \sq{-1}{3};   \node at (-0.5,2.5) {\footnotesize $1$};
%%\end{tikzpicture} & . \end{tabular} \end{center}
%%    To write $\w$ in cycle notation, we send each generator $i$ to $(i~i+1)$. Then we have, multiplying transpositions right to left, $$12434 \mapsto (12)(23)(45)(34)(45) = (1235).$$
%%    So, $\w = 12434$ in $W(A_4)$ corresponds to the cycle $(1235)$ in $S_5$.
%%    For the one-line notation, we get $$(1235) \equiv [23541]$$ since 1 is sent to 2, 2 is sent to 3, 3 is sent to 5, 4 is sent to itself, and 5 is sent to 1.
%%    
%%    Now consider the string diagram for $[23541]$. We have two rows of nodes $\{1,2,3,4,5\}$, top and bottom, and we connect nodes from bottom to top using the one-line notation. Since 1 is sent to 2, we connect the first node in the bottom row to the second node in the top row.
%%    Then a string diagram for $[23541]$ is
%%\begin{center} \begin{tabular}{m{4.75cm} m{0.5cm} m{4cm}}
%%\scalebox{0.8}{$$\xymatrix{
%%    \bullet           & \bullet           & \bullet           & \bullet           & \bullet \\
%%    \bullet\ar@{-}@[purple][ur] & \bullet\ar@{-}@[purple][ul] & \bullet\ar@{-}[u] & \bullet\ar@{-}[u] & \bullet\ar@{-}[u] \\
%%    \bullet\ar@{-}[u] & \bullet\ar@{-}@[magenta][ur] & \bullet\ar@{-}@[magenta][ul] & \bullet\ar@{-}@[orange][ur]& \bullet\ar@{-}@[orange][ul] \\
%%    \bullet\ar@{-}[u] & \bullet\ar@{-}[u] & \bullet\ar@{-}@[blue][ur]&  \bullet\ar@{-}@[blue][ul] & \bullet\ar@{-}[u] \\
%%    \bullet\ar@{-}[u] & \bullet\ar@{-}[u] & \bullet\ar@{-}[u] & \bullet\ar@{-}@[ggreen][ur] & \bullet\ar@{-}@[ggreen][ul]
%%}$$} & $=$ &
%%\scalebox{0.9}{$$\xymatrix{
%%    \bullet & \bullet & \bullet & \bullet & \bullet  \\ &&& \\
%%    \bullet\ar@{-}[uur] & \bullet\ar@{-}[uur] & \bullet\ar@{-}[uurr] & \bullet\ar@{-}[uu] & \bullet\ar@{-}[uullll]
%%}$$}
%%\end{tabular} \end{center}
%%    because $(1235) = \textcolor{purple}{(12)} \textcolor{magenta}{(23)} \textcolor{orange}{(45)} \textcolor{blue}{(34)} \textcolor{ggreen}{(45)}$ and composition goes up in the picture.
%%    It is known that the crossings in the string diagram correspond to blocks in the heap \cite{jones}. The transpositions $(23)$ and $(45)$ are disjoint, and hence they commute, so they are on the same level in the string diagram, just as the $s_2$ and $s_4$ blocks are on the same level of the heap. %Notice that there are five crossings and five blocks in the heap.
%%    The triangle created by crossings in the string diagram corresponds to the subheap
%%\begin{center} \begin{tabular}{m{1.5cm} m{0.5cm}} \centering
%%\begin{tikzpicture}[scale=0.85]
%%    \sq{0.5}{2}; \node at (1,1.5)   {\footnotesize $4$};
%%    \sq{0}{1};   \node at (0.5,0.5) {\footnotesize $3$};
%%    \sq{0.5}{0}; \node at (1,-0.5)  {\footnotesize $4$};
%%\end{tikzpicture} & .
%%\end{tabular} \end{center}
%%    Notice that there is another choice for the crossings in the string diagram. We could have drawn the $5 \to 1$ \emph{above} the crossing created by the $3 \to 5$ to $4 \to 4$ strings. This new string diagram corresponds to another heap for $w$.
%%    The other string diagram and corresponding heap are
%%\begin{center} \begin{tabular}{m{6cm} m{1cm} m{3cm} m{0.5cm}}
%%$$\xymatrix{
%%    \bullet & \bullet & \bullet & \bullet & \bullet  \\ &&& \\
%%    \bullet\ar@{-}[uur] & \bullet\ar@{-}[uur] & \bullet\ar@/_1.1pc/@{-}[uurr] & \bullet\ar@{-}[uu] & \bullet\ar@/_0.85pc/@{-}[uullll]
%%}$$
%%& and &
%%\begin{tikzpicture}
%%    \sq{0}{-1};   \node at (0.5,-1.5) {\footnotesize $3$};
%%    \sq{0.5}{0};  \node at (1,-0.5)   {\footnotesize $4$};
%%    \sq{0}{1};    \node at (0.5,0.5)  {\footnotesize $3$};
%%    \sq{-0.5}{2}; \node at (0,1.5)    {\footnotesize $2$};
%%    \sq{-1}{3};   \node at (-0.5,2.5) {\footnotesize $1$};
%%\end{tikzpicture} & .
%%\end{tabular} \end{center}
%%
%%%\textcolor{blue}{(I don't know how to do the curved strings yet, so that's (obviously) still the same heap from before. I'll change it when I figure out the string stuff.)}
%%\end{example}
%%
%%\begin{remark} It follows that an element $w \in W$ is FC if and only if there is exactly one heap for $w$ if and only if there is a unique crossing pattern for the string diagram. That is, there are no choices for crossings in the string diagram for $w$.
%%\end{remark}
%%
%%\begin{example} Let $w \in W(A_5)$ have reduced expression $\w = 14235$. Then, clearly, $w \in \FC(A_5)$ since $w$ is a Coxeter element.
%%    Now we have that $\w$ corresponds to $(12)(45)(23)(34)(56) = (123564)$ in $S_6$, which is $[235164]$ in one-line notation.
%%    Then the string diagram for $w$ is
%%$$\xymatrix{
%%    \bullet & \bullet & \bullet & \bullet & \bullet & \bullet  \\ &&& \\
%%    \bullet\ar@{-}[uur] & \bullet\ar@{-}[uur] & \bullet\ar@{-}[uurr] & \bullet\ar@{-}[uulll] & \bullet\ar@{-}[uur] & \bullet\ar@{-}[uull]
%%}$$
%%    As expected, there were no choices for crossings since $w$ is FC. Thus there is a unique heap for $w$, which is
%%\begin{center} \begin{tabular}{m{3cm} m{0.5cm}} \begin{tikzpicture}
%%    \sq{0}{2};    \node at (0.5,1.5)  {\footnotesize $1$};
%%    \sq{0.5}{1};  \node at (1,0.5)    {\footnotesize $2$};
%%    \sq{1.5}{1};  \node at (2,0.5)    {\footnotesize $4$};
%%    \sq{1}{0};    \node at (1.5,-0.5) {\footnotesize $3$};
%%    \sq{2}{0};    \node at (2.5,-0.5) {\footnotesize $5$};
%%\end{tikzpicture} & .
%%\end{tabular} \end{center}
%%\end{example}
%
%\chapter{Representations of permutations from Coxeter groups of type $A_n$}
%\section{Pattern avoidance for CFC elements}
%    In this section, $W$ refers to the Coxeter group of type $A_n$. Recall that $W$ is isomorphic to the symmetric group $S_{n+1}$ via the mapping that sends $s_i$ to the adjacent transposition $(i~i+1)$.
%    We will not make a distinction between an element from $W(A_n)$ and the corresponding permutation in $S_{n+1}$.
%%    We will consider different representations of permutations from the symmetric group such as cycle notation and one-line notation.
%
%%\section{One-line notation}
%    As a convention, we will multiply (compose) permutations right to left.
%    Recall that if $w \in S_n$, then $[w(1)~w(2) \cdots w(n)]$ is the \emph{one-line notation} corresponding to $w$. Note the use of brackets.
%
%\begin{example} Let $W$ be the Coxeter graph of type $A_4$. Let $w \in W$ have reduced expression $\w = 12342$.
%    Then the corresponding permutation in $S_5$ is $$(12)(23)(34)(45)(23) = (1245).$$
%    Then, in one-line notation, we have $$(1245) = [24351]$$ since 1 is sent to 2, 2 is sent to 4, 3 is sent to itself, 4 is sent to 5, and 5 is sent back to 1.
%\end{example}
%
%    We can depict the one-line notation of a permutation $w$ as a graph to see its shape. A \emph{permutation line graph} has line segments joining $(i,w(i))$ to $(i+1,w(i+1))$ for each $1 \leq i \leq n-1$.
%
%\begin{example}\label{ex:linegraphs} We consider some permutation line graphs. \begin{enumerate}[label=(\alph*)]
%\item Consider the permutation $w = [2413]$. Then the permutation line graph is
%\begin{center} \begin{tabular}{m{5.75cm} m{1cm}}
%\begin{tikzpicture}[scale=0.85] \begin{scriptsize}
%\draw (0,0)--(4.5,0); \draw (0,0)--(0,4.5);
%\foreach \x in {1,2,3,4} \draw[shift={(\x,0)},color=black] (0pt,2pt)--(0pt,-2pt);
%\foreach \y in {1,2,3,4} \draw[shift={(0,\y)},color=black] (2pt,0pt)--(-2pt,0pt);
%    \draw (1,-0.25) node {$1$};
%    \draw (2,-0.25) node {$2$};
%    \draw (3,-0.25) node {$3$};
%    \draw (4,-0.25) node {$4$};
%    \draw (-0.8,1)  node {$w(3) = 1$};
%    \draw (-0.8,2)  node {$w(1) = 2$};
%    \draw (-0.8,3)  node {$w(4) = 3$};
%    \draw (-0.8,4)  node {$w(2) = 4$};
%    \draw[fill=black] (1,2) circle (1pt);
%    \draw[fill=black] (2,4) circle (1pt);
%    \draw[fill=black] (3,1) circle (1pt);
%    \draw[fill=black] (4,3) circle (1pt);
%    \draw (1,2)--(2,4)--(3,1)--(4,3);
%\end{scriptsize} \end{tikzpicture} & .
%\end{tabular} \end{center}
%
%\item Consider the permutation $w = [315462]$. Then the permutation line graph is
%\begin{center} \begin{tabular}{m{6.5cm} m{1cm}}
%\begin{tikzpicture}[scale=0.85] \begin{scriptsize}
%\draw (0,0)--(6.5,0); \draw (0,0)--(0,6.5);
%\foreach \x in {1,2,3,4,5,6} \draw[shift={(\x,0)},color=black] (0pt,2pt)--(0pt,-2pt);
%\foreach \y in {1,2,3,4,5,6} \draw[shift={(0,\y)},color=black] (2pt,0pt)--(-2pt,0pt);
%    \draw (1,-0.25) node {$1$};
%    \draw (2,-0.25) node {$2$};
%    \draw (3,-0.25) node {$3$};
%    \draw (4,-0.25) node {$4$};
%    \draw (5,-0.25) node {$5$};
%    \draw (6,-0.25) node {$6$};
%    \draw (-0.25,1) node {$1$};
%    \draw (-0.25,2) node {$2$};
%    \draw (-0.25,3) node {$3$};
%    \draw (-0.25,4) node {$4$};
%    \draw (-0.25,5) node {$5$};
%    \draw (-0.25,6) node {$6$};
%    \draw[fill=black]   (1,3) circle (1pt);
%    \draw[fill=black]   (2,1) circle (1pt);
%    \draw[fill=black]   (3,5) circle (1pt);
%    \draw[fill=black]   (4,4) circle (1pt);
%    \draw[fill=black]   (5,6) circle (1pt);
%    \draw[fill=black]   (6,2) circle (1pt);
%%    \draw[color=ggreen] (3,5) circle (4pt);
%%    \draw[color=ggreen] (4,4) circle (4pt);
%%    \draw[color=ggreen] (6,2) circle (4pt);
%    \draw (1,3)--(2,1)--(3,5)--(4,4)--(5,6)--(6,2);
%\end{scriptsize} \end{tikzpicture} & .
%\end{tabular} \end{center}
%%    Note that the circled points constitute the 321 pattern in $[31\textcolor{ggreen}{54}6\textcolor{ggreen}{2}]$, highlighted in \textcolor{ggreen}{green}.
%\end{enumerate}
%\end{example}
%
%
%%\section{Pattern avoidance and CFC elements}
%%    To classify a group element of $W(A_n)$ as fully commutative or cyclically fully commutative, we consider the one-line notation of the corresponding permutation in $S_{n+1}$ and determine its \emph{pattern avoidance}.
%    Using the notion of \emph{pattern avoidance}, we can determine whether a permutation is FC or CFC by inspecting its one-line notation.
%    If $w \in W(A_n)$, then $w$ avoids the pattern $321$ if there is no subset $\{i,j,k\} \subseteq \{1,\ldots,n+1\}$ with $i < j < k$ and $w(k) < w(j) < w(i)$.
%    Similarly, a permutation $w$ avoids the pattern $3412$ if there is no subset $\{i,j,k,\ell\} \subseteq \{1,\ldots,n+1\}$ with $i < j < k < \ell$ and $w(k) < w(\ell) < w(i) < w(j)$.
%    
%    Note that the elements that constitute the 321 and 3412 patterns need not be consecutive. To have a 321 pattern, the one-line notation must have a strictly descending subsequence of three elements.
%    The permutation line graphs of the patterns 321 and 3412 are shown in Figure~\ref{fig:patternmountains}.
%
%\begin{example}\label{ex:patterns} Let $w \in W(A_5)$ have reduced expression $\w = 234513$. Then $w$ corresponds to $$(23)(34)(45)(56)(12)(34) = (13562)$$ in $S_6$. The one-line notation for $w$ is $[31\textcolor{ggreen}{54}6\textcolor{ggreen}{2}]$. 
%    There is a 321 pattern in the one-line notation, highlighted in \textcolor{ggreen}{green}, but there is no 3412 pattern. The permutation line graph for $[315462]$ is shown in part (b) of Example~\ref{ex:linegraphs}.
%\end{example}
%
%\begin{proposition}[Billey, \cite{billey07}]\label{prop:321FC} An element $w \in W(A_n)$ is FC if and only if $w$ is 321-avoiding. \qed
%\end{proposition}
%
%\begin{center} \begin{figure}[h!]
%\begin{subfigure}{3in} \centering
%\begin{tikzpicture}%[scale=0.85]
%\draw (0,0)--(3.5,0); \draw (0,0)--(0,3.5);
%\foreach \x in {1,2,3} \draw[shift={(\x,0)},color=black] (0pt,2pt)--(0pt,-2pt);
%\foreach \y in {1,2,3} \draw[shift={(0,\y)},color=black] (2pt,0pt)--(-2pt,0pt);
%\begin{scriptsize}
%    \draw (1,-0.3) node {$i$};
%    \draw (2,-0.3) node {$j$};
%    \draw (3,-0.3) node {$k$};
%    \draw (-0.4,1) node {$w(k)$};
%    \draw (-0.4,2) node {$w(j)$};
%    \draw (-0.4,3) node {$w(i)$};
%    \draw[fill=black] (1,3) circle (1.5pt);
%    \draw[fill=black] (2,2) circle (1.5pt);
%    \draw[fill=black] (3,1) circle (1.5pt);
%    \draw (0.5,-0.3) node {$\cdots$};
%    \draw (1.5,-0.3) node {$\cdots$};
%    \draw (2.5,-0.3) node {$\cdots$};
%    \draw (-0.4,0.5) node {$\vdots$};
%    \draw (-0.4,1.55) node {$\vdots$};
%    \draw (-0.4,2.55) node {$\vdots$};
%\end{scriptsize}\end{tikzpicture}
%\caption{The pattern 321}\label{fig:321}
%\end{subfigure}
%\begin{subfigure}{3in} \centering
%\begin{tikzpicture}[scale=0.8]
%\draw (0,0)--(4.5,0); \draw (0,0)--(0,4.5);
%\foreach \x in {1,2,3,4} \draw[shift={(\x,0)},color=black] (0pt,2pt)--(0pt,-2pt);
%\foreach \y in {1,2,3,4} \draw[shift={(0,\y)},color=black] (2pt,0pt)--(-2pt,0pt);
%\begin{scriptsize}
%    \draw (1,-0.3) node {$i$};
%    \draw (2,-0.3) node {$j$};
%    \draw (3,-0.3) node {$k$};
%    \draw (4,-0.3) node {$\ell$};
%    \draw (-0.5,1) node {$w(k)$};
%    \draw (-0.5,2) node {$w(\ell)$};
%    \draw (-0.5,3) node {$w(i)$};
%    \draw (-0.5,4) node {$w(j)$};
%    \draw[fill=black] (1,3) circle (2pt);
%    \draw[fill=black] (2,4) circle (2pt);
%    \draw[fill=black] (3,1) circle (2pt);
%    \draw[fill=black] (4,2) circle (2pt);
%    \draw (0.5,-0.3) node {$\cdots$};
%    \draw (1.5,-0.3) node {$\cdots$};
%    \draw (2.5,-0.3) node {$\cdots$};
%    \draw (3.5,-0.3) node {$\cdots$};
%    \draw (-0.5,0.5) node {$\vdots$};
%    \draw (-0.5,1.55) node {$\vdots$};
%    \draw (-0.5,2.55) node {$\vdots$};
%    \draw (-0.5,3.55) node {$\vdots$};
%\end{scriptsize} \end{tikzpicture}
%\caption{The pattern 3412}\label{fig:3412}
%\end{subfigure}
%\caption{The shapes of the 321 and 3412 patterns.} \label{fig:patternmountains}
%\end{figure} \end{center}
%
%    The following proposition about pattern avoidance is from \cite{CFC}.
%    
%\begin{proposition}[Boothby, et al., \cite{CFC}]\label{prop:patterns} An element $w \in W(A_n)$ is CFC if and only if $w$ is $321$- and $3412$-avoiding. \qed
%\end{proposition}
%
%\begin{example}\label{ex:permscyclic} We will now explore a few examples. \begin{enumerate}[leftmargin=0.75in, label=(\alph*)]
%\item Let $W$ be the Coxeter group of type $A_3$. Then $W \cong S_4$. Let $w \in W$ have reduced expression $\w = 132$. Then $w$ is CFC since $w$ is a Coxeter element. In this case, its unique heap is
%\begin{center} \begin{tabular}{m{2cm} m{0.5cm}}
%\begin{tikzpicture}
%    \sq{0}{1};   \node at (0.5,0.5) {$1$};
%    \sq{1}{1};   \node at (1.5,0.5) {$3$};
%    \sq{0.5}{0}; \node at (1,-0.5)  {$2$};
%\end{tikzpicture} & . \end{tabular} \end{center}
%    We see that $w$ corresponds to the permutation $(12)(34)(23) = (1243) = [2413]$ in $S_4$ in cycle notation and one-line notation.
%%    The string diagram is
%%\begin{center} \begin{tabular}{m{4.5cm} m{0.5cm}}
%%$$\xymatrix{
%%    \bullet & \bullet & \bullet & \bullet  \\ &&& \\
%%    \bullet\ar@{-}[uur] & \bullet\ar@{-}[uurr] & \bullet\ar@{-}[uull] &\bullet\ar@{-}[uul]
%%}$$ & .
%%\end{tabular} \end{center}
%%    Notice that the crossings of the string diagram have the same shape as the heap, and we had no choices for crossings, so there is only one heap, which is also clear from the heap representation.
%%    Hence $w$ is fully commutative.
%    The permutation line graph is shown in part (a) of Example~\ref{ex:linegraphs}.
%    Since there are only four elements and is not $[3412]$ exactly, $w$ is clearly 3412-avoiding. It is also 321-avoiding because there is not a strictly decreasing subsequence of three elements in the one-line notation. We can also see this in the permutation line graph.
%    These conclusions agree with Proposition~\ref{prop:patterns}.
%	
%\item Let $W$ be the Coxeter group of type $A_3$. Then $W \cong S_4$. Let $w \in W$ have reduced expression $\w = 3213$. Then, in cycle and one-line notations, we have that $w$ corresponds to $$(34)(23)(12)(34) = (124) = [2431]$$ in $S_4$.
%    Since $431$ in the one-line notation is a 321 pattern, $[2431]$ is not 321-avoiding. %Again, since the one-line notation has  only four generators but is not $[3412]$ exactly, it is 3412-avoiding. Hence $w$ is not cyclically fully commutative.
%    %Now, we consider the string diagram to determine if $w$ is FC. The string diagram is
%%\begin{center} 
%%\begin{tabular}{m{4.5cm} m{0.5cm}}
%%$$\xymatrix{
%%    \bullet\ar@/^1pc/@{.}[ddrrr] & \bullet & \bullet & \bullet  \\ &&& \\
%%    \bullet\ar@{-}[uur] & \bullet\ar@/_/@{-}[uurr] & \bullet\ar@{-}[uu] &\bullet\ar@/^1pc/@{-}[uulll]
%%}$$ & .
%%\end{tabular} \end{center}
%%    Note that there are two choices for crossings in the string diagram. We could have drawn the $4 \to 1$ string \emph{over} the crossing of the $3 \to 3$ and $2 \to 4$ strings, as denoted by a dotted string in the string diagram above.
%    We can see this in the permutation line graph
%\begin{center} \begin{tabular}{m{5cm} m{1cm}}
%\begin{tikzpicture}[scale=0.85] \begin{scriptsize}
%\draw (0,0)--(4.5,0); \draw (0,0)--(0,4.5);
%\foreach \x in {1,2,3,4} \draw[shift={(\x,0)},color=black] (0pt,2pt)--(0pt,-2pt);
%\foreach \y in {1,2,3,4} \draw[shift={(0,\y)},color=black] (2pt,0pt)--(-2pt,0pt);
%    \draw (1,-0.25) node {$1$};
%    \draw (2,-0.25) node {$2$};
%    \draw (3,-0.25) node {$3$};
%    \draw (4,-0.25) node {$4$};
%    \draw (-0.8,1)  node {$1$};
%    \draw (-0.8,2)  node {$2$};
%    \draw (-0.8,3)  node {$3$};
%    \draw (-0.8,4)  node {$4$};
%    \draw[fill=black] (1,2) circle (1pt);
%    \draw[fill=black] (2,4) circle (1pt); \draw[color=ggreen] (2,4) circle (3.5pt);
%    \draw[fill=black] (3,3) circle (1pt); \draw[color=ggreen] (3,3) circle (3.5pt);
%    \draw[fill=black] (4,1) circle (1pt); \draw[color=ggreen] (4,1) circle (3.5pt);
%    \draw (1,2)--(2,4)--(3,3)--(4,1);
%\end{scriptsize} \end{tikzpicture} & ,
%\end{tabular} \end{center}
%    where the circled points correspond to the elements that constitute the 321 pattern.
%    Then, by Proposition~\ref{prop:321FC}, $w$ is not FC, and hence not CFC. In this case, the heaps for $w$ are
%\begin{center} \begin{tabular}{m{2.4cm} m{1cm} m{2.2cm} m{0.5cm}} \centering
%\begin{tikzpicture}
%    \sq{1}{2};   \node at (1.5,1.5) {$3$};
%    \sq{0.5}{1}; \node at (1,0.5)   {$2$};
%    \sq{0}{0};   \node at (0.5,-0.5){$1$};
%    \sq{1}{0};   \node at (1.5,-0.5){$3$};
%\end{tikzpicture} & and &
%\begin{tikzpicture}
%    \sq{0.5}{3}; \node at (1,2.5)   {$2$};
%    \sq{1}{2};   \node at (1.5,1.5) {$3$};
%    \sq{0.5}{1}; \node at (1,0.5)   {$2$};
%    \sq{0}{0};   \node at (0.5,-0.5){$1$};
%\end{tikzpicture} & .
%\end{tabular} \end{center}
%%    Thus, $w$ is not CFC, and hence it is not even FC.
%    It is clear from the heaps that $w$ is not FC because each heap contains a convex subheap corresponding to the braid relation $323 = 232$.
%
%\item Let $W$ be the Coxeter group of type $A_n$ where $n$ is at least 7. Let $w \in W$ correspond to the element $[\cdots \textcolor{magenta}{5}1\textcolor{magenta}{83}2\textcolor{magenta}{4} \cdots]$ of $S_{n+1}$.
%    Note that this element is not 3412-avoiding because the pink elements create a 3412 pattern. So, $w$ is not CFC.
%    Moreover, $w$ is not even FC by Proposition~\ref{prop:321FC} because 832 exhibits a 321 pattern.
%	
%\item Let $W$ be the Coxeter group of type $A_3$. Then $W \cong S_4$. Let $w \in W$ have reduced expression $\w = 2132$. Then $w$ is FC and the heap of $w$ is 
%\begin{center} \begin{tabular}{m{2.5cm} m{0.5cm}}
%\begin{tikzpicture}
%    \sq{0.5}{1}; \node at (1,0.5)   {$2$};
%    \sq{0}{0};   \node at (0.5,-0.5){$1$};
%    \sq{1}{0};   \node at (1.5,-0.5){$3$};
%    \sq{0.5}{-1};\node at (1,-1.5)  {$2$};
%\end{tikzpicture} & .
%\end{tabular} \end{center}
%%    and the string diagram is
%%\begin{center} \begin{tabular}{m{5.5cm} m{0.5cm}}
%%    $$\xymatrix{
%%        \bullet & \bullet & \bullet & \bullet  \\ &&& \\
%%        \bullet\ar@{-}[uurr] & \bullet\ar@{-}[uurr] & \bullet\ar@{-}[uull] &\bullet\ar@{-}[uull]
%%    }$$ & .
%%\end{tabular} \end{center}
%%    Notice that there were no choices for crossings in the string diagram, so there is only one heap for $w$.
%    Inspecting the heap makes it clear that $w$ is not CFC. Moreover, we see that the corresponding permutation is $$(23)(12)(34)(23) = (13)(24) = [3412],$$ which is obviously not 3412-avoiding but is 321-avoiding.
%%    We can see this in the permutation line graph
%%\begin{center} \begin{tabular}{m{5cm} m{1cm}}
%%\begin{tikzpicture}[scale=0.85] \begin{scriptsize}
%%\draw (0,0)--(4.5,0); \draw (0,0)--(0,4.5);
%%\foreach \x in {1,2,3,4} \draw[shift={(\x,0)},color=black] (0pt,2pt)--(0pt,-2pt);
%%\foreach \y in {1,2,3,4} \draw[shift={(0,\y)},color=black] (2pt,0pt)--(-2pt,0pt);
%%    \draw (1,-0.25) node {$1$};
%%    \draw (2,-0.25) node {$2$};
%%    \draw (3,-0.25) node {$3$};
%%    \draw (4,-0.25) node {$4$};
%%    \draw (-0.8,1)  node {$1$};
%%    \draw (-0.8,2)  node {$2$};
%%    \draw (-0.8,3)  node {$3$};
%%    \draw (-0.8,4)  node {$4$};
%%    \draw[fill=black] (1,3) circle (1pt); \draw[color=ggreen] (1,3) circle (3.5pt);
%%    \draw[fill=black] (2,4) circle (1pt); \draw[color=ggreen] (2,4) circle (3.5pt);
%%    \draw[fill=black] (3,1) circle (1pt); \draw[color=ggreen] (3,1) circle (3.5pt);
%%    \draw[fill=black] (4,2) circle (1pt); \draw[color=ggreen] (4,2) circle (3.5pt);
%%    \draw (1,3)--(2,4)--(3,1)--(4,2);
%%\end{scriptsize} \end{tikzpicture} & ,
%%\end{tabular} \end{center}
%%    where the circled points correspond to the elements that constitute the 3412 pattern.
%
%\item \label{ex:cyclicshifts} Let $w \in W(A_4)$ have reduced expression $\w = 1234$. Then $w$ is FC and the heap of $w$ is
%\begin{center} \begin{tabular}{m{2.4cm} m{0.5cm}} \centering \begin{tikzpicture}%[scale=0.85]
%    \sq{0}{2};    \node at (0.5,1.5)  {\footnotesize $1$};
%    \sq{0.5}{1};  \node at (1,0.5)    {\footnotesize $2$};
%    \sq{1}{0};    \node at (1.5,-0.5) {\footnotesize $3$};
%    \sq{1.5}{-1}; \node at (2,-1.5)   {\footnotesize $4$};
%\end{tikzpicture} & . \end{tabular} \end{center}
%    Then $w$ corresponds to $(12)(23)(34)(45) = (12345)$. Consider all possible sequences of cyclic shifts of $H(w)$ and their corresponding permutations in $S_5$. We have 
%\begin{center} \begin{tabular}{m{2.8cm} m{0.75cm} m{5cm}}     
%    \centering
%\begin{tikzpicture}[scale=0.85]
%    \sq{0}{0};    \node at (0.5,-0.5) {\footnotesize $1$};
%    \sqm{0.5}{1}; \node at (1,0.5)    {\footnotesize $2$};
%    \sq{1}{0};    \node at (1.5,-0.5) {\footnotesize $3$};
%    \sq{1.5}{-1}; \node at (2,-1.5)   {\footnotesize $4$};
%\end{tikzpicture} & $\mapsto$ &
%    $(23)(12)(34)(45) = (13452)$ \\ && \\
%
%\begin{tikzpicture}[scale=0.85]
%    \sq{0}{0};    \node at (0.5,-0.5) {\footnotesize $1$};
%    \sq{0.5}{-1}; \node at (1,-1.5)   {\footnotesize $2$};
%    \sqm{1}{0};   \node at (1.5,-0.5) {\footnotesize $3$};
%    \sq{1.5}{-1}; \node at (2,-1.5)   {\footnotesize $4$};
%\end{tikzpicture} & $\mapsto$ &
%    $(12)(34)(23)(45) = (12453)$ \\ && \\
%
%\begin{tikzpicture}[scale=0.85]
%    \sqm{0}{0};   \node at (0.5,-0.5) {\footnotesize $1$};
%    \sq{0.5}{-1}; \node at (1,-1.5)   {\footnotesize $2$};
%    \sq{1}{-2};   \node at (1.5,-2.5) {\footnotesize $3$};
%    \sq{1.5}{-1}; \node at (2,-1.5)   {\footnotesize $4$};
%\end{tikzpicture} & $\mapsto$ &
%    $(12)(23)(45)(34) = (12354)$ \\ && \\
%
%\begin{tikzpicture}[scale=0.85]
%    \sq{0}{0};    \node at (0.5,-0.5) {\footnotesize $1$};
%    \sqm{0.5}{1}; \node at (1,0.5)    {\footnotesize $2$};
%    \sq{1}{0};    \node at (1.5,-0.5) {\footnotesize $3$};
%    \sq{1.5}{1};  \node at (2,0.5)    {\footnotesize $4$};
%\end{tikzpicture} & $\mapsto$ &
%    $(23)(45)(12)(34) = (13542)$ \\ && \\
%
%\begin{tikzpicture}[scale=0.85]
%    \sqm{0}{0};   \node at (0.5,-0.5) {\footnotesize $1$};
%    \sq{0.5}{-1}; \node at (1,-1.5)   {\footnotesize $2$};
%    \sq{1}{0};    \node at (1.5,-0.5) {\footnotesize $3$};
%    \sq{1.5}{1};  \node at (2,0.5)    {\footnotesize $4$};
%\end{tikzpicture} & $\mapsto$ &
%    $(45)(12)(34)(23) = (14532)$ \\ && \\
%
%\begin{tikzpicture}[scale=0.85]
%    \sq{0}{-2};   \node at (0.5,-2.5) {\footnotesize $1$};
%    \sq{0.5}{-1}; \node at (1,-1.5)   {\footnotesize $2$};
%    \sq{1}{0};    \node at (1.5,-0.5) {\footnotesize $3$};
%    \sqm{1.5}{1}; \node at (2,0.5)    {\footnotesize $4$};
%\end{tikzpicture} & $\mapsto$ &
%    $(45)(34)(23)(12) = (15432)$ \\ && \\
%
%\begin{tikzpicture}[scale=0.85]
%    \sq{0}{0};    \node at (0.5,-0.5) {\footnotesize $1$};
%    \sq{0.5}{1};  \node at (1,0.5)    {\footnotesize $2$};
%    \sq{1}{2};    \node at (1.5,1.5)  {\footnotesize $3$};
%    \sq{1.5}{1};  \node at (2,0.5)    {\footnotesize $4$};
%\end{tikzpicture} & $\mapsto$ &
%    $(34)(23)(45)(12) = (14532)$
%\end{tabular} \end{center}
%    by shifting the \textcolor{magenta}{pink} blocks to obtain the heap that follows. Since $w$ is a Coxeter element, it is CFC, and every cyclic shift of $w$ is also CFC by Remark~\ref{rem:CoxCFC}. 
%\end{enumerate} \end{example}
%
%    Recall that two permutations are conjugate if and only if they have the same cycle type.
%    Since Coxeter elements in $W(A_n)$ correspond to $(n+1)$-cycles in $S_{n+1}$, all Coxeter elements are conjugate as they have the same cycle type. So, all Coxeter elements are cyclically equivalent by Theorem~\ref{thm:e2} and as seen in Figure~\ref{fig:A4boxes}, but not all $(n+1)$-cycles correspond to Coxeter elements.
%     
%    For example, $1234 = (12)(23)(34)(45) = (12345)$. There are seven other Coxeter elements conjugate to $1234$, as shown in Figure~\ref{fig:coxeltsinA4}, but there are 24 distinct 5-cycles in $S_5$.
%
%%    Using cycle type to determine conjugacy for other types of elements is not as straightforward.
%    Given a product of disjoint cycles, we want to be able to determine if the group element corresponding to the permutation is a CFC element.
%    For example, which 4-cycles in $S_4$ correspond to CFC elements in $A_3$?
%    In order to attempt to answer this question, we need a couple definitions.
%
%    Let $(\cdots i~w(i)~w^2(i) \cdots)$ be a disjoint cycle in the permutation corresponding to $w \in W(A_n)$. Then there is a \emph{direction change at $w(i)$} if $i < w(i)$ and $w(i) > w^2(i)$ or $i > w(i)$ and $w(i) < w^2(i)$.
%
%\begin{example} Consider the symmetric group $S_5$. Let $w = (12435)$ and $y = (135)(246)$ in $S_5$. Then $y$ does not have a direction change, but there is a direction change at 4 in $w$ since $i = 2 < w(i) = 4$ and $4 = w(i) > w^2(i) = 3$.
%\end{example}
%
%    We define the \emph{support of a disjoint cycle $c$ of $w$} to be the set of numbers appearing in the cycle, denoted by $\cyclesupp(c)$.
%    Note that $\cyclesupp(c)$ is not the same set as $\supp(w)$, even in the case when $w$ corresponds to a single cycle.
%    We say a cycle $w$ has \emph{connected support} if the support of $w$ is a set of consecutive numbers.
%
%\begin{example} The support of the permutation $(1357)$ is $\{1,3,5,7\}$, so $(1357)$ does not have connected support. However, the permutation $(234)$ does have connected support, namely $\cyclesupp((234)) = \{2,3,4\}$.
%\end{example}
%
%\begin{conjecture} Let $w \in W(A_n)$ correspond to a permutation with disjoint cycles $c_1, c_2, \ldots, c_k$ in $S_{n+1}$. Assume each $c_j$ is written with the smallest number first.
%    Then $w \in \CFC(A_n)$ if and only if %there does not exist an $i \in \cyclesupp(w_j)$ such that $i > w_j(i)$ and $w_j(i) < w_j^2(i)$, where each $w_j$ has support $\{k,\ldots,k+\ell\}$.
%    each $c_j$ has connected support and at most one direction change.
%\end{conjecture}
%
%%\begin{example} The element of the Coxeter group corresponding to the transposition $(13)$ does not have connected support. %agree with the conjecture. In particular, the support of $(13)$ is not a set of consecutive numbers.
%%\end{example}
%
%\begin{example} We return to part (e) of Example~\ref{ex:permscyclic}. In that example, we have a 5-cycle corresponding to each cyclic shift of the heap of $1234$.
%    Note that each of the 5-cycles agrees with both conditions of the conjecture. %Every permutation corresponds to a CFC element of $A_4$.
%    However, the cycle $w = (14352)$ has two direction changes since $1 = i < w(i) = 4$ and $4 = w(i) > w^2(i) = 3$ is a direction change and $4 = i > w(i) = 3$ and $3 = w(i) < w^2(i) = 5$ is another direction change.
%    Then the heap of $w$ is 
%\begin{center} \begin{tikzpicture}
%    \sq{1}{4};   \node at (1.5,3.5) {\footnotesize $3$};
%    \sq{0.5}{3}; \node at (1,2.5)   {\footnotesize $2$};
%    \sq{1.5}{3}; \node at (2,2.5)   {\footnotesize $4$};
%    \sq{0}{2};   \node at (0.5,1.5) {\footnotesize $1$};
%    \sq{1}{2};   \node at (1.5,1.5) {\footnotesize $3$};
%    \sq{1.5}{1}; \node at (2,0.5)   {\footnotesize $4$};
%\end{tikzpicture} \end{center}
%    which is not CFC, in agreement with the conjecture.
%\end{example}
%
%%    For a cycle to have at most one direction change, the permutation line graph cannot contain
%
%%    Cycle type addresses our goal for the conjugacy classes for Coxeter groups of type $A_n$ but does not generalize to Coxeter groups of other types.
%%    We need another way to determine conjugacy in order to generalize to Coxeter groups of types other than $A_n$.
%    
%    Cycle type provides insight into the structure of the sets of conjugate CFC elements. However, our ultimate goal is to generalize to other types of Coxeter groups, where cycle type is not available.
%    
%    
%    
%    
%    