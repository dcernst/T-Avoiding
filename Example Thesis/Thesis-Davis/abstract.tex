\begin{abstract} 
\doublespacing

The Temperley--Lieb algebra, invented by Temperley and Lieb in 1971, is a finite dimensional associative algebra that arose in the context of statistical mechanics. Later in 1971, Penrose showed that this algebra can be realized in terms of certain diagrams. Then in 1987, Jones showed that the Temperley--Lieb algebra occurs naturally as a quotient of the Hecke algebra arising from a Coxeter group of type $A$. This realization of the Temperley--Lieb algebra as a Hecke algebra quotient was later generalized to the case of an arbitrary Coxeter group by Graham.

Cellular algebras were introduced by Graham and Lehrer, and are a class of finite dimensional associative algebras defined in terms of a ``cell datum" and three axioms. The axioms allow one to define a set of modules for the algebra known as cell modules, and one of the main strengths of the theory is that it is relatively straightforward to construct and to classify the irreducible modules for a cellular algebra in terms of quotients of the cell modules. In this thesis, we present an associative diagram algebra that is a faithful representation of a particular quotient of the Temperley--Lieb algebra of type $D$, which has a basis indexed by the so-called type II fully commutative elements of the Coxeter group of type $D$. By explicitly constructing a cell datum for the corresponding diagram algebra, we show that the quotient is cellular.


\end{abstract}

