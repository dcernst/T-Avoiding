\documentclass{article}
\pagenumbering{gobble}
\usepackage{amsmath}
\usepackage{amssymb}
\usepackage{amsfonts}
\usepackage{amsthm}
\usepackage[left=1in,right=1in,top=1in,bottom=1in]{geometry}
\newcommand{\gen}[1]{\langle #1 \rangle}
\newcommand{\CFC}{\text{CFC}(W)}
\providecommand{\abs}[1]{\left\lvert#1\right\rvert}

\theoremstyle{definition}
    \newtheorem{Lemma}{Lemma}
    \newtheorem*{note}{Note}
    \newtheorem*{ex}{Example(s)}
\theoremstyle{theorem}
    \newtheorem{theorem}{Theorem}
    \newtheorem*{prop}{Proposition}

\begin{document}
\noindent Brooke Fox \\
ACGT Outline \\
21 January 2014 \\

\section{Motivation}
    Big question: Given $w \in \CFC$, to which CFC elements is it conjugate? It depends on the chunk/block type. For example, a heap with block type (1,2,2,3,7) is conjugate to all others of the same block type.
    Claim: CFC elements conjugate if and only if they have the same block type.

\section{Basic definitions} 
\begin{itemize}
	\item[$\circ$] A \emph{Coxeter system} is a group $W$ together with a generating set $S$ such that $W = \gen{S : \underbrace{(st)(st) \cdots (st)}_{m(s,t) ~\text{times}} = (st)^{m(s,t)} = e}$ where $s,t \in S$ and $m(s,s) = 1$, $m(s,t) = m(t,s)$, and $2 \leq m(s,t) < \infty$ for $s \neq t$. $m(s,t)$ is called the bond strength, and $\abs{st} = m(s,t)$.
	
	\item[$\circ$] A \emph{Coxeter graph} consists of vertex set $S$ and edge set $\{\{s,t\} : m(s,t) \geq 3\}$. We label edge $\{s,t\}$ with $m(s,t)$. There is a one-to-one correspondence between Coxeter systems and Coxeter graphs.
	
	\item[$\circ$] Types: $A_n$ (straight line, $n$ vertices, all bond strength 3), $W(A_n) \cong S_{n+1}$ via $s_i \mapsto (i \; i+1)$; $I_2(m)$ (straight line, 2 vertices, bond strength $m$); $B_n$ (straight line, $n$ vertices, $s_1s_2s_1s_2 = s_2s_1s_2s_1$ and the rest are bond strength 3); others.
	
		\item[$\circ$] Braids: $sts = tst$ for bond strength 3, $stst = tsts$ for bond strength 4, etc.
	
	\item[$\circ$] Word versus element. A \emph{Coxeter element} is an element of the group in which every generator appears exactly once in each reduced expression.
	
	\item[$\circ$] Fully commutative: An element $w \in W$ is \emph{fully commutative} (FC) if all of its reduced expressions are commutation equivalent.
	\begin{prop} FC if and only if no reduced expression for the group element has braids. \end{prop}
	\begin{ex} Coxeter elements are FC. In $A_3$, the element $s_2s_1s_3s_2$ is a reduced expression for an FC element. \end{ex}
	
	\item[$\circ$] An element $w \in W$ is \emph{cyclically fully commutative} (CFC) if every cyclic shift of every reduced expression for $w$ is a reduced expression for an FC element. That is, the reduced expressions, when written in a circle, avoid $\gen{s,t}_{m(s,t)}$ subwords for $m(s,t) \geq 3$.
	\begin{ex} Coxeter elements are CFC since they are FC and any cyclic shift of a Coxeter element is a Coxeter element. In $A_3$, $s_2s_1s_3s_2$ is not CFC because it has a cyclic shift $s_1s_3s_2s_2$ which reduces to $s_1s_3$.
	 \end{ex}
	
	\item[$\circ$] Conjugates: We define $\sim_\kappa$ to be the equivalence relations generated by cyclic shifts of words. That is, $s_{x_1}s_{x_2} \cdots s_{x_m} \mapsto s_{x_2}s_{x_3} \cdots s_{x_m}s_{x_1}$.
	\begin{theorem}[Eriksson--Erikkson] Two Coxeter elements are \emph{conjugate} if and only if they are kappa-equivalent (cyclic shifts of each other). \end{theorem}
	
%	\item[$\circ$] rotation, rotation equivalent: For example, if $w = 123$ and we conjugate by 1, we get $1w1 = 11231 = 231$, which has the effect of moving the first letter to the end of the word, or \emph{rotation}. Two words are \emph{rotation equivalent} if one can be obtained from the other by a series of rotations and commutations.

	\item[$\circ$] Heaps. Show 123, 321, 213, 132. Show conjugacy classes for $A_3$ with block types: $\{1,2,3\}$, (1); $\{12,21,23,32\}$, (2); $\{13\}$, (1,1); $\{123,132,213,321\}$, (3) -- they all look the same when you wrap. Block types for $A_4$: (1), (1,1), (1,2), (3), (4).
\pagebreak
\end{itemize}

\section{Theorem}
\begin{Lemma} Every CFC chunk is cyclically equivalent to a diagonal heap. \end{Lemma}
\begin{Lemma}[Squeeze operation] Starting from the left, determine the first two elements which commute. (They should be $k+m+1$ and $k$.) Move the higher element, by commutation, as far to the right as possible. Now find the rightmost occurrence of that element, and move it, by commutation, as far to the left as possible. \end{Lemma}
\begin{Lemma} If $m(s,t)=3$, then $stst = ts$. Proof; $stst \Rightarrow tstt \Rightarrow ts$. \end{Lemma}
\begin{note} Squeeze is used to create long braids. \end{note}
\begin{Lemma} To translate a CFC chunk by space -- that is, to move $k \, \cdots \, k+m$ to $k+1 \, \cdots \, k+m+1$ -- conjugate by $k \, \cdots \, k+m+1$. \end{Lemma}
\begin{Lemma} To permute two CFC chunks -- say, $1 \, \cdots \, k \,\, k+2 \, \cdots k+m$ to $1 \, \cdots \, k-m \,\, k-m+2 \, \cdots \, k+m$ -- conjugate by $k-1 \, \cdots \, k+m \,\, k-2 \, \cdots \, k+m-1 \,\, 1 \, \cdots \, m+2$. \end{Lemma}

\section{Open problems and to do} \begin{enumerate}
\item If $w \in \CFC$, then is $xwx^{-1} \in \CFC$ (if not, ignore) not on our list? Data says no, but we need to finish the proof.
\item Theorem in other types? $B_n$?
\item How many block types?
\item Size of conjugacy class for each block type?
\end{enumerate}

\end{document}