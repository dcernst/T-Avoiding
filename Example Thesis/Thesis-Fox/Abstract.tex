\begin{abstract}
\doublespacing

A fundamental result of Coxeter groups, known as Matsumoto's theorem, states that any two reduced expressions of the same element differ by a sequence of commutations and braid moves. If two elements have expressions that are cyclic shifts of each other (as words), then they are conjugate (as group elements). We say that an expression is cyclically reduced if every cyclic shift of it is reduced, and ask the following question, where an affirmative answer would be a ``cyclic version'' of Matsumoto's theorem. \emph{Do two cyclically reduced expressions of conjugate elements differ by a sequence of braid relations and cyclic shifts}? While the answer is, in general, ``no,'' understanding when the answer is ``yes'' is a central focus of a broad ongoing research project. It was recently shown to hold for all Coxeter elements.  

A Coxeter element is a special case of a fully commutative element, which is any element with the property that any two reduced expressions are equivalent by only commutations. In this thesis, we study the cyclically fully commutative elements. These are the elements for which every cyclic shift of any reduced expression is a reduced expression of a fully commutative element. In this light, the cyclically fully commutative elements are the ``cyclic version'' of the fully commutative elements. In particular, the cyclic version of Matsumoto's theorem for the cyclically fully commutative elements asks when two reduced expressions for conjugate elements are equivalent via only commutations and cyclic shifts.  

In this thesis, we study the combinatorics of cyclically fully commutative elements in Coxeter groups of type $A$ as it relates to conjugacy. In particular, we introduce the notion of cylindrical heaps and ring equivalence in order to state our main result, which says that two cyclically fully commutative elements of a Coxeter group of type $A$ are conjugate if and only if their corresponding cylindrical heaps are ring equivalent.

\end{abstract}