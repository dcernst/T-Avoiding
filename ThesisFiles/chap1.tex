\chapter{Preliminaries}

\section{Introduction}
To be written once we know where everything is going to go.

%%%%%%%%%%%%%%%%%%%%%%%%%

\section{Coxeter System}\label{sec:coxeter}
A \emph{Coxeter System} is a pair $(W,S)$ consisting of a finite set $S$ of generating involutions and a group $W$, called a \emph{Coxeter Group}, with presentation 
\[ 
W = \langle S \mid (st)^{m(s, t)} = e \text{ for } m(s, t) < \infty \rangle,
\]
where $e$ is the identity, $m(s,t) = 1$ if and only if $s = t$, and $m(s,t) = m(t,s)$. It turns out that the elements of $S$ are distinct as group elements and that $m(s,t)$ is the \emph\emph{order} of $st$~\cite{Humphreys1990}. We call $m(s,t)$ the \emph{bond strength} of $s$ and $t$.\\

Since $s$ and $t$ are elements of order 2, the relation $(st)^{m(s,t)}=e$ can be written as
\begin{equation}\label{braid} 
	\underbrace{sts \cdots}_{m(s,t)}=\underbrace{tst\cdots}_{m(s,t)}
\end{equation}
with $m(s,t) \geq 2$ factors. If $m(s,t)=2$, then $st=ts$ is called a \emph{short braid relation} and $s$ and $t$ commute. Otherwise, if $m(s,t) \geq 3$, then the relation in \eqref{braid} is called a \emph{long braid relation}. Replacing $\underbrace{sts\cdots}_{m(s,t)}$ with $\underbrace{tst\cdots}_{m(s,t)}$ will be referred to as a \emph{braid move}.\\

We can represent a Coxeter System, $(W,S)$, with a unique \emph{Coxeter graph}, $\Gamma$, having
\begin{enumerate}
\item vertex set $S=\{s_1, s_2, \ldots, s_n\}$ and
\item edges $\{s_i, s_j\}$ for each $m(s,t) \geq 3$.	
\end{enumerate}
Each edge $\{s_i, s_j\}$ is labeled with its corresponding bond strength $m(s,t)$. Since $m(s,t)=3$ occurs most frequently, it is customary to leave the edge unlabeled. If $(W,S)$ is a Coxeter group with corresponding Coxeter system $\Gamma$, we may denote the group as $W(\Gamma)$ for emphasis. There is a one-to-one correspondence between Coxeter systems and Coxeter graphs. Given a Coxeter graph $\Gamma$, we can uniquely reconstruct the corresponding Coxeter system. Note that $s$ and $t$ are not connected in the graph if and only if $m(s,t)=2$. Also, the Coxeter group $W(\Gamma)$ is said to be \emph{irreducible} if and only if $\Gamma$ is connected. Otherwise, $W(\Gamma)$ is said to be \emph{reducible}. Furthermore, if the graph is disconnected, the connected components correspond to factors in a direct product of irreducible Coxeter groups~\cite{Humphreys1990}.\\

\begin{example}
~
\begin{itemize}
\item[(a)~] The Coxeter graph of type $A_n$ is seen in Figure \textcolor{red}{reference the proper figure}. Given $A_n$, we can construct the corresponding Coxeter system, $W(A_n)$, with generating set $S=\{s_1, s_2, \ldots s_n\}$ and defining relations
\begin{enumerate}
	\item $s_i^2=e$ for all $i$
	\item $s_is_j=s_js_i$ when $|i-j|>1$
	\item $s_is_js_i=s_js_is_j$ when $|i-j|=1.$
\end{enumerate}
The Coxeter group $W(A_n)$ is isomorphic to the symmetric group $S_{n+1}$ elements sending $s_i \rightarrow (i, i+1)$. This thesis will briefly touch on Coxeter systems of type $A_n$.
\item[(b)~] The Coxeter graph of type $B_n$ is seen in Figure \textcolor{red}{reference the proper figure}. From $B_n$, we can construct the Coxeter group $W(B_n)$ with generating set $S=~\{s_0,s_1, \ldots s_{n-1}\}$ and defining relations
\begin{enumerate}
	\item $s_i^2=e$ for all $i$
	\item $s_is_j=s_js_i$  when $|i-j|>1$
	\item $s_is_js_i=s_js_is_j$ when $|i-j|=1$ for $i,j \in \{1,2,\ldots, n-1\}$
	\item $s_0s_1s_0s_1=s_1s_0s_1s_0$.
\end{enumerate}
The Coxeter group $W(B_n)$ is isomorphic to $S_n^B$, where $S_n^B$ is the group of all signed permutations. This thesis will also touch on Coxeter systems of type $B_n$.
\item[(c)~] The Coxeter graph of Type $\widetilde C_n$ is seen in Figure \textcolor{red}{reference the proper figure}. From $\widetilde C_n$ we can construct the Coxeter group $W(\widetilde C_n)$ with generating set $S=~\{s_0, s_1, \ldots s_n\}$ and defining relations 
\begin{enumerate}
	\item $s_i^2=e$ for all $i$
	\item $s_is_j=s_js_i$ when $|i-j|>1$ for $i \in \{1,2, \ldots, n-1\}$
	\item $s_is_js_i=s_js_is_j$ 	when $|i-j|=1$ for $i \in \{1,2, \ldots, n-1\}$
	\item $s_0s_1s_0s_1=s_1s_0s_1s_0$
	\item $s_ns_{n-1}s_ns_{n-1}=s_{n-1}s_ns_{n-1}s_n.$
\end{enumerate}
The Coxeter group $W(\widetilde C_n)$ will be touched upon in this thesis as well.
\end{itemize}
\end{example}

Given a Coxeter system, $(W,S)$, a word $s_{x_1}s_{x_2} \cdots s_{x_m}$ in the free monoid $S^*$ on $S$, is called an \emph{expression} for $w \in W$ if it is equal to $w$ when considered as a group element. If $m$ is minimal over all expressions for $w$, the corresponding word is called a \emph{reduced expression} for $w$. In this case, we define the \emph{length} of $w$ to be $l(w):= m$. Each element $w \in W$ may have multiple reduced expressions to represent it. If we wish to emphasize a specific possibly reduced expression for $w \in W$ we will represent it as $\overline{w}=s_{x_1}s_{x_2}\cdots s_{x_m}.$ The following theorem tells us more about how reduced expressions relate.

\begin{theorem} [Matsumoto, \cite{Geck2000}]
	If $w \in W$, then every reduced expression for $w$ can be obtained by a sequence of braid moves of the form
	\[\underbrace{sts\cdots}_{m(s,t)} \rightarrow \underbrace{tst\cdots}_{m(s,t)}\]
	where $s,t \in S$ and $m(s,t) \geq 2$. \qed
\end{theorem}
 
It follows from Matsumoto's Theorem that any reduced expression for $w \in W$ contains the same number of generators in the expression. Let $w \in W$ and fix a reduced expression $\overline{w}$ for $w$. Then the \emph{support} of $w$, denoted $\supp(\overline{w})$, is the set of all generators of that appear in $\overline{w}$. It follows from Matsumoto's Theorem that $s$ appears in $\supp(\overline{w})$ if and only if $s$ appears in the support of all reduced expressions for $w$. If $\supp(\overline{w})=S$, we say that $s$ has \emph{full support}. Given $w \in W$ and a fixed reduced expression $\overline{w}$ for $w$, any subsequence of $\overline{w}$ is called a \emph{subexpression} of $\overline{w}$. \\

\begin{example}
Let $w \in W(A_7)$ and let $\overline{w}=s_7s_2s_4s_5s_3s_2s_3s_6$ be a fixed expression for $w$. Then we have
\begin{align*}
s_7s_2s_4s_5s_3s_2s_3s_6&=s_7s_4s_2s_5s_3s_2s_3s_6\\
&=s_7s_4s_5s_2s_3s_2 s_3s_6\\
&=s_7s_4s_5s_3s_2s_3s_3s_6\\
&=s_7s_4s_5s_3s_2s_6.
\end{align*}
This shows that $\overline{w}$ was not reduced. However, $s_7s_4s_5s_3s_2s_6$ is reduced. Thus $l(w)=6$ and $\supp(w)=\{s_2, s_3, s_4, s_5, s_6, s_7\}$.
\end{example}

Let $w \in W(\Gamma)$. We define the \emph{left descent set} and \emph{right descent set} of $w$ as follows:
\[\mathcal{L}(w):=\{s \in S \mid l(sw) < l(w)\}\]
and
\[\mathcal{R}(w):=\{s \in S \mid l(ws) < l(w)\}.\]
It should be noted that $s \in \mathcal{L}(w)$ if and only if there is a reduced expression for $w$ that begins with $s$ and $s \in \mathcal{R}(w)$ if and only if there is a reduced expression for $w$ that ends with $s$.

\begin{example}
Let $w \in W(B_4)$ and let $\overline{w}=s_0s_1s_2s_1s_3$ be a reduced expression for $w$. Note that all reduced expressions for $w$ are as follows 
$$\begin{array}{ll}
s_0s_1s_2s_1s_3 & s_0s_2s_1s_2s_3\\
s_0s_1s_2s_3s_1 & s_2s_0s_1s_2s_3.	
\end{array}$$
We see that $l(w)=5$, and $w$ has full support. Note that $\mathcal{L}(w)=\{s_0, s_2\}$ and $\mathcal{R}(w)=\{s_1, s_3\}$.	
\end{example}


%%%%%%%%%%%%%%%%%%%%%%%%%%%


\section{Fully Commutative Elements}\label{sec:FC}
Let $(W,S)$ be a Coxeter system of type $\Gamma$ and let $w \in W$. Following~\cite{Stembridge1996}, we define a relation $\sim$ on the set of reduced expressions for $w$. Let $\overline{w}_1$ and $\overline{w}_2$ be two reduced expressions for $w$. We define $\overline{w}_1 \sim \overline{w}_2$ if we can obtain $\overline{w}_2$ from $\overline{w}_1$ by applying a single braid move of the form $s_is_j \mapsto s_js_i$ where $m(s_i,s_j)=2$. Now, define the equivalence relation $\approx$ by taking the reflexive transitive closure of $\sim$. Each equivalence class under $\approx$ is called a \emph{commutation class}. If $w$ has a single commutation class, then we say that $w$ is \emph{fully commutative}, or just FC. 

The set of fully commutative elements of $W(\Gamma)$ is denoted by $\FC(\Gamma)$. We say that a reduced expression $\overline{w}$ is $\FC$ if it is a reduced expression for $w \in \FC(\Gamma)$. Given some $w \in \FC(\Gamma)$, the definition of fully commutative tells us that to obtain all the reduced expressions for $w$, one must only perform short braid moves. The following theorem tells us that the we can not obtain a reduced expression for $w$ using long braid relations.

\begin{theorem}[Stembridge,~\cite{Stembridge1996}]
	An element $w \in W$ is FC if and only if no reduced expression for $w$ contains $\underbrace{sts\cdots}_{m(s,t)}$ as a subword for all $s_i \neq s_j$ when $m(s,t) \geq 3$. \qed
\end{theorem}

\begin{example}
	Let $w \in W(\widetilde{C}_4)$ and let $\overline{w}=s_0s_1s_2s_0s_3s_1$ be a reduced expression for $w$. We see that
	\[s_0s_1\textcolor{purple}{s_2s_0}s_3s_1=s_0s_1s_0s_2\textcolor{purple}{s_3s_1}=s_0s_1s_0s_2s_1s_3,\]
	where the purple indicates applying a short braid relation. Note that there is no way possible to perform a long braid relation. Hence $w$ is $\FC$.
\end{example}

\begin{example}
Let $w \in W(\widetilde{C}_4)$ and let $\overline{w}=s_0s_1s_2s_0s_1s_2$ be a reduced expression for $w$. We see that
\[s_0s_1\textcolor{purple}{s_3s_0}s_1s_2=s_0s_1s_0\textcolor{purple}{s_3s_1}s_2=\textcolor{blue}{s_0s_1s_0s_1}s_3s_2,\]
where the purple indicates applying a short braid relation and the blue indicates applying a long braid relation. Thus $w$ is not $\FC$ since a long braid relation can be applied.  	
\end{example}

Stembridge classified the irreducible Coxeter groups that contain a finite number of fully commutative elements, the so-called \emph{FC-finite Coxeter groups}. This thesis is mainly concerned with $W(A_n),~W(B_n),~W(\widetilde{C}_n)$. Both $W(A_n),~W(B_n)$ are finite Coxeter groups, and thus are $\FC$ finite. On the other hand $W(\widetilde{C}_n)$ is infinite and has infinitely many $\FC$ elements. However, there exist some infinite Coxeter groups that contain finitely many $\FC$ elements. For example, $E_n$ for $n \geq 9$ are infinite \textcolor{red}{cite coxeter graph figure}, but contain only finitely many fully commutative elements.

\begin{theorem}[Stembridge,~\cite{Stembridge1996}]
\label{thm:FCfinite} The FC-finite irreducible Coxeter groups are of type $A_n$ with $n \geq 1$, $B_n$ with $n \geq 2$, $D_n$ with $n \geq 4$, $E_n$ with $n \geq 6$, $F_n$ with $n \geq 4$, $H_n$ with $n \geq 3$, and $I_2(m)$ with $5 \leq m < \infty$. The corresponding Coxeter graphs are shown in \textcolor{red}{reference coxeter graph figure}. \qed
\end{theorem}
  


%%%%%%%%%%%%%%%%%%%%%%%%%%%


\section{Heaps}\label{sec:Heaps}

We can now discuss another representation of Coxeter group elements. Each reduced expression can be associated with a labeled partially ordered set (poset) called a heap.  Heaps provide a visual representation of a reduced expression while preserving the relations among the generators. We follow the development of heaps of straight line Coxeter groups in~\cite{Billey2007},~\cite{Ernst2010} and~\cite{Stembridge1996}. 

Let $(W,S)$ be a Coxeter system. Suppose $\overline{w}=s_{x_1}s_{x_2}\cdots s_{x_r}$ is a fixed reduced expression for $w \in W$. As in~\cite{Stembridge1996}, we define a partial ordering on the indices $\{1, 2, \ldots, r\}$ by the transitive closure of the relation $\lessdot$
