\chapter{T-Avoiding Elements in Coxeter groups of Types $\C_n$ and $B_n$}\label{chap:Cn}



%%%%%%%%%%%%%%%%%%%%%%%%%%
\section{Classification of T-Avoiding Elements in Coxeter Groups of Type $\C_n$}\label{sec:TAC}

In this section we will classify the T-avoiding elements in Coxeter systems of type $\C_n$, a new result. We will first show that there are no $\tII$ elements that are not FC in $W(\C_n)$.

Before we begin the proof we must first define the notion of a pushed-down representation of a heap. First recall that there are potentially many ways to draw the lattice point representation of a heap, each differing by the amount of vertical space between blocks. We wish to fix one such representation. Let $\w$ be a reduced expression for $w \in W(\C_n)$. We construct the \emph{pushed-down representation} of $H(\w)$ by first giving all blocks fully exposed to the bottom the same vertical position, and then all other blocks are as low as possible in the heap. That is, the heap has been constructed by placing all blocks in the lowest possible vertical position of the heap. Notice that we can now label the rows in the heap from bottom to top where the bottom-most row is row 1 and proceed naturally upward from there. 

We now define the height of a braid. Given the presence of a braid in the heap of an reduced expression $\w$, we define the \emph{height of the braid} to be the row number in which the uppermost block involved in the braid is located in the pushed-down representation. It is important to note that in the pushed-down representation, a braid may not appear in consecutive rows. That is, some of the blocks involved in a braid may be lower in the heap and the braid may not be immediately apparent. 

\begin{example}
Let $\w=s_0s_1s_3s_2s_1s_0s_1s_3$ be a reduced expression for $w \in W(C_3)$. The pushed-down representation of the heap for $\w$ is given in Figure~\ref{fig:pusheddownheap}. The height of the braid $s_1s_2s_1$, which is highlighted in \textcolor{teal}{teal} in Figure~\ref{fig:pusheddownheap}, is 5 since the upper block for $s_1$ is located in the fifth row of the pushed-down representation heap. Notice that the block for $s_3$ can slide up higher in the heap. If we were to slide the block for $s_3$ up until it hits the block for $s_2$ we would obtain the braid $s_3s_2s_3$ in the heap. In this case, the height for the braid $s_3s_2s_3$ is also 5.   

\begin{figure}[h!] \centering
\begin{tikzpicture}[scale=0.40]
	\heapblock{0}{10}{0}{purple}
	\heapblock{3}{8}{3}{purple}
	\heapblock{1}{8}{1}{teal}
	\heapblock{2}{6}{2}{teal}
	\heapblock{1}{4}{1}{teal}
	\heapblock{0}{2}{0}{purple}
	\heapblock{1}{0}	{1}{purple}
	\heapblock{3}{0}{3}{purple}
%	\heapblock{0}{0}{0}{purple}
%	\heapblock{2}{0}{2}{purple}
%	\heapblock{4}{0}{4}{purple}
\end{tikzpicture}	
\caption{Pushed-down representation of a heap.}\label{fig:pusheddownheap}
\end{figure}
\end{example}

We now show that there are no $\tII$ elements in $W(\C_n) \setminus \FC(\C_n)$. We first need the following lemma.

\begin{lemma}\label{lem:zigzag}
Let $w=uv$ (reduced) in a Coxeter system of type $\C_n$ such that $v$ is FC and has a reduced expression beginning with $s_0s_1\cdots s_{n-1}s_ns_{n-1}$. Then $w$ has Property T on the right.
\begin{proof}
	Let $w=uv$ (reduced) in a Coxeter system of type $\C_n$ such that $v$ is FC and has a reduced expression beginning with $s_0s_1 \cdots s_{n-1}s_ns_{n-1}$. The top of the heap for $v$ is shown in Figure~\ref{fig:lemzigzag1} which must be a convex subheap of the heap for some reduced expression for $w$. If $H(v)$ is equal to the heap in Figure~\ref{fig:lemzigzag1}, then $v$, and hence $w$, have Property T on the right. If the heap given in Figure~\ref{fig:lemzigzag1} is not the complete heap for $v$, then since $v$ is FC, it must be the case that the heap in Figure~\ref{fig:lemzigzag2} is a convex subheap of $H(v)$. Iterating this process we see that $H(v)$ must have a zigzagging shape that changes direction every time it encounters $s_0$ or $s_n$. That is, 
	\[
		v=(s_0s_1 \cdots s_{n-1}s_ns_{n-1} \cdots s_1)^m 
		\begin{cases}
 			s_0s_1 \cdots s_{j-1}s_j, & \text{for } j \leq n\\
 			s_0s_1 \cdots s_{n-1}s_ns_{n-1} \cdots s_j, & \text{for } j> 1.
		\end{cases}
	\]  
	From this we see that $w$ ends with $s_{j-1}s_j$ or $s_{j+1}s_j$ which implies $w$ has Property T on the right.
\end{proof}
\end{lemma}

\begin{figure}[h!]\centering
\begin{tabular}{m{7cm} m{7cm}}
\begin{subfigure}{0.5\textwidth} \centering
\begin{tikzpicture}[scale=0.40]
	\sheapblock{0}{-18}{}{white}
	\dheapblock{0}{-4}{}{black}
	\dheapblock{1}{-6}{}{black}
	\dheapblock{2}{-8}{}{black}

	\heapblock{0}{0}{0}{purple}
	\heapblock{1}{-2}{1}{purple}
	\heapblock{2}{-4}{2}{purple}
	\heapblock{3}{-6}{3}{purple}
	\heapblock{4}{-8}{4}{purple}
	
	\node[] at (6,-10){$\ddots$}; 
	\node[] at (3,-10){$\ddots$};
	
	\dheapblock{5}{-12}{}{black}
	\dheapblock{6}{-14}{}{black}
	\dheapblock{5}{-16}{}{black}
	
	\heapblock{7}{-12}{n-1}{purple}
	\heapblock{8}{-14}{n}{purple}
	\heapblock{7}{-16}{n-1}{purple}
\end{tikzpicture}
\caption{}\label{fig:lemzigzag1}
\end{subfigure} &

\begin{subfigure}{0.5\textwidth} \centering
\begin{tikzpicture}[scale=0.40]
	\dheapblock{0}{-4}{}{black}
	\dheapblock{1}{-6}{}{black}
	\dheapblock{2}{-8}{}{black}

	\heapblock{0}{0}{0}{purple}
	\heapblock{1}{-2}{1}{purple}
	\heapblock{2}{-4}{2}{purple}
	\heapblock{3}{-6}{3}{purple}
	\heapblock{4}{-8}{4}{purple}
	
	\node[] at (6,-10){$\ddots$}; 
	\node[] at (3,-10){$\ddots$};
	
	\dheapblock{5}{-12}{}{black}
	\dheapblock{6}{-14}{}{black}
	\dheapblock{5}{-16}{}{black}
	\dheapblock{4}{-18}{}{black}
	
	\heapblock{7}{-12}{n-1}{purple}
	\heapblock{8}{-14}{n}{purple}
	\heapblock{7}{-16}{n-1}{purple}
	\heapblock{6}{-18}{n-2}{purple}
\end{tikzpicture}
\caption{}\label{fig:lemzigzag2}	
\end{subfigure}
\end{tabular}
\caption{The subheap of $w$ discussed in Lemma~\ref{lem:zigzag}.}\label{fig:lemzigzag}
\end{figure}


\begin{theorem}\label{thm:TavoidC}
There are no $\tII$ elements in $W(\C_n)\setminus \FC(\C_n)$. 	
\begin{proof}
	We proceed by contradiction. Let $w \in W(\C_n)\setminus\FC(\C_n)$ and $w$ is $\tII$. Consider all possible pushed-down representations for heaps of $w$. Choose a representation  that has a minimal height braid among all braids appearing in all heaps for $w$ and let $k$ represent that minimum height. There may be a tie, in which case choose your favorite. Without loss of generality we assume the generators involved in the braid are $s_{i-1}$ and $s_{i},$ where the bond strength is case specific. In the following cases we give the height of a braid and unless we indicate otherwise whenever we refer to a block being in a specific row, we are considering the pushed-down representation of the heap. In order to consider the braids that we are looking for we need to allow some flexibility when referring to the vertical position of a given block. In the following cases, all subheaps are assumed to be convex.
	
	Case (1): In this case, we are assuming that the braid is located in consecutive rows with the upper-most block in either row 3 or row 4 depending on the bond strength and the lowest block involved in the braid is located in the bottom-most row of the heap.
	
	Subcase (1.1): Suppose $k=3$. This implies that $m(s_{i-1}, s_i)=3$. Without loss of generality assume $s_{i-1}$ is in the bottom-most row of the heap. Clearly, the block for $s_{i+1}$ must be in the bottom-most row of the heap as well, otherwise $w$ has Property T, which is a contradiction to the original choice of $w$. Restricting our focus to the subheap of $w$ that contains the braid we are considering,  we see that this subheap of $w$ has the following form
	\begin{center}
	\begin{tikzpicture}[scale=0.40]
		%\sheapblock{2}{6}{i}{purple}
		\dheapblock{0}{2}{}{black}
		\heapblock{1}{0}{i-1}{purple}
		\heapblock{3}{0}{i+1}{purple}
		\heapblock{2}{2}{i}{purple}
		\heapblock{1}{4}{i-1}{purple}
	\end{tikzpicture}	
	\end{center}
 	where the blocks for $s_{i-1}$ and $s_{i+1}$ are in the bottom-most row of the heap and are thus fully exposed. Applying the braid move we get the subheap seen here
 	\begin{center}
 	\begin{tikzpicture}[scale=0.40]
 		\dheapblock{0}{0}{}{black}
 		%\dheapblock{2}{4}{}{black}
 		\heapblock{2}{0}{i+1}{purple}
 		\heapblock{1}{2}{i}{purple}
 		\heapblock{0}{4}{i-1}{purple}
 		\heapblock{1}{6}{i}{purple}
 		%\sheapblock{0}{8}{i-1}{purple}
 	\end{tikzpicture}
 	\end{center}
	which clearly has Property T since $s_{i-1}$ is now in the first row of the pushed-down representation. This is a contradiction to the way in which we chose $w$. 
	
	Subcase (1.2): Suppose $k=4$. This implies that $m(s_{i-1}, s_i)=4$. Without loss of generality assume $s_{i-1}=s_0$ and $s_{i}=s_1$. The other case being $s_{i-1}=s_{n-1}$ and $s_i=s_n$ which we could handle with a symmetric argument.
	
	Subcase (1.2.1): First we take $s_0$ to be the topmost block in the braid. Then we obtain the subheap seen here
	\begin{center}
	\begin{tikzpicture}[scale=0.40]
		\dheapblock{2}{2}{}{black}
		\heapblock{1}{0}{1}{purple}
		\heapblock{0}{2}{0}{purple}
		\heapblock{1}{4}{1}{purple}
		\heapblock{0}{6}{0}{purple}
	\end{tikzpicture}	
	\end{center}
	which clearly indicates that $w$ has Property T on the right. This is a contradiction to the way in which we chose $w$.
	
	Subcase (1.2.2): Now we take $s_1$ to be the topmost block involved in the braid. Clearly, the block for $s_2$ must be in the bottom-most row of the heap as well, otherwise $w$ has Property T, which is a contradiction to the orginal choice of $w$. Restricting our focus to the subheap of $w$ that contains the braid we are considering, we see that the subheap of $w$ has the following form
	\begin{center}
	\begin{tikzpicture}[scale=0.40]
		\heapblock{2}{6}{i}{purple}
		\dheapblock{0}{2}{}{black}
		\heapblock{1}{0}{i-1}{purple}
		\heapblock{3}{0}{i+1}{purple}
		\heapblock{2}{2}{i}{purple}
		\heapblock{1}{4}{i-1}{purple}
	\end{tikzpicture}	
	\end{center}
	where the blocks for $s_{i-1}$ and $s_{i+1}$ are in the bottom-most row of the heap and are thus fully exposed. Applying the braid move we get the subheap seen here
 	\begin{center}
 	\begin{tikzpicture}[scale=0.40]
 		\dheapblock{0}{0}{}{black}
 		\dheapblock{2}{4}{}{black}
 		\heapblock{2}{0}{2}{purple}
 		\heapblock{1}{2}{1}{purple}
 		\heapblock{0}{4}{0}{purple}
 		\heapblock{1}{6}{1}{purple}
 		\heapblock{0}{8}{0}{purple}
 	\end{tikzpicture}
 	\end{center}
	which shows that $w$ has Property T since $s_{i-1}$ is now in the first row of a pushed-down representation for $w$. This is a contradiction to the way in which we chose $w$. 
 
	For the rest of the cases we will assume that $k \geq 4$.
	
	Case (2): Suppose the braid has height $k$ and assume the braid does not involve $s_0, s_1, s_{n-1}$ or $s_n$. This implies that $m(s_{i-1},s_i)=3$, $m(s_{i-2},s_{i-1})=3$ and $m(s_i,s_{i+1})=3$. Without loss of generality assume $s_i$ is in the $k$th row of the heap and if necessary we have brought the blocks for $s_{i-1}$ and $s_i$ up next to $s_i$ in row $k$. We now consider which blocks may occur in the $(k-3)$th row of the heap in two cases. 
	
	Subcase (2.1): Assume that the block for $s_{i-1}$ is in the heap in the $(k-3)$th row and we allow for the block for $s_{i+1}$ to be in the same row as well, but it does not necessarily have to be. In the following figures of subheaps, the block for $s_{i+1}$ will be represented in a \textcolor{purple}{purple}-striped block to indicate that it could be present but it does not have to be. The following is the subheap that we are considering
	\begin{center}
	\begin{tikzpicture}[scale=0.40]
		\dheapblock{4}{4}{}{black}
		\heapblock{2}{0}{i-1}{purple}
		\sheapblock{4}{0}{i+1}{purple}
		\heapblock{1}{2}{i-2}{purple}
		\heapblock{3}{2}{i}{teal}
		\heapblock{2}{4}{i-1}{teal}
		\heapblock{3}{6}{i}{teal}
	\end{tikzpicture}
	\end{center}
	where we have highlighted the braid in \textcolor{teal}{teal}. Notice that the block for $s_{i-2}$ is present in the $(k-2)$th row otherwise there is a braid with height less than $k$ and $w$ would not be reduced. Applying the braid move to the heap we get the following subheap
	\begin{center}
	\begin{tikzpicture}[scale=0.40]
		\dheapblock{3}{2}{}{black}
		\dheapblock{1}{6}{}{black}
		\heapblock{2}{0}{i-1}{rred}
		\sheapblock{4}{0}{i+1}{purple}
		\heapblock{1}{2}{i-2}{rred}
		\heapblock{2}{4}{i-1}{rred}
		\heapblock{3}{6}{i}{teal}
		\heapblock{2}{8}{i-1}{teal}
	\end{tikzpicture}
	\end{center}
	which has a new braid in it. This braid, which we have highlighted in \textcolor{rred}{red} for emphasis, has height $k-1$. In applying the braid move we have obtained a heap which has a braid with height less than $k$ our original choice. This is a contradiction to the way in which we chose our heap.
	
	Subcase (2.2): Assume that the block for $s_{i+1}$ is in the $(k-3)$th row of the heap and the block for $s_{i-1}$ does not appear in the $(k-3)$th row. The following is the subheap we are considering
	\begin{center}
	\begin{tikzpicture}[scale=0.40]
		\heapblock{3}{0}{i+1}{purple}
		\dheapblock{1}{0}{}{black}
		%\dheapblock{0}{2}{}{black}
		\heapblock{2}{2}{i}{teal}
		\heapblock{1}{4}{i-1}{teal}
		\heapblock{2}{6}{i}{teal}
	\end{tikzpicture}
	\end{center}
 	where the dotted square represents that $s_{i-1}$ may not occupy the indicated position and the braid is highlighted in \textcolor{teal}{teal}. Applying the braid move in the subheap we obtain the following heap
 	\begin{center}
 	\begin{tikzpicture}[scale=0.40]
 		\dheapblock{-1}{2}{}{black}
 		\heapblock{2}{0}{i+1}{purple}
 		\heapblock{0}{0}{i-1}{teal}
 		\heapblock{1}{2}{i}{teal}
 		\heapblock{0}{4}{i-1}{teal}
 	\end{tikzpicture}
 	\end{center}
	where the height of the new braid is $k-1$. This contradicts the way in which we chose the heap of $w$. From this we gather that the braid must contain $s_0,s_1, s_{n-1},$ or $s_n$.
	
	Case (3): Suppose the braid has height $k$ and assume the braid contains $s_2$ or $s_{n-2}$. Without loss of generality we assume that the braid contains $s_2$ as the other argument is symmetric to the one presented here. Notice that if the braid contains $s_2s_3s_2$, we are in Case (2), as a result we assume our braid is of the form $s_1s_2s_1$ or $s_2s_1s_2$. 
	
	Subcase (3.1): Assume the block for $s_1$ is in row $k$, then as we are not in Case (1) we know that at least one of the blocks for $s_0, s_2$ are in the $(k-3)$th row. 
	
	Subcase (3.1.1): Assume that the block for $s_0$ is in row $k-3$ but the block for $s_2$ is not in row $k-3$. The following is the subheap we are considering
	\begin{center}
	\begin{tikzpicture}[scale=0.40]
		\dheapblock{2}{0}{}{black}
		\dheapblock{0}{4}{}{black}
		\heapblock{0}{0}{0}{purple}
		\heapblock{1}{2}{1}{teal}
		\heapblock{2}{4}{2}{teal}
		\heapblock{1}{6}{1}{teal}
	\end{tikzpicture}	
	\end{center}
	where we have highlighted the braid in \textcolor{teal}{teal} and have indicated with dotted blocks positions which cannot be occupied. Applying the braid move to the heap we get the following subheap
	\begin{center}
	\begin{tikzpicture}[scale=0.40]
		\dheapblock{3}{2}{}{black}
		\heapblock{0}{0}{0}{purple}
		\heapblock{2}{0}{2}{teal}
		\heapblock{1}{2}{1}{teal}
		\heapblock{2}{4}{2}{teal}
	\end{tikzpicture}	
	\end{center}
	where the new braid now has height $k-1$. This contradicts the way in which we chose the heap of $w$. 
	
	Subcase (3.1.2): Assume that the block for $s_2$ is in row $k$. Potentially the block for $s_0$ is in row $k$, however it does not have to be, so we emphasize this in the following heap with a \textcolor{purple}{purple}-striped block.  Then the subheap we are considering is as follows
	\begin{center}
	\begin{tikzpicture}[scale=0.40]
		\dheapblock{0}{4}{}{black}
		\heapblock{2}{0}{2}{purple}
		\sheapblock{0}{0}{0}{purple}
		\heapblock{1}{2}{1}{teal}
		\heapblock{2}{4}{2}{teal}
		\heapblock{1}{6}{1}{teal}
	\end{tikzpicture}	
	\end{center}
	where the braid we are considering in the heap is highlighted in \textcolor{teal}{teal}. Notice that if this was the heap for $w$, the expression for $w$ would not be reduced. This forces the block for $s_3$ to be in the heap as follows:
	\begin{center}
	\begin{tikzpicture}[scale=0.40]
		\dheapblock{0}{4}{}{black}
		\heapblock{2}{0}{2}{purple}
		\sheapblock{0}{0}{0}{purple}
		\heapblock{1}{2}{1}{teal}
		\heapblock{2}{4}{2}{teal}
		\heapblock{1}{6}{1}{teal}
		\heapblock{3}{2}{3}{purple}
	\end{tikzpicture}	
	\end{center}
	Applying the braid move to the heap we get the following subheap
	\begin{center}
	\begin{tikzpicture}[scale=0.40]
		\dheapblock{1}{2}{}{black}
		\sheapblock{0}{0}{0}{purple}
		\heapblock{2}{0}{2}{rred}
		\heapblock{3}{2}{3}{rred}
		\heapblock{2}{4}{2}{rred}
		\heapblock{1}{6}{1}{purple}
		\heapblock{2}{8}{2}{purple}
	\end{tikzpicture}	
	\end{center}
	where a new braid has appeared which we have highlighted in \textcolor{rred}{red}. Notice that the height of this new braid is $k-1$, which is a contradiction to the original choice of $w$.
	
	Subcase (3.2): Assume $s_2$ is in row $k$. This implies that at least one of the blocks for $s_1$ or $s_3$ is in the $(k-3)$th row. We proceed in cases.
	
	Subcase (3.2.1): Assume that only the block for $s_3$ is in row $k-3$. Then we have the subheap
	\begin{center}
	\begin{tikzpicture}[scale=0.40]
		\dheapblock{1}{0}{}{black}
		\dheapblock{3}{4}{}{black}
		\heapblock{3}{0}{3}{purple}
		\heapblock{2}{2}{2}{teal}
		\heapblock{1}{4}{1}{teal}
		\heapblock{2}{6}{2}{teal}
	\end{tikzpicture}	
	\end{center}
	where we have highlighted the heap in \textcolor{teal}{teal}. Applying the braid move to this subheap we obtain the following subheap 
	\begin{center}
	\begin{tikzpicture}[scale=0.40]
		\dheapblock{0}{2}{}{black}
		\heapblock{1}{0}{1}{teal}
		\heapblock{3}{0}{3}{purple}
		\heapblock{2}{2}{2}{teal}
		\heapblock{1}{4}{1}{teal}
	\end{tikzpicture}	
	\end{center}
	where the braid has height $k-1$, which is a contradiction to the way in which we chose the heap for $w$.
	
	Subcase (3.2.2): Assume that the block for $s_1$ is in row $k-3$ and the block for $s_3$ is not in row $k-3$. Then we have the subheap
	\begin{center}
	\begin{tikzpicture}[scale=0.40]
		\dheapblock{3}{0}{}{black}
		\dheapblock{3}{4}{}{black}
		\heapblock{1}{0}{1}{purple}
		%\sheapblock{3}{0}{3}{purple}
		\heapblock{2}{2}{2}{teal}
		\heapblock{1}{4}{1}{teal}
		\heapblock{2}{6}{2}{teal}
	\end{tikzpicture}
	\end{center}
	where we have highlighted the braid in teal. Notice that if this is the actual subheap, the corresponding expression for $w$ is not reduced, so we know that $s_0$ must appear in the heap which we illustrate here
	\begin{center}
	\begin{tikzpicture}[scale=0.40]
		\dheapblock{3}{0}{}{black}
		\dheapblock{3}{4}{}{black}
		\heapblock{0}{2}{0}{purple}
		\heapblock{1}{0}{1}{purple}
		%\sheapblock{3}{0}{3}{purple}
		\heapblock{2}{2}{2}{teal}
		\heapblock{1}{4}{1}{teal}
		\heapblock{2}{6}{2}{teal}
	\end{tikzpicture}
	\end{center}
	applying the braid move to the heap we obtain the following subheap
	\begin{center}
	\begin{tikzpicture}[scale=0.40]
		\dheapblock{0}{6}{}{black}
		\heapblock{1}{0}{1}{purple}
		%\sheapblock{3}{0}{3}{purple}
		\heapblock{0}{2}{0}{purple}
		\heapblock{1}{4}{1}{teal}
		\heapblock{2}{6}{2}{teal}
		\heapblock{1}{8}{1}{teal}
	\end{tikzpicture}	
	\end{center}
	Notice that there are no new braids. However, we know that this cannot be the bottom row of our subheap as then $w$ has Property T on the right. This implies that the block for $s_2$ is in the row beneath the blocks for $s_1$ and $s_3$. Notice that $s_0$ cannot be in this row since otherwise the heap would contain a lower braid. This leads to the heap 
	\begin{center}
	\begin{tikzpicture}[scale=0.40]
		\dheapblock{0}{-2}{}{black}
		\dheapblock{0}{6}{}{black}
		\heapblock{1}{0}{1}{purple}
		%\sheapblock{3}{0}{3}{purple}
		\heapblock{0}{2}{0}{purple}
		\heapblock{1}{4}{1}{teal}
		\heapblock{2}{6}{2}{teal}
		\heapblock{1}{8}{1}{teal}
		\heapblock{2}{-2}{2}{rred}
	\end{tikzpicture}	
	\end{center}
	where we have emphasized that $s_0$ cannot be in the position with a dotted block. Again this cannot be the bottom row of the subheap and utilizing the same argument we get the following heap
	\begin{center}
	\begin{tikzpicture}[scale=0.40]
		\dheapblock{0}{-2}{}{black}
		\dheapblock{1}{-4}{}{black}
		\dheapblock{0}{6}{}{black}
		\heapblock{1}{0}{1}{purple}
		%\sheapblock{3}{0}{3}{purple}
		\heapblock{0}{2}{0}{purple}
		\heapblock{1}{4}{1}{teal}
		\heapblock{2}{6}{2}{teal}
		\heapblock{1}{8}{1}{teal}
		\heapblock{2}{-2}{2}{rred}
		\heapblock{3}{-4}{3}{rred}
	\end{tikzpicture}	
	\end{center}
	Again this cannot be the bottom row of the heap. Iterating this argument, we get the following subheap:
	\begin{center}
	\begin{tikzpicture}[scale=0.40]
		\dheapblock{0}{4}{}{black}
		\dheapblock{1}{2}{}{black}
		\heapblock{1}{6}{1}{purple}
		%\sheapblock{3}{0}{3}{purple}
		\heapblock{0}{8}{0}{purple}
		\heapblock{1}{10}{1}{teal}
		\heapblock{2}{12}{2}{teal}
		\heapblock{1}{14}{1}{teal}
		\heapblock{2}{4}{2}{rred}
		\heapblock{3}{2}{3}{rred}
		\heapblock{4}{0}{4}{rred}
		%\node[] at (7,-1.5){$\ddots$};
		\node[] at (5,-1.5){$\ddots$};
		\node[] at (3,-1.5){$\ddots$};
		
		%\dheapblock{10}{-3.5}{}{black}
		%\dheapblock{11}{-5.5}{}{black}
		%\dheapblock{10}{-7.5}{}{black}
		\dheapblock{6}{-7.5}{}{black}
		\dheapblock{7}{-5.5}{}{black}
		\dheapblock{6}{-3.5}{}{black}
		\heapblock{8}{-3.5}{n-1}{rred}
		\heapblock{9}{-5.5}{n}{rred}
		\heapblock{8}{-7.5}{n-1}{rred}
		
		
%		\node[] at (6,-9){$\iddots$};
%		\node[] at (4,-9){$\iddots$};
%		\node[] at (8,-9){$\iddots$};
	\end{tikzpicture}	
	\end{center}
	But then by Lemma~\ref{lem:zigzag} $w$ has Property T on the right. This is a contradiction to the way in which we chose $w$.
	
	Subcase (3.2.3): Assume that blocks for $s_1$ and $s_3$ are in row $k-3$. Then the subheap we are considering is
	\begin{center}
	\begin{tikzpicture}[scale=0.40]
		\dheapblock{3}{4}{}{black}
		\heapblock{1}{0}{1}{purple}
		\heapblock{3}{0}{3}{purple}
		\heapblock{2}{2}{2}{teal}
		\heapblock{0}{2}{0}{purple}
		\heapblock{1}{4}{1}{teal}
		\heapblock{2}{6}{2}{teal}
	\end{tikzpicture}	
	\end{center}
	where the braid is highlighted in \textcolor{teal}{teal}. Applying the braid move we get the following subheap
	\begin{center}
	\begin{tikzpicture}[scale=0.40]
		\dheapblock{0}{6}{}{black}
		\heapblock{1}{0}{1}{purple}
		\heapblock{3}{0}{3}{purple}
		\heapblock{0}{2}{0}{purple}
		\heapblock{1}{4}{1}{teal}
		\heapblock{2}{6}{2}{teal}
		\heapblock{1}{8}{1}{teal}
	\end{tikzpicture}
	\end{center}
	where no new braids have appeared, and in fact the original braid is higher now with height $k+1$. Notice however, that the $(k-3)$th row will not be the bottom row of our heap because $w$ would have Property T with respect to $s_1$ and $s_0$, a contradiction to our assumption. With this in mind we consider the $(k-4)$th row. Notice that $s_0$ will not appear in the $(k-4)$th row as we would have a lower braid. This implies that the $(k-4)$th row contains at least one of $s_2$ or $s_4$. This leads us to represent this with the following in the following 3 subcases. 
	
	Subcase (3.2.3.1): Suppose $s_2$ is in the $(k-4)$th row but $s_4$ is not. Then we have the subheap
	\begin{center}
	\begin{tikzpicture}[scale=0.40]
		\dheapblock{0}{8}{}{black}
		\dheapblock{0}{0}{}{black}
		\heapblock{2}{0}{2}{rred}
		\dheapblock{4}{0}{}{black}
		\heapblock{1}{2}{1}{purple}
		\heapblock{3}{2}{3}{purple}
		\heapblock{0}{4}{0}{purple}
		\heapblock{1}{6}{1}{teal}
		\heapblock{2}{8}{2}{teal}
		\heapblock{1}{10}{1}{teal}
	\end{tikzpicture}	
	\end{center}
	where we have highlighted the additions in \textcolor{rred}{red}. We have also placed dotted blocks in positions where no blocks may appear. Notice that if this new row is row 1, then $w$ would have Property T with respect to $s_2$ and $s_1$. Again, this implies that this will not be the bottom row of the heap. Repeating this process again we see that having $s_1$ or $s_3$ will create a braid in the $(k-5)$th row. Then we have the subheap 
	\begin{center}
	\begin{tikzpicture}[scale=0.40]
		\dheapblock{0}{8}{}{black}
		\dheapblock{0}{0}{}{black}
		\heapblock{2}{0}{2}{rred}
		\dheapblock{4}{0}{}{black}
		\heapblock{1}{2}{1}{rred}
		\heapblock{3}{2}{3}{rred}
		\heapblock{0}{4}{0}{purple}
		\heapblock{1}{6}{1}{teal}
		\heapblock{2}{8}{2}{teal}
		\heapblock{1}{10}{1}{teal}
		\sheapblock{3}{-2}{3}{rred}
		\sheapblock{1}{-2}{1}{rred}
	\end{tikzpicture}	
	\end{center}
	where we have highlighted the additions in \textcolor{rred}{red}-striped blocks and the new lower braid is also highlighted in \textcolor{rred}{red}. This is a contradiction to the way in which we chose $w$.
	
	Subcase (3.2.3.2): Suppose $s_4$ is in the $(k-4)$th row but $s_2$ is not. Then we have the subheap
	\begin{center}
	\begin{tikzpicture}[scale=0.40]
		\dheapblock{0}{0}{}{black}
		\heapblock{4}{0}{4}{rred}
		\dheapblock{2}{0}{}{black}
		\heapblock{1}{2}{1}{purple}
		\heapblock{3}{2}{3}{purple}
		\heapblock{0}{4}{0}{purple}
		\heapblock{1}{6}{1}{teal}
		\heapblock{2}{8}{2}{teal}
		\heapblock{1}{10}{1}{teal}
	\end{tikzpicture}	
	\end{center}
	where we have highlighted the additions in \textcolor{rred}{red}.  Notice that if this new row is row 1, then $w$ would have Property T with respect to $s_3$ and $s_4$. Again, this implies that this is not the bottom row of the heap. Repeating this process again we obtain the subheap
	\begin{center}
	\begin{tikzpicture}[scale=0.40]
		\sheapblock{3}{-2}{3}{rred}
		\sheapblock{5}{-2}{5}{rred}
		\dheapblock{0}{0}{}{black}
		\heapblock{4}{0}{4}{rred}
		\dheapblock{2}{0}{}{black}
		\heapblock{1}{2}{1}{purple}
		\heapblock{3}{2}{3}{purple}
		\heapblock{0}{4}{0}{purple}
		\heapblock{1}{6}{1}{teal}
		\heapblock{2}{8}{2}{teal}
		\heapblock{1}{10}{1}{teal}
	\end{tikzpicture}	
	\end{center}
	where we have highlighted the possible additions as \textcolor{rred}{red}-striped blocks. Notice that if the block for $s_3$ is in that position we have a  braid with height less than $k$. So assume the block for $s_3$ is not in that position. Then we have the subheap
	\begin{center}
	\begin{tikzpicture}[scale=0.40]
		\dheapblock{3}{-2}{}{black}
		\heapblock{5}{-2}{5}{rred}
		\dheapblock{0}{0}{}{black}
		\heapblock{4}{0}{4}{rred}
		\dheapblock{2}{0}{}{black}
		\heapblock{1}{2}{1}{purple}
		\heapblock{3}{2}{3}{purple}
		\heapblock{0}{4}{0}{purple}
		\heapblock{1}{6}{1}{teal}
		\heapblock{2}{8}{2}{teal}
		\heapblock{1}{10}{1}{teal}
	\end{tikzpicture}
	\end{center}
	where we have highlighted the additions in \textcolor{rred}{red} and have placed a dotted block in the position for $s_3$ indicating that no block may appear there. Iterating this process we end up with the subheap 
	\begin{center}
	\begin{tikzpicture}[scale=0.40]
		\dheapblock{3}{-2}{}{black}
%		\dheapblock{7}{-2}{}{black}
%		\dheapblock{6}{0}{}{black}
		\heapblock{5}{-2}{5}{rred}
		\dheapblock{0}{0}{}{black}
		\heapblock{4}{0}{4}{rred}
		\dheapblock{2}{0}{}{black}
		\heapblock{1}{2}{1}{purple}
		\heapblock{3}{2}{3}{purple}
		\heapblock{0}{4}{0}{purple}
		\heapblock{1}{6}{1}{teal}
		\heapblock{2}{8}{2}{teal}
		\heapblock{1}{10}{1}{teal}
		
		%\node[] at (8,-4){$\ddots$};
		\node[] at (6.5,-4){$\ddots$};
		\node[] at (5, -4){$\ddots$};
		
		%\dheapblock{10}{-6.5}{}{black}
		\dheapblock{5}{-6.5}{}{black}
		\heapblock{7}{-6.5}{n-2}{rred}
		\dheapblock{6}{-12.5}{}{black}
		\dheapblock{7}{-10.5}{}{black}
		\dheapblock{6}{-8.5}{}{black}
		\heapblock{8}{-8.5}{n-1}{rred}
		\heapblock{9}{-10.5}{n}{rred}
		\heapblock{8}{-12.5}{n-1}{rred}
	\end{tikzpicture}
	\end{center}
	where we have highlighted the additions in \textcolor{rred}{red} and have placed dotted blocks in positions that blocks may not occupy. Notice that if the zig-zag was to end before reaching $s_0$, the heap would have Property T, which is a contradiction to the way in which we chose $w$. Suppose that the zig-zag continues on after reaching $s_0$. Then we are able to drop the block for $s_1$ down and create a lower braid. This is a contradiction to the way in which we chose $w$.
	
	Subcase (3.2.3.3): Suppose the blocks for $s_2$ and $s_4$ are in the $(k-4)$th row. Then we have the subheap
	\begin{center}
	\begin{tikzpicture}[scale=0.40]
		\dheapblock{0}{2}{}{black}
		\dheapblock{0}{10}{}{black}
		%\dheapblock{2}{-2}{}{black}
		%\sheapblock{3}{0}{3}{rred}
		%\sheapblock{5}{0}{5}{rred}
		\heapblock{2}{2}{2}{rred}
		\heapblock{4}{2}{4}{rred}
		\heapblock{1}{4}{1}{purple}
		\heapblock{3}{4}{3}{purple}
		\heapblock{0}{6}{0}{purple}
		\heapblock{1}{8}{1}{teal}
		\heapblock{2}{10}{2}{teal}
		\heapblock{1}{12}{1}{teal}
	\end{tikzpicture}	
	\end{center}
	where we have highlighted the additions in \textcolor{rred}{red} and have placed a dotted block in the position for $s_0$ in the same row, as if it were to appear the heap would have a lower braid. This subheap has Property T in the bottom with respect to $s_2$ and $s_1$. Thus this cannot be the bottom row of our heap. Repeating this process, we see that $s_1$ will not be in row $k-5$ since this would create a lower braid. Thus we must have $s_3$ or $s_5$ in the $(k-5)$th row. We represent this with the following subheap 
	\begin{center}
	\begin{tikzpicture}[scale=0.40]
		\dheapblock{0}{2}{}{black}
		\dheapblock{1}{0}{}{black}
		%\dheapblock{2}{-2}{}{black}
		\heapblock{3}{0}{3}{rred}
		\heapblock{5}{0}{5}{rred}
		\heapblock{2}{2}{2}{rred}
		\heapblock{4}{2}{4}{rred}
		\heapblock{1}{4}{1}{purple}
		\heapblock{3}{4}{3}{purple}
		\heapblock{0}{6}{0}{purple}
		\heapblock{1}{8}{1}{teal}
		\heapblock{2}{10}{2}{teal}
		\heapblock{1}{12}{1}{teal}
	\end{tikzpicture}	
	\end{center}
	where again we have highlighted the additions in \textcolor{rred}{red} and put dotted blocks where no block may appear. Notice that if we were to only place one of the blocks for $s_3$ or $s_5$ we would quickly be in Subcase (3.2.3.1) or (3.2.3.2) above. Thus we know we must place both $s_3$ and $s_5$ in the $(k-5)$th row. Again, if the $(k-5)$th row is the first row, then  $w$ would have Property T with respect to $s_3$ and $s_2$. This implies that the $(k-5)$th row is not the bottom-most row in our heap. Iterating this process we obtain the following subheap
	\begin{center}
	\begin{tikzpicture}[scale=0.40] 
		\dheapblock{0}{2}{}{black}
		\dheapblock{1}{0}{}{black}
		\dheapblock{2}{-2}{}{black}
		\heapblock{1}{12}{1}{teal}
		\heapblock{2}{10}{2}{teal}
		\heapblock{1}{8}{1}{teal}
		\heapblock{0}{6}{0}{purple}
		\heapblock{1}{4}{1}{purple}
		\heapblock{3}{4}{3}{purple}
		\heapblock{2}{2}{2}{rred}
		\heapblock{3}{0}{3}{rred}
		\heapblock{4}{2}{4}{rred}
		\heapblock{5}{0}{5}{rred}
		\heapblock{4}{-2}{4}{rred}
		
		\node[] at (8,-4){$\ddots$};
		\node[] at (6.5,-4){$\ddots$};
		\node[] at (5, -4){$\ddots$};
		
		\dheapblock{7}{-7}{}{black}
		\dheapblock{8}{-9}{}{black}
		%\dheapblock{7}{-11}{}{black}
		\heapblock{9}{-7}{n-3}{rred}
		\heapblock{11}{-7}{n-1}{rred}
		\heapblock{10}{-9}{n-2}{rred}
		\heapblock{12}{-9}{n}{rred}
	\end{tikzpicture}
	\end{center}
	where again we see that if the row containing the blocks for $s_{n-2}$ and $s_n$ corresponds to row 1, then $w$ would have Property T with respect to $s_{n-2}$ and $s_{n-3}$. Continuing in this manner, we obtain the following subheap
	\begin{center}
	\begin{tikzpicture}[scale=0.400] 
%		\draw[white] (-1,5) circle (2pt);
%		\draw[white] (-1,-14) circle (2pt);
%		\draw[white] (5,-7) circle (2pt);
%		\draw{ (0.5,10) node{}
%		[-] (-1,5)--(0,3)--(1,1)--(2,-1)--(3,-3)--(8.5,-7)--(0,-13)--(-1,-15)--(-1,5)
%		(2,-7) node{\huge$\emptyset$}
%		%[-] (-1,4)--(5,-7)
%		%[-] (-1,-14)--(5,-7)
%		(1,10) node{}};
		\dheapblock{0}{2}{}{black}
		\dheapblock{1}{0}{}{black}
		\dheapblock{2}{-2}{}{black}
		\heapblock{1}{12}{1}{teal}
		\heapblock{2}{10}{2}{teal}
		\heapblock{1}{8}{1}{teal}
		\heapblock{0}{6}{0}{purple}
		\heapblock{1}{4}{1}{purple}
		\heapblock{3}{4}{3}{purple}
		\heapblock{2}{2}{2}{rred}
		\heapblock{3}{0}{3}{rred}
		\heapblock{4}{2}{4}{rred}
		\heapblock{5}{0}{5}{rred}
		\heapblock{4}{-2}{4}{rred}
		
		\node[] at (7,-3){$\ddots$};
		\node[] at (6.2,-4){$\ddots$};
		\node[] at (5, -5){$\ddots$};
		
		\dheapblock{7}{-7}{}{black}
		\dheapblock{8}{-9}{}{black}
		\dheapblock{7}{-11}{}{black}
		\heapblock{9}{-7}{n-3}{orange}
		\heapblock{11}{-7}{n-1}{rred}
		\heapblock{10}{-9}{n-2}{orange}
		\heapblock{12}{-9}{n}{rred}
		\heapblock{11}{-11}{n-1}{rred}
		\heapblock{9}{-11}{n-3}{orange}
%		\sheapblock{10}{-11}{n-2}{rred}
%		
%		\node[] at (7, -11){$\iddots$};
%		\node[] at (7,-13){$\iddots$};
%		
%		\sheapblock{3}{-14}{3}{rred}
%		\sheapblock{1}{-14}{1}{rred}
%		\sheapblock{2}{-16}{2}{rred}
%		\sheapblock{0}{-16}{0}{rred}
%		\sheapblock{4}{-16}{4}{rred}
	\end{tikzpicture}

	\end{center}
	where the \textcolor{rred}{red} blocks correspond to a portion of an FC element. Recall that as we created this subheap we placed blocks in a manner to prevent a braid from appearing with height less than $k$ we prevented blocks from being placed inside the double diagonal (signified by the dotted blocks). However the \textcolor{orange}{orange} blocks create a new braid with height less than $k$. This is a contradiction to the way in which we chose $w$.  
%	\begin{center}
%	\begin{tikzpicture}[scale=0.40] 
%		\heapblock{1}{12}{1}{teal}
%		\heapblock{2}{10}{2}{teal}
%		\heapblock{1}{8}{1}{orange}
%		\heapblock{0}{6}{0}{orange}
%		\heapblock{1}{4}{1}{orange}
%		\heapblock{3}{4}{3}{purple}
%		\heapblock{0}{2}{0}{orange}
%		\sheapblock{2}{2}{2}{rred}
%		\sheapblock{3}{0}{3}{rred}
%		\sheapblock{4}{2}{4}{rred}
%		\sheapblock{5}{0}{5}{rred}
%		\sheapblock{4}{-2}{4}{rred}
%		
%		\node[] at (7,-2){$\ddots$};
%		\node[] at (7,-4){$\ddots$};
%		
%		\sheapblock{9}{-3}{n-3}{rred}
%		\sheapblock{11}{-3}{n-1}{rred}
%		\sheapblock{10}{-5}{n-2}{rred}
%		\sheapblock{12}{-5}{n}{rred}
%		\sheapblock{11}{-7}{n-1}{rred}
%		\sheapblock{9}{-7}{n-3}{rred}
%		\sheapblock{0}{-5}{}{rred}
%		\sheapblock{2}{-5}{}{rred}
%	
%		\node[] at (7, -7){$\iddots$};
%		\node[] at (7,-9){$\iddots$};
%		\node[] at (5,-5){$\cdots$};
%		\node[] at (0,-1.5){$\vdots$};
%		\node[] at (0,-7){$\vdots$};
%		
%		\sheapblock{3}{-9}{3}{rred}
%		\sheapblock{1}{-9}{1}{rred}
%		\sheapblock{2}{-11}{2}{rred}
%		\sheapblock{0}{-11}{0}{rred}
%		\sheapblock{4}{-11}{4}{rred}
%	\end{tikzpicture}
%	\end{center}
%	where we see that this has led to $s_0$ being located in the $(k-4)$th row. We emphasize here with the not labeled \textcolor{rred}{red}-striped blocks that every possible brick is present in the area that was outlined in the previous heap. Notice that a new braid has now appeared in the heap. We have highlighted this in \textcolor{orange}{orange} above. This new braid has height $k-1$. This is a contradiction as the braid appears lower than the original braid we chose. 
		
	Case (4): Suppose the braid has height $k$ and assume the braid contains $s_1$ or $s_{n-1}$. Without loss of generality we assume that the braid contains $s_1$, as the the other argument is symmetric to the one presented here. Notice that if the braid is  $s_1s_2s_1$, then we are in Case (3), so assume the braid consists of $s_0$ and $s_1$. 
	
	Subcase (4.1): In this case, we take our braid to be $s_1s_0s_1s_0$. Assume that, if necessary, the blocks that complete the braid have been brought up next to $s_1$ in the $k$th row. We now consider which blocks can occur in the $(k-3)$th row and $(k-4)$th row in two cases. We know that row $k-3$ is not the bottom row of our heap which implies $s_1$ must be in row $k-4$. Notice that if the block for $s_2$ is not in the $(k-3)th$ row then the expression for $w$ was not reduced. Assume the block for $s_2$ is located in the $(k-3)$th row. Then the subheap we are considering is
	\begin{center}
	\begin{tikzpicture}[scale=0.40]
		\dheapblock{2}{4}{}{black}
		\heapblock{0}{0}{0}{orange}
		\heapblock{2}{0}{2}{purple}
		\heapblock{1}{2}{1}{orange}
		\heapblock{0}{4}{0}{orange}
		\heapblock{1}{6}{1}{orange}
	\end{tikzpicture}	
	\end{center}
	where the braid we mentioned is highlighted in \textcolor{orange}{orange}. Applying the braid move we get the following subheap
	\begin{center}
	\begin{tikzpicture}[scale=0.40]
		\dheapblock{2}{4}{}{black}
		\heapblock{2}{0}{2}{purple}
		\heapblock{1}{2}{1}{orange}
		\heapblock{0}{4}{0}{orange}
		\heapblock{1}{6}{1}{orange}
		\heapblock{0}{8}{0}{orange}
	\end{tikzpicture}
	\end{center}
	in which we see that the original braid is now located higher in the heap with height $k+1$. Since $k > 4$ (otherwise we are in Case (1)), we know that in the original heap, $s_0$ and $s_2$ are located above row 1. This implies that the heap for $w$ has more rows underneath, which we will now systematically fill in. 
	
	Subcase (4.1.1): We first consider if the block for $s_1$ is located in row $k-4$ in the heap immediately above and $s_3$ is allowed but not required to be there. This leads to the following heap
	\begin{center}
	\begin{tikzpicture}[scale=0.40]
		\dheapblock{2}{6}{}{black}
		\sheapblock{3}{0}{3}{purple}
		\heapblock{1}{0}{1}{rred}
		\dheapblock{3}{0}{}{black}
		\heapblock{2}{2}{2}{rred}
		\heapblock{1}{4}{1}{rred}
		\heapblock{0}{6}{0}{orange}
		\heapblock{1}{8}{1}{orange}
		\heapblock{0}{10}{0}{orange}
	\end{tikzpicture}	
	\end{center}
	where a new braid appears which we have highlighted in \textcolor{rred}{red}. Notice that the block for $s_0$ cannot be in between the blocks for $s_1$ which are highlighted in \textcolor{rred}{red}, since otherwise the corresponding expression for $w$ is not reduced. This implies that we have
	\begin{center}
	\begin{tikzpicture}[scale=0.40]
		\dheapblock{0}{2}{}{black}
		\dheapblock{2}{6}{}{black}
		\sheapblock{3}{0}{3}{purple}
		\heapblock{1}{0}{1}{rred}
		\dheapblock{3}{0}{}{black}
		\heapblock{2}{2}{2}{rred}
		\heapblock{1}{4}{1}{rred}
		\heapblock{0}{6}{0}{orange}
		\heapblock{1}{8}{1}{orange}
		\heapblock{0}{10}{0}{orange}
	\end{tikzpicture}	
	\end{center}
	where we have indicated the absence of the block for $s_0$ as usual. This new braid has height $k-2$ which is has height lower thank $k$, the height of the original braid that we chose. This is a contradiction to the way in which we chose $w$. 
	
	Subcase (4.1.2): Now we consider the case where the block for $s_3$ is in the $(k-4)$th row and the block for $s_1$ is not. This leads to the following subheap:
	\begin{center}
	\begin{tikzpicture}[scale=0.40]
		%\dheapblock{3}{4}{}{black}
		\dheapblock{1}{0}{}{black}
		\dheapblock{0}{2}{}{black}
		\heapblock{3}{0}{3}{rred}
		\heapblock{2}{2}{2}{purple}
		\heapblock{1}{4}{1}{orange}
		\heapblock{0}{6}{0}{orange}
		\heapblock{1}{8}{1}{orange}
		\heapblock{0}{10}{0}{orange}
	\end{tikzpicture}
	\end{center}
	Although there are no new braids present, the bottom row of the subheap is not the bottom row of the heap for $w$ since otherwise $w$ would have Property T. Repeating the above argument we extend our heap to look like
	\begin{center}
	\begin{tikzpicture}[scale=0.40]
		%\dheapblock{3}{6}{}{black}
		%\dheapblock{4}{4}{}{black}
		%\dheapblock{2}{0}{}{black}
		%\dheapblock{5}{2}{}{black}
		\dheapblock{0}{2}{}{black}
		\dheapblock{2}{0}{}{black}
		\dheapblock{1}{2}{}{black}
		\heapblock{4}{0}{4}{rred}
		\heapblock{3}{2}{3}{rred}
		\heapblock{2}{4}{2}{purple}
		\heapblock{1}{6}{1}{orange}
		\heapblock{0}{8}{0}{orange}
		\heapblock{1}{10}{1}{orange}
		\heapblock{0}{12}{0}{orange}
	\end{tikzpicture}
	\end{center} 
	where again we see no new braids. Again, we know that the bottom row of the subheap above is not the bottom row of the heap for $w$ since otherwise $w$ would have Property T. Iterating this process we obtain a heap that looks like:
	\begin{center}
	\begin{tikzpicture}[scale=0.40]
		%\dheapblock{3}{6}{}{black}
		%\dheapblock{4}{4}{}{black}
		%\dheapblock{5}{2}{}{black}
		%\dheapblock{6}{0}{}{black}
		\heapblock{0}{12}{0}{orange}
		\heapblock{1}{10}{1}{orange}
		\heapblock{0}{8}{0}{orange}
		\heapblock{1}{6}{1}{orange}
		\heapblock{2}{4}{2}{purple}
		\dheapblock{1}{2}{}{black}
		\heapblock{3}{2}{3}{rred}
		\dheapblock{2}{0}{}{black}
		\heapblock{4}{0}{4}{rred}
		
		\node[] at (7,-1.5){$\ddots$};
		\node[] at (5,-1.5){$\ddots$};
		\node[] at (3,-1.5){$\ddots$};
		
		%\dheapblock{10}{-3.5}{}{black}
		%\dheapblock{11}{-5.5}{}{black}
		%\dheapblock{10}{-7.5}{}{black}
		\dheapblock{6}{-7.5}{}{black}
		\dheapblock{7}{-5.5}{}{black}
		\dheapblock{6}{-3.5}{}{black}
		\heapblock{8}{-3.5}{n-1}{rred}
		\heapblock{9}{-5.5}{n}{rred}
		\heapblock{8}{-7.5}{n-1}{rred}
		
		
%		\node[] at (6,-9){$\iddots$};
%		\node[] at (4,-9){$\iddots$};
%		\node[] at (8,-9){$\iddots$};
	\end{tikzpicture}	
	\end{center}
	Then by Lemma~\ref{lem:zigzag} we know that $w$ has Property T on the right. This is a contradiction to the way in which we chose the heap for $w$. 
	
	Subcase (4.2): Assume the braid of height $k$ is $s_0s_1s_0s_1$. We now consider which blocks may occur in row $k-4$. Notice that $s_0$ cannot be in the $(k-4)$th row as the corresponding expression for $w$ would not be reduced. This implies that $s_2$ is in the $(k-4)$th row as the $(k-3)$th row cannot be row 1, otherwise we are in Case (1.2.1). From this we get the subheap
	\begin{center}
	\begin{tikzpicture}[scale=0.40]
		\heapblock{2}{0}{2}{purple}
		\dheapblock{0}{0}{}{black}
		\heapblock{1}{2}{1}{orange}
		\heapblock{0}{4}{0}{orange}
		\heapblock{1}{6}{1}{orange}
		\heapblock{0}{8}{0}{orange}
	\end{tikzpicture}	
	\end{center}
	where we have highlighted the braid in \textcolor{orange}{orange}. Applying the braid move we get the following subheap
	
	\begin{center}
	\begin{tikzpicture}[scale=0.40]
		\heapblock{0}{0}{0}{orange}
		\heapblock{2}{0}{2}{purple}
		\heapblock{1}{2}{1}{orange}
		\heapblock{0}{4}{0}{orange}
		\heapblock{1}{6}{1}{orange}
	\end{tikzpicture}	
	\end{center}
	where the height of the braid is now $k-1$. This is a contradiction to our original assumption that the heap we started with contains the lowest braid.
	
	Therefore, it follow that $W(\C_n)$ does not contain any not FC $\tII$ elements. 
\end{proof}
\end{theorem}

We will now classify the $\tII$ elements in $W(\C_n)$. We first classify $\tII$ elements in $W(\C_n)$ for $n$ odd and then proceed to the classification for $n$ even.

\begin{theorem}
	If $n$ is odd, then there are no $\tII$ elements in the Coxeter system of type $\C_n$.
	\begin{proof}
		Consider the Coxeter system of type $\C_n$. By Theorem~\ref{thm:TavoidC} we know that $W(\C_n)$ contains no $\tII$ elements that are not FC. Recall that $W(\C_n)$ is a star reducible Coxeter group, which implies that $W(\C_n)$ contains no $\tII$ elements that are FC. Thus, ,as $W(\C_n)$ has no $\tII$ elements that are FC and no $\tII$ elements that are not FC, $W(\C_n)$ has no $\tII$ elements.
	\end{proof}
\end{theorem}

We next classify the $\tII$ elements in the Coxeter system of type $\C_n$ for $n$ even. Recall that $W(\C_n)$ for $n$ even is not a star reducible Coxeter group. In Theorem~\ref{thm:TavoidC} we showed that $W(\C_n)$ does not have $\tII$ elements that are not FC. This leaves us with only the FC elements to check.

\begin{theorem}
	If $n$ is even, then the only $\tII$ elements in $W(\C_n)$ are sandwich stacks.
	\begin{proof}
		Let $w \in W(\C_n)$ such that $w$ is $\tII$. By Theorem~\ref{thm:TavoidC}, we know that $w$ is an FC element. Further, we can restrict our search down to the subset of non-cancellable elements that are not star reducible. Specifically we can consider the non-cancellable elements that do not contain Property T. Recall that in Remark~\ref{rem:noncancel} we stated that the only $\tII$ elements with full support are sandwich stacks. Thus, the only $\tII$ elements in $W(\C_n)$ for $n$ odd are sandwich stacks.
	\end{proof}
\end{theorem}

%%%%%%%%%%%%%%%%%%%
\section{Classification of T-Avoiding Elements of Type $B_n$}\label{sec:typeB}

Unlike the type $\C_n$ case, classifying T-avoiding elements in Coxeter systems of type $B_n$ is straight forward. 

\begin{theorem}\label{thm:typeB}
There are no $\tII$ elements in Coxeter systems of type $B_n$.
\begin{proof}
	Each $W(B_n)$ is a parabolic subgroup of $W(\C_k)$ for $k \geq n$ and $k$ odd. Since $W(\C_k)$ for $k$ odd has no $\tII$ elements, this forces $W(B_n)$ to not have any $\tII$ elements.
\end{proof}
\end{theorem}
   
In the next Chapter, we characterize Property T and T-avoiding in Coxeter systems in terms of signed pattern avoidance.   