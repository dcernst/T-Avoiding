\begin{abstract}
\doublespacing

	Motivated by a desire to understand the Kazhdan-Lusztig theory of the Hecke algebra of the underlying Coxeter group, R.M. Green classified the so-called star reducible Coxeter groups, which have the property that all fully commutative elements (in the sense of Stembridge) can be sequentially reduced via star operations to a product of commuting generators. It turns out that in some Coxeter groups there are elements, called T-avoiding elements, which cannot be systematically dismantled in this way. More specifically an element $w$ is called T-avoiding if $w$ does not have a reduced expression beginning or ending with a pair of non-commuting generators. Clearly, a product of commuting generators is trivially T-avoiding. However, sometimes there are more interesting T-avoiding elements. We define two different types of T-avoiding elements, type I T-avoiding elements and type II T-avoiding elements. All Coxeter groups have type I T-avoiding elements. However,  it has been shown that some Coxeter groups have type II T-avoiding elements and others do not. A natural question that arises from this is \emph{which Coxeter groups have type II T-avoiding elements and which do not}. 
	
	In this thesis we state the already known results regarding T-avoiding elements in certain Coxeter groups. We also present a proof regarding the T-avoiding elements in Coxeter systems of types $B_n$ and $\C_n$. 	
 
	
\end{abstract}

%A star reducible Coxeter group is one in which every fully commutative element can be reduced to a product of commuting generators through repeated star reductions. A complete list of these star re
