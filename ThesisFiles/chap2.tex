\chapter{T-Avoiding Elements in Coxeter groups of Type $B_n$}\label{chap:BnandCn}

In this chapter we classify the T-avoiding elements in Coxeter systems of type $B_n$, which is an original result. We start by introducing some combinatorial tools for type $B_n$ and then finish with a proof of the classification in type $B_n$. Note that the proof for Coxeter systems of type $B_n$ closely follows the classification of T-avoiding elements of type $D_n$ seen in~\cite{Gern2013a}. 

\section{Tools for the Classification}\label{sec:Btools}

Recall from Example~\ref{ex:B} that $W(B_n) \cong \Sym_n^B$ (also called the hyperoctahedral group). We define $\Sym_n^B$ to be the group of all bijections $w$ of the set $\{-n, \ldots, -1, 0, 1, 2, \ldots, n\}$ such that 
\[w(-a)=-w(a)\] for all $a \in \{-n, \ldots, -1, 0, 1, 2, \ldots, n\}$. For $w \in \Sym_n^B$ we write $w=[a_1, a_2, \ldots, a_n]$, to mean that $w(i)=a_i$ for $i \in \{1,2, \ldots n\}$ and call this the signed permutation notation of $w$. That is, we can write $w \in W(B_n)$ using signed permutation notation 
\[ w=[w(1),w(2), \ldots, w(n-1), w(n)], \]
where we write a bar underneath a number in place of a negative sign in order to simplify notation. 

As a set of generators for $\Sym_n^B$ we take $S(B_n)=\{s_0,s_1,s_2, \ldots, s_{n-1}\}$, where for each $i \in \{1,2,\ldots n-1\}$, we have
\[s_i=[1,2, \ldots i-1, i+1,i,i+2, \ldots, n-1,n] \] and we identify $s_0$ with
\[s_0=[\underline{1}, 2 \ldots, n].\] Further $w(-i)=-w(i)$ for $|i| \in \{1,2, \ldots, n\}$. The following propositions provide insight into what happens to a given signed permutation when we multiply by $s_i$ on the right or the left.

\begin{proposition}
	Let $w \in W(B_n)$ with corresponding signed permutation 
	\[w=[w(1),w(2), \ldots ,w(n)].\] Suppose $s_i \in S(B_n)$. If $i \geq 1$, then multiplying $w$ on the right by $s_i$ has the effect of interchanging $w(i)$ and $w(i+1)$ in  the signed permutation notation. If $i=0$, then multiplying $w$ on the right by $s_i$ has the effect of switching the sign of $w(1)$. 	
	\begin{proof}
	This follows from~\cite[Section 8.1 and A3.1]{Bjorner2005}.	
	\end{proof}
\end{proposition}

\begin{proposition}
Let $w \in W(B_n)$ with corresponding signed permutation 
	\[w=[w(1),w(2), \ldots ,w(n)].\] Suppose $s_i \in S(B_n)$. If $i \geq 1$, then multiplying on the left by $s_i$ has the effect of interchanging the entries whose absolute values are $i$ and $i+1$ in the signed permutation notation. If $i=0$, then multiplying $w$ on the left by $s_i$ has the effect of switching the sign of the entry whose absolute value is $1$.
	\begin{proof}
	This follows from~\cite[Section 8.1 and A3.1]{Bjorner2005}.	
	\end{proof}
\end{proposition}

Suppose $w \in W(B_n)$ has reduced expression $\w=s_{x_1}s_{x_2}\cdots s_{x_n}$. We may construct the signed permutation of $w$ from left to right as it is the easier way to multiply based upon the above propositions. However, note that our convention is still composition from left to right. We provide an example of this construction below.

\begin{example}
Let $w \in W(B_6)$ with a given reduced expression $\w=s_0s_1s_3s_4s_5s_2$. Then we iteratively build the signed permutation as follows. First, $s_0=[\underbar{1},2,3,4,5,6]$ by definition. Next $s_0s_1=[2,\underbar{1},3,4,5,6]$ since multiplying by $s_1$ on the right hand side switches the values in position 1 and position 2. Repeating this we get $s_0s_1s_3=[2,\underbar{1},4,3,5,6]$ and ultimately we end with $w=[2,4,\underbar{1},5,6,3]$. 

%Notice that if we were to construct the signed permutation for $w$ from right to left, we would start with $s_2=[1,3,2,4,5,6]$. Next we would have $s_5s_2=[1,3,2,4,6,5]$. However, $s_4s_5s_2=[1,3,2,5,6,4]$. Notice this time we were not able to just switch $w(i)$ and $w(i+1)$ instead we found $4$ and $5$ and switched their relative positions, which is more difficult than constructing the signed permutation notation notation left to right, which is why we choose to construct left to right.
\end{example}

Given the signed permutation notation for an element $w \in W(B_n)$ we can easily calculate the left and right descent sets of $w$. The following proposition explains how.

\begin{proposition}\label{prop:descent}
Let $w \in W(B_n)$. Then 
\[ \mathcal{R}(w)=\{s_i \in S \mid w(i) > w(i+1)\} \]
where $w(0)=0$ by definition.
\begin{proof}
	This is~\cite[Proposition 8.1.2]{Bjorner2005}.
\end{proof}
\end{proposition}

We now will introduce the concept of signed pattern avoidance, which will help with the classification of the T-avoiding elements in Coxeter systems of type $B_n$. Our approach mimics the one found in~\cite{Gern2013a}. Let $w \in W(B_n)$, and let $a,b,c \in \mathbb{Z}$. We say that $w$ \emph{contains the signed consecutive pattern} $abc$ if there is some $i \in \{1,2, \ldots, n-2\}$ such that $(|w(i)|,|w(i+1)|,|w(i+2)|)$ is in the same relative order as $(|a|,|b|,|c|)$ and $\sgn(w(i))=\sgn(a), \sgn(w(i+1))=\sgn(b)$, and $\sgn(w(i+2))=\sgn(c)$, where typically one takes $a,b,c$ to be a subset of the set $\{\pm1,\pm2,\pm3\}$. We say that $w$ \emph{avoids the signed consecutive pattern} $abc$ if there is no $i \in \{1,2, \ldots, n-2\}$ such that $\left(|w(i)|, |w(i+1)|, |w(i+2)|\right)$ is in the same consecutive order as $\left(|a|, |b|, |c| \right)$ such that $\sgn(w(i))=\sgn(a)$, $\sgn(w(i+1))=\sgn(b)$, and $\sgn(w(i+2))=\sgn(c)$.

\begin{example}
Let $w \in W(B_4)$ with signed permutation \[w=[\underline{2},4, \underline{1}, 3].\] We see that $w$ has the signed consecutive pattern $\underline{2} 3 \underline{1}$, since $(|w(1)|, |w(2)|, |w(3)|)$ are in the same relative order as $(|-2|, |3|, |-1|)$, and $\sgn(w(1))=\sgn(-2)$, $\sgn(w(2))=\sgn(3)$, and $\sgn(w(3))=\sgn(-1)$. However, $w$ avoids the signed consecutive pattern $1\underline{2}3$.
\end{example}

Occasionally, we will need to factor $w \in W(B_n)$ in a specific manner. Let $I=\{s,t\}$ for $s, t \in S(B_n)$ such that $s$ and $t$ do not commute. Define $W^I$ as the set of all $w \in W(B_n)$ such that $\LD(w) \cap I= \emptyset$ and define $W_I=\langle s,t \rangle$. In~\cite{Humphreys1990}, it is shown that any element $w \in W(B_n)$ can be written as $w=w^Iw_I$ (reduced) where $w^I \in W^I$ and $w_I \in W_I$.
%%%%%%%%%%%%%%%%%%%%%%%

\section{Classification of T-Avoiding Elements in Type $B_n$}\label{sec:TAB}

%\textcolor{red}{Time permitting we will streamline the corollaries into the lemmas themselves or as a remark at the end of the lemmas}

In this section we classify the T-avoiding elements in Coxeter systems of type $B_n$. Our main result in this section is Theorem~\ref{thm:classificationofB}. Notice that if $n=2$, $W(B_2) \cong I_2(4)$, which by Theorem~\ref{thm:i2m} we know has no $\tII$ elements. We proceed with $n \geq 3$. First we need some preparatory lemmas. 

\begin{lemma}\label{lem:sts}
An element $w \in W(B_n)$ has a reduced expression ending in $s_is_{i+1}s_i$ where $m(s_i,s_{i+1})=3$ if and only if $w$ contains one of the following signed consecutive patterns beginning in position $i$:

\begin{center}
\begin{tabular}{llll}
$321$             & $32\underbar{1}$ & $1 \underbar{23}$ & $\underbar{123}$  \\
$31 \underbar{2}$ & $3 \underbar{12}$ & $21\underbar{3}$  & $2 \underbar{13}$ 

\end{tabular}	
\end{center}


\begin{proof}
	Suppose $w$ contains one of the signed consecutive patterns seen above. Then there is some $i$ such that $w(i) > w(i+1) > w(i+2)$. By Proposition~\ref{prop:descent}, $s_i,s_{i+1} \in \mathcal{R}(w)$. Since $m(s_i,s_{i+1})=3$ and $s_i, s_{i+1} \in \RD(w)$, $w$ ends in $s_is_{i+1}s_{i}$ or $s_{i+1}s_is_{i+1}$.
	
	Conversely, suppose $w$ has a reduced expression ending in $s_is_{i+1}s_i$ where $m(s_i,s_{i+1})=3$. This implies that $w_I=s_is_{i+1}s_i$ which implies that $s_i,s_{i+1} \in \mathcal{R}(w)$. Since $s_i,s_{i+1} \in \mathcal{R}(w)$, we see that $w(i)>w(i+1)>w(i+2)$ by Proposition~\ref{prop:descent}. Thus, $w$ contains one of the signed consecutive patterns seen above.
	
	Therefore, $w$ has a reduced expression ending in $s_is_{i+1}s_i$ if and only if $w$ contains one of the signed consecutive patterns seen above beginning in position $i$. 
\end{proof}	
\end{lemma}

\begin{corollary}\label{lem:endswithsts}
	An element $w \in W(B_n)$ has a reduced expression beginning in $s_is_{i+1}s_i$ where $m(s_i,s_{i+1})=3$ if and only if $w$ contains one of the following consecutive patterns beginning in position $i$:

\begin{center}
\begin{tabular}{llll}
$321$             & $32\underbar{1}$ & $1 \underbar{23}$ & $\underbar{123}$  \\
$31 \underbar{2}$ & $3 \underbar{12}$ & $21\underbar{3}$  & $2 \underbar{13}$ 

\end{tabular}	
\end{center}

	\begin{proof}
		 We know that $w$ has no reduced expressions beginning with $s_is_{i+1}s_i$ if and only if $w^{-1}$ has no reduced expression ending with $s_is_{i+1}s_i$ which by Lemma~\ref{lem:sts} happens only if $w^{-1}$ avoids the signed consecutive patterns seen above beginning in position $i$.
	\end{proof}
\end{corollary}

\begin{lemma}\label{lem:ts}
An element $w \in W(B_n)$ has a reduced expression ending in $s_is_{i+1}$ where $m(s_i,s_{i+1})=3$ if and only if $w$ contains one of the following consecutive patterns beginning in position $i$:

\begin{center}
\begin{tabular}{llll}
$231$             & $23\underbar{1}$ & $12 \underbar{3}$ & $\underbar{1}2\underbar{3}$  \\
$\underline{1}32$ & $13 \underbar{2}$ & $\underbar{2}1\underbar{3}$  & $\underbar{213}$ 

\end{tabular}	
\end{center}

\begin{proof}	
	Suppose $w$ contains one of the signed consecutive patterns seen above. Then there is some $i$ such that $w(i+1)>w(i)>w(i+2)$. By Proposition~\ref{prop:descent}, $s_{i+1} \in \mathcal{R}(w)$. Now multiplying on the right by $s_{i+1}$ we see that $ws_{i+1}(i+1)=w(i+2)$ and $ws_{i+1}(i)=w(i)$. We know that $w(i+2)<w(i)$, which implies that $s_i \in \mathcal{R}(ws_{i+1})$, and hence $w$ has a reduced expression that ends in $s_is_{i+1}$.
	
	 Conversely, suppose that $w$ has a reduced expression ending in $s_is_{i+1}$ where $m(s_i,s_{i+1})=3$. Then $w(i+2)<w(i+1)$ and $w(i)<w(i+1)$. Since $s_i \in \mathcal{R}(ws_{i+1})$ we have $w(i+2)=ws_{i+1}(i+1)<ws_{i+1}(i)=w(i)$. Thus, we have that $w(i+1) > w(i) > w(i+2)$. Hence $w$ contains one of the signed consecutive patterns from above.
	
	Therefore, $w$ has a reduced expression ending in $s_is_{i+1}$ if and only if $w$ contains the signed consecutive patterns from above beginning in position $i$.
\end{proof}	
\end{lemma}

\begin{corollary}\label{lem:endswithst}
	An element $w \in W(B_n)$ has a reduced expression beginning in $s_is_{i+1}$ where $m(s_i,s_{i+1})=3$ if and only if $w$ contains one of the following consecutive patterns beginning in position $i$:

\begin{center}
\begin{tabular}{llll}
$231$             & $23\underbar{1}$ & $12 \underbar{3}$ & $\underbar{1}2\underbar{3}$  \\
$\underline{1}32$ & $13 \underbar{2}$ & $\underbar{2}1\underbar{3}$  & $\underbar{213}$ 

\end{tabular}	
\end{center}.
	\begin{proof}
		We know that $w$ has no reduced expressions beginning with $s_is_{i+1}$ if and only if $w^{-1}$ has no reduced expression ending with $s_is_{i+1}$ which by Lemma~\ref{lem:ts} happens only if $w^{-1}$ avoids the signed consecutive patterns seen above in position $i$. 
	\end{proof}
\end{corollary}

\begin{lemma}\label{lem:st}
An element $w \in W(B_n)$ has a reduced expression ending in $s_{i+1}s_i$ where $m(s_i, s_{i+1})=3$ if and only if $w$ contains one of the following consecutive patterns beginning in position $i$:

\begin{center}
\begin{tabular}{llll}
$312$             & $3\underbar{1}2$ & $3 \underbar{2}1$ & $3\underbar{21}$  \\
$2\underbar{3}1$ & $2\underbar{31}$ & $1\underbar{32}$  & $\underbar{132}$ 

\end{tabular}	
\end{center}
\begin{proof}
	Suppose that $w$ contains one of the signed consecutive patterns seen above.  Then there is some $i$ such that $w(i)>w(i+2)>w(i+1)$. By Proposition~\ref{prop:descent} we see that $s_i \in \mathcal{R}(w)$. Multiplying on the right by $s_i$ we get $ws_i(i+1)=w(i)$ and $ws_i(i+2)=w(i+2)$. By above $w(i)>w(i+2)$, and by Proposition~\ref{prop:descent} $s_{i+1} \in \mathcal{R}(ws_i)$. This implies that $w$ has a reduced expression ending in $s_{i+1}s_i$. 
	
	Conversely suppose $w$ ends in a reduced expression with $s_{i+1}s_i$ where $m(s_i,s_{i+1})=3$. Then $w_I=s_{i+1}s_i$. We see that $w(i)>w(i+1)$ and $w(i+2)>w(i+1)$. Since $s_{i+1} \in \mathcal{R}(ws_i)$, we have $w(i+2)=ws_i(i+2)<ws_i(i+1)=w(i)$. From this we have $w(i)>w(i+2)$, so $w(i)>w(i+2)>w(i+1)$. Hence, $w$ contains one of the signed consecutive patterns seen above. 
	
	Therefore, $w$ has a reduced expression ending in $s_{i+1}s_i$ if and only if $w$ contains one of the signed consecutive patterns seen above beginning in position $i$.
\end{proof}
\end{lemma}

\begin{corollary}\label{lem:endswithts}
An element $w \in W(B_n)$ has a reduced expression beginning in $s_{i+1}s_i$ where $m(s_i, s_{i+1})=3$ if and only if $w$ contains one of the following consecutive patterns beginning in position $i$:

\begin{center}
\begin{tabular}{llll}
$312$             & $3\underbar{1}2$ & $3 \underbar{2}1$ & $3\underbar{21}$  \\
$2\underbar{3}1$ & $2\underbar{31}$ & $1\underbar{32}$  & $\underbar{132}$ 

\end{tabular}	
\end{center}
	\begin{proof}
		Let $s_i,s_{i+1} \in S(B_n)$ such that $m(s,t)=3$ and $s_0 \notin\{s_i,s_{i+1}\}$. We know that $w$ has no reduced expression beginning with $s_{i+1}s_i$ if and only if $w^{-1}$ has no reduced expression ending with $s_{i+1}s_i$ which by Lemma~\ref{lem:st} happens only if $w^{-1}$ avoids the signed consecutive patterns seen above beginning in position $i$.
		\end{proof}
\end{corollary}

\begin{lemma}\label{lem:endswiths0}
Let $w \in W(B_n)$. Then $w$ has a reduced expression ending in $s_1s_0$ if and only if $w(0)>w(1)$ and $-w(1)>w(2)$.
\begin{proof}
	Suppose $w \in W(B_n)$ such that $w$ ends with $s_1s_0$. Then $s_0 \in \RD(w)$ and $s_1 \in \RD(ws_0)$. This implies that $ws_0(1)>ws_0(2)$ by~Proposition~\ref{prop:descent}. We see that $ws_0(1)=w(-1)=-w(1)$ and $ws_0(2)=2$. Hence $-w(1)=ws_0(1)>ws_0(2)=w(2)$. Further, since $s_0 \in \RD(w)$, we see that $w(0)>w(1)$.
	
	Conversely, suppose $w \in W(B_n)$ such that $w(0)>w(1)$ and $-w(1)>w(2)$. Since $w(0)>w(1)$, we know that $s_0 \in \RD(w)$. Multiplying on the right by $s_0$ we see that $ws_0(1)=-w(1)$ and $ws_0(2)=w(2)$. Note that since $ws_0(1)=-w(1)>w(2)=ws_0(2)$, $s_1 \in \RD(ws_0)$. Thus, $w$ ends with $s_1s_0$. 
	
	Therefore, $w$ has a reduced expression ending in $s_1s_0$ if and only if $w(0)>w(1)$ and $-w(1)>w(2)$.
\end{proof}
\end{lemma}

\begin{corollary}\label{lem:beginswiths0}
	Let $w \in W(B_n)$. Then $w$ has a reduced expression beginning in $s_0s_1$ if and only if $w^{-1}(0)>w^{-1}(1)$ and $-w^{-1}(1)>w^{-1}(2)$.
	\begin{proof}
		Let $w \in W(B_n)$. We know that $w$ has no reduced expressions beginning in $s_0s_1$ if and only if $w^{-1}$ has no reduced expressions ending in $s_0s_1$. By Lemma~\ref{lem:endswiths0} we know that this occurs if and only if $w^{-1}(0)>w^{-1}(1)$ and $-w^{-1}(1)>w^{-1}(2)$.
	\end{proof}
\end{corollary}

\begin{lemma}\label{lem:endswiths_1}
Let $w \in W(B_n)$. Then $w$ has a reduced expression ending in $s_0s_1$ if and only if $w(0)>w(2)$ and $w(1)>w(2)$.
\begin{proof}
	Suppose $w \in W(B_n)$ such that $w$ ends with $s_0s_1$. Then $s_1 \in \RD(w)$ and $s_0 \in \RD(ws_1)$. Then $ws_1(0)>ws_1(1)$. We see that $ws_1(0)=0$ and $ws_1(1)=w(2)$. This implies that $0=ws_1(0)>ws_1(1)=2$. Further, since $s_1 \in \RD(w)$ this implies that $w(1) > w(2)$. Thus, if $w$ ends with $s_0s_1$, then $w(1)>w(2)$ and $w(0)>w(2)$.
	
	Conversely, suppose $w \in W(B_n)$ such that $w(1)>w(2)$ and $w(0)>w(2)$. This implies that $s_1 \in \RD(W)$. Multiplying $w$ on the right by $s_1$ we see that $ws_1(0)=w(0)$ and $ws_1(1)=w(2)$. Note that since $ws_1(0)=w(0)>w(2)=ws_1(1)$, $s_0 \in \RD(ws_1)$. Thus, $w$ ends with $s_0s_1$. 
	
	Therefore, $w$ has a reduced expression ending in $s_0s_1$ if and only if $w(1)>w(2)$ and $w(0)>w(2)$.
\end{proof}	
\end{lemma}

\begin{corollary}\label{lem:beginswiths1}
	Let $w \in W(B_n)$. Then $w$ has a reduced expression beginning in $s_1s_0$ if and only if $w^{-1}(0)>w^{-1}(2)$ and $w^{-1}(1)>w^{-1}(2)$.
	\begin{proof}
		Let $w \in W(B_n)$. We know that $w$ has no reduced expressions beginning in $s_1s_0$ if and only if $w^{-1}$ has no reduced expressions ending in $s_1s_0$. By Lemma~\ref{lem:endswiths_1} we know that this occurs if and only if $w^{-1}(0)>w^{-1}(2)$ and $w^{-1}(1)>w^{-1}(2)$.
	\end{proof}
\end{corollary}

\begin{lemma}\label{lem:prodofcommA}
Let $w \in W(B_n)$ such that each entry for $w$ in the signed permutation notation is positive and both $w$ and $w^{-1}$ avoid the consecutive patterns $321$, $231$, and $312$. Then $w$ is a product of commuting generators.
\begin{proof}
	This follows from an appropriate translation of~\cite[Lemma 2.2.9]{Gern2013a}.
\end{proof}	
\end{lemma}

\begin{lemma}\label{lem:prodofCommB}
Let $w \in W(B_n)$ be $\tI$ and let $i \in \{1,2, \ldots, n\}$. Then $w$ satisfies all the following conditions:
\begin{enumerate}[leftmargin=2cm]
\item $w(j) > \min\{w(i-1), w(i)\}$ for all $j >i$;\label{it:trivT1}
\item $w(k) < \max\{w(i-1), w(i)\}$ for all $k < i-1$;\label{it:trivT2}
\item If $w(i), w(i+1) > 0$, then $w(j)>0$ for all $j \geq i$;\label{it:trivT3}
\item If $w(i), w(i+1) < 0$, then $w(j)<0$ for all $j \leq i+1$.\label{it:trivT4}
\end{enumerate}
\begin{proof}
	Suppose there is some least $j>i$ such that $w(j) \leq \min\{w(i-1), w(i)\}$. Note that $j>i$ so $w(j) \neq w(i)$, and $w(j) \neq w(i-1)$ so $w(j) < \min\{w(i-1), w(i)\}$. Then $w(j-2) \geq \min\{w(i+1), w(i)\}>w(j)$. This implies that either $w(j-1)>w(j-2)>w(j)$ or $w(j-2)>w(j-1)>w(j)$, which implies $w$ contains the consecutive pattern $231$ or $321$, which is a contradiction to $w$ being a $\tI$ element by Lemmas~\ref{lem:sts} and~\ref{lem:st}. Thus, proving~\ref{it:trivT1}.
	
	Suppose there exists a maximal $k<i-1$ such that $w \geq \max\{w(i-1), w(i)\}$. Note that $k < i-1$ so $k \neq i$ and $k \neq i-1$. Then $w(k)> \max\{w(i-1), w(i)\}$. Since $k$ is maximal $w(k+1) \leq  \max\{w(i-1), w(i)\}$ and $w(k+2) \leq \max\{w(i-1), w(i)\}$. This implies that either $w(k+2)<w(k+1)<w(k)$ or $w(k+1)<w(k+2)<w(k)$, which implies $w$ contains the consecutive pattern $321$ or $312$, which is a contradiction to $w$ being a $\tI$ element by Lemmas~\ref{lem:sts} and~\ref{lem:ts}. Thus, proving~\ref{it:trivT2}.
	
	It is easy to see that Assertion~\ref{it:trivT1} implies~\ref{it:trivT3} and Assertion~\ref{it:trivT2} implies~\ref{it:trivT4}.
\end{proof}	
\end{lemma}

\begin{lemma}\label{lem:231}
Let $w \in W(B_n)$ such that $w$ contains the consecutive pattern $\underline{2}31$. Then $w$ is not $\tII$.
\begin{proof}
	Let $w \in W(B_n)$ such that $w$ contains the consecutive pattern $\underline{2}31$.
	
	Case (1): Suppose $w$ has the signed permutation notation $w=[\underline{2},3,1]$. This implies that $w=s_1s_0s_2$. Clearly, some reduced expression for $w$ begins with a product of non-commuting generators. Thus, $w$ has Property T.
	
	Case (2): Suppose that $w$ has the signed permutation notation $w=[\underline{a},b,c, \ast, \ldots, \ast]$ where $\underline{a}bc$ corresponds to the signed consecutive pattern $\underline{2}31$, and $\ast$ indicates unknown values for $w(i)$ for $i=4,5, \ldots, n$. We now consider the possible signed consecutive pattern $bc \ast$. The possibilities are: $312$, $31\underline{2}$, $321$, $32\underline{1}$, $213$, or $21\underline{3}$. We know that $b$ and $c$ must be positive since they are positive in $w$ and we also know that $b>c$ by the original signed consecutive pattern. Note that by Lemmas~\ref{lem:sts},~\ref{lem:st}, and~\ref{lem:prodofcommA} and Corollaries~\ref{lem:endswithsts},and~\ref{lem:endswithst} all of these patterns imply that $w$ has a reduced expression that begins or ends with a product of non-commuting generators. Thus, $w$ has Property T.
	
	Case (3): Suppose that $w$ has the signed permutation notation $w=[\ast, \ldots, \ast, \underline{a},b,c]$ where $\underline{a}bc$ corresponds to the signed consecutive pattern $\underline{2}31$, and $\ast$ indicates unknown values for $w(i)$ for $i=1,2, \ldots ,n-3$. We now consider the possible signed consecutive pattern $\ast \underline{a}b$. The possibilities are: $1 \underline{2} 3$, $\underline{1} \underline{2}3$, $2 \underline{1} 3$, $\underline{2} \underline{1} 3$, $3 \underline{1} 2$, or $\underline{3} \underline{1} 2$. Note that by Lemmas~\ref{lem:st},~\ref{lem:endswiths0}, and~\ref{lem:endswiths_1} and Corollaries~\ref{lem:endswithst},~\ref{lem:beginswiths0} and~\ref{lem:beginswiths1} all of these patterns imply that $w$ has a reduced expression that begins or ends with a product of non-commuting generators. Thus, $w$ has Property T. 
	
	Case (4): Suppose that $w$ has the signed permutation notation $w=[\ast, \ldots, \ast, \underline{a},b,c, \ast, \ldots, \ast]$ where $\underline{a}bc$ corresponds to the signed consecutive pattern $\underline{2}31$, and $\ast$ indicates unknown values for $w(i)$ for $|w(i)|\neq a,b,c$. In this case we can apply either Case (2) or Case (3) and we can conclude that $w$ has a reduced expression that begins or ends with a product of non-commuting generators. Thus, $w$ has Property T.

	Hence, if $w \in W(B_n)$ contains the consecutive pattern $\underline{2}31$, then $w$ is not $\tII$.
\end{proof}	
\end{lemma}

\begin{lemma}\label{lem:2312}
Let $w \in W(B_n)$ such that $w$ contains the consecutive pattern $\underline{2}3\underline{1}$. Then $w$ is not $\tII$.
\begin{proof}
	Let $w \in W(B_n)$ such that $w$ contains the consecutive pattern $\underline{2}3\underline{1}$.
	
	Case (1): Suppose $w$ has the signed permutation notation $w=[\underline{2},3,\underline{1}]$. This implies that $w=s_0s_1s_0s_2$. Clearly, some reduced expression for $w$ begins with a product of non-commuting generators. Thus, $w$ has Property T.
	
	Case (2): Suppose that $w$ has the signed permutation notation $w=[\underline{a},b,\underline{c}, \ast, \ldots, \ast]$ where $\underline{a}b\underline{c}$ corresponds to the signed consecutive pattern $\underline{2}3\underline{1}$, and $\ast$ indicates unknown values for $w(i)$ for $i=4,5, \ldots n$. We now consider the possible signed consecutive pattern $b \underline{c} \ast$. The possibilities are: $3\underline{1}2$, $3\underline{1}\underline{2}$, $3\underline{2}1$, $3 \underline{2}\underline{1}$, $2\underline{1}3$, or $2\underline{1}\underline{3}$. We know that $b$ must be positive since it is positive in $w$, $c$ must be negative since it is negative in $w$, and we also know that $|b|>|c|$ by the original signed consecutive pattern. Note that by Lemmas~\ref{lem:sts},~\ref{lem:st}, and~\ref{lem:endswiths_1} and Corollaries~\ref{lem:endswithsts},\ref{lem:endswithst} and~\ref{lem:beginswiths1} all of these patterns imply that $w$ has a reduced expression that begins or ends with a product of non-commuting generators. Thus, $w$ has Property T.
	
	Case (3): Suppose that $w$ has the signed permutation notation $w=[\ast, \ldots, \ast, \underline{a},b,\underline{c}]$ where $\underline{a}b\underline{c}$ corresponds to the signed consecutive pattern $\underline{2}3\underline{1}$, and $\ast$ indicates unknown values for $w(i)$ for $i=1,2, \ldots ,n-3$. We now consider the possible signed consecutive pattern $\ast ab$. The possibilities are: $1 \underline{2} 3$, $\underline{1} \underline{2} 3$, $2 \underline{1} 3$, $\underline{2} \underline{1} 3$, $3 \underline{1} 2$, or $\underline{3} \underline{1} 2$. We know that $a$ must be negative, $b$ must be positive and $|a|<|b|$ by the original signed permutation. Note that by Lemmas~\ref{lem:st},~\ref{lem:endswiths0}, and~\ref{lem:endswiths_1} and Corollaries~\ref{lem:endswithst},~\ref{lem:beginswiths0} and~\ref{lem:beginswiths1} all of these patterns imply that $w$ has a reduced expression that begins or ends with a product of non-commuting generators. Thus, $w$ has Property T. 

	Case (4): Suppose that $w$ has the signed permutation notation $w=[\ast, \ldots,\ast, \underline{a},b,\underline{c}, \ast, \ldots, \ast]$ where $\underline{a}b\underline{c}$ corresponds to the signed consecutive pattern $\underline{2}3\underline{1}$, and $\ast$ indicates unknown values for $w(i)$ for $|w(i)|\neq a,b,c$. In this case we can apply either Case (2) or Case (3) and we can conclude that $w$ has a reduced expression that begins or ends with a product of non-commuting generators. Thus, $w$ has Property T.

	Hence, if $w \in W(B_n)$ contains the consecutive pattern $\underline{2}3\underline{1}$, then $w$ is not $\tII$.
\end{proof}	
\end{lemma}

\begin{lemma}\label{lem:123}
Let $w \in W(B_n)$ such that $w$ contains the consecutive pattern $\underline{1}23$. Then either $w$ is not $\tII$.
\begin{proof}
	Let $w \in W(B_n)$ such that $w$ contains the consecutive pattern $\underline{1}23$.
	
	Case (1): Suppose $w$ has the signed permutation notation $w=[\underline{1}23]$. This implies that $w=s_0$. Clearly, $w$ is a $\tI$ element as it is a single generator.
	
	Case (2): Suppose that $w$ has the signed permutation notation $w=[\underline{a},b,c, \ast, \ldots, \ast]$ where $\underline{a}bc$ corresponds to the signed consecutive pattern $\underline{1}23$, and $\ast$ indicates unknown values for $w(i)$ for $i=4,5, \ldots n$. We now consider the possible signed consecutive patterns $bc \ast$. The possibilities are: $231$, $23 \underline{1}$, $132$, $13 \underline{2}$, $123$, $12 \underline{3}$. We know that $b$ and $c$ are positive, and we also know that $|b|<|c|$ by the original signed consecutive pattern. Note that by Lemmas~\ref{lem:ts},~\ref{lem:endswiths1} and~\ref{lem:prodofcommA} and Corollaries~\ref{lem:endswithts},~and\ref{lem:beginswiths1} all of these patterns imply that $w$ has a reduced expression that begins or ends with a product of non-commuting generators. Thus, $w$ has Property T.
	
	Case (3): Suppose that $w$ has the signed permutation notation $w=[\ast, \ldots, \ast, \underline{a},b,c]$ where $\underline{a}bc$ corresponds to the signed consecutive pattern $\underline{1}23$, and $\ast$ indicates unknown values for $w(i)$ for $i=1,2, \ldots ,n-3$. We now consider the possible signed consecutive patterns $\ast \underline{a} b$. The possibilities are: $3 \underline{1} 2$, $\underline{3} \underline{1} 2$, $2 \underline{1} 3$, $\underline{2} \underline{1} 3$, $1 \underline{2} 3$, or $\underline{1} \underline{2} 3$. We know that $a$ must be negative, $b$ must be positive and $|a|<|b|$ by the original signed permutation. Note that by Lemmas~\ref{lem:st},~\ref{lem:endswiths0}, and~\ref{lem:endswiths_1} and Corollaries~\ref{lem:endswithst},~\ref{lem:beginswiths0} and~\ref{lem:beginswiths1}, all of these patterns imply that $w$ has a reduced expression that begins or ends with a product of non-commuting generators. Thus, $w$ has Property T. 
	
	Case (4): Suppose that $w$ has the signed permutation notation $w=[\ast, \ldots, \ast, \underline{a},b,c, \ast, \ldots, \ast]$ where $\underline{a}bc$ corresponds to the signed consecutive pattern $\underline{1}23$, and $\ast$ indicates unknown values for $w(i)$ for $|w(i)|\neq a,b,c$. In this case we can apply either Case (2) or Case (3) and we can conclude that $w$ has a reduced expression that begins or ends with a product of non-commuting generators. Thus, $w$ has Property T.

	Hence, if $w \in W(B_n)$ contains the consecutive pattern $\underline{1}23$, then $w$ is not $\tII$.
\end{proof}	
\end{lemma}

\begin{lemma}\label{lem:132}
Let $w \in W(B_n)$ such that $w$ contains the consecutive pattern $\underline{1}32$. Then either $w$ is not $\tII$.
\begin{proof}
	Let $w \in W(B_n)$ such that $w$ contains the consecutive pattern $\underline{1}32$.
	
	Case (1): Suppose $w$ has the signed permutation notation $w=[\underline{1}32]$. This implies that $w=s_0s_2$. Clearly, $w$ is a $\tI$ element as it is a product of commuting generators.
	
	Case (2): Suppose that $w$ has the signed permutation notation $w=[\underline{a},b,c, \ast, \ldots, \ast]$ where $\underline{a}bc$ corresponds to the signed consecutive pattern $\underline{1}32$, and $\ast$ indicates unknown values for $w(i)$ for $i=4,5, \ldots n$. We now consider the possible signed consecutive pattern $bc \ast$. The possibilities are: $231$, $23 \underline{1}$, $13 2$, $13 \underline{2}$, $123$, or $12\underline{3}$. We know that $b$ and $c$ are positive, and we also know that $|b|<|c|$ by the original signed consecutive pattern. Note that by Lemmas~\ref{lem:sts},~\ref{lem:st}, and~\ref{lem:prodofcommA} and Corollaries~\ref{lem:endswithsts} and~\ref{lem:endswithst} all of these patterns imply that $w$ has a reduced expression that begins or ends with a product of non-commuting generators. Thus, $w$ has Property T.
	
	Case (3): Suppose that $w$ has the signed permutation notation $w=[\ast, \ldots, \ast, \underline{a},b,c]$ where $\underline{a}bc$ corresponds to the signed consecutive pattern $\underline{1}32$, and $\ast$ indicates unknown values for $w(i)$ for $i=1,2, \ldots ,n-3$. We now consider the possible signed consecutive pattern $\ast \underline{a} b$. The possibilities are: $3 \underline{1} 2$, $\underline{3} \underline{1} 2$, $2 \underline{1} 3$, $\underline{2} \underline{1} 3$, $3 \underline{2} 1$, or $\underline{3} \underline{2} 1$. We know that $a$ must be negative, $b$ must be positive and $|a|<|b|$ by the original signed permutation. Note that by Lemmas~\ref{lem:st},~\ref{lem:endswiths0}, and~\ref{lem:endswiths_1} and Corollaries~\ref{lem:endswithst},~\ref{lem:beginswiths0} and~\ref{lem:beginswiths1} all of these patterns imply that $w$ has a reduced expression that begins or ends with a product of non-commuting generators. Thus, $w$ has Property T. 
	
	Case (4): Suppose that $w$ has the signed permutation notation $w=[\ast, \ldots, \ast, \underline{a},b,c, \ast, \ldots, \ast]$ where $\underline{a}bc$ corresponds to the signed consecutive pattern $\underline{1}32$, and $\ast$ indicates unknown values for $w(i)$ for $w(i)\neq a,b,c$. In this case we can apply either Case (2) or Case (3) and we can conclude that $w$ has a reduced expression that begins or ends with a product of non-commuting generators. Thus, $w$ has Property T.
	
	Hence, if $w \in W(B_n)$ contains the consecutive pattern $\underline{1}32$, then $w$ is not $\tII$.
\end{proof}	
\end{lemma}

We now are ready to tackle one of the main results of this thesis.

\begin{theorem}\label{thm:classificationofB}
There are no $\tII$ elements in $W(B_n)$.	


\begin{proof}
We proceed by contradiction. Suppose that $w \in W(B_n)$ is a $\tII$ element. There are $2^3 \cdot 3!=48$ possible choices of signed consecutive patterns for $w(1)w(2)w(3)$ where $w=[w(1), w(2), w(3), \ast, \ldots, \ast]$. These 48 signed consecutive patterns are seen in the table below. We only consider these signed consecutive patterns in the first three entries of the signed permutation representation, as if we can eliminate all possibilities we have a contradiction to $w$ being a $\tII$ element.

\begin{center}
\begin{tabular}{|l|l|l|l|l|l|l|l|}
\hline
\cellcolor{blue!30}$123$ & \cellcolor{orange2!40}$\underline{1}23$ & \cellcolor{brown!50}$1\underline{2}3$ & \cellcolor{red!25}$12\underline{3}$ & \cellcolor{brown!50}$\underline{12}3$ & \cellcolor{red!25}$\underline{1}2\underline{3}$ & \cellcolor{turq!40}$1\underline{23}$ & \cellcolor{turq!40}$\underline{123}$ \\
\hline
\cellcolor{blue!30}$132$ & \cellcolor{orange2!40}$\underline{1}32$ & \cellcolor{brown!50}$1\underline{3}2$ & \cellcolor{red!25}$13\underline{2}$ & \cellcolor{brown!50}$\underline{13}2$ & \cellcolor{red!25}$\underline{1}3\underline{2}$ & \cellcolor{ggreen!50}$1\underline{32}$ & \cellcolor{ggreen!50}$\underline{132}$ \\
\hline
\cellcolor{blue!30}$213$ & \cellcolor{yellow!50}$\underline{2}13$ & \cellcolor{brown!50}$2\underline{1}3$ & \cellcolor{turq!40}$21\underline{3}$ & \cellcolor{yellow!50}$\underline{21}3$ & \cellcolor{red!25}$\underline{2}1\underline{3}$ & \cellcolor{turq!40}$2\underline{13}$ & \cellcolor{red!25}$\underline{213}$ \\
\hline
\cellcolor{red!25}$231$ & \cellcolor{purple2!50}$\underline{2}31$ & \cellcolor{ggreen!50}$2\underline{3}1$ & \cellcolor{red!25}$23\underline{1}$ & \cellcolor{brown!50}$\underline{23}1$ & \cellcolor{purple2!50}$\underline{2}3\underline{1}$ & \cellcolor{ggreen!50}$2\underline{31}$ & \cellcolor{brown!50}$\underline{231}$ \\
\hline
\cellcolor{ggreen!50}$312$ & \cellcolor{yellow!50}$\underline{3}12$ & \cellcolor{ggreen!50}$3\underline{1}2$ &\cellcolor{turq!40}$31\underline{2}$ & \cellcolor{yellow!50}$\underline{31}2$ & \cellcolor{yellow!50}$\underline{3}1\underline{2}$ & \cellcolor{turq!40}$3\underline{12}$ & \cellcolor{yellow!50}$\underline{312}$ \\
\hline
\cellcolor{turq!40}$321$ & \cellcolor{yellow!50}$\underline{3}21$ & \cellcolor{ggreen!50}$3\underline{2}1$ & \cellcolor{turq!40}$32\underline{1}$ & \cellcolor{yellow!50}$\underline{32}1$ & \cellcolor{yellow!50}$\underline{3}2\underline{1}$ & \cellcolor{ggreen!50}$3\underline{21}$ & \cellcolor{yellow!50}$\underline{321}$\\
\hline
\end{tabular}
\end{center}

We can use Lemma~\ref{lem:sts} and Corollary~\ref{lem:endswithsts} to eliminate the signed consecutive patterns highlighted in \textcolor{turq}{turquoise}. In addition, using Lemma~\ref{lem:st} and Corollary~\ref{lem:endswithst} to eliminate the signed consecutive patterns highlighted in \textcolor{red}{red}. From Lemma~\ref{lem:ts} and Corollary~\ref{lem:endswithts} we eliminate the consecutive patterns highlighted in \textcolor{ggreen}{green}.  Using Lemma~\ref{lem:endswiths0} and Corollary~\ref{lem:beginswiths0} we are able to eliminate the signed consecutive patterns highlighted in \textcolor{yellow}{yellow}. Also Lemma~\ref{lem:endswiths_1} and Corollary~\ref{lem:beginswiths1} show that $w$ will not have the signed consecutive patterns highlighted in \textcolor{brown}{brown}. We also use Lemma~\ref{lem:prodofcommA} to eliminate the signed consecutive patterns highlighted in \textcolor{blue}{blue}. From Lemmas~\ref{lem:231} and~\ref{lem:2312} we are able to eliminate signed consecutive patterns highlighted in \textcolor{purple}{purple}. Finally, we can use Lemmas~\ref{lem:123} and~\ref{lem:132} to eliminate signed consecutive patterns highlighted in \textcolor{orange2}{orange}. Since all of the above patterns are eliminated as possibilities for $w(1)w(2)w(3)$ and there are no other signed consecutive patterns that are possible for these positions, and hence $w$ is not a $\tII$ element in the Coxeter group of type $B_n$.
\end{proof}
\end{theorem}


The upshot of Theorem~\ref{thm:classificationofB} is that the only T-avoiding elements in Coxeter systems of type $B_n$ are products of commuting generators and the identity.