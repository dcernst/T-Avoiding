\documentclass[9pt]{beamer}

\usepackage{amsfonts}
\usepackage{subfigure}
\usepackage{spot}
\usepackage{amsmath}
\usepackage{latexsym}
\usepackage{amssymb}
\usepackage{graphicx}
\usepackage{pgf,tikz}
\usepackage{pgfplots}
\usetikzlibrary{spy}
\usetikzlibrary{backgrounds}
\usetikzlibrary{decorations}
\usepackage{pstricks}
\usetikzlibrary{decorations.markings}
\usetikzlibrary{arrows,shapes,positioning}
\usetikzlibrary{calc}
\usepackage{marvosym}
\usepackage{color}
\definecolor{darkblue}{rgb}{0, 0, .6}
\definecolor{midnightblue}{rgb}{.2, .2, .7}
\definecolor{grey}{rgb}{.7, .7, .7}
\hypersetup{
	colorlinks=true, 
	linkcolor=midnightblue,
	anchorcolor=midnightblue,
	citecolor=midnightblue,
	pagecolor=midnightblue,
	urlcolor=midnightblue,
	pdftitle={},
	pdfauthor={}
}

\newcommand{\w}{\overline{w}}
\newcommand{\sigmabar}{\overline{\sigma}}

\mode<presentation>

%\usetheme{Pittsburgh}
\usetheme{Warsaw}

\usecolortheme{rose}
\usecolortheme{seahorse}
%\usecolortheme{beaver}
%\usecolortheme{beetle}

\useoutertheme{split}

\setbeamertemplate{footline}[split, frame number]
\setbeamertemplate{enumerate items}[default]
\setbeamertemplate{itemize items}[circle]

\setbeamersize{text margin left=6mm}
\setbeamersize{text margin right=6mm}
\setbeamersize{sidebar width right=0mm}
\setbeamersize{sidebar width left=0mm}
\setbeamertemplate{navigation symbols}{}

\newtheorem{comments}{Comments}
\newtheorem{question}{Question}
\newtheorem{goal}{Goal}
\newtheorem{remark}{Remark}
\newtheorem{proposition}{Proposition}
\newtheorem{conjecture}{Conjecture}

\newcommand*\oldmacro{}
\let\oldmacro\insertshorttitle
\renewcommand*\insertshorttitle{
\oldmacro\hfill
\insertframenumber\,/\,\inserttotalframenumber}



\begin{document}

\setspotlightstyle{rectangle, rounded corners,fill=blue!40}

\title[Classification of the T-avoiding permutations]{\textbf{Classification of the T-avoiding permutations and generalizations to other Coxeter groups}}
\author[Cormier, Ernst, Goldenberg, Kelly, Malbon]{\spot{J.~Cormier}\,, \spot{Z.~Goldenberg}\,, J.~Kelly, C.~Malbon\\
Directed by D.C.~Ernst}
\institute[PSU]{Plymouth State University\\
Mathematics Department}

\date[April 10, 2011]{{\bf 2011 AMS Spring Eastern Sectional Meeting}\\
Combinatorics of Coxeter Groups Special Session\\
College of the Holy Cross, April 9--10, 2011}

\frame{\titlepage}

%% ----------------------------------------------------------------------

\begin{frame}{The symmetric group} 

\begin{definition}
The \spot{symmetric group} $S_{n}$ is the collection of bijections from $\{1,2, \ldots, n\}$ to $\{1, 2, \dots, n\}$ where the operation is function composition (left $\leftarrow$ right).  Each element of $S_{n}$ is called a \spot{permutation}\,.
\end{definition}

\begin{block}{Comment}
We can think of $S_{n}$ as the group that acts by rearranging $n$ coins.
\end{block}

\medskip

One way of representing permutations is via \spot{cycle notation}\,, which we will illustrate by way of example.

\begin{example}
Consider $\sigma=(1\ 3\ 5\ 2)(4\ 6)$.  This means $\sigma(1)=3$, $\sigma(3)=5$, $\sigma(5)=2$, $\sigma(2)=1$, $\sigma(4)=6$, and $\sigma(6)=4$.
\end{example}

\end{frame}

%% ----------------------------------------------------------------------

\begin{frame}{String diagrams}

A second way of representing permutations is via \spot{string diagrams}\,, which we again introduce by way of example.

\begin{example}
Consider $\sigma=(1\ 3\ 5\ 2)(4\ 6)$ from previous example. 

\begin{figure}
\begin{tikzpicture}[scale=.5]
\draw[fill=black] \foreach \y in {0,1,2,3,4,5} {(0,\y) circle (2pt)};
\draw[fill=black] \foreach \y in {0,1,2,3,4,5} {(6,\y) circle (2pt)};
\node at (0,0){%
\psset{unit=0.5cm}
\pscurve{-}(0,1)(1.5,1.5)(3,2.5)(6,3)
};
\node at (0,0){%
\psset{unit=0.5cm}
\pscurve{-}(0,0)(3,.5)(4.5,1.5)(6,2)
};
\node at (0,0){%
\psset{unit=0.5cm}
\pscurve{-}(0,2)(1.5,1.5)(3,.5)(6,0)
};
\node at (0,0){%
\psset{unit=0.5cm}
\pscurve{-}(0,4)(1.5,3.5)(3,2.5)(4.5,1.5)(6,1)
};
\node at (0,0){%
\psset{unit=0.5cm}
\pscurve{-}(0,5)(3,4.5)(6,4)
};
\node at (0,0){%
\psset{unit=0.5cm}
\pscurve{-}(0,3)(1.5,3.5)(3,4.5)(6,5)
};
\end{tikzpicture}
\end{figure}
\end{example}

%\vspace{-1em}

\begin{block}{Comment}
Given a permutation $\sigma$, there are many ways to draw the associated string diagram.  However, we adopt the following conventions:
\begin{enumerate}
\item no more than two strings cross each other at a given point,
\item strings are drawn so as to minimize crossings.
\end{enumerate}
\end{block}

%\begin{theorem}
%Strings cross at most once.
%\end{theorem}

\end{frame}

%% ----------------------------------------------------------------------

\begin{frame}{Generators and relations for $S_{n}$}

$S_{n}$ is generated by the \spot{adjacent 2-cycles}\,:
\[
(1\ 2), (2\ 3), \dots, (n-1\ n).
\]
That is, every element of $S_{n}$ can be written as a product of the adjacent 2-cycles.

\bigskip

Define
\[
\spot{s_{i}=(i\ i+1)},
\]
so that $s_{1}, s_{2}, \dots, s_{n-1}$ generate $S_{n}$.

\begin{block}{Comments}
$S_{n}$ satisfies the following relations:
\begin{enumerate}

\item $s_{i}^{2}=1$ for all $i$  (2-cycles have order 2)

\item \spot{short braid relations}\,: $s_{i}s_{j}=s_{j}s_{i}$, for $|i-j|\geq 2$ (disjoint cycles commute)

\item \spot{long braid relation}\,: $s_{i}s_{j}s_{i}=s_{j}s_{i}s_{j}$, for $|i-j|=1$.

\end{enumerate}

\end{block}

\end{frame}

%% ----------------------------------------------------------------------

\begin{frame}{Reduced expressions \& Matsumoto's theorem}

\begin{block}{Definition}
If $s_{x_1}s_{x_2}\cdots s_{x_m}$ is an expression for $\sigma\in S_{n}$ and $m$ is minimal, then we say that the expression is \spot{reduced}\,.  %If we wish to emphasize a fixed, possibly reduced, expression for $\sigma\in S_{n}$, we represent it as
%\[
%\spot{\sigmabar=s_{x_1}s_{x_2}\cdots s_{x_m}}.
%\]
\end{block}

%\vspace{-1em}

\begin{example}
Consider $\sigma=s_{2}s_{1}s_{2}s_{3}s_{1}s_{2}\in S_{4}$.  We see that
\[
\spot(first){s_{2}s_{1}s_{2}}\,s_{3}s_{1}s_{2} =  \spot(second){s_{1}s_{2}s_{1}}\,s_{3}s_{1}s_{2}= s_{1}s_{2}s_{1}\,\spot(third){s_{3}s_{1}}\,s_{2}= s_{1}s_{2}s_{1}\,\spot(fourth){s_{1}s_{3}}\,s_{2}=s_{1}s_{2}\,\spot(fifth){s_{1}s_{1}}\,s_{3}s_{2}=s_{1}s_{2}s_{3}s_{2}.\tikz[remember picture, overlay]{ \draw (first) to[bend right,<->, red,thick] (second);\draw (third) to[bend right,<->, red,thick] (fourth);}
\]
So, the original expression was \emph{not} reduced, but it turns out that the last expression on the right is reduced.
\end{example}

\begin{theorem}[Matsumoto]
Any two reduced expressions for $\sigma\in S_{n}$ differ by a sequence of braid relations.
\end{theorem}

\begin{example}[continued]
The only reduced expressions for $\sigma$ are: $s_{1}s_{2}s_{3}s_{2}$, $s_{1}s_{3}s_{2}s_{3}$, and $s_{3}s_{1}s_{2}s_{3}$.
\end{example}

\end{frame}

%% ----------------------------------------------------------------------

%\begin{frame}{Fully commutative elements}
%
%\begin{definition}
%Let $\sigma=s_{i_{1}}\dots s_{i_{r}} \in S_{n}$ be a reduced expression.  We say that $\sigma$ is \spot{fully commutative} (\spot{FC}) if any two reduced expressions for $\sigma$ may be obtained from each other by iterated commutation. 
%\end{definition}
%
%\begin{theorem}[Stembridge]
%Equivalently (but not immediately obvious), $\sigma$ is FC iff it has no reduced expression containing $s_{i}s_{j}s_{i}$ for $|i-j|=1$ (i.e., there are no opportunities to apply a braid relation).
%\end{theorem}
%
%\begin{example}
%The element $s_{1}s_{2}s_{3}$ is FC. However, $\spot{s_{3}s_{2}s_{3}}s_{1}$ is \spot{not} FC because we have an opportunity to apply a braid relation.
%\end{block}
%
%\end{frame}

%% ----------------------------------------------------------------------

\begin{frame}{Heaps}

A third way of representing permutations is via \spot{heaps}\,.  Fix a reduced expression $s_{x_1}s_{x_2}\cdots s_{x_m}$ for $\sigma\in S_{n}$.  Loosely speaking, the heap for this expression is a set of lattice points (called \spot{nodes}\,), one for each $s_{x_{i}}$, embedded in $\mathbb{N}\times \mathbb{N}$ such that:

\begin{itemize}
\item The node corresponding to $s_{x_{i}}$ has vertical component equal to $n+1-x_{i}$ (smaller numbers at the top), 
\item If $i<j$ and $s_{x_{i}}$ does not commute with $s_{x_{j}}$, then $s_{x_{i}}$ occurs to the left of $s_{x_{j}}$.
\end{itemize}

\begin{example}
Consider \spot[fill=purple]{$s_{1}s_{2}s_{3}s_{2}$}\,, \spot{$s_{1}s_{3}s_{2}s_{3}$}\,, and \spot{$s_{3}s_{1}s_{2}s_{3}$} from the previous example.  It turns out, there are two distinct heaps.
\begin{columns}
\begin{column}{.4\linewidth}
\begin{center}
\begin{tikzpicture}[scale=.5]
%\draw[fill=black] \foreach \y in {1,2,3,4,} {(-1,\y) circle (2pt)};
%\draw[fill=black] \foreach \y in {1,2,3,4,} {(6,\y) circle (2pt)};
\node[left] at (3, 1.5){$s_3$};
\node[left] at (1.5,2.5){$s_2$};
\node[left] at (0,3.5){$s_1$};
\node[left] at (4.5,2.5){$s_2$};
\filldraw[purple] (3,1.5) circle (5pt);
\filldraw[purple] (1.5,2.5) circle (5pt);
\filldraw[purple] (0,3.5) circle (5pt);
\filldraw[purple](4.5,2.5)circle(5pt);
\end{tikzpicture}
\end{center}
\end{column}

\begin{column}{.1\linewidth}
\begin{center}
and
\end{center}
\end{column}

\begin{column}{.4\linewidth}
\begin{center}
\begin{tikzpicture}[scale=.5]
%\draw[fill=black] \foreach \y in {1,2,3,4,} {(-1,\y) circle (2pt)};
%\draw[fill=black] \foreach \y in {1,2,3,4,} {(6,\y) circle (2pt)};
\node[left] at (3, 1.5){$s_3$};
\node[left] at (1.5,2.5){$s_2$};
\node[left] at (0,3.5){$s_1$};
\node[left] at (0,1.5){$s_3$};
\filldraw[blue!40] (3,1.5) circle (5pt);
\filldraw[blue!40] (1.5,2.5) circle (5pt);
\filldraw[blue!40] (0,3.5) circle (5pt);
\filldraw[blue!40](0,1.5)circle(5pt);
\end{tikzpicture}
\end{center}
\end{column}
\end{columns}

\end{example}
\begin{block}{Comment}
If two reduced expressions for $\sigma$ differ by a sequence of short braid relations, then they have the same heap.  In particular, if no reduced expression contains an opportunity to provide a long braid relation, then $\sigma$ has a unique heap.
\end{block}

\end{frame}

%% ----------------------------------------------------------------------

\begin{frame}{Combining string diagrams and heaps}

The points at which two strings cross correspond to nodes in the heap. Hence, we may overlay strings on top of a heap by drawing the strings from right to left so that they cross at each entry in the heap where they meet and bounce at each lattice point not in the heap. 

\vspace{1em}

Conversely, each string diagram corresponds to a heap by taking all of the points where the strings cross as the nodes of the heap.


\begin{example}
\begin{figure}
\begin{tikzpicture}[scale=.5]
\draw[fill=black] \foreach \y in {1,2,3,4,5} {(0,\y) circle (2pt)};
\draw[fill=black] \foreach \y in {1,2,3,4,5} {(6,\y) circle (2pt)};
\node at (0,0){%
\psset{unit=0.5cm}
\pscurve{-}(0,1)(1.5,1.5)(3,2.5)(6,3)
};

%\node at (0,0){%
%\psset{unit=0.5cm}
%\pscurve{-}(0,0)(3,.5)(4.5,1.5)(6,2)
%};

\node at (0,0){%
\psset{unit=0.5cm}
\pscurve{-}(0,2)(1.5,1.5)(3,.75)(4.5,1.5)(6,2)
};

\node at (0,0){%
\psset{unit=0.5cm}
\pscurve{-}(0,4)(1.5,3.5)(3,2.5)(4.5,1.5)(6,1)
};

\node at (0,0){%
\psset{unit=0.5cm}
\pscurve{-}(0,5)(3,4.5)(6,4)
};

\node at (0,0){%
\psset{unit=0.5cm}
\pscurve{-}(0,3)(1.5,3.5)(3,4.5)(6,5)
};
\filldraw[purple] (4.5,1.5) circle (5pt);
%\filldraw[white] (3,0.5) circle (5pt);
\filldraw[purple] (1.5,1.5) circle (5pt);
\filldraw[purple] (3,2.5) circle (5pt);
\filldraw[purple] (3,4.5) circle (5pt);
\filldraw[purple] (1.5,3.5) circle (5pt);
\end{tikzpicture}
\end{figure}
\end{example}

\end{frame}

%% ----------------------------------------------------------------------
\begin{frame}{Property T}

\begin{definition}
We say that a permutation $\sigma$ has \spot{Property T} iff there exists $i$ such that either

\begin{columns}
\begin{column}{0.4\textwidth}
  
\begin{enumerate}
\item $\sigma(i)>\sigma(i+1),\sigma(i+2)$,

\vspace{.5em}

\begin{tikzpicture}[scale=.5]
\foreach \y in {0,1,2}\draw[fill=black] (6,\y) circle (2pt);
\node[right] at (6,2) {$i$};
\node[right] at (6,1) {$i+1$};
\node[right] at (6,0) {$i+2$};
\node at (0,0){%
\psset{unit=0.5cm}
\pscurve{<-}(0,0)(2,.5)(4,1.5)(6, 2)
};
\node at (0,0){%
\psset{unit=0.5cm}
\pscurve{<-}(0,1)(2,.5)(6,0)
};
\node at (0,0){%
\psset{unit=0.5cm}
\pscurve{<-}(0,2)(4, 1.5)(6,1)
};
\end{tikzpicture}
\end{enumerate}
\end{column}

\begin{column}{.1\linewidth}
\begin{center}
or
\end{center}
\end{column}

\begin{column}{0.4\linewidth}
\begin{enumerate}
\item[2.] $\sigma(i+2)<\sigma(i), \sigma(i+1)$.

\vspace{.5em}

\begin{tikzpicture}[scale=.5]
\draw[fill=black]\foreach \y in {0,1,2}{(6,\y) circle (2pt)};
\node[right] at (6,2) {$i$};
\node[right] at (6,1) {$i+1$};
\node[right] at (6,0) {$i+2$};
\node at (0,0){%
\psset{unit=0.5cm}
\pscurve{<-}(0,2)(2,1.5)(4,.5)(6, 0)
};
\node at (0,0){%
\psset{unit=0.5cm}
\pscurve{<-}(0,1)(2,1.5)(6,2)
};
\node at (0,0){%
\psset{unit=0.5cm}
\pscurve{<-}(0,0)(4,.5)(6,1)
};
\end{tikzpicture}
\end{enumerate}
\end{column}

\end{columns}
\end{definition}

\begin{example}
Consider the following permutation $\sigma=(1\ 3\ 5\ 2)(4\ 6)$.

\begin{figure}
\begin{tikzpicture}[scale=.5,decoration={brace,amplitude=3pt}]
\draw[fill=black] \foreach \y in {0,1,2,3,4,5} {(0,\y) circle (2pt)};
\draw[fill=black] \foreach \y in {0,1,2,3,4,5} {(6,\y) circle (2pt)};
\node at (0,0){%
\psset{unit=0.5cm}
\pscurve{-}(0,1)(1.5,1.5)(3,2.5)(6,3)
};
\node at (0,0){%
\psset{unit=0.5cm}
\pscurve{-}(0,0)(3,.5)(4.5,1.5)(6,2)
};
\node at (0,0){%
\psset{unit=0.5cm}
\pscurve{-}(0,2)(1.5,1.5)(3,.5)(6,0)
};
\node at (0,0){%
\psset{unit=0.5cm}
\pscurve{-}(0,4)(1.5,3.5)(3,2.5)(4.5,1.5)(6,1)
};
\node at (0,0){%
\psset{unit=0.5cm}
\pscurve{-}(0,5)(3,4.5)(6,4)
};
\node at (0,0){%
\psset{unit=0.5cm}
\pscurve{-}(0,3)(1.5,3.5)(3,4.5)(6,5)
};
\draw [decorate,color= red,thick] (6.25,2.10) -- (6.25,-0.10);
\draw [decorate,color= blue,thick] (6.75,3.10) -- (6.75,0.90);
\draw [decorate,color= yellow,thick] (-.25,-.10)--(-.25,2.10);
\draw [decorate,color= orange,thick] (-.25,2.90)--(-.25,5.10);
\end{tikzpicture}
\end{figure}
\vspace{-.5em}
We see that $\sigma$ and $\sigma^{-1}$ each have Property T in two spots.

\end{example}

\end{frame}

%% ----------------------------------------------------------------------

\begin{frame}{T-avoiding}

\begin{columns}
\begin{column}{.47\linewidth}
\begin{definition}
We say that a permutation $\sigma$ is \spot{T-avoiding} iff neither $\sigma$ or $\sigma^{-1}$ has Property T.
\end{definition}

\begin{theorem}[CEGKM]
A permutation $\sigma$ is T-avoiding iff $\sigma$ is a product of disjoint adjacent 2-cycles.
\end{theorem}

\end{column}

\begin{column}{.47\linewidth}

\begin{example}
The permutation $\sigma=(2\ 3)(5\ 6)$ is T-avoiding.
\begin{figure}
\begin{tikzpicture}[scale=.5]
\draw[fill=black] \foreach \y in {0,1,2,3,4,5} {(0,\y) circle (2pt)};
\draw[fill=black] \foreach \y in {0,1,2,3,4,5} {(3,\y) circle (2pt)};
\node at (0,0){%
\psset{unit=0.5cm}
\pscurve{-}(0,5)(1.5,5)(3,5)
};
\node at (0,0){%
\psset{unit=0.5cm}
\pscurve{-}(0,4)(1.5,3.5)(3,3)
};
\node at (0,0){%
\psset{unit=0.5cm}
\pscurve{-}(0,3)(1.5,3.5)(3,4)
};
\node at (0,0){%
\psset{unit=0.5cm}
\pscurve{-}(0,2)(1.5,2)(3,2)
};
\node at (0,0){%
\psset{unit=0.5cm}
\pscurve{-}(0,1)(1.5,.5)(3,0)
};
\node at (0,0){%
\psset{unit=0.5cm}
\pscurve{-}(0,0)(1.5,.5)(3,1)
};
\end{tikzpicture}
\end{figure}
\end{example}

\end{column}

\end{columns}

\begin{block}{Sketch of proof}
Fix a reduced expression for $\sigma$, say $s_{x_1}s_{x_2}\cdots s_{x_m}$, and consider the heap for this reduced expression.  The reverse implication of the theorem is trivial.  For the forward direction, consider the contrapositive: 

\vspace{1em}

If $\sigma$ is not a product of disjoint adjacent 2-cycles, then $\sigma$ or $\sigma^{-1}$ has Property T.
\end{block}

\end{frame}

%% ----------------------------------------------------------------------

\begin{frame}{Sketch of proof (continued)}

%\begin{block}{Note}
%The easy case occurs when a node in the second column on either side is "blocked" by at most one node in the first column.
%\end{block}


\begin{block}{The easy case}


The easy case occurs when a node in the second column on either side is "blocked" by at most one node in the first column.



\begin{columns}
\begin{column}{0.4\textwidth}
  
\begin{enumerate}
\item $\sigma$ has the property:

\vspace{1em}

\begin{tikzpicture}[scale=.5]
\foreach \y in {0,1,2}\draw[fill=black] (6,\y) circle (2pt);
\node at (0,0){%
\psset{unit=0.5cm}
\pscurve{<-}(0,0)(2,.5)(4,1.5)(6, 2)
};
\node at (0,0){%
\psset{unit=0.5cm}
\pscurve{<-}(0,1)(2,.5)(6,0)
};
\node at (0,0){%
\psset{unit=0.5cm}
\pscurve{<-}(0,2)(4, 1.5)(6,1)
};
\filldraw[purple] (2,.5) circle (5pt);
\filldraw[purple] (4,1.5) circle (5pt);
\end{tikzpicture}

\vspace{3em}

\begin{tikzpicture}[scale=.5]
\draw[fill=black]\foreach \y in {0,1,2}{(6,\y) circle (2pt)};
\node at (0,0){%
\psset{unit=0.5cm}
\pscurve{<-}(0,2)(2,1.5)(4,.5)(6, 0)
};
\node at (0,0){%
\psset{unit=0.5cm}
\pscurve{<-}(0,1)(2,1.5)(6,2)
};
\node at (0,0){%
\psset{unit=0.5cm}
\pscurve{<-}(0,0)(4,.5)(6,1)
};
\filldraw[purple] (2,1.5) circle (5pt);
\filldraw[purple] (4,.5) circle (5pt);
\end{tikzpicture}
\end{enumerate}
\end{column}

\begin{column}{.1\linewidth}
\begin{center}
or
\end{center}
\end{column}

\begin{column}{0.4\linewidth}
\begin{enumerate}
\item[2.] $\sigma^{-1}$ has the property:

\vspace{1em}

\begin{tikzpicture}[scale=.5]
\foreach \y in {0,1,2}\draw[fill=black] (0,\y) circle (2pt);
\node at (0,0){%
\psset{unit=0.5cm}
\pscurve{->}(0,0)(2,.5)(4,1.5)(6, 2)
};
\node at (0,0){%
\psset{unit=0.5cm}
\pscurve{->}(0,1)(2,.5)(6,0)
};
\node at (0,0){%
\psset{unit=0.5cm}
\pscurve{->}(0,2)(4, 1.5)(6,1)
};
\filldraw[purple] (2,.5) circle (5pt);
\filldraw[purple] (4,1.5) circle (5pt);
\end{tikzpicture}

\vspace{3em}

\begin{tikzpicture}[scale=.5]
\draw[fill=black]\foreach \y in {0,1,2}{(0,\y) circle (2pt)};
\node at (0,0){%
\psset{unit=0.5cm}
\pscurve{->}(0,2)(2,1.5)(4,.5)(6, 0)
};
\node at (0,0){%
\psset{unit=0.5cm}
\pscurve{->}(0,1)(2,1.5)(6,2)
};
\node at (0,0){%
\psset{unit=0.5cm}
\pscurve{->}(0,0)(4,.5)(6,1)
};
\filldraw[purple] (2,1.5) circle (5pt);
\filldraw[purple] (4,.5) circle (5pt);
\end{tikzpicture}

\end{enumerate}
\end{column}

\end{columns}


\end{block}

\end{frame}

%% ----------------------------------------------------------------------

\begin{frame}{Sketch of proof (continued)}

\begin{block}{The hard case}
\begin{figure}
\begin{tikzpicture}[scale=.25]
\draw[fill=black] \foreach \y in {11,13,15} {(16,\y) circle (2pt)};
\node[right] at (16,15) {$i$};
\node[right] at (16,13) {$i+1$};
\node[right] at (16,11) {$i+2$};
\node at (0,0){%
\psset{unit=0.25cm}
\pscurve{-}(16,15)(14,14)(12,12)(10,10)
};
\node at (0,0){%
\psset{unit=0.25cm}
\pscurve{-}(16,11)(14,10)(13,9)
};
\node at (0,0){%
\psset{unit=0.25cm}
\pscurve{->}(16,13)(14,14)(12,16)
};
\node at (0,0){%
\psset{unit=0.25cm}
\pscurve{->}(9,5)(8,4)(6,2)(4,0.5)(2,2)(0,4)
};
\node at (0,0){%
\psset{unit=0.25cm}
\pscurve{->}(6,6)(4,4)(2,2)(0,0)
};
\filldraw[purple](2,2)circle(7.5pt);
\filldraw[purple](4,4)circle(7.5pt);
\filldraw[purple](6,2)circle(7.5pt);
\filldraw[purple](8,4)circle(7.pt);
\filldraw[black](7,7)circle(3pt);
\filldraw[black](8,8)circle(3pt);
\filldraw[black](9,9)circle(3pt);
\filldraw[black](10,6)circle(3pt);
\filldraw[black](11,7)circle(3pt);
\filldraw[black](12,8)circle(3pt);
\filldraw[purple](12,12)circle(7.5pt);
\filldraw[purple](14,10)circle(7.5pt);
\filldraw[purple](14,14)circle(7.5pt);
\end{tikzpicture}
\end{figure}\end{block}

\begin{theorem}
By applying a sequence of long braid relations, you can convert a heap in the hard case to a heap in the easy case.
\end{theorem}

The previous theorem provides the necessary motivation for generalizing the definition of Property T in other Coxeter groups.

\end{frame}

%% ----------------------------------------------------------------------

\begin{frame}{Coxeter groups}

\begin{definition}
A \spot{Coxeter group} consists of a  group $W$  together with a generating set $S$ consisting of elements of order 2  with presentation 
\[
W = \langle S : s^{2}=1, (st)^{m(s, t)} = 1 \rangle,
\] 
 where $m(s, t) \geq 2$ for $s\neq t$. 

\end{definition}

\begin{block}{Comment}

Since $s$ and $t$ are elements of order 2, the relation $(st)^{m(s,t)}=1$ can be rewritten as 

\begin{center}
\begin{tabular}{ll}
$\left.\begin{array}{lcc}m(s,t)=2 & \implies  &\ \ \, st=ts\ \   \end{array}  \right\}$&  \spot{short braid relations}\\
\\ 
$\left.\begin{array}{lcc}m(s,t)=3 & \implies & sts=tst \\
& & \\
m(s,t)=4 & \implies & stst=tsts \\
 & \vdots &  \end{array}  \right\}$ & \spot{long braid relations}
\end{tabular}
\end{center}
We can uniquely encode the generators and relations using a \spot{Coxeter graph}\,.

\end{block}

\end{frame}

%% ----------------------------------------------------------------------

\begin{frame}{Types $A$ and $B$}

\begin{example}[Type $A$]
The symmetric group $S_{n+1}$ with the adjacent 2-cycles as a generating set is a Coxeter group of type $A_{n}$.

\vspace{-.5em}
\begin{figure}
\begin{tikzpicture}
\draw[fill=black] \foreach \x in {1,2,3,4,5} {(\x,10) circle (1pt)};
\draw \foreach \x in {1,2,3} {(\x,10) node[label=below:$s_{\x}$]{}};
\draw {(4,10) node[label=below:$s_{n-1}$]{}};
\draw {(5,10) node[label=below:$s_{n}$]{}};
\draw {(3.5,10) node[]{$\cdots$}};
\draw[-] (1,10) -- (3,10);
\draw[-] (4,10) -- (5,10);
\end{tikzpicture}
\end{figure}
\end{example}

\begin{example}[Type $B$]
Coxeter groups of type $B_{n}$ ($n\geq 2$) are defined by:
\begin{figure}
\begin{tikzpicture}
\draw[fill=black] \foreach \x in {1,2,3,4,5} {(\x,10) circle (1pt)};
\draw \foreach \x in {1,2,3} {(\x,10) node[label=below:$s_{\x}$]{}};
\draw {(4,10) node[label=below:$s_{n-1}$]{}};
\draw {(5,10) node[label=below:$s_{n}$]{}};
\draw {(3.5,10) node[]{$\cdots$}};
\draw {(1.5,10) node[label=above:$4$]{}};
\draw[-] (1,10) -- (3,10);
\draw[-] (4,10) -- (5,10);
\end{tikzpicture}
\end{figure}

$W(B_{n})$ is generated by $S(B_{n})=\{s_{1}, s_{2}, 
\cdots, s_{n}\}$ and is subject to
\begin{enumerate}
\item $s_{i}^{2}=1$ for all $i$,
\item $s_{i}s_{j}=s_{j}s_{i}$ if $|i-j|>1$,
\item $s_{i}s_{j}s_{i}=s_{j}s_{i}s_{j}$ if $|i-j|=1$ and $1<i,j\leq n$, 
\item $s_{1}s_{2}s_{1}s_{2}=s_{2}s_{1}s_{2}s_{1}$.
\end{enumerate}
\end{example}

\end{frame}

%% ----------------------------------------------------------------------

\begin{frame}{Generalization of T-avoiding}

\begin{definition}
Let $(W,S)$ be a Coxeter group and let $w\in W$. Then $w$ has \spot{Property T} iff $w$ has a reduced expression of the form $stu$ or $uts$, where $m(s,t)\geq 3$ and $u\in W$.
\end{definition}

\begin{theorem}[CEGKM]
In type $A$ and $B$, $w\in W$ is T-avoiding iff $w$ is a product of commuting generators.
\end{theorem}

\begin{block}{Comment}
The answer isn't so simple in other Coxeter groups.
\begin{itemize}
\item We have also classified the T-avoiding elements in type affine $C$, which consists of more than just products of commuting generators.

\item Similarly, Tyson Gern has recently classified the T-avoiding elements in type $D$, and again, the classification is more complicated than just products of commuting generators.

\end{itemize}
\end{block}

\end{frame}

\end{document}