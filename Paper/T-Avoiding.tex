\documentclass[12pt]{amsart}                       

\usepackage{tikz}
\usetikzlibrary{3d}
\usepackage{amsmath,amssymb,amsfonts,amscd, array,amsthm}
\usepackage[margin=1in]{geometry}
\usepackage{caption}
\usepackage[labelformat=simple,labelfont={}]{subcaption}
	\renewcommand\thesubfigure{(\alph{subfigure})}
\usepackage[table]{colortbl}
\usepackage{mathdots}
\usepackage{setspace}
\usepackage{float}

\theoremstyle{definition}
\newtheorem{theorem}{Theorem}[section] 
\newtheorem{lemma}[theorem]{Lemma}
\newtheorem{proposition}[theorem]{Proposition}
\newtheorem{corollary}[theorem]{Corollary}
\newtheorem{conjecture}[theorem]{Conjecture}
\newtheorem{definition}[theorem]{Definition}
\newtheorem{example}[theorem]{Example}
\newtheorem{examples}[theorem]{Examples} 
\newtheorem{problem}[theorem]{Problem} 
\newtheorem{remark}[theorem]{Remark}

\renewcommand{\qed}{\hfill \mbox{$\Box$}}
\newcommand{\newold}[2]{{\bf new:}{\color{blue}#1}{\bf old:}{\color{red}#2}}%

%%%%%%%%%%%%%% Heap Code %%%%%%%%%%%%%%
\usetikzlibrary{patterns}

\newcommand\heapblock[4]{\fill[fill=#4, fill opacity=0.35, draw=#4, line width=1.1pt, rounded corners,shift={(\xxaxis:#1)},shift={(\yyaxis:#2)}] (-1,-1) rectangle (1,1);\node at (#1,#2) {\scriptsize $#3$};}

\newcommand\dheapblock[4]{\draw[dotted, draw=#4, line width=1.1pt, rounded corners,shift={(\xxaxis:#1)},shift={(\yyaxis:#2)}] (-1,-1) rectangle (1,1);\node at (#1,#2) {\footnotesize $#3$};}

\newcommand\sheapblock[4]{\draw[pattern= north west lines, pattern color=#4, draw=#4, line width=1.1pt, rounded corners,shift={(\xxaxis:#1)},shift={(\yyaxis:#2)}] (-1,-1) rectangle (1,1);\node at (#1,#2) {\scriptsize $#3$};}

\newcommand\xxaxis{0}
\newcommand\yyaxis{90}

\definecolor{orange}{RGB}{255,102,0}
\definecolor{ggreen}{RGB}{0,153,0}
\definecolor{darkblue}{RGB}{0,0,255}
\definecolor{purple}{RGB}{153,51,255}
\definecolor{turq}{RGB}{72,209,204}
\definecolor{gray}{RGB}{220,220,220}
\definecolor{orange2}{RGB}{255,100,0}
\definecolor{purple2}{RGB}{159,51,250}
\definecolor{rred}{rgb}{0.9, 0.17, 0.31}

%%%%%%%%%%%%%%% Extra Things To make Typing Easier %%%%%%%%%%%%%%

\newcommand{\supp}{\mathrm{supp}}
\newcommand{\FC}{\mathrm{FC}}
\newcommand{\Sym}{\mathrm{Sym}}
\newcommand{\w}{\mathsf{w}}
\newcommand{\C}{\widetilde{C}}
\newcommand{\sgn}{\mathrm{sgn}}
\newcommand{\RD}{\mathcal{R}}
\newcommand{\LD}{\mathcal{L}}

\usepackage{enumitem}
\setenumerate[1]{label={(\arabic*)}}
	\setenumerate{listparindent=\parindent}
		
\DeclareMathOperator{\tI}{T_1-\mathrm{avoiding}}
\DeclareMathOperator{\tII}{T_2-\mathrm{avoiding}}

\title{A Study of T-Avoiding Elements of Coxeter Groups}

\author{Dana C.~Ernst}%Swap order???
\author{Taryn M.~Laird}

\address{
Dana C.~Ernst and Taryn M.~Laird,
Department of Mathematics and Statistics,
Northern Arizona University PO Box 5717,
Flagstaff, AZ 86011-5717, USA
}
\email{Dana.Ernst@nau.edu, taryn.m.laird@gmail.com}

\subjclass[2000]{???}
\keywords{????}

\begin{document}

\maketitle

\begin{abstract}
Kazhdan--Lusztig polynomials arise in the context of Hecke algebras associated to Coxeter groups. The computation of these polynomials is very difficult for examples even in relatively small groups. Motivated by a desire to understand the Kazhdan--Lusztig theory of the Hecke algebra of the underlying Coxeter group, R.M. Green classified the so-called star reducible Coxeter groups, which have the property that all fully commutative elements (in the sense of Stembridge) can be sequentially reduced via star operations to a product of commuting generators. It turns out that in some Coxeter groups there are elements, called T-avoiding elements, which cannot be systematically dismantled in this way. More specifically an element $w$ is called T-avoiding if $w$ does not have a reduced expression beginning or ending with a pair of non-commuting generators. Clearly, a product of commuting generators is trivially T-avoiding. However, sometimes there are more interesting non-trivially T-avoiding elements. A natural question is \emph{which Coxeter groups have non-trivially T-avoiding elements?} Computation of Kazhdan--Lusztig polynomials involving the non-trivially T-avoiding elements is difficult in general. However, having an understanding of the T-avoiding elements provides valuable information about possible obstructions to determining the Kazhdan--Lusztig polynomials. In this thesis we  begin by summarizing the previously known results regarding T-avoiding elements in certain Coxeter groups and then classify the T-avoiding elements in Coxeter groups of types $B_n$ and $\C_n$. 
\end{abstract}

\section{Preliminaries}

%%%%%%%%%%%%%%%%%%%%%%

\subsection{Introduction}
In mathematics, one uses groups to study symmetry.  In particular, a reflection group can be used to study the reflection and rotational symmetry of an object.  A Coxeter group can be thought of as a generalized reflection group, where the group is generated by a set of elements of order two (i.e., reflections) and there are rules for how the generators interact with each other.  Every element of a Coxeter group can be written as an expression in the generators, and if the number of generators in an expression (including multiplicity) is minimal, we say that the expression is reduced. 

Kazhdan--Lusztig polynomials arise in the context of Hecke algebras associated to Coxeter groups. The computation of these polynomials is very difficult even for relatively small groups. Motivated by the desire to understand the Kazhdan--Lusztig theory of the Hecke algebra of the underlying Coxeter group, Green~\cite{Green2006a} classified the so-called star reducible Coxeter groups which have the property that all fully commutative elements (in the sense of Stembridge) can be sequentially reduced via star operations to a product of commuting generators. 

It turns out that in some Coxeter groups there are elements, called T-avoiding elements, which cannot be systematically dismantled in the way described above. More specifically an element $w$ is called \emph{T-avoiding} if $w$ does not have a reduced expression beginning or ending with a pair of non-commuting generators. Clearly, a product of commuting generators is trivially T-avoiding. However, sometimes there are more interesting T-avoiding elements, which we will refer to as type 2 T-avoiding elements. 

Our interest in T-avoiding elements is motivated by a desire to compute Kahzdan--Lusztig polynomials, denoted $P_{x,w}$, where $x$ and $w$ are elements of a fixed Coxeter group. A bound on the degree of $P_{x,w}$ is known, but in general it is not known when this bound is achieved. Of particular interest are the coefficients $\mu(x,w)$ that appear when the maximum degree of $P_{x,w}$ is attained. The polynomials $P_{x,w}$ and coefficients $\mu(x,w)$ are determined through recurrence relations, however no closed form is known for calculating either in an efficient matter. Calculations involving type 2 T-avoiding elements are generally more difficult as the descent sets of these elements have undesirable properties.   In addition, knowing which elements are T-avoiding often provides us with the base case for inductive arguments involving star operations.

In his PhD thesis~\cite{Gern2013a}, Gern classified the T-avoiding elements in Coxeter groups of type $D_n$. Unlike in types $A_n$ and $B_n$, it turns out that the classification in type $D_n$ includes type 2 T-avoiding elements. The T-avoiding elements are rich in combinatorics and are interesting in their own right. The focus of this thesis is classifying T-avoiding elements in certain Coxeter groups.

This thesis is organized as follows. After necessary background information is presented in Section~\ref{sec:coxeter}, we introduce the class of fully commutative elements in Section~\ref{sec:FC}. Next in Section~\ref{sec:Heaps} we discuss a visual representation for elements of Coxeter groups, called heaps. In Section~\ref{sec:star}, we introduce the concept of a star reduction and star reducible Coxeter groups and in Section~\ref{sec:Tavoid} we formally introduce the notion of a T-avoiding element. In Section~\ref{sec:noncancel} we define non-cancellable elements in Coxeter groups, as well as remark upon a specific family of non-cancellable elements in $W(\C_n)$ when $n$ is odd. We then state classifications and conjectures regarding T-avoiding elements in several Coxeter groups in Chapter~\ref{chap:TandTavoid}. All of the results in Chapter~\ref{chap:TandTavoid}, barring Section~\ref{sec:tavoidI}, are previously known. Chapter~\ref{chap:Cn} contains the main results of this thesis, namely the classification of T-avoiding elements in Coxeter groups of types $\C_n$ and $B_n$, which are new results. Section~\ref{sec:Btools} introduces signed permutations and signed patterns in the context of Coxeter systems of type $B_n$. We characterize Property T and T-avoiding in Coxeter systems of type $B_n$ in terms of consecutive signed patterns in Section~\ref{sec:TAB}. We conclude with some open questions in Section~\ref{sec:open}.


%%%%%%%%%%%%%%%%%%%%%%%%%

\subsection{Coxeter Systems}\label{sec:coxeter}

A \emph{Coxeter system} is a pair $(W,S)$ consisting of a finite set $S$ of generating involutions and a group $W$, called a \emph{Coxeter group}, with presentation 
\[ 
W = \langle S \mid (st)^{m(s, t)} = e  \rangle,
\]
where $e$ is the identity, $m(s,t) = 1$ if and only if $s = t$, and $m(s,t) = m(t,s) \geq 2$ for $s \neq t$. If there is no relation between $s,t \in S$, then we define $m(s,t)=\infty$. However, in this thesis we assume that all $m(s,t)$ are finite. It turns out that the elements of $S$ are distinct as group elements and that $m(s,t)$ is the \emph\emph{order} of $st$~\cite{Humphreys1990}. We call $m(s,t)$ the \emph{bond strength} of $s$ and $t$.\\

Since $s$ and $t$ are elements of order 2, the relation $(st)^{m(s,t)}=e$ can be rewritten as
\begin{equation}\label{braid} 
	\underbrace{sts \cdots}_{m(s,t)}=\underbrace{tst\cdots}_{m(s,t)}
\end{equation}
with $m(s,t) \geq 2$ factors. If $m(s,t)=2$, then $st=ts$ is called a \emph{commutation relation}. Otherwise, if $m(s,t) \geq 3$, then the relation in \eqref{braid} is called a \emph{braid relation}. The replacement \[\underbrace{sts\cdots}_{m(s,t)} \mapsto  \underbrace{tst\cdots}_{m(s,t)}\] will be referred to as a \emph{commutation} if $m(s,t)=2$ and a \emph{braid move} if $m(s,t) \geq 3$.\\

We can represent a Coxeter system $(W,S)$ with a \emph{Coxeter graph} $\Gamma$ having
\begin{enumerate}[leftmargin=2cm]
\item vertex set $S$ and
\item edges $\{s, t\}$ for each $m(s,t) \geq 3$.	
\end{enumerate}

Each edge $\{s,t\}$ is labeled with its corresponding bond strength. Since $m(s,t)=3$ occurs frequently, it is customary to omit this label. Note that $s$ and $t$ are not connected by an edge in the graph if and only if $m(s,t)=2$. There is a one-to-one correspondence between Coxeter systems and Coxeter graphs. That is, given a Coxeter graph $\Gamma$, we can uniquely reconstruct the corresponding Coxeter system. If $(W,S)$ is a Coxeter system with corresponding Coxeter graph $\Gamma$, we may denote the Coxeter group as $W(\Gamma)$ and the generating set as $S(\Gamma)$ for clarity. Also, the Coxeter system $(W,S)$ is said to be \emph{irreducible} if and only if $\Gamma$ is connected. If the graph $\Gamma$ is disconnected, the connected components correspond to factors in a direct product of the corresponding Coxeter groups~\cite{Humphreys1990}. The Coxeter graphs given in Figure~\ref{fig:labeledgraphs} correspond to the Coxeter systems that will be primarily addressed in this thesis. 

\begin{figure}[h]
\begin{tabular}{m{7cm} m{7cm}}
\begin{subfigure}{0.5\textwidth} \centering
\begin{tikzpicture}[scale=1.0]%A_{n}
\draw[fill=black] \foreach \x in {1,2,...,6} {(\x,10) circle (2pt)};
\fill[fill=white] (2,11) circle (2pt);
\draw {(.5,10) node{}
(1.5,10) node[label=above:\textcolor{white}{$4$}]{}
(4.5,10) node{$\cdots$}
(1,10) node[label=below:$s_1$]{}
(2,10) node[label=below:$s_2$]{}
(3,10) node[label=below:$s_3$]{}
(4,10) node[label=below:$s_4$]{}
(5,10) node[label=below:$s_{n-1}$]{}
(6,10) node[label=below:$s_{n}$]{}
[-] (1,10) -- (4,10)
[-] (5,10) -- (6,10)
(1,10) node{}}; 
\end{tikzpicture}
\caption{$A_{n}$} \label{fig:labeledA}
\end{subfigure} &

\begin{subfigure}{0.5\textwidth} \centering
\begin{tikzpicture}[scale=1.0]%\widetilde{A}_{n}
\draw [fill=black] \foreach \x in {1,...,4} {(\x,7.5) circle (2pt)};
\draw [fill=black] (2.5, 8.5) circle (2pt);
\draw {(.5,8.5) node{}
(2.5,7.5) node{$\cdots$}
(1,7.5) node[label=below:$s_1$]{}
(2,7.5) node[label=below:$s_2$]{}
(3,7.5) node[label=below:$s_{n-1}$]{}
(4,7.5) node[label=below:$s_{n}$]{}
(2.5,8.5) node[label=right:$s_{n+1}$]{}
[-] (2.5,8.5) -- (1, 7.5)
[-] (2.5,8.5) -- (4, 7.5)
[-] (1,7.5) -- (2,7.5)
[-] (3,7.5) -- (4,7.5)
(2,8.5) node{}}; 
\end{tikzpicture}
\caption{$\widetilde{A}_{n}$} \label{fig:labeledaffAn}
\end{subfigure} \\

    & \\ 

\begin{subfigure}{0.5\textwidth} \centering
\begin{tikzpicture}[scale=1.0]%B_{n}
\draw [fill=black] \foreach \x in {1,2,...,6} {(\x,8.5) circle (2pt)};
%\draw [fill=white] (1,10) circle(2pt);
\draw {(.5,8.5) node{}
(1.5,8.5) node[label=above:$4$]{}
(6,8.5) node[label=above:\textcolor{white}{$4$}]{}
(4.5,8.5) node{$\cdots$}
(1,8.5) node[label=below:$s_0$]{}
(2,8.5) node[label=below:$s_1$]{}
(3,8.5) node[label=below:$s_2$]{}
(4,8.5) node[label=below:$s_3$]{}
(5,8.5) node[label=below:$s_{n-2}$]{}
(6,8.5) node[label=below:$s_{n-1}$]{}
[-] (1,8.5) -- (4,8.5)
[-] (5,8.5) -- (6,8.5)
(2,8.5) node{}}; 
\end{tikzpicture}
\caption{$B_{n}$} \label{fig:labeledB}
\end{subfigure} &

\begin{subfigure}{0.5\textwidth} \centering
\begin{tikzpicture}[scale=1.0]
\draw[fill=black] \foreach \x in {1,2,...,6} {(\x,5) circle (2pt)};%\widetild{C}_{n}
%\fill[white] (1,6) circle (2pt);
\draw {(.5,5) node{}
(4.5,5) node{$\cdots$}
(5.5,5) node[label=above:$4$]{}
(1.5,5) node[label=above:$4$]{}
(1,5) node[label=below:$s_0$]{}
(2,5) node[label=below:$s_1$]{}
(3,5) node[label=below:$s_2$]{}
(4,5) node[label=below:$s_3$]{}
(5,5) node[label=below:$s_{n-1}$]{}
(6,5) node[label=below:$s_{n}$]{}
[-] (1,5) -- (4,5)
[-] (5,5) -- (6,5)
(2,5) node{}};
\end{tikzpicture}
\caption{$\widetilde{C}_{n}$} \label{fig:labeledaffC}
\end{subfigure}  \\

& \\

\begin{subfigure}{0.5\textwidth} \centering
\begin{tikzpicture}[scale=1.0]
\draw[fill=black] \foreach \x in {1,2,...,6} {(\x,6.5) circle (2pt)};%D_{n}
\draw[fill=black] (2,7.5) circle (2pt);
\draw {(.5,6.5) node{}
(4.5,6.5) node{$\cdots$}
(1,6.5) node[label=below:$s_1$]{}
(2,6.5) node[label=below:$s_2$]{}
(3,6.5) node[label=below:$s_3$]{}
(4,6.5) node[label=below:$s_4$]{}
(5,6.5) node[label=below:$s_{n-2}$]{}
(6,6.5) node[label=below:$s_{n-1}$]{}
(2,7.5) node[label=right:$s_0$]{}
[-] (1,6.5) -- (4,6.5)
[-] (5,6.5) -- (6,6.5)
[-] (2,6.5) -- (2,7.5)
(2,6.5) node{}};
\end{tikzpicture}
\caption{$D_{n}$} \label{fig:labeledD}
\end{subfigure}&

\begin{subfigure}{0.5\textwidth} \centering
\begin{tikzpicture}[scale=1.0]%F_{n}
\draw[fill=black] \foreach \x in {1,2,...,6} {(\x,3) circle (2pt)};
\fill[white] (1,4) circle (2pt);
\draw {(.5,3) node{}
(2.5,3) node[label=above:$4$]{}
(4.5,3) node{$\cdots$}
(1,3) node[label=below:$s_1$]{}
(2,3) node[label=below:$s_2$]{}
(3,3) node[label=below:$s_3$]{}
(4,3) node[label=below:$s_4$]{}
(5,3) node[label=below:$s_{n-1}$]{}
(6,3) node[label=below:$s_{n}$]{}
[-] (1,3) -- (4,3)
[-] (5,3) -- (6,3)
(3,3) node{}};
\end{tikzpicture}
\caption{$F_{n}$} \label{fig:labeledFn}
\end{subfigure}  \\
\end{tabular}

\begin{subfigure}{1.0\textwidth} \centering
\begin{tikzpicture}[scale=1.0]
\draw[fill=black] \foreach \x in {1,2} {(\x,0) circle (2pt)};
\fill[fill=white] (2,1) circle (2pt);
\draw {(.25,0) node{}
(1.5,0) node[label=above:$m$]{}
(1,0) node[label=below:$s_1$]{}
(2,0) node[label=below:$s_2$]{}
[-] (1,0) -- (2,0)
(2,0) node{}};
\end{tikzpicture}
\caption{$I_{2}(m)$} \label{fig:labeledI}
\end{subfigure}

\caption{Examples of a few Coxeter graphs.}\label{fig:labeledgraphs}
\end{figure}


\begin{example}
~
\begin{itemize}
\item[(a)~] The Coxeter system of type $A_n$ is given by the graph in Figure~\ref{fig:labeledA}. We can construct the corresponding Coxeter group $W(A_n)$ with generating set $S(A_n)=\{s_1, s_2, \ldots ,s_n\}$ and defining relations
\begin{enumerate}[leftmargin=2cm]
	\item $s_i^2=e$ for all $i$;
	\item $s_is_j=s_js_i$ when $|i-j|>1$;
	\item $s_is_js_i=s_js_is_j$ when $|i-j|=1.$
\end{enumerate}
The Coxeter group $W(A_n)$ is isomorphic to the symmetric group $\Sym_{n+1}$ under the correspondence $s_i \mapsto (i, i+1)$, where $(i, i+1)$ is the adjacent transposition that swaps $i$ and $i+1$.
\item[(b)~]\label{ex:B} The Coxeter system of type $B_n$ is given by the graph in Figure~\ref{fig:labeledB}. We can construct the corresponding Coxeter group $W(B_n)$ with generating set $S(B_n)=\{s_0,s_1, \ldots ,s_{n-1}\}$ and defining relations
\begin{enumerate}[leftmargin=2cm]
	\item $s_i^2=e$ for all $i$;
	\item $s_is_j=s_js_i$  when $|i-j|>1$;
	\item $s_is_js_i=s_js_is_j$ when $|i-j|=1$ for $i,j \in \{1,2,\ldots, n-1\}$;
	\item $s_0s_1s_0s_1=s_1s_0s_1s_0$.
\end{enumerate}
The Coxeter group $W(B_n)$ is isomorphic to the group, $\Sym_n^B$, of signed permutations on the set $\{1,2, \ldots,n\}$. We discuss $\Sym_n^B$ in more detail in Section~\ref{sec:Btools}.
\item[(c)~] The Coxeter system of type $\widetilde C_n$ is given by the graph in Figure~\ref{fig:labeledaffC}. We can construct the corresponding Coxeter group $W(\widetilde C_n)$ with generating set $S(\widetilde{C}_n)=~\{s_0, s_1, \ldots ,s_n\}$ and defining relations 
\begin{enumerate}[leftmargin=2cm]
	\item $s_i^2=e$ for all $i$;
	\item $s_is_j=s_js_i$ when $|i-j|>1$ for $i \in \{0,2, \ldots, n\}$;
	\item $s_is_js_i=s_js_is_j$ 	when $|i-j|=1$ for $i \in \{1,2, \ldots, n-1\}$;
	\item $s_0s_1s_0s_1=s_1s_0s_1s_0$;
	\item $s_ns_{n-1}s_ns_{n-1}=s_{n-1}s_ns_{n-1}s_n.$
\end{enumerate}
Note that $W(\widetilde{C}_n)$ has $n+1$ generators. It turns out that $W(\C_n)$ is an infinite group.
\end{itemize}
\end{example}



The Coxeter graphs given in Figure~\ref{fig:fincoxgraphs} correspond to the collection of irreducible finite-type Coxeter systems, whose corresponding Coxeter groups are finite, while the Coxeter graphs given in Figure~\ref{fig:infincoxgraphs} are the so-called irreducible \emph{affine Coxeter systems}, whose corresponding Coxeter groups are infinite~\cite{Humphreys1990}. From now on we will refer to a finite Coxeter system to be a system where $W(\Gamma)$ is finite. Note that $W(B_n)$ is one of the irreducible finite Coxeter groups, so it is finite, while $W(\C_n)$ is one of the affine groups making it infinite. The irreducible affine Coxeter systems are unique in that if a vertex is removed along with the corresponding edges from the Coxeter graph, the newly created graph will result in a Coxeter system with a finite Coxeter group. 

\begin{figure}[h!]
\begin{tabular}{m{7cm} m{7cm}}
\begin{subfigure}{0.5\textwidth} \centering
\begin{tikzpicture}[scale=1.0]%A_{n}
\draw[fill=black] \foreach \x in {1,2,...,6} {(\x,10) circle (2pt)};
\draw {(.5,10) node{}
(1.5,10) node[label=above:\textcolor{white}{$4$}]{}
(4.5,10) node{$\cdots$}
%(1,10) node[label=below:$s_1$]{}
%(2,10) node[label=below:$s_2$]{}
%(3,10) node[label=below:$s_3$]{}
%(4,10) node[label=below:$s_4$]{}
%(5,10) node[label=below:$s_{n-1}$]{}
%(6,10) node[label=below:$s_{n}$]{}
[-] (1,10) -- (4,10)
[-] (5,10) -- (6,10)
(1,10) node{}}; 
\end{tikzpicture}
\caption{$A_{n}$} \label{fig:A}
\end{subfigure} &

\begin{subfigure}{0.5\textwidth} \centering
\begin{tikzpicture}[scale=1.0]%B_{n}
\draw [fill=black] \foreach \x in {1,2,...,6} {(\x,8.5) circle (2pt)};
\draw {(.5,8.5) node{}
(1.5,8.5) node[label=above:$4$]{}
(4.5,8.5) node{$\cdots$}
%(1,8.5) node[label=below:$s_0$]{}
%(2,8.5) node[label=below:$s_1$]{}
%(3,8.5) node[label=below:$s_2$]{}
%(4,8.5) node[label=below:$s_3$]{}
%(5,8.5) node[label=below:$s_{n-2}$]{}
%(6,8.5) node[label=below:$s_{n-1}$]{}
[-] (1,8.5) -- (4,8.5)
[-] (5,8.5) -- (6,8.5)
(2,8.5) node{}}; 
\end{tikzpicture}
\caption{$B_{n}$} \label{fig:B}
\end{subfigure} \\

    & \\ 

\begin{subfigure}{0.5\textwidth} \centering
\begin{tikzpicture}[scale=1.0]
\draw[fill=black] \foreach \x in {1,2} {(\x,0) circle (2pt)};
\fill[fill=white] (2,1) circle (2pt);
\draw {(.25,0) node{}
(1.5,0) node[label=above:$m$]{}
%(1,0) node[label=below:$s_1$]{}
%(2,0) node[label=below:$s_2$]{}
[-] (1,0) -- (2,0)
(2,0) node{}};
\end{tikzpicture}
\caption{$I_{2}(m)$} \label{fig:I}
\end{subfigure} &

\begin{subfigure}{0.5\textwidth} \centering
\begin{tikzpicture}[scale=1.0]
\draw[fill=black] \foreach \x in {1,2,...,6} {(\x,6.5) circle (2pt)};%D_{n}
\draw[fill=black] (2,7.5) circle (2pt);
\draw {(.5,6.5) node{}
(4.5,6.5) node{$\cdots$}
%(1,6.5) node[label=below:$s_1$]{}
%(2,6.5) node[label=below:$s_2$]{}
%(3,6.5) node[label=below:$s_3$]{}
%(4,6.5) node[label=below:$s_4$]{}
%(5,6.5) node[label=below:$s_{n-2}$]{}
%(6,6.5) node[label=below:$s_{n-1}$]{}
%(2,7.5) node[label=right:$s_0$]{}
[-] (1,6.5) -- (4,6.5)
[-] (5,6.5) -- (6,6.5)
[-] (2,6.5) -- (2,7.5)
(2,6.5) node{}};
\end{tikzpicture}
\caption{$D_{n}$} \label{fig:D}
\end{subfigure} \\

    & \\ 
    
\begin{subfigure}{0.5\textwidth} \centering
\begin{tikzpicture}[scale=1.0]%E_{6}
\draw[fill=black] \foreach \x in {1,2,...,5} {(\x,4.5) circle (2pt)};
\draw[fill=black] (3,5.5) circle (2pt);
\draw {
[-] (1,4.5) -- (5,4.5)
[-] (3,4.5) -- (3,5.5)
(3,4.5) node{}};
\end{tikzpicture}
\caption{$E_{6}$} \label{fig:E6}
\end{subfigure} &



\begin{subfigure}{0.5\textwidth} \centering
\begin{tikzpicture}[scale=1.0]%E_{7}
\draw[fill=black] \foreach \x in {1,2,...,6} {(\x,4.5) circle (2pt)};
\draw[fill=black] (3, 5.5) circle (2pt);
\draw {
[-] (1,4.5) -- (5,4.5)
[-] (5,4.5) -- (6,4.5)
[-] (3,4.5) -- (3,5.5)
(3,4.5) node{}};
\end{tikzpicture}
\caption{$E_{7}$} \label{fig:E7}
\end{subfigure} \\

    & \\ 


\begin{subfigure}{0.5\textwidth} \centering
\begin{tikzpicture}[scale=1.0]%E_{8}
\draw[fill=black] \foreach \x in {1,2,...,7} {(\x,4.5) circle (2pt)};
\draw[fill=black] (3,5.5) circle (2pt);
\draw {
[-] (1,4.5) -- (7,4.5)
[-] (3,4.5) -- (3,5.5)
(3,4.5) node{}};
\end{tikzpicture}
\caption{$E_{8}$} \label{fig:E6}
\end{subfigure} &

\begin{subfigure}{0.5\textwidth} \centering
\begin{tikzpicture}[scale=1.0]%F_{4}
\draw[fill=black] \foreach \x in {1,2,...,4} {(\x,3) circle (2pt)};
\fill[white] (1,4) circle (2pt);
\draw {(.5,3) node{}
(2.5,3) node[label=above:$4$]{}
[-] (1,3) -- (4,3)
(3,3) node{}};
\end{tikzpicture}
\caption{$F_{4}$} \label{fig:F4}
\end{subfigure} \\

&\\

\begin{subfigure}{0.5\textwidth} \centering
\begin{tikzpicture}[scale=1.0]
\draw[fill=black] \foreach \x in {1,2,...,3} {(\x,1.5) circle (2pt)};%H_{3}
\draw {(.5,1.5) node{}
(1.5,1.5) node[label=above:$5$]{}
[-] (1,1.5) -- (3,1.5)
(2,1.5) node{}}; 
\end{tikzpicture}
\caption{$H_{3}$} \label{fig:H}
\end{subfigure} &

\begin{subfigure}{0.5\textwidth} \centering
\begin{tikzpicture}[scale=1.0]
\draw[fill=black] \foreach \x in {1,2,...,4} {(\x,1.5) circle (2pt)};%H_{4}
\draw {(.5,1.5) node{}
(1.5,1.5) node[label=above:$5$]{}
[-] (1,1.5) -- (4,1.5)
(2,1.5) node{}}; 
\end{tikzpicture}
\caption{$H_{4}$} \label{fig:H}
\end{subfigure}
\end{tabular}
\caption{Irreducible finite Coxeter systems.}
\label{fig:fincoxgraphs}
\end{figure}

%%%%%%%%%%%%%%%%%%%%%%%%%%%

\begin{figure}[h!]
\begin{tabular}{m{7cm} m{7cm}}
\begin{subfigure}{0.5\textwidth} \centering
\begin{tikzpicture}[scale=1.0]%\widetilde{A}_{2}
\draw[fill=black] \foreach \x in {1,2} {(\x,10) circle (2pt)};
\fill[white] (1,11) circle (2pt);
\draw { (.5,10) node{}
(1.5,10) node[label=above:$\infty$]{}
%(1,10) node[label=below:$s_1$]{}
%(2,10) node[label=below:$s_2$]{}
[-] (1,10) -- (2,10)
(1,10) node{}}; 
\end{tikzpicture}
\caption{$\widetilde{A}_{2}$} \label{fig:affA2}
\end{subfigure} &

\begin{subfigure}{0.5\textwidth} \centering
\begin{tikzpicture}[scale=1.0]%\widetilde{A}_{n}
\draw [fill=black] \foreach \x in {1,...,4} {(\x,7.5) circle (2pt)};
\draw [fill=black] (2.5, 8.5) circle (2pt);
\draw {(.5,8.5) node{}
(2.5,7.5) node{$\cdots$}
%(1,7.5) node[label=below:$s_1$]{}
%(2,7.5) node[label=below:$s_2$]{}
%(3,7.5) node[label=below:$s_{n-1}$]{}
%(4,7.5) node[label=below:$s_{n}$]{}
%(2.5,8.5) node[label=right:$s_{n+1}$]{}
[-] (2.5,8.5) -- (1, 7.5)
[-] (2.5,8.5) -- (4, 7.5)
[-] (1,7.5) -- (2,7.5)
[-] (3,7.5) -- (4,7.5)
(2,8.5) node{}}; 
\end{tikzpicture}
\caption{$\widetilde{A}_{n}$} \label{fig:affAn}
\end{subfigure} \\

    & \\ 

\begin{subfigure}{0.5\textwidth} \centering
\begin{tikzpicture}[scale=1.0]%\widetilde{B}_{n}
\draw[fill=black] \foreach \x in {1,2,...,6} {(\x,0) circle (2pt)};
\draw[fill=black] (5,1) circle (2pt);
\draw {(.25,0) node{}
(1.5,0) node[label=above:$4$]{}
(4.5, 0) node{$\cdots$}
%(1,0) node[label=below:$s_0$]{}
%(2,0) node[label=below:$s_1$]{}
%(3,0) node[label=below:$s_2$]{}
%(4,0) node[label=below:$s_3$]{}
%(5,0) node[label=below:$s_{n-2}$]{}
%(6,0) node[label=below:$s_{n-1}$]{}
%(5,1) node[label=right:$s_n$]{}
[-] (1,0) -- (4,0)
[-] (5,0) -- (6,0)
[-] (5,1) -- (5,0)
(2,0) node{}};
\end{tikzpicture}
\caption{$\widetilde{B}_{n}$} \label{fig:affB}
\end{subfigure} &

\begin{subfigure}{0.5\textwidth} \centering
\begin{tikzpicture}[scale=1.0]
\draw[fill=black] \foreach \x in {1,2,...,6} {(\x,5) circle (2pt)};%\widetild{C}_{n}
\fill[white] (1,6) circle (2pt);
\draw {(.5,5) node{}
(4.5,5) node{$\cdots$}
(5.5,5) node[label=above:$4$]{}
(1.5,5) node[label=above:$4$]{}
%(1,5) node[label=below:$s_0$]{}
%(2,5) node[label=below:$s_1$]{}
%(3,5) node[label=below:$s_2$]{}
%(4,5) node[label=below:$s_3$]{}
%(5,5) node[label=below:$s_{n-1}$]{}
%(6,5) node[label=below:$s_{n}$]{}
[-] (1,5) -- (4,5)
[-] (5,5) -- (6,5)
(2,5) node{}};
\end{tikzpicture}
\caption{$\widetilde{C}_{n}$} \label{fig:affC}
\end{subfigure} \\

    & \\ 
    
\begin{subfigure}{0.5\textwidth} \centering
\begin{tikzpicture}[scale=1.0]%\widetilde{D}_{6}
\draw[fill=black] \foreach \x in {1,2,...,6} {(\x,3.5) circle (2pt)};
\draw[fill=black] (2,4.5) circle (2pt);
\fill[white] (2,5.5) circle (2pt);
\draw[fill=black] (5,4.5) circle (2pt);
\draw {
(3.5,3.5) node{$\cdots$}
[-] (1,3.5) -- (3,3.5)
[-] (4, 3.5) --(6, 3.5)
[-] (2,3.5) -- (2,4.5)
[-] (5,3.5)-- (5,4.5)
(3,4.5) node{}};
\end{tikzpicture}
\caption{$\widetilde{D}_{n}$} \label{fig:E6}
\end{subfigure} &



\begin{subfigure}{0.5\textwidth} \centering
\begin{tikzpicture}[scale=1.0]%\widetilde{E}_{6}
\draw[fill=black] \foreach \x in {1,2,...,5} {(\x,4.5) circle (2pt)};
\draw[fill=black] (3, 5.5) circle (2pt);
\draw[fill=black] (3, 6.5) circle (2pt);
\draw {
[-] (1,4.5) -- (5,4.5)
[-] (3,4.5) -- (3,6.5)
(3,4.5) node{}};
\end{tikzpicture}
\caption{$\widetilde{E}_{6}$} \label{fig:affE6}
\end{subfigure} \\

    & \\ 


\begin{subfigure}{0.5\textwidth} \centering
\begin{tikzpicture}[scale=1.0]%\widetilde{E}_{7}
\draw[fill=black] \foreach \x in {1,2,...,7} {(\x,4.5) circle (2pt)};
\draw[fill=black] (4,5.5) circle (2pt);
\draw {
[-] (1,4.5) -- (7,4.5)
[-] (4,4.5) -- (4,5.5)
(3,4.5) node{}};
\end{tikzpicture}
\caption{$\widetilde{E}_{7}$} \label{fig:affE7}
\end{subfigure} &

\begin{subfigure}{0.5\textwidth} \centering
\begin{tikzpicture}[scale=1.0]%\widetilde{E}_{8}
\draw[fill=black] \foreach \x in {1,2,...,8} {(\x,3) circle (2pt)};
\draw[fill=black] (3,4) circle (2pt);
\draw {(.5,3) node{}
[-] (3,4) -- (3,3)
[-] (1,3) -- (8,3)
(3,3) node{}};
\end{tikzpicture}
\caption{$\widetilde{E}_{8}$} \label{fig:affE8}
\end{subfigure} \\

&\\

\begin{subfigure}{0.5\textwidth} \centering
\begin{tikzpicture}[scale=1.0]
\draw[fill=black] \foreach \x in {1,2,...,5} {(\x,1.5) circle (2pt)};%\widetilde{F}_{4}
\draw {(.5,1.5) node{}
(2.5,1.5) node[label=above:$4$]{}
[-] (1,1.5) -- (5,1.5)
(2,1.5) node{}}; 
\end{tikzpicture}
\caption{$\widetilde{F}_{4}$} \label{fig:H}
\end{subfigure} &

\begin{subfigure}{0.5\textwidth} \centering
\begin{tikzpicture}[scale=1.0]
\draw[fill=black] \foreach \x in {1,2,...,3} {(\x,1.5) circle (2pt)};%\widetilde{G}_{2}
\draw {(.5,1.5) node{}
(2.5,1.5) node[label=above:$6$]{}
[-] (1,1.5) -- (3,1.5)
(2,1.5) node{}}; 
\end{tikzpicture}
\caption{$\widetilde{G}_{4}$} \label{fig:H}
\end{subfigure}
\end{tabular}
\caption{Irreducible affine Coxeter systems.}
\label{fig:infincoxgraphs}
\end{figure}


Given a Coxeter system $(W,S)$, a word $s_{x_1}s_{x_2} \cdots s_{x_m}$ in the free monoid $S^*$ on $S$ is called an \emph{expression} for $w \in W$ if it is equal to $w$ when considered as a group element. If $m$ is minimal among all expressions for $w$, the corresponding word is called a \emph{reduced expression} for $w$. In this case, we define the \emph{length} of $w$ to be $l(w):= m$. Each element $w \in W$ may have multiple reduced expressions that represent it. If we wish to emphasize a specific, possibly reduced, expression for $w \in W$ we will represent it as $\w=s_{x_1}s_{x_2}\cdots s_{x_m}$ (using {\sf{sans serif font}}). If $u,v \in W$, we say that the product $uv$ is \emph{reduced} if $l(uv)=l(u)+l(v)$. Matsumoto's Theorem, which follows, tells us more about how reduced expressions for a given group element are related.

\begin{proposition} [Matsumoto, \cite{Geck2000}]
	Let $(W,S)$ be a Coxeter system. If $w \in W$, then given a reduced expression for $w$ we can obtain every other reduced expression for $w$ by a sequence of braid moves and commutations of the form
	\[\underbrace{sts\cdots}_{m(s,t)} \rightarrow \underbrace{tst\cdots}_{m(s,t)}\]
	where $s,t \in S$ and $m(s,t) \geq 2$. \qed
\end{proposition}
 
It follows from Matsumoto's Theorem that if a generator $s$ appears in a reduced expression for $w \in W$, then $s$ appears in all reduced expressions for $w$. Let $w \in W$ and define the \emph{support} of $w$, denoted $\supp(w)$, to be the set of all generators that appear in any reduced expression for $w$. If $\supp(w)=S$, we say that $w$ has \emph{full support}. 

Given $w \in W$ and a fixed reduced expression $\w$ for $w$, any subsequence of $\w$ is called a \emph{subexpression} of $\w$. We will refer to a subexpression consisting of a consecutive subsequence of $\w$ as a \emph{subword} of $\w$. 

\begin{example}
Let $\w=s_7s_2s_4s_5s_3s_2s_3s_6$ be an expression for $w \in W(A_7)$. Then we have 
\begin{align*}
s_7\textcolor{purple}{s_2s_4}s_5s_3s_2s_3s_6&=s_7s_4\textcolor{purple}{s_2s_5}s_3s_2s_3s_6\\
&=s_7s_4s_5\textcolor{teal}{s_2s_3s_2} s_3s_6\\
&=s_7s_4s_5s_3s_2\textcolor{rred}{s_3s_3}s_6\\
&=s_7s_4s_5s_3s_2s_6,
\end{align*}
where the \textcolor{purple}{purple}-highlighted text corresponds to a commutation, the \textcolor{teal}{teal}-highlighted text corresponds to a braid move, and the \textcolor{rred}{red}-highlighted text corresponds to cancellation. This shows that the original expression $\w$ is not reduced. However, it turns out that $s_7s_4s_5s_3s_2s_6$ is reduced. Thus, $l(w)=6$ and $\supp(w)=\{s_2, s_3, s_4, s_5, s_6, s_7\}$.
\end{example}

Let $(W,S)$ be a Coxeter system and let $w \in W$. We define the \emph{left descent set} and \emph{right descent set} of $w$ as follows:
\[\mathcal{L}(w):=\{s \in S \mid l(sw) < l(w)\}\]

\[\mathcal{R}(w):=\{s \in S \mid l(ws) < l(w)\}.\]
In~\cite{Bjorner2005} it is shown that $s \in \mathcal{L}(w)$ (respectively, $\mathcal{R}(w)$) if and only if there is a reduced expression for $w$ that begins (respectively, ends) with $s$.

\begin{example}
The following list consists of all reduced expressions for a particular $w \in W(B_4)$:
$$\begin{array}{ll}
s_0s_1s_2s_1s_3 & s_0s_2s_1s_2s_3\\
s_0s_1s_2s_3s_1 & s_2s_0s_1s_2s_3	
\end{array}$$
We see that $l(w)=5$ and $w$ has full support. Also, we see that $\mathcal{L}(w)=\{s_0, s_2\}$ while $\mathcal{R}(w)=\{s_1, s_3\}$.	
\end{example}

Given a Coxeter system $(W,S)$, for any subset $I \subseteq S$, define $W_I$ to be the subgroup of $W$ generated by all $s \in I$. Such a subgroup is called a \emph{parabolic subgroup} of $W$. By Section 5.5 of~\cite{Humphreys1990}, for $I \subseteq S$, the corresponding parabolic subgroup forms a Coxeter system $(W_I,I)$ with the given values $m(s,t)$.
%%%%%%%%%%%%%%%%%%%%%%%%%%%


\subsection{Fully Commutative Elements}\label{sec:FC}
Let $(W,S)$ be a Coxeter system of type $\Gamma$ and let $w \in W(\Gamma)$. Following~\cite{Stembridge1996}, we define a relation $\sim$ on the set of reduced expressions for $w$. Let $\w_1$ and $\w_2$ be two reduced expressions for $w$. We define $\w_1 \sim \w_2$ if we can obtain $\w_2$ from $\w_1$ by applying a single commutation move of the form $st \mapsto ts$ where $m(s,t)=2$. Now, define the equivalence relation $\approx$ by taking the reflexive transitive closure of $\sim$. Each equivalence class under $\approx$ is called a \emph{commutation class}. If there is a single commutation class for the set of reduced expressions for $w$, then we say that $w$ is \emph{fully commutative} (FC). 

The set of FC elements of $W(\Gamma)$ is denoted by $\FC(\Gamma)$. Given some $w \in \FC(\Gamma)$ and a starting reduced expression for $w$, observe that the definition of FC states that one only needs to perform commutations to obtain all reduced expressions for $w$, but the following result due to Stembridge~\cite{Stembridge1996} states that when $w$ is FC, performing commutations is the only possible way to obtain another reduced expression for $w$.

\begin{proposition}[Stembridge,~\cite{Stembridge1996}]\label{thm:Stembridge}
	An element $w \in \FC(\Gamma)$ is FC if and only if no reduced expression for $w$ contains $\underbrace{sts\cdots}_{m(s,t)}$ as a subword for all $m(s,t) \geq 3$. \qed
\end{proposition}

In other words, $w$ is FC if and only if no reduced expression provides the opportunity to apply a braid move. For example, in a Coxeter system of type $B_n$ an element is $\FC$ if no reduced expression contains the subwords $s_0s_1s_0s_1$, $s_1s_0s_1s_0$, $s_ks_{k+1}s_k$, and $s_{k+1}s_ks_{k+1}$ where $0<k\leq n-2$. In a Coxeter system of type $\C_n$, an element is $\FC$ if no reduced expression for the element contains the subwords seen above with $0<k\leq n-1$ and does not contain the subwords $s_{n-1}s_ns_{n-1}s_n$ and $s_{n}s_{n-1}s_ns_{n-1}$.

%\begin{example}
%	\textcolor{green}{Let $\w=s_0s_1s_2s_0s_3s_1$ be a reduced expression for $w \in W(\C_4)$}. Although it is not immediately obvious, there is no possible way to perform a braid move in any reduced expression for $w$. Hence $w$ is $\FC$.
%\end{example}

\begin{example}
Let $\w_1=s_1s_0s_1s_3s_4s_5s_2s_4s_6$ be a reduced expression for $w \in W(\widetilde{C}_6)$. Applying the commutation $s_2s_4 \mapsto s_4s_2$, we can obtain another reduced expression for $w$, namely $\w_2=s_1s_0s_1s_3s_4s_5s_4s_2s_6$, which is in the same commutation class as $\w_1$. However, applying the braid move $s_4s_5s_4 \mapsto s_5s_4s_5$, we obtain another reduced expression $\w_3=s_1s_0s_1s_3s_5s_4s_5s_2s_6$. Note that since $\w_3$ was obtained by applying a braid move, $\w_3$ is in a different commutation class from $\w_1$ and $\w_2$. Since $w$ has at least two commutation classes, one containing $\w_1$ and $\w_2$ and another containing $\w_3$, $w$ is not FC by Proposition~\ref{thm:Stembridge}.
\end{example}

%\begin{example}
%Let $w \in W(\widetilde{C}_4)$ and let $\w=s_0s_1s_2s_0s_1s_2$ be a reduced expression for $w$. We see that
%\[s_0s_1\textcolor{purple}{s_3s_0}s_1s_2=s_0s_1s_0\textcolor{purple}{s_3s_1}s_2=\textcolor{orange}{s_0s_1s_0s_1}s_3s_2,\]
%where the \textcolor{purple}{purple}-highlighted text indicates applying a commutation and the \textcolor{orange}{orange}-highlighted text indicates applying a braid move. Thus, $w$ is not $\FC$ by Theorem~\ref{thm:Stembridge}.  	
%\end{example}

Stembridge classified the Coxeter systems whose groups contain a finite number of FC elements, the so-called \emph{FC-finite Coxeter groups}. Both $W(A_n)~\mathrm{and}~W(B_n)$ are finite Coxeter groups, and Thus, are $\FC$-finite. On the other hand, $W(\widetilde{C}_n)$ is infinite and happens to also contain infinitely many $\FC$ elements. There exist infinite Coxeter groups that contain finitely many $\FC$ elements. For example, $W(E_n$) for $n \geq 9$ (see Figure~\ref{fig:FCfincoxgraphs}) is infinite, but contains only finitely many FC elements.

\begin{proposition}[Stembridge,~\cite{Stembridge1996}]
\label{thm:FCfinite} The irreducible FC-finite Coxeter systems are of type $A_n$ with $n \geq 1$, $B_n$ with $n \geq 2$, $D_n$ with $n \geq 4$, $E_n$ with $n \geq 6$, $F_n$ with $n \geq 4$, $H_n$ with $n \geq 3$, and $I_2(m)$ with $5 \leq m < \infty$. \qed
\end{proposition} 
 
The irreducible FC-finite Coxeter graphs are given in Figure~\ref{fig:FCfincoxgraphs}. Note that the irreducible finite Coxeter systems given in Figure~\ref{fig:fincoxgraphs} certainly have only a finite number of FC elements. So the irreducible FC-finite Coxeter systems contain the irreducible finite Coxeter systems. However, notice there are a few graphs in Figure~\ref{fig:fincoxgraphs} that we have not yet encountered. Specifically, we have not yet encountered the Coxeter groups determined by graphs in Figures~\ref{fig:FCEn} for $n \geq 9$,~\ref{fig:FCFn} for $n \geq 5$,~\ref{fig:FCHn} for $n \geq 5$. All of these Coxeter systems have corresponding infinite groups for sufficiently large $n$, yet contain only finitely many FC elements.

\begin{figure}[h!]
\begin{tabular}{m{7cm} m{7cm}}
\begin{subfigure}{0.5\textwidth} \centering
\begin{tikzpicture}[scale=1.0]%A_{n}
\draw[fill=black] \foreach \x in {1,2,...,6} {(\x,10) circle (2pt)};
\draw {(.5,10) node{}
(1.5,10) node[label=above:\textcolor{white}{$4$}]{}
(4.5,10) node{$\cdots$}
[-] (1,10) -- (4,10)
[-] (5,10) -- (6,10)
(1,10) node{}}; 
\end{tikzpicture}
\caption{$A_{n}$} \label{fig:FCA}
\end{subfigure} &

\begin{subfigure}{0.5\textwidth} \centering
\begin{tikzpicture}[scale=1.0]%B_{n}
\draw [fill=black] \foreach \x in {1,2,...,6} {(\x,8.5) circle (2pt)};
\draw {(.5,8.5) node{}
(1.5,8.5) node[label=above:$4$]{}
(4.5,8.5) node{$\cdots$}
[-] (1,8.5) -- (4,8.5)
[-] (5,8.5) -- (6,8.5)
(2,8.5) node{}}; 
\end{tikzpicture}
\caption{$B_{n}$} \label{fig:FCB}
\end{subfigure} \\

    & \\ 

\begin{subfigure}{0.5\textwidth} \centering
\begin{tikzpicture}[scale=1.0]
\draw[fill=black] \foreach \x in {1,2,...,6} {(\x,6.5) circle (2pt)};%D_{n}
\draw[fill=black] (2,7.5) circle (2pt);
\draw {(.5,6.5) node{}
(4.5,6.5) node{$\cdots$}
[-] (1,6.5) -- (4,6.5)
[-] (5,6.5) -- (6,6.5)
[-] (2,6.5) -- (2,7.5)
(2,6.5) node{}};
\end{tikzpicture}
\caption{$D_{n}$} \label{fig:FCD}
\end{subfigure} &
    
\begin{subfigure}{0.5\textwidth} \centering
\begin{tikzpicture}[scale=1.0]%E_{6}
\draw[fill=black] \foreach \x in {1,2,...,6} {(\x,4.5) circle (2pt)};
\draw[fill=black] (3,5.5) circle (2pt);
%\fill[white] (3,5.9) circle (2pt);
\draw {(.5, 4.5) node{}
(4.5,4.5) node{$\cdots$}
[-] (1,4.5) -- (4,4.5)
[-] (5,4.5) -- (6,4.5)
[-] (3,4.5) -- (3,5.5)
(3,4.5) node{}};
\end{tikzpicture}
\caption{$E_{n}$} \label{fig:FCEn}
\end{subfigure} \\

&\\

\begin{subfigure}{0.5\textwidth} \centering
\begin{tikzpicture}[scale=1.0]%F_{n}
\draw[fill=black] \foreach \x in {1,2,...,6} {(\x,3) circle (2pt)};
\fill[white] (1,4) circle (2pt);
\draw {(.5,3) node{}
(2.5,3) node[label=above:$4$]{}
(4.5,3) node{$\cdots$}
[-] (1,3) -- (4,3)
[-] (5,3) -- (6,3)
(3,3) node{}};
\end{tikzpicture}
\caption{$F_{n}$} \label{fig:FCFn}
\end{subfigure} &


\begin{subfigure}{0.5\textwidth} \centering
\begin{tikzpicture}[scale=1.0]
\draw[fill=black] \foreach \x in {1,2,...,6} {(\x,1.5) circle (2pt)};%H_{4}
\fill[white] (1,2.5) circle (2pt);
\draw {(.5,1.5) node{}
(1.5,1.5) node[label=above:$5$]{}
(4.5,1.5) node{$\cdots$}
[-] (1,1.5) -- (4,1.5)
[-] (5,1.5) -- (6,1.5)
(2,1.5) node{}}; 
\end{tikzpicture}
\caption{$H_{n}$} \label{fig:FCHn}
\end{subfigure}\\

&\\
\end{tabular}

\begin{subfigure}{1.0\textwidth} \centering
\begin{tikzpicture}[scale=1.0]
\draw[fill=black] \foreach \x in {1,2} {(\x,0) circle (2pt)};
\fill[fill=white] (2,1) circle (2pt);
\draw {(.25,0) node{}
(1.5,0) node[label=above:$m$]{}
[-] (1,0) -- (2,0)
(2,0) node{}};
\end{tikzpicture}
\caption{$I_{2}(m)$} \label{fig:FCI}
\end{subfigure}

\caption{Irreducible FC-finite Coxeter systems.}
\label{fig:FCfincoxgraphs}
\end{figure}

%%%%%%%%%%%%%%%%%%%%%%%%%%%%%%%%

\subsection{Heaps}\label{sec:Heaps}

We now discuss a visual representation of Coxeter group elements. Each reduced expression can be associated with a labeled partially ordered set (poset) called a heap.  Heaps provide a visual representation of a reduced expression while preserving the relations among the generators. We follow the development of heaps for straight-line Coxeter groups found in~\cite{Billey2007},~\cite{Ernst2010}, and~\cite{Stembridge1996}. 

Let $(W,S)$ be a Coxeter system of type $\Gamma$. Suppose $\w=s_{x_1}s_{x_2}\cdots s_{x_r}$ is a fixed reduced expression for $w \in W(\Gamma)$. As in~\cite{Stembridge1996}, we define a partial ordering on the indices $\{1, 2, \ldots, r\}$ by the transitive closure of the relation $\lessdot$ defined via $j \lessdot i$ if $i < j$ and $s_{x_i}$ and $s_{x_j}$ do not commute. In particular, since $\w$ is reduced, $j \lessdot i$ if $s_{x_i}=s_{x_j}$ by transitivity. This partial order is referred to as the \emph{heap} of $\w$, where $i$ is labeled by $s_{x_i}$. Note that for simplicity we are omitting the labels of the underlying poset yet retaining the labels of the corresponding generators.

It follows from~\cite{Stembridge1996} that heaps are well-defined up to commutation class. That is, given two reduced expressions $\w_1$ and $\w_2$ for $w \in W$ that are in the same commutation class, the heaps for $\w_1$ and $\w_2$ will be equal. In particular, if $w \in \FC(\Gamma)$, then $w$ has one commutation class, and Thus, $w$ has a unique heap. Conversely, if $\w_1$ and $\w_2$ are in different commutation classes, then the heap of $\w_1$ will be distinct from the heap of $\w_2$.

\begin{example}\label{ex:word}
Let $\w=s_6s_4s_2s_5s_3s_1s_4s_0s_1$ be a reduced expression for $w \in \FC(\widetilde{C}_6).$ We see that $\w$ is indexed by $\{1,2,3,4,5,6,7,8,9\}$. As an example, $9 \lessdot 8$ since $8 <9$ and $s_0$ and $s_1$ do not commute. The labeled Hasse diagram for the heap poset is seen in Figure~\ref{fig:Hasse}.
\begin{figure}[h]
\centering
\begin{tikzpicture}[scale=0.5]
	\node[scale=0.6, label=left:$s_{2}$] at (2,5.5) {$\bullet$};
	\node[scale=0.6, label=left:$s_{4}$] at (4,5.5) {$\bullet$};
	\node[scale=0.6, label=right:$s_{6}$] at (6,5.5) {$\bullet$};
	\node[scale=0.6, label=left:$s_{1}$] at (1,4.5) {$\bullet$};
	\node[scale=0.6, label=left:$s_{3}$] at (3,4.5) {$\bullet$};
	\node[scale=0.6, label=right:$s_{5}$] at (5,4.5) {$\bullet$};
	\node[scale=0.6, label=left:$s_{0}$] at (0,3.5) {$\bullet$};
	\node[scale=0.6, label=left:$s_{4}$] at(4,3.5) {$\bullet$};
	\node[scale=0.6, label=left:$s_{1}$] at (1,2.5) {$\bullet$};
\draw (1,2.5)--(0,3.5)--(1,4.5)--(2,5.5)--(3,4.5)--(4,3.5);
\draw (3,4.5)--(4,5.5)--(5,4.5)--(4,3.5);
\draw (6,5.5)--(5,4.5);
\end{tikzpicture}
\caption{Labeled Hasse diagram for the heap of an element in $\FC(\widetilde{C}_6).$}
\label{fig:Hasse}	
\end{figure}
\end{example}

Let $\w$ be a reduced expression for an element $w \in W(\widetilde{C}_n)$. As in~\cite{Billey2007} and~\cite{Ernst2010} we can represent a heap of $\w$ as a set of lattice points embedded in $\{0,1,2,\ldots, n\} \times \mathbb{N}$. To do so, we assign coordinates (not unique) $(x,y) \in \{0,1,2,\ldots, n\} \times \mathbb{N}$ to each entry of the labeled Hasse diagram for the heap of $\w$ in such a way that:
\begin{enumerate}[leftmargin=2cm]
\item An entry with coordinates $(x,y)$ is labeled $s_i$ (or $i$) in the heap if and only if $x = i$; 

\item If an entry with coordinates $(x,y)$ is greater than an entry with coordinates $(x',y')$ in the heap then $y > y'$.
\end{enumerate}

Although the above is specific to $W(\widetilde{C}_n)$, the same construction works for any straight-line Coxeter graph with the appropriate adjustments made to the label set and assignment of coordinates. Specifically, for type $A_n$ our label set is $\{1,2, \ldots, n\}$ and for type $B_n$ our label set is $\{0,1, \ldots, n-1\}$.

In the case of any straight-line Coxeter graph, it follows from the definition that $(x,y)$ covers $(x',y')$ in the heap if and only if $x = x' \pm 1$, $y'< y$, and there are no entries $(x'', y'')$ such that $x'' \in \{x, x'\}$ and $y'< y'' < y$. This implies that we can completely reconstruct the edges of the Hasse diagram and the corresponding heap poset from a lattice point representation. The lattice point representation can help us visualize arguments that are potentially complex. Note that in our heaps the entries fully exposed to the top (respectively, bottom) correspond to the generators occurring in the left (respectively, right) descent set of the corresponding reduced expression.

Let $\w$ be a reduced expression for $w \in W(\widetilde{C}_n)$. We denote the lattice point representation of the heap poset in $\{0,1,2, \ldots n\} \times \mathbb{N}$ described in the preceding paragraphs via $H(\w)$. If $w$ is $\FC$, then the choice of reduced expression for $w$ is irrelevant and we will often write $H(w)$ and we refer to $H(w)$ as the heap of $w$. Note that we will use the same notation for heaps in Coxeter groups of all types with straight-line Coxeter graphs.

Let $\w=s_{x_1}s_{x_2}\cdots s_{x_r}$ be a reduced expression for $w \in W(\widetilde{C}_n)$. If $s_{x_i}$ and $s_{x_j}$ are adjacent generators in the Coxeter graph with $i<j$, then we must place the point labeled by $s_{x_i}$ at a level that is \emph{above} the level of the point labeled by $s_{x_j}$. Because generators in a Coxeter graph that are not adjacent do commute, points whose $x$-coordinates differ by more than one can slide past each other or land in the same level. To emphasize the covering relations of the lattice point representation we will enclose each entry in the heap in a square with rounded corners (called a block) in such a way that if one entry covers another the blocks overlap halfway. In addition, we will also label each square for $s_i$ with $i$.

There are potentially many ways to illustrate a heap of an arbitrary reduced expression, each differing by the vertical placement of the blocks. For example, we can place blocks in vertical positions as high as possible, as low as possible, or some combination of low/high. In this thesis, we choose what we view to be the best representation of the heap of each example and when illustrating the heaps of arbitrary reduced expressions we will discuss the relative position of the entries but never the absolute coordinates.

We say that a block in the heap for a reduced expression is \emph{fully exposed} to the top (respectively, bottom) to mean that the top (respectively, bottom) edge of a heap block is not covered by any blocks above (respectively, below) in the heap. That is, there are no blocks that cover part of the top or bottom edge of the heap. Since there are multiple heap representations when $w \in W(\Gamma)$ is not FC, it is possible that a block that is fully exposed in one heap may not be fully exposed in a different heap representing $w$. 

\begin{example}
	Let $\w=s_6s_4s_2s_5s_3s_1s_4s_0s_1$ be a reduced expression for $w \in \FC(\widetilde{C}_6)$ as seen in Example~\ref{ex:word}. Figure~\ref{fig:FC heap} shows a possible lattice point representation for $H(\w)$. Since $w$ is FC this is the unique heap representation for $w$. Because $\w$ has a unique heap, we can obtain $\LD(w)$ (respectively, $\RD(w)$) from the blocks that are fully exposed to the top (respectively, bottom) of the heap. We see that $\LD(w)=\{s_2,s_4,s_6\}$ and $\RD(w)=\{s_1,s_4\}$.
\begin{figure}[h]
\centering
\begin{tikzpicture}[scale=0.45]
\heapblock{2}{6}{2}{purple}
\heapblock{4}{6}{4}{purple}
\heapblock{6}{6}{6}{purple}
\heapblock{1}{4}{1}{purple}
\heapblock{3}{4}{3}{purple}
\heapblock{5}{4}{5}{purple}
\heapblock{0}{2}{0}{purple}
\heapblock{4}{2}{4}{purple}
\heapblock{1}{0}{1}{purple}
\end{tikzpicture}
\caption{A lattice point representation for the heap of an FC element in $W(\widetilde{C}_6)$.}
\label{fig:FC heap}
\end{figure}
\end{example}

\begin{example}
Let $\w_1=s_0s_2s_4s_3s_2s_1$ be a reduced expression for $w \in W(\widetilde{C}_4)$. Applying the commutation move $s_2s_4\mapsto s_4s_2$, we can obtain another reduced expression for $w$, namely $\w_2=s_0s_4s_2s_3s_2s_1$, which is in the same commutation class as $\w_1$, and hence has the same heap. However, applying the braid move $s_2s_3s_2 \mapsto s_3s_2s_3$, we obtain another reduced expression $\w_3=s_0s_4s_3s_2s_3s_1$. Note that since $\w_3$ was obtained by applying a braid move, $\w_3$ is in a different commutation class than $\w_1$ and $\w_2$. Representations of $H(\w_1), H(\w_2)$, and $H(\w_3)$ are seen in Figure~\ref{fig:not FC}, where the braid relation is colored in \textcolor{teal}{teal}. From the heaps we see that $\LD(w)=\{s_0,s_2,s_4\}$ and $\RD=\{s_1,s_3\}$. However, if we only had one heap or the other, we would miss some elements in the left and right descent sets as $s_3$ is not fully exposed to the bottom of the heap in Figure~\ref{fig:notFC2} and $s_2$ is not fully exposed to the top of the heap in Figure~\ref{fig:notFC3}. 

\begin{figure}[h]
\centering
\begin{subfigure}[b]{0.3\textwidth}	
\centering
\begin{tikzpicture}[scale=0.45]
\heapblock{0}{6}{0}{purple}
\heapblock{2}{6}{2}{teal}
\heapblock{4}{6}{4}{purple}	
\heapblock{3}{4}{3}{teal}
\heapblock{2}{2}{2}{teal}
\heapblock{1}{0}{1}{purple}
\end{tikzpicture}
\caption{$H(\w_1)$ and $H(\w_2)$.}\label{fig:notFC2}
\end{subfigure}
\begin{subfigure}[b]{0.3\textwidth}	
\centering
\begin{tikzpicture}[scale=0.45]
\heapblock{10}{6}{0}{purple}
\heapblock{14}{6}{4}{purple}
\heapblock{13}{4}{3}{teal}
\heapblock{12}{2}{2}{teal}
\heapblock{11}{0}{1}{purple}
\heapblock{13}{0}{3}{teal}
\end{tikzpicture}
\caption{$H(\w_3)$}\label{fig:notFC3}
\end{subfigure}
\caption{Two heaps of a non-FC element in $W(\widetilde{C}_4)$.}	
\label{fig:not FC}
\end{figure}
\end{example}

As for expressions, it will be helpful to have the notion of a subheap. Let $\w=s_{x_1}s_{x_2} \cdots s_{x_r}$ be a reduced expression for $w \in W(\Gamma)$. We define a heap $H'$ to be a \emph{subheap} of $H(\w)$ if $H'=H(\w')$ where $\w'=s_{y_1}s_{y_2} \cdots s_{y_k}$ is a subexpression of $\w$. We emphasize that the subexpression need not be a subword (i.e., a consecutive subexpression).

Recall that a subposet $Q$ of $P$ is called convex if $y \in Q$ whenever $x< y<z$ in $P$ and $x,z \in Q$. We will refer to a subheap as a \emph{convex subheap} if the underlying subposet is convex.

\begin{example}
Let $\w=s_3s_2s_1s_2s_5s_4s_6s_5$ be a reduced expression for $w \in W(\C_7)$. Now let $\w'=s_5s_4s_5$ be the subexpression of $\w$ that results from deleting all but the fifth, sixth, and last generators of $\w$. Then the subheap $H(\w')$ is seen in Figure~\ref{fig:subheap}. However, $H(\w')$ is not convex since there is an entry in $H(w)$ labeled by $s_6$ occurring between the two consecutive occurrences of $s_5$ that does not occur in $H(w')$. However, if we do include the entry labeled by $s_6$, then we get the subheap seen in Figure~\ref{fig:convex}, which is convex.

\begin{figure}[h]
\centering
\begin{subfigure}[b]{0.3\textwidth}	
\centering
\begin{tikzpicture}[scale=0.45]
%	\heapblock{0}{6}{0}{purple}
%	\heapblock{2}{6}{2}{teal}
%	\heapblock{4}{6}{4}{purple}	
	\heapblock{3}{4}{5}{purple}
	\heapblock{2}{2}{4}{purple}
	\heapblock{3}{0}{5}{purple}
\end{tikzpicture}
\caption{}\label{fig:subheap}
\end{subfigure}
\begin{subfigure}[b]{0.3\textwidth}	
\centering
\begin{tikzpicture}[scale=0.45]
%	\heapblock{10}{6}{0}{purple}
%	\heapblock{14}{6}{4}{purple}
	\heapblock{13}{4}{5}{purple}
	\heapblock{12}{2}{4}{purple}
	\heapblock{14}{2}{6}{purple}
	\heapblock{13}{0}{5}{purple}
\end{tikzpicture}
\caption{}\label{fig:convex}
\end{subfigure}
\caption{Subheap and convex subheap of the heap for an element in $W(\C_7)$.}	
\end{figure}
\end{example}


It will be extremely useful for us to be able to quickly determine whether a heap corresponds to an element in $\FC(B_n)$ or $\FC(\widetilde{C}_n)$. The next proposition is a special case of \cite[Proposition 3.3]{Stembridge1996} and follows easily when one considers the consecutive subwords that are impermissible in reduced expressions for elements in $\FC(B_n)$ and $\FC(\widetilde{C}_n)$ as discussed in Section~\ref{sec:FC}.

\begin{proposition}\label{prop:impermiss}
Let $(W,S)$ be a Coxeter system of type $\C_n$. If $w \in \FC(\widetilde{C}_n)$, then $H(w)$ cannot contain any of the configurations seen in Figure~\ref{fig:impermiss heaps}, where $0 < k < n-1$ and we use a square with a dotted boundary to emphasize that no element of the heap may occupy the corresponding position. \qed
\end{proposition}

\begin{figure}[h]
\begin{tabular}{m{4cm} m{4cm} m{4cm}}
	\begin{subfigure}{0.33\textwidth} \centering
	\begin{tikzpicture}[scale=0.45]
		\heapblock{1}{6}{1}{purple}
		\dheapblock{2}{4}{}{black}
		\heapblock{0}{4}{0}{purple}
		\heapblock{1}{2}{1}{purple}
		\heapblock{0}{0}{0}{purple}
	\end{tikzpicture}
	\caption{}\label{fig:C&B}
	\end{subfigure} &
	
	\begin{subfigure}{0.33\textwidth} \centering
	\begin{tikzpicture}[scale=0.45]
		\heapblock{0}{6}{0}{purple}
		\heapblock{1}{4}{1}{purple}
		\heapblock{0}{2}{0}{purple}
		\dheapblock{2}{2}{}{black}
		\heapblock{1}{0}{1}{purple}
	\end{tikzpicture}
	\caption{}\label{fig:C&B2}
	\end{subfigure} &

	\begin{subfigure} {0.33\textwidth} \centering
	\begin{tikzpicture}[scale=0.45]
		\heapblock{1}{8}{}{white}
		\heapblock{1}{6}{k+1}{purple}
		\heapblock{0}{4}{k}{purple}
		\dheapblock{2}{4}{}{black}
		\heapblock{1}{2}{k+1}{purple}
	\end{tikzpicture}
	\caption{}\label{fig:a,b,c}
	\end{subfigure}\\
	
	&\\
	
	\begin{subfigure}{0.33\textwidth} \centering
	\begin{tikzpicture}[scale=0.45]
		\heapblock{1}{8}{}{white}
		\heapblock{2}{6}{k}{purple}
		\heapblock{3}{4}{k+1}{purple}
		\dheapblock{1}{4}{}{black}
		\heapblock{2}{2}{k}{purple}
	\end{tikzpicture}
	\caption{}\label{fig:a,b,c2}
	\end{subfigure}&
	
	\begin{subfigure}{0.33\textwidth} \centering
	\begin{tikzpicture}[scale=0.45]
		\heapblock{2}{6}{n}{purple}
		\heapblock{1}{4}{n-1}{purple}
		\heapblock{2}{2}{n}{purple}
		\dheapblock{0}{2}{}{black}
		\heapblock{1}{0}{n-1}{purple}
	\end{tikzpicture}
	\caption{}\label{fig:cimpermiss}	
	\end{subfigure} &
	
	\begin{subfigure}{0.33\textwidth} \centering
	\begin{tikzpicture}[scale=0.45]
		\heapblock{1}{6}{n-1}{purple}
		\dheapblock{0}{4}{}{black}
		\heapblock{2}{4}{n}{purple}
		\heapblock{1}{2}{n-1}{purple}
		\heapblock{2}{0}{n}{purple}
	\end{tikzpicture}	
	\caption{}\label{fig:cimpermiss2}
	\end{subfigure}
\end{tabular}	
\caption{Impermissible configurations for heaps of $\FC(\widetilde{C}_n)$.}\label{fig:impermiss heaps}
\end{figure}

Since $W(B_n)$ is a parabolic subgroup of $W(\widetilde{C}_n)$, we can use Figure~\ref{fig:impermiss heaps} to classify the impermissible configurations for elements of $\FC(B_n)$. In particular, the impermissible configurations for elements of $\FC(B_n)$ are those seen in Figures~\ref{fig:C&B},~\ref{fig:C&B2}~\ref{fig:a,b,c}, and~\ref{fig:a,b,c2}.

%%%%%%%%%%%%%%%%%%%%%%%%%%%%%%%%

\section{Star Reductions and Property T}

\subsection{Star Reductions}\label{sec:star}

The notion of a star operation was originally introduced by Kazhdan and Lusztig in~\cite{Kazhdan1979} for simply-laced Coxeter systems (i.e., $m(s,t) \leq 3$ for all $s,t \in S$), and was later generalized to all Coxeter systems in~\cite{Lusztig1985}. If $I=\{s,t\}$ is a pair of non-commuting generators of a Coxeter group $W$, then $I$ induces four partially defined maps from $W$ to itself, known as \emph{star operations}. A star operation, when it is defined, increases or decreases the length of an element to which it is applied by 1. For our purposes it is enough to only define the star operations that decrease the length of an element by 1, and as a result we will not develop the notion in full generality.

Let $(W,S)$ be a Coxeter system of type $\Gamma$ and let $I=\{s,t\}\subseteq S$ be a pair of generators with $m(s,t) \geq 3$. Let $w \in W(\Gamma)$ such that $s \in \mathcal{L}(w)$. We say $w$ is \emph{left star reducible by $s$ with respect to $t$} if $m(s,t) \geq 3$, $s \in \LD(w)$, and $t \in \mathcal{L}(sw)$. We analogously define $w$ to be \emph{right star reducible by $s$ with respect to $t$}. Observe that $w$ is left (respectively, right) star reducible if and only if $w=stu$ (respectively, $w=uts$), where the product on the right hand side of the equation is reduced with $u \in W(\Gamma)$ and $m(s,t) \geq 3$. We say that $w$ is \emph{star reducible} if it is either left or right star reducible.

\begin{example}\label{ex:starred}
Let $\w=s_0s_1s_0s_2$ be a reduced expression for $w \in W(B_3)$. We see that $w$ is left star reducible by $s_0$ with respect to $s_1$ to $s_1s_0s_2$ since $m(s_0,s_1)=4$ and $s_0 \in \mathcal{L}(w)$ while $s_1 \in \mathcal{L}(s_0w)$. Notice that $w$ is FC and $\RD(w)=\{s_2,s_0\}$ since $s_0$ and $s_2$ commute. We see that $ws_2=s_0s_1s_0$ and $ws_0=s_0s_1s_2$. Note that in both instances $s_1 \notin \RD(ws_2)=\{s_0\}$ and $s_1 \notin \LD(ws_0)=\{s_2\}$. Because of this $w$ is not right star reducible. 
\end{example}

It may be helpful to visualize star reductions in terms of heaps. Let $(W,S)$ be a Coxeter system with straight-line Coxeter graph $\Gamma$ and let $I=\{s,t\}\subseteq S$ be a pair of generators with $m(s,t) \geq 3$. Suppose $w$ is left star reducible by $s$ with respect to $t$. Then there exists a heap for some reduced expression of $w$ where the block for $s$ is fully exposed to the top such that removing the block for $s$ off of the top allows for $t$ to now be fully exposed to the top of the heap. Similarly, if $w$ is right star reducible by $s$ with respect to $t$, then there exists a heap for some reduced expression of $w$ where the block for $s$ is fully exposed to the bottom of the heap such that removing the block for $s$ off the bottom allows for $t$ to now be fully exposed to the bottom. Conversely, if a heap of $w \in W(\Gamma)$ has this property, then $w$ is star reducible. In Figure~\ref{fig:heapwithT} we see the top portion of two possible heap representations of an element that is left star reducible by $s$ with respect to $t$, where the dotted square indicates that no block may occupy this position.  Notice that flipping the heap upside down in Figure~\ref{fig:heapwithT} will result in a heap that is right star reducible. 

\begin{figure*}[h!]
\begin{tabular}{m{7cm} m{7cm}}
\begin{subfigure}{0.5\textwidth} \centering
\begin{tikzpicture}[scale=0.455]
	\dheapblock{2}{2}{}{black}
	\heapblock{0}{2}{s}{purple}
	\heapblock{1}{0}{t}{purple}
\end{tikzpicture}
\caption{}\label{fig:starleft}
\end{subfigure} &

\begin{subfigure}{0.5\textwidth} \centering
\begin{tikzpicture}[scale=0.455]
	\dheapblock{1}{2}{}{black}
	\heapblock{3}{2}{s}{purple}
	\heapblock{2}{0}{t}{purple}
\end{tikzpicture}
\caption{}\label{fig:starright}	
\end{subfigure}
\end{tabular}
\caption{A visual representation of an element that is left star reducible by $s$ with respect to $t$.}\label{fig:heapwithT}
\end{figure*}   

The following example utilizes heaps to show that an element is star reducible.

\begin{example}\label{ex:starredheap}
Let $\w=s_0s_1s_0s_2$ be a reduced expression for $w \in W(B_4)$. Note that $w$ is FC. By Example~\ref{ex:starred} we know that $w$ is left star reducible by $s_0$ with respect to $s_1$. In Figure~\ref{fig:heapy}, we see the heap of $w$. Notice that the block for $s_0$ is fully exposed to the top of the heap. Removing the block for $s_0$ gives the heap in Figure~\ref{fig:multiplied}. Notice that the block for $s_1$ is now fully exposed to the top of the heap. Hence, $w$ is left star reducible by $s_0$ with respect to $s_1$. However, notice that the blocks for $s_0$ and $s_2$ are fully exposed to the bottom. In removing either of these blocks individually we are unable to fully expose $s_1$ to the bottom. Thus, we can see that $w$ is not right star reducible.  
\end{example}


%It may be helpful to visualize star reductions in terms of heaps. Figure~\ref{fig:heapy} represents $H(\w)$. Note that we can see $s_0$ is in the left descent set of $w$ since $s_0$ is in the top row of the heap. Furthermore, multiplying on the left by $s_0$ we get the heap in Figure~\ref{fig:multiplied}. Again, since $s_1$ is in the top row of the heap, $s_1 \in \mathcal{L}(s_1w)$. In Figure~\ref{fig:heapy} we also see that $s_3$ is in the right descent set of $w$ since $s_3$ is in the bottom row of the heap. Multiplying on the left by $s_3$ we can see that $s_2$ would be in the bottom level of the heap so $s_2 \in \mathcal{R}(ws_3)$. From this we can interpret visually an element $w \in W(\Gamma)$ is right star reducible (respectively, left star reducible) if there exists a heap that we can pull a block off the bottom row of the heap (respectively, top of the heap) and a new block that wasn't previously in the bottom row (respectively, top row) is now in the bottom row (respectively, top row) of the heap. That is, we can systematically dismantle the heap for a given element by pulling blocks off of the bottom row (respectively, top row) of the heap and have the heap decrease in height, meaning there are fewer rows of the heap than there were in the original, or a block that was previously trapped in the second (respectively, second to last) row of the heap is now free to be in either the first or second (respectively, second to last or last) row. If the heap has the same number of rows as the original and if now blocks that were in the second or second to last row of the heap can now be in the first or last row of the heap when we attempt pulling a single block off the top and a single block off the bottom and we try all possible combination of single blocks in the top row and bottom row, then the heap is not star reducible.

\begin{figure}[h!]
\begin{tabular}{m{7cm} m{7cm}}
\begin{subfigure}{0.5\textwidth} \centering
\begin{tikzpicture}[scale=0.455]
\heapblock{2}{2}{2}{purple}
\heapblock{0}{2}{0}{purple}
\heapblock{1}{4}{1}{purple}
\heapblock{0}{6}{0}{purple}
\end{tikzpicture}
\caption{} \label{fig:heapy}
\end{subfigure} &

\begin{subfigure}{0.5\textwidth} \centering
\begin{tikzpicture}[scale=0.455]
\heapblock{2}{2}{2}{purple}
\heapblock{0}{2}{0}{purple}
\heapblock{0}{6}{}{white}
\heapblock{1}{4}{1}{purple}
\end{tikzpicture}
\caption{} \label{fig:multiplied}
\end{subfigure}
\end{tabular}
\caption{Visualization of Example~\ref{ex:starred}.}
\label{fig:starred}
\end{figure}

It is important to note that for non-FC group elements, when we are evaluating for star reducibility we must consider all heap representations for the element before concluding that it is not star reducible. That is, if $w$ is not FC, then we are not be able to say that $w$ is not star reducible when viewing a single heap as there could be a different heap for $w$ in which we are able to fully expose a block that was previously blocked in a different heap.

\begin{example}
Let $\w=s_3s_1s_2s_1s_0s_1s_3s_0s_2s_4$ be a reduced expression for $w \in W(\C_3)$. The heap of $\w$ is given in Figure~\ref{fig:starrednfc1}, where we have highlighted a braid in \textcolor{teal}{teal}. Notice that this heap appears to not be star reducible since if we were to remove the block for $s_1$ or $s_3$ individually we would not fully expose $s_2$ to the top of the heap. The same goes for attempting to fully expose blocks in the bottom of the heap. However, when we perform the braid move, resulting in the heap seen in Figure~\ref{fig:starrednfc2}, it is now obvious that the element is star reducible. Thus, when considering a non-FC element for star reducibility via the heap, it is very important to consider all heaps for that element.

\begin{figure}[h!]
\begin{tabular}{m{7cm} m{7cm}}
\begin{subfigure}{0.5\textwidth} \centering
\begin{tikzpicture}[scale=0.455]
	\heapblock{1}{10}{1}{teal}
	\heapblock{3}{10}{3}{purple}
	\heapblock{2}{8}{2}{teal}
	\heapblock{1}{6}{1}{teal}
	\heapblock{0}{4}{0}{purple}
	\heapblock{1}{2}{1}{purple}
	\heapblock{3}{2}{3}{purple}
	\heapblock{0}{0}{0}{purple}
	\heapblock{2}{0}{2}{purple}
	\heapblock{4}{0}{4}{purple}	
\end{tikzpicture}
\caption{}\label{fig:starrednfc1}	
\end{subfigure}&

\begin{subfigure}{0.5\textwidth} \centering
\begin{tikzpicture}[scale=0.455]
	\heapblock{2}{10}{2}{teal}
	\heapblock{3}{12}{3}{purple}
	\heapblock{1}{8}{1}{teal}
	\heapblock{2}{6}{2}{teal}
	\heapblock{0}{6}{0}{purple}
	\heapblock{1}{4}{1}{purple}
	\heapblock{3}{4}{3}{purple}
	\heapblock{0}{2}{0}{purple}
	\heapblock{2}{2}{2}{purple}
	\heapblock{4}{2}{4}{purple}	
\end{tikzpicture}
\caption{}\label{fig:starrednfc2}	
\end{subfigure}
\end{tabular}
\caption{Visualization of Example 2.1.3.}\label{fig:starrednfc}
\end{figure}
\end{example}


We say that $w \in W(\Gamma)$ is \emph{star reducible to a product of commuting generators} if there is a sequence
\[w_1=w \mapsto w_2 \mapsto \cdots \mapsto w_n\]
where for each $1 \leq i \leq n$, $w_{i}$ is left star reducible or right star reducible to $w_{i+1}$ with respect to some pair $\{s_i, t_i\}$, and $w_n$ is a product of commuting generators. Using the notion of star reduction we are now able to introduce the concept of a star reducible Coxeter group. Let $(W,S)$ be a Coxeter group of type $\Gamma$. We say that $(W,S)$ or $W(\Gamma)$ is \emph{star reducible} if every element of $\FC(\Gamma)$ is star reducible to a product of commuting generators. Notice that we are restricting to just the FC elements in $W(\Gamma)$. Visually a star reducible Coxeter group can be thought of in the following way. Given a heap for an element in $\FC(\Gamma)$, we are able to systematically remove fully exposed blocks from the top or bottom of the heap and at each step have a block that was previously not fully exposed become fully exposed until we are left with a heap that can be drawn as a single row. 

In~\cite{Green2006a}, Green classified all star reducible Coxeter groups.

%we are able to pull the top or bottom most block off of the heap and have the heap decrease in height (the number of rows that it consists of) or have a new block come into the top or bottom most rows. We can perform this pulling of blocks off the top and bottom of the heap systematically until the heap consists of one row, corresponding to an element that is a product of commuting generators. For example in Figure~\ref{fig:heapy}, we were able to apply a star reduction to remove the topmost block corresponding to the generator $s_0$ and obtain the new heap seen in Figure~\ref{fig:multiplied}. We see that the heap in Figure~\ref{fig:multiplied} has one less row than Figure~\ref{fig:heapy}. We could do the same with the block on bottom of Figure~\ref{fig:multiplied} corresponding to $s_3$ and have the heap again decrease in the number of rows it has. Performing one more star reduction in pulling the brick off the top of Figure~\ref{fig:multiplied} which corresponds to $s_1$ in the reduced expression would leave us with a heap that is only 1 row which is a product of commuting generators. Thus, Figure~\ref{fig:heapy} can be star reduced to a product of commuting generators.  In~\cite{Green2006a}, Green classified all star reducible Coxeter groups. 
%The Coxeter groups $W(A_n)$, $W(B_n)$ and $W(\widetilde{C}_n)$ are star reducible. However, $W(A_n)$ and $W(B_n)$ don't have $\tII$ elements, while $W(\widetilde{C}_n)$ in one parity does have $\tII$ elements.
\begin{proposition}[Green,~\cite{Green2006a}]\label{prop:starredcoxgrp}
	Let $(W,S)$ be a Coxeter system of type $\Gamma$. Then $(W,S)$ is star reducible if and only if each component of $\Gamma$ is either a complete graph with labels $m(s,t)\geq 3$ or is one of the following types: type $A_n$ $(n \geq 1)$, type $B_n$ $(n \geq 2)$, type $D_n$ $(n \geq 4)$, type $F_n$ $(n \geq 4)$, type $H_n$ $(n \geq 2)$, type $I_2(m)$ $(m \geq 3)$, type $\widetilde{A}_{n}$ $(n \geq 3 \textrm{ and } n \textrm{ even})$, type $\widetilde{C}_{n}$ $(n\geq 3 \textrm{ and } n \textrm{ odd})$, type $\widetilde{E}_6$, or type $\widetilde{F}_5$. \qed
\end{proposition}    

%%%%%%%%%%%%%%%%%

\subsection{Property T}\label{sec:Tavoid}

In~\cite{Green2006a}, Green utilizes the following theorem to help classify the star reducible Coxeter groups. 
\begin{proposition}[Green,~\cite{Green2006a}, Theorem 4.1]\label{thm:starred}
	Let $(W,S)$ be a star reducible Coxeter system of type $\Gamma$, and let $w \in W$. Then one of the following possibilities occurs for some Coxeter generators $s,t, u$ with $m(s,t) \neq 2$, $m(t,u) \neq 2$, and $m(s,u)=2$:
	\begin{enumerate}[leftmargin=2cm]
	\item $w$ is a product of commuting generators;\label{it:triv}
	\item $w$ has a reduced product $w=stu$;\label{it:proptend}
	\item $w$ has a reduced product $w=uts$;\label{it:proptbeg}
	\item $w$ has a reduced product $w=sutv$.\label{it:tavoid}	\qed
	\end{enumerate}
\end{proposition}

Notice that Items~\ref{it:proptend} and~\ref{it:proptbeg} indicate an element that is left or right star reducible, respectively. Also notice that an element $w$ that has the form of Item~\ref{it:triv} does not meet the conditions of Items~\ref{it:proptend} and~\ref{it:proptbeg}. In particular, $w$ is not star reducible if it satisfies the condition of Item~\ref{it:triv}. Lastly, note that if an element $w$ is of the form of Item~\ref{it:tavoid} and not of the form of Items~\ref{it:proptend} and~\ref{it:proptbeg}, then $w$ is not star reducible. Note that Items~\ref{it:proptend},~\ref{it:proptbeg}, and~\ref{it:tavoid} are not mutually exclusive.

Motivated by the proposition above, we define the notions of Property T and T-avoiding. Let $(W,S)$ be a Coxeter system of type $\Gamma$ and let $w \in W$. We say that $w$ has \emph{Property T} if and only if there exists a reduced product for $w$ such that $w=stu$ or $w=uts$ where $m(s,t)\geq 3$ and $u \in W$. That is, $w$ has Property T if there exists a reduced expression for $w$ that begins or ends with a product of non-commuting generators. An element $w \in W(\Gamma)$ is called \emph{T-avoiding} if $w$ does not have Property T. This implies that a T-avoiding element is not star reducible.

 Since elements that are star reducible also have Property T we already know how to visualize Property T in terms of heaps. Visually a product of commuting generators be made into a single row heap by pushing all the blocks into the same vertical position. It is clear that a single row heap will not portray the characteristic of Property T as seen in Figure~\ref{fig:heapwithT} and Thus, a product of commuting generators is T-avoiding, which we state as a proposition.


\begin{proposition}\label{thm:trivTavoid}
Let $(W,S)$ be a Coxeter system of type $\Gamma$. If $w \in W(\Gamma)$ such that $w$ is a product of commuting generators, then $w$ is T-avoiding. \qed	
\end{proposition}

We will call the identity or an element that is a product of commuting generators \emph{type I} T-avoiding, which we abbreviate as \emph{$\tI$}. If $w$ is T-avoiding and not a of type I, we will say that $w$ is \emph{type II} T-avoiding, which we abbreviate as \emph{$\tII$}. It is not clear that $\tII$ elements exist. Referring back to Green's classification (Proposition~\ref{thm:starred}) of what elements in star reducible Coxeter groups look like, we see that Item~\ref{it:triv} corresponds to an element $w$ being $\tI$, Items~\ref{it:proptend} and~\ref{it:proptbeg} refer to the element $w$ having Property T on the left and right, respectively and Item~\ref{it:tavoid} refers to an element being $\tII$ provided no reduced expression for the element exhibits Items~\ref{it:proptend} and~\ref{it:proptbeg}. In star reducible Coxeter systems, every FC element is star reducible to a product of commuting generators, which implies that no FC element can be $\tII$ in such groups. For example, as will be seen in Chapters~\ref{chap:TandTavoid} and~\ref{chap:BnandCn}, the Coxeter systems of type $A_n$ and $B_n$ have no $\tII$ elements, while the Coxeter systems of type $D_n$ do.

%Visually this is seen in Example~\ref{fig:heapnoT} elements in $W(\Gamma)$ that are products of commuting generators are always going to be one row in the heap. This implies that we are not able to remove generators and have elements come into rows that they were previously in as the heap is only one row and there can be no lateral movement when we remove bricks. 

\begin{example}\label{ex:tavoid}
Let $\w=s_1s_3s_5$ be a reduced expression for $w \in W(A_5)$.  Since $w$ is a product of commuting generators, by Proposition~\ref{thm:trivTavoid} we know that $w$ is $\tI$. %In Figure~\ref{fig:heapnoT} we see the heap for $w_2$. %Note that as the heap is only one row and $w_2$ is FC, it is clear that $w_2$ does not have Property T.
\end{example}
\begin{example}\label{ex:prop-T}
Let $\w_1=s_5s_3s_2s_4s_1$ be a reduced expression for $w \in W(A_5)$. At first glance it may appear that $w$ does not have Property T since both $s_1$ and $s_4$ commute as well as $s_3$ and $s_5$. However, note that applying the commutation move $s_4s_2 \mapsto s_2s_4$ results in $\w_2=s_1s_2s_4s_3s_5$. Hence $w$ has Property T since $m(s_1,s_2)=3$ and there is a reduced expression for $w$ that begins with $s_1s_2$. In Figure~\ref{fig:heapw/T} we see the heap of $w$. Note that we can see Property T in the bottom of the heap highlighted in \textcolor{orange}{orange}. In addition to the \textcolor{orange}{orange} highlighted subheap, $w$ also has Property T with respect to $s_3$ and $s_2$ in the top of the heap, and $s_4$ and $s_5$ in the bottom of the heap.
\end{example}

\begin{figure}[h!]\centering
%\begin{tabular}{m{7cm} m{7cm}}
%\begin{subfigure}{0.5\textwidth} \centering
\begin{tikzpicture}[scale=0.455]
\heapblock{5}{6}{5}{purple}
\heapblock{3}{6}{3}{purple}
\heapblock{2}{4}{2}{orange}
\heapblock{4}{4}{4}{purple}
\heapblock{1}{2}{1}{orange}
\end{tikzpicture}
\caption{Heap of an element with Property T.} \label{fig:heapw/T}	
\end{figure}

%\begin{subfigure}{0.5\textwidth} \centering
%\begin{tikzpicture}[scale=0.455]
%\heapblock{3}{4}{}{white}
%\heapblock{3}{8}{}{white}
%\heapblock{1}{6}{1}{purple}
%\heapblock{3}{6}{3}{purple}
%\heapblock{5}{6}{5}{purple}
%\end{tikzpicture}
%\caption{Heap of a T-Avoiding element}\label{fig:heapnoT}
%\end{subfigure}
%\end{tabular}
%\caption{Heaps of an element with Property T and a T-Avoiding element}\label{fig:prptheaps}
%\end{figure}
%As with star reducible elements it may be helpful to visualize Property T through heaps. Figure~\ref{fig:heapw/T} provides a representation of an element in $W(\Gamma)$ with Property T and Figure~\ref{fig:heapnoT} provides a representation of an element without Property T. Notice that if we were to remove the block for $s_1$ in the bottom row of Figure~\ref{fig:heapw/T}, the heap would become one less row in height and we would have a new bottom row in the heap. However, in Figure~\ref{fig:heapnoT}, we are not able to remove able to remove any bricks and have a new brick come to the top or bottom row as the heap is just one row. From this we can gather that when we are using heaps to visualize whether or not an element of $W(\Gamma)$ has Property T we must observe either of the following things. The first of these is that the heap decreased in height. That is, there is one less row than the original heap. The second thing that we could observe is that when we remove a block from the heap a new element, that was originally trapped in the second (respectively, second to last) row is now able to move into the first (respectively, last) row of the heap. This implies that an element that does not have Property-T, has every element in the second and second to last row of the heap blocked by two blocks in the first and last rows of the heap, as this would imply that we could not remove a block and have a new element come to the first or last row as it would still be blocked by the other element that remained.

%\begin{example}
%Let $\w_1=s_1s_4s_2s_3s_5$ be a reduced expression for $w_1 \in W(A_5)$ as seen in Example~\ref{ex:prop-T}	and let $\w_2=s_1s_3s_5$ be a reduced expression for $w_2 \in W(A_5)$.  
%\end{example}
%

\begin{example}
Let $\w=s_0s_2s_4s_1s_3s_0s_2s_4$ be a reduced expression for $w \in W(\C_4)$. The heap of $w$ is seen in Figure~\ref{fig:sandwich1}. It turns out that $w$ is FC and $\tII$. Notice that no matter which single block we remove that is fully exposed to the top of the heap no new element becomes fully exposed. The same applies to the bottom of the heap. Thus, $w$ is $\tII$. 
\begin{figure}[h!]
\centering
\begin{tikzpicture}[scale=0.455]
\heapblock{0}{6}{0}{purple}
\heapblock{2}{6}{2}{purple}
\heapblock{4}{6}{4}{purple}
\heapblock{1}{4}{1}{purple}
\heapblock{3}{4}{3}{purple}
\heapblock{0}{2}{0}{purple}
\heapblock{2}{2}{2}{purple}
\heapblock{4}{2}{4}{purple}
\end{tikzpicture}
\caption{Heap of a $\tII$ element in $W(\widetilde{C}_4)$.}\label{fig:sandwich1}	
\end{figure}
\end{example}

%We will now extend the definition of Property T to Coxeter groups. Let $(W,S)$ be a Coxeter group of type $\Gamma$. We say that $W(\Gamma)$ has \emph{Property T} if all elements in $W(\Gamma)$ that are not the products of commuting generators have Property T. It is clear that if a Coxeter group has Property T, then it also has Property S. It remains an open question whether or not a Coxeter group with Property S, also has Property T, but we conjecture this is true. If this is the case, then by a remark following the definition of Property S in~\cite{Green2007}, the classification of the T-avoiding elements for any connected, nonbranching Coxeter graph of finite or affine type, except the Coxeter system of type $\widetilde{F}_4$, would be complete. 

One thing to notice here is that all Coxeter groups have $\tI$ elements as the identity is $\tI$ and  they also contain products of commuting generators, since individual elements of $S$ are considered products of commuting generators. The more interesting $\tII$ elements do not appear in all Coxeter groups. In Chapter~\ref{chap:TandTavoid} we will summarize what is known about the T-avoiding elements in Coxeter systems of types $\widetilde{A}_n$, $A_n$, $D_n$, $F_n$, and $I_2(m)$, and in Chapters~\ref{chap:BnandCn} and~\ref{chap:Cn} we classify the T-avoiding elements in Coxeter systems of types $B_n$ and $\widetilde{C}_n$. 


%%%%%%%%%%%%%%


\subsection{Non-Cancellable Elements}\label{sec:noncancel}
 
We now introduce the concept of weak star reducibility, which is related to the notion of cancellable in~\cite{Fan1997}. Let $(W,S)$ be a Coxeter system of type $\Gamma$ and let $I=\{s,t\} \subseteq S$ be a pair of non-commuting generators. If $w  \in \FC(\Gamma)$, then $w$ is \emph{left weak star reducible by $s$ with respect to $t$ to $sw$} if
\begin{enumerate}[leftmargin=2cm]
\item $w$ is left star reducible by $s$ with respect to $t$, and
\item $tw \notin \FC(\Gamma)$.	
\end{enumerate}
Notice that Condition (2) implies that $l(tw)>l(w)$. Also note that we are restricting our definition of weak star reducible to the set of $\FC$ elements of $W(\Gamma)$. We analogously define \emph{right weak star reducible by $s$ with respect to $t$ to $ws$}. We say that $w$ is \emph{weak star reducible} if $w$ is either left or right weak star reducible. Otherwise, we say that $w$ is \emph{non-cancellable}. Notice that from this we know that if $w \in W(\Gamma)$ is weak star reducible, the $w$ is star reducible. However, $w$ being star reducible does not imply that $w$ is weak star reducible.

%\begin{example}\label{ex:noncancel}
% Let $\w=s_0s_1s_0s_2$ be a reduced expression for $w \in W(B_4)$. From Example~\ref{ex:starred} we know that $w$ is left star reducible. Also, $tw=s_1s_0s_1s_0s_2$, which is not in $\FC(B_4)$. Thus, we see that $w$ is left weak star reducible by $s_0$ with respect to $s_1$ to $s_1s_0s_2$. In addition, Example~\ref{ex:starred} showed that $w$ is not right star reducible and hence $w$ is not right weak star reducible. 
%\end{example}

Again it might be useful to visualize the concept of weak star reducibility in terms of heaps. Recall that in Section~\ref{sec:star} we described what a star reduction looks like in terms of heaps. Since the definition of weak star reducible includes that the element a heap represents is star reducible we again need to have those properties illustrated in Figure~\ref{fig:heapw/T}. In addition, for a heap to be weak star reducible, adding the block that becomes fully exposed when a block is removed from the heap must create a braid in the heap forcing the new larger heap to not be FC. That is, one of the impermissible configurations seen in Section~\ref{sec:Heaps} will appear at the top or bottom of the heap.

\begin{example}
	Let $\w=s_0s_1s_0s_2$ be a reduced expression for $w \in W(B_4)$ as in Example~\ref{ex:noncancel}. Figure~\ref{fig:heapin2.3.2} shows the heap of $w$. Notice that in the heap we can clearly see that $w$ is left star reducible by $s_0$ with respect to $s_1$. In Figure~\ref{fig:weakstarbraid} we see that adding $s_1$ to the top of the heap creates a braid which is highlighted in \textcolor{orange}{orange}. Therefore, $w$ is left weak star reducible by $s_0$ with respect to $s_1$ to $\w=s_1s_0s_2$. In addition, Example~\ref{ex:starred} showed that $w$ is not right star reducible and hence $w$ is not right weak star reducible.
\end{example}
  

%Recall in Figure~\ref{fig:heapy} we have a representation for $w$ as described in Example~\ref{ex:noncancel}. In Section~\ref{sec:star}, we described how  In Figure~\ref{fig:noncancel} we can see that when we multiply $w$ by $s_1$ on the right we end up with a braid, which is  highlighted in orange. Since the heap in Figure~\ref{fig:noncancel} has the impermissible subheap seen in Figure~\ref{fig:C&B2}, $s_1w \notin \FC(B_4)$.  When using heaps to identify whether an element $w$ of $\FC(\Gamma)$ is Non-Cancellable or not there are two key properties that must be observed. The first of the properties is that the $H(w)$ must be able to be dismantled from the top or bottom so that the heap resulting from pulling the top block off or the bottom block off is one row shorter or allows for a new brick to be in the top most or bottom most row of the heap. The second property that must be observed is that given the block that appears in the top or bottom of the heap when the star operation is performed, adding another of that block to the top or bottom of the original heap respectively will result in an impermissible subheap appearing. If both of these properties occur, then the element that corresponds to the heap is not Non-Cancellable.

\begin{figure}[h!]
\begin{tabular}{m{7cm} m{7cm}}
\begin{subfigure}{0.5\textwidth}\centering
\begin{tikzpicture}[scale=0.455]
	\heapblock{1}{6}{}{white}
	\heapblock{0}{0}{0}{purple}
	\heapblock{2}{0}{2}{purple}
	\heapblock{1}{2}{1}{purple}
	\heapblock{0}{4}{0}{purple}
\end{tikzpicture}	
\caption{Heap of $w$}\label{fig:heapin2.3.2}
\end{subfigure}&


\begin{subfigure}{0.5\textwidth}\centering
\begin{tikzpicture}[scale=0.455]
\heapblock{2}{0}{2}{purple}
\heapblock{0}{0}{0}{orange}
\heapblock{1}{2}{1}{orange}
\heapblock{0}{4}{0}{orange}
\heapblock{1}{6}{1}{orange}
\end{tikzpicture}
\caption{Heap of $s_1w$}\label{fig:weakstarbraid}
\end{subfigure}
\end{tabular}
\caption{Heap of a weak star reducible element of $\FC(B_4)$.} \label{fig:noncancel}
\end{figure}

\begin{example}
Let $w \in \FC(B_4)$ and let $\w=s_0s_1$ be a reduced expression for $w$. Note that $w$ is left (respectively, right) star reducible by $s_0$ with respect to $s_1$ (respectively, by $s_1$ with respect to $s_0$). However, $s_1s_0s_1 \in \FC(B_4)$ (respectively, $s_0s_1s_0 \in \FC(B_4)$).  The corresponding heap for $w$ appears in Figure~\ref{fig:noncancelvisual}. Clearly when $s_0$ is added to the bottom of the heap, the new heap is still in $\FC(B_4)$ and the same can be said when $s_1$ is added to the top of the heap. Thus, $w$ is non-cancellable.
\end{example}

\begin{figure*}[h!] \centering
\begin{tikzpicture}[scale=0.455]
\heapblock{0}{2}{0}{purple}
\heapblock{1}{0}{1}{purple}	
\end{tikzpicture}	
\caption{Heap of a non-cancellable element of $\FC(B_4)$.}\label{fig:noncancelvisual}
\end{figure*}

In~\cite{Ernst2010}, Ernst classified the non-cancellable elements in Coxeter systems of type $W(B_n)$ and $W(\C_n)$. We will state part of the classification here as it is important to the development of the $\tII$ elements in $W(\C_n)$ for $n$ odd. For the full classification see~\cite[Sections 4.2 and 5]{Ernst2010}. 

Before we state the classification we first define a specific group element in $W(\C_n)$ for $n$ odd which we will refer to as a \emph{sandwich stack}, an example of which is seen in Figure~\ref{fig:singsandstack}. Notice that this element has full support, is FC, and is $\tII$.

\begin{figure}[h!] \centering
\begin{tikzpicture}[scale=0.455]
	\heapblock{0}{4}{0}{purple}
	\heapblock{2}{4}{2}{purple}
	\node[] at (4,4){$\cdots$};
	\heapblock{6}{4}{n-2}{purple}
	\heapblock{8}{4}{n}{purple}
	\heapblock{1}{2}{1}{purple}
	\heapblock{3}{2}{3}{purple}
	\node[] at (5,2){$\cdots$};
	\heapblock{7}{2}{n-1}{purple}
	\heapblock{0}{0}{0}{purple}
	\heapblock{2}{0}{2}{purple}
	\node[] at (4,0){$\cdots$};
	\heapblock{6}{0}{n-2}{purple}
	\heapblock{8}{0}{n}{purple}
\end{tikzpicture}
\caption{Heap of a sandwich stack in $\FC(\C_n)$ for $n$ odd.}\label{fig:singsandstack}
\end{figure}

We can extend this pattern to the heap seen in Figure~\ref{fig:stacksandstack}. Like the smaller example in Figure~\ref{fig:singsandstack} the element that corresponds to this heap has full support, is FC, and is $\tII$. We define \emph{sandwich stacks} to be the elements in $W(\C_n)$ for $n$ odd that have a heap of the form given in Figure~\ref{fig:stacksandstack}.

\begin{figure}[h!] \centering
\begin{tikzpicture}[scale=0.455]
	\heapblock{0}{4}{0}{purple}
	\heapblock{2}{4}{2}{purple}
	\node[] at (4,4){$\cdots$};
	\heapblock{6}{4}{n-2}{purple}
	\heapblock{8}{4}{n}{purple}
	\heapblock{1}{2}{1}{purple}
	\heapblock{3}{2}{3}{purple}
	\node[] at (5,2){$\cdots$};
	\heapblock{7}{2}{n-1}{purple}
	\heapblock{0}{0}{0}{purple}
	\heapblock{2}{0}{2}{purple}
	\node[] at (4,0){$\cdots$};
	\heapblock{6}{0}{n-2}{purple}
	\heapblock{8}{0}{n}{purple}
	\node[] at (5,-2){$\vdots$};
	\heapblock{1}{-4}{1}{purple}
	\heapblock{3}{-4}{3}{purple}
	\node[] at (5,-4){$\cdots$};
	\heapblock{7}{-4}{n-1}{purple}
	\heapblock{0}{-6}{0}{purple}
	\heapblock{2}{-6}{2}{purple}
	\node[] at (4,-6){$\cdots$};
	\heapblock{6}{-6}{n-2}{purple}
	\heapblock{8}{-6}{n}{purple}
\end{tikzpicture}	
\caption{Heap of a sandwich stack in $\FC(\C_n)$ for $n$ odd.}\label{fig:stacksandstack}
\end{figure}

\begin{remark}\label{rem:noncancel}
	 In Coxeter systems of type $\C_n$ for $n$ odd, the sandwich stacks are the only $\tII$ non-cancellable elements with full support. All other types of non-cancellable elements in $W(\C_n)$ ($n$ odd) that were classified in~\cite{Ernst2010} do not have full support or have Property T. This is important to our classification of T-avoiding elements in $W(\C_n)$ for $n$ odd.
\end{remark}



%\begin{theorem}
%Let $w \in \FC(B_n)$. Then $w$ is non-cancellable if and only if $w$ is either a product of commuting generators, $s_0s_1u$, and $s_1s_0u$ where $u$ is a product of commuting generators with $s_1, s_2,s_3 \in \	supp(w)$. \textcolor{red}{Dana this isn't related to the thesis but to satisfy my curiosity why can't $s_3$ be in $supp(w)$} \qed
%\end{theorem}
%
%\begin{figure}[h!]
%\begin{tabular}{m{7cm} m{7cm}}
%\begin{subfigure}{0.5\textwidth} \centering
%\begin{tikzpicture}[scale=0.455]
%	\heapblock{0}{6}{0}{rred}
%	\heapblock{1}{4}{1}{rred}
%	\heapblock{3}{4}{}{purple}
%	\heapblock{5}{4}{}{purple}
%	
%	\node[] at (7,4){$\cdots$};
%	
%	\heapblock{9}{4}{}{purple}
%\end{tikzpicture}
%%\caption{}	
%\end{subfigure}&
%
%\begin{subfigure}{0.5\textwidth}\centering
%	\begin{tikzpicture}[scale=0.455]
%	\heapblock{1}{6}{1}{rred}
%	\heapblock{0}{4}{0}{rred}
%	\heapblock{2}{4}{}{purple}
%	\heapblock{4}{4}{}{purple}
%	
%	\node[] at (6,4){$\cdots$};
%	
%	\heapblock{8}{4}{}{purple}
%\end{tikzpicture}
%%\caption{}
%\end{subfigure}
%\end{tabular}
%\caption{Visualization of all non-cancellable elements in $W(B_n)$.}
%\end{figure}
%\end{theorem}

%%%%%%%%%%%%%%%%%%%%%%%%%%%%%%%%

\section{T-Avoiding Elements in Coxeter groups of Types $\widetilde{A}_n, A_n, D_n, F_n$, and $I_2(m)$}\label{chap:TandTavoid}

%%%%%%%%%%%%%%%%%%%%%%%%%%
In this chapter we classify the T-avoiding elements in Coxeter systems of types $\widetilde{A}_n, A_n, D_n, F_n$, and $I_2(m)$. All of the results presented in this Chapters are previously known with the exception of $I_2(m)$.


\subsection{Types $\widetilde{A}_n$ and $A_n$}\label{sec:tavoidA}
In this section we state what is already known classification regarding T-avoiding elements in Coxeter systems of type $\widetilde{A}_n$ and $A_n$ and we present a conjecture regarding the $\tII$ elements in $W(\widetilde{A}_n$ for $n$ even. We first focus on $W(\widetilde{A}_n)$.

\begin{proposition}
 If $n \geq 2$ and $n$ is odd, then there are no $\tII$ elements in $W(\widetilde{A}_n)$. Otherwise, if $n \geq 2$ and $n$ is even, then $W(\widetilde{A}_n)$ contains $\tII$ elements.
\begin{proof}
	This is~\cite[Proposition~3.1.2]{Fan1999} after a translation of terminology.\qedhere
\end{proof}
\end{proposition}

%Now we will classify the $\tII$ elements in $W(A_n)$. Recall that $W(A_n)$ is a star reducible Coxeter group which by our observations above implies that if $W(A_n)$ has $\tII$ elements, these elements will be not FC. In~\cite{Fan1999}, the $\tII$ elements were classified. The classification is seen in the following theorem.

%We will now proceed into the classification of T-avoiding elements in $W(\widetilde{A}_n)$. Similar to $W(A_n)$, $W(\widetilde{A}_n)$ will also have $\tI$ elements since $W(\widetilde{A}_n)$ contains elements that are products of commuting generators. We will now proceed to classifying $\tII$ elements in $W(\widetilde{A_n})$. The following classification is a result from~\cite{Fan1999}.

The classification seen in~\cite{Fan1999} did not specifically classify the $\tII$ elements for type $\widetilde{A}_n$ for $n$ even. The following is our conjecture regarding what the $\tII$ elements are in $W(\widetilde{A}_n)$ for $n$ even.
\begin{conjecture}
	The only $\tII$ elements in $W(\widetilde{A}_n)$ for $n$ even are of the form $(s_0s_2 \cdots s_{n-2}s_ns_1s_3 \cdots s_{n-3}s_{n-1})^k$  for $k \in \mathbb{Z}^+$. 
\end{conjecture} 

Recall that $W(\widetilde{A}_n)$, for $n$ even, is not a star reducible Coxeter group (Proposition~\ref{prop:starredcoxgrp}). Hence it is possible that the $\tII$ elements in $W(\widetilde{A}_n)$, for $n$ even, are FC. Further, as $W(A_n)$ is a parabolic subgroup of $W(\widetilde{A}_n)$ and $W(A_n)$ is a star reducible Coxeter group, any FC $\tII$ elements must have full support. First notice that 
\[w=(s_0s_2 \cdots s_{n-2}s_ns_1s_3 \cdots s_{n-3}s_{n-1})^k\] 
is a reduced product. Also, $w$ is FC, and has full support. In addition, $w$ is in fact T-avoiding. Since $W(\widetilde{A}_n)$ does not have a straight-line Coxeter graph, the heaps in $W(\widetilde{A}_n)$ are more appropriately viewed as three-dimensional. We can envision the element above as a ``castle turret" in which every block is in the wall. As stated in the conjecture we believe that these are the only $\tII$ elements. However, it remains an open question as to whether there are any $\tII$ elements in $W(\widetilde{A}_n)\setminus \FC(\widetilde{A}_n)$. Classifying these $\tII$ elements remains an open problem. We now proceed with the classification of T-avoiding elements in Coxeter groups of type $A_n$. 

\begin{theorem}
There are no $\tII$ elements in $W(A_n)$. 
\begin{proof}
Notice that the Coxeter graph of type $A_n$ can be obtained from the Coxeter graph of type $\widetilde{A}_k$, for $k > n$. This is done by removing the appropriate number of vertices and edges from the Coxeter graph of type $\widetilde{A}_k$. Since $W(\widetilde{A}_k)$ for $k$ even has no $\tII$ elements, this forces $W(A_n)$ to not have $\tII$ elements. Thus, $W(A_n)$ does not have any $\tII$ elements.  
%Since $W(A_n)$ is a parabolic subgroup of $W(\widetilde{A}_n)$ this is a consequence of~\cite[Proposition 3.1.2.]{Fan1999}. Specifically, we can obtain the Coxeter graph of type $A_n$ from the Coxeter graph of type $\widetilde{A}_n$ for $n$ even by removing the appropriate number of vertices and edges. From this we can see that if $W(A_n)$ was to have $\tII$ elements, this would imply that $W(\widetilde{A}_n)$ for $n$ even would also have $\tII$ elements as well. Thus, $W(A_n)$ can not have bad elements.
\end{proof}
\end{theorem}


%%%%%%%%%%%%%%%%

\subsection{Type $D_n$}

In this section we summarize the previously known classification of the T-avoiding elements in Coxeter systems of type $D_n$, seen in~\cite{Gern2013a}. Recall that $W(D_n)$ is a star reducible Coxeter group and as a result any potential $\tII$ elements are not FC.

\begin{proposition}
 There are $\tII$ elements in $W(D_n)$ for $n \geq 4$.
\begin{proof}
	This is a consequence of~\cite[Section 2.2]{Gern2013a}. 
\end{proof}
\end{proposition}

We now will classify these elements as seen in~\cite{Gern2013a}. Before we do so we define interval notation useful to the classification from~\cite[Definition 2.3.1]{Gern2013a}. For $2 \leq i \leq j$ denote the element $s_{i}s_{i+1} \cdots s_{j-1}s_j$ by $[i,j]$. For $i \geq 3$, denote $s_1s_3s_4\cdots s_i$ by $[1,i]$ and for $j \geq 2$ denote $s_1s_2s_3 \cdots s_j$ by $[0,j]$. If $0 \leq j <i$ and $i \geq 2$ define $[j,i]=[i,j]^{-1}$. Finally, for $i,j \geq 3$ denote $s_is_{i-1}s_{i-2} \cdots s_4s_3s_1s_2s_3s_4 \cdots s_j$ by $[-i,j]$. The following determines the classification for T-avoiding elements in $W(D_n)$. 

\begin{proposition}
	Let $w \in W(D_n)$ be $\tII$. 
%	Then $w$ has the signed permutation notation 
%	\[ w_m= \begin{cases}
% 			[(-1)^{\frac{m}{2}},\underline{m},3, \underline{m-2}, 5, \ldots, \underline{4}, m-1, \underline{2}, m+1, m+2, \ldots, n] & \text{if, $m$ is even} \\
% 			[(-1)^{\frac{m-1}{2}}, \underline{m-1}, 3, \underline{m-3}, 5, \ldots, \underline{4}, m-2, \underline{2}, m, m+1, \ldots n] & \text{if, $m$ is odd}	
% \end{cases}
%\]
Then $w=w_ku$ (reduced product) for some $m \leq k \leq n$, where $u$ is the identity or is a product of commuting generators  such that $\supp(u) \subseteq\{s_{m+2}, s_{m+3}, s_{m+4}, \ldots, s_n\}$ and 
\[ w_n=
\begin{cases}
	[2,0][4,0] \cdots [k-2,0][k,0][k-j,k-2j] \cdots [k-1,k-2][k,k]  &  \text{$n$ even}\\
	[2,0][4,0] \cdots [m-2,0][m,0][m-k,m-2k] \cdots [m-1,m-2][m,m] &  \text{$n$ odd}
\end{cases}
\] where $m=k-1$ and 
\[
j= 
\begin{cases}
\frac{k}{2}-2 & \text{if $k$ is even}\\
\frac{k-1}{2}-2 & \text{if $k$ is odd.}
\end{cases}
\]
\begin{proof}
	This is~\cite[Lemmas 2.2.18 and 2.3.4]{Gern2013a}. Although it is not immediately obvious, $w_n$ is reduced and not FC.
\end{proof}
\end{proposition}

In Figure~\ref{fig:Dtavoid}, we see two different elements that are T-avoiding in $W(D_5)$. Notice that the blocks that are highlighted in \textcolor{rred}{red} alternate, this prevents the \textcolor{teal}{teal}-highlighted braid from forcing its way to the top or the bottom of the heap. Due to the fork in the graph we must make slight alterations to heaps for $W(D_n)$. Specifically we allow $s_0$ and $s_1$ to occupy the same horizontal placement. 

\begin{figure}[h!]\centering
%\begin{tabular}{m{7cm} m{7cm}}
%\begin{subfigure}{0.5\textwidth} \centering
%\begin{tikzpicture}[scale=0.45]
%	\heapblock{1}{10}{}{white}
%	\heapblock{1}{12}{}{white}
%	\heapblock{1}{0}{}{white}
%	\heapblock{1}{8}{1}{rred}
%	\heapblock{3}{8}{3}{teal}
%	\heapblock{2}{6}{2}{teal}
%	\heapblock{1}{4}{0}{rred}
%	\heapblock{3}{4}{3}{teal}
%\end{tikzpicture}	
%\caption{}
%\end{subfigure} &

%\begin{subfigure}{0.5\textwidth} \centering
\begin{tikzpicture}[scale=0.45]
	\heapblock{1}{12}{0}{rred}
	\heapblock{3}{12}{3}{purple}
	\heapblock{5}{12}{5}{purple}
	\heapblock{2}{10}{2}{purple}
	\heapblock{4}{10}{4}{purple}
	\heapblock{1}{8}{1}{rred}
	\heapblock{3}{8}{3}{teal}
	\heapblock{2}{6}{2}{teal}
	\heapblock{1}{4}{0}{rred}
	\heapblock{3}{4}{3}{teal}
	\heapblock{2}{2}{2}{purple}
	\heapblock{4}{2}{4}{purple}
	\heapblock{1}{0}{1}{rred}
	\heapblock{3}{0}{3}{purple}
	\heapblock{5}{0}{5}{purple}
\end{tikzpicture}
%\caption{}	
%\end{subfigure}
%\end{tabular}
\caption{Visual representation of a $\tII$ element in $W(D_5)$.}\label{fig:Dtavoid}
\end{figure}

%%%%%%%%%%%%%%%

\subsection{Type $F_n$}

In this section we state the known but unpublished classification of T-avoiding elements in Coxeter systems of type $F_4$ and $F_5$. Note that these results are not needed in Chapters~\ref{chap:BnandCn} and~\ref{chap:Cn}. %Note that all of the following results are unpublished. %Recall that $W(F_n)$ is a star reducible Coxeter group so any $\tII$ element in $W(F_n)$ will not be FC.

We start with the Coxeter system of type $F_5$.  Recall that $W(F_5)$ is a star reducible Coxeter group so any $\tII$ elements will not be FC. Before we begin the classification we introduce the notion of a specific element in $W(F_5)$ called a \emph{bowtie}, which is given by the heap in Figure~\ref{fig:singbowtie}. Note that in Figure~\ref{fig:singbowtie4}, the \textcolor{orange}{orange} blocks correspond to the elements that have bond strength 4. It turns out that the expression determined by this heap is in fact reduced. Looking at the heap in Figure~\ref{fig:singbowtiebraid}, we have highlighted a braid in \textcolor{teal}{teal}. We can obtain a ``stack of bowties" by removing the top-most layer of the given heap of the bowtie and adding a new bowtie to the stack. Doing this repeatedly results in the heap seen in Figure~\ref{fig:stackobowties}. Similar to a single bowtie, the expression that corresponds to a stack of bowties is reduced and not FC. These heaps are referenced in the following unpublished theorem by Cross, Ernst, Hills-Kimball, and Quaranta (2012), which classifies the T-avoiding elements in the Coxeter systems of type $F_5$.

\begin{figure*}[h!]
\begin{tabular}{m{7cm} m{7cm}}
\begin{subfigure}{0.5\textwidth} \centering
\begin{tikzpicture}[scale=0.45]
	\heapblock{1}{10}{1}{purple}
	\heapblock{3}{10}{3}{orange}
	\heapblock{5}{10}{5}{purple}
	\heapblock{2}{8}{2}{orange}
	\heapblock{4}{8}{4}{purple}
	\heapblock{3}{6}{3}{orange}
	\heapblock{2}{4}{2}{orange}
	\heapblock{4}{4}{4}{purple}
	\heapblock{1}{2}{1}{purple}
	\heapblock{3}{2}{3}{orange}
	\heapblock{5}{2}{5}{purple}
\end{tikzpicture}
\caption{}\label{fig:singbowtie4}
\end{subfigure}&

\begin{subfigure}{0.5\textwidth}\centering
\begin{tikzpicture}[scale=0.45]
	\heapblock{1}{10}{1}{purple}
	\heapblock{3}{10}{3}{purple}
	\heapblock{5}{10}{5}{purple}
	\heapblock{2}{8}{2}{purple}
	\heapblock{4}{8}{4}{teal}
	\heapblock{3}{6}{3}{teal}
	\heapblock{2}{4}{2}{purple}
	\heapblock{4}{4}{4}{teal}
	\heapblock{1}{2}{1}{purple}
	\heapblock{3}{2}{3}{purple}
	\heapblock{5}{2}{5}{purple}
\end{tikzpicture}
\caption{}\label{fig:singbowtiebraid}
\end{subfigure}
\end{tabular}
\caption{Heap of a single bowtie in $W(F_5)$.}\label{fig:singbowtie}	
\end{figure*}

\begin{figure*}[h!] \centering
\begin{tikzpicture}[scale=0.45]
\heapblock{1}{10}{1}{purple}
	\heapblock{3}{10}{3}{orange}
	\heapblock{5}{10}{5}{purple}
	\heapblock{2}{8}{2}{orange}
	\heapblock{4}{8}{4}{purple}
	\heapblock{3}{6}{3}{orange}
	\heapblock{2}{4}{2}{orange}
	\heapblock{4}{4}{4}{purple}
	\heapblock{1}{2}{1}{purple}
	\heapblock{3}{2}{3}{orange}
	\heapblock{5}{2}{5}{purple}
	\heapblock{2}{0}{2}{orange}
	\heapblock{4}{0}{4}{purple}
	\heapblock{3}{-2}{3}{orange}
	\heapblock{2}{-4}{2}{orange}
	\heapblock{4}{-4}{4}{purple}
	\heapblock{1}{-6}{1}{purple}
	\heapblock{3}{-6}{3}{orange}
	\heapblock{5}{-6}{5}{purple}
	
	\node[] at (3,-8){$\vdots$};
	
	\heapblock{1}{-10}{1}{purple}
	\heapblock{3}{-10}{3}{orange}
	\heapblock{5}{-10}{5}{purple}
	\heapblock{2}{-12}{2}{orange}
	\heapblock{4}{-12}{4}{purple}
	\heapblock{3}{-14}{3}{orange}
	\heapblock{2}{-16}{2}{orange}
	\heapblock{4}{-16}{4}{purple}
	\heapblock{1}{-18}{1}{purple}
	\heapblock{3}{-18}{3}{orange}
	\heapblock{5}{-18}{5}{purple}
\end{tikzpicture}
\caption{Heap of a stack of bowties in $W(F_5)$.}\label{fig:stackobowties}
\end{figure*}

\begin{proposition}
The only $\tII$ elements in $W(F_5)$ are stacks of bowties. \qed	
\end{proposition}


As a result of the classification in type $F_5$, Cross et al.~were also able to classify the T-avoiding elements in $W(F_4)$. 

\begin{corollary}
There are no $\tII$ elements in the Coxeter system of type $F_4$. 
\begin{proof}
	Since there are no $\tII$ elements in $W(F_5)$ that do not have full support, we know that there are not any $\tII$ elements in $W(F_4)$. Because if there were $\tII$ elements they would also be $\tII$ in $W(F_5)$.
\end{proof}
\end{corollary}

Cross et al.~conjectured that in Coxeter systems of type $F_n$ for $n \geq 5$, an element is $\tII$ if and only if it is a stack of bowties multiplied by a product of commuting generators, where the product of commuting generators does not contain the the generators appearing in the stack of bowties. In 2013, Gilbertson and Ernst worked with this conjecture and quickly found it to be false. The heap seen in Figure~\ref{fig:f6bat} corresponds to a $\tII$ element in the Coxeter group of type $F_6$ that is not a bowtie. It turns out that like the bowties discussed above these elements can also be stacked to create an infinite number of $\tII$ elements. In addition, as $n$ gets large there are a number of modifications that can be made that result in additional $\tII$ elements. From this we conjecture that the classification of T-avoiding elements in Coxeter systems of type $F_n$ for $n \geq 6$ gets complicated very quickly. Classifying T-avoiding elements in $W(F_n)$ for $n \geq 6$ remains an open problem. 

\begin{figure}[h!]\centering
\begin{tikzpicture}[scale=0.45]
	\heapblock{2}{10}{2}{orange}
	\heapblock{4}{10}{4}{purple}
	\heapblock{6}{10}{6}{purple}
	\heapblock{3}{8}{3}{orange}
	\heapblock{5}{8}{5}{purple}
	\heapblock{2}{6}{2}{orange}
	\heapblock{4}{6}{4}{purple}
	\heapblock{1}{4}{1}{purple}
	\heapblock{3}{4}{3}{orange}
	\heapblock{2}{2}{2}{orange}
	\heapblock{1}{0}{1}{purple}
	\heapblock{3}{0}{3}{orange}
	\heapblock{2}{-2}{2}{orange}
	\heapblock{4}{-2}{4}{purple}
	\heapblock{3}{-4}{3}{orange}
	\heapblock{5}{-4}{5}{purple}
	\heapblock{2}{-6}{2}{orange}
	\heapblock{4}{-6}{4}{purple}
	\heapblock{6}{-6}{6}{purple}
\end{tikzpicture}
\caption{Heap of a $\tII$ element in $W(F_6)$.}\label{fig:f6bat}
\end{figure}

%%%%%%%%%%%%%%%%

\subsection{Type $I_2(m)$}\label{sec:tavoidI}

In this section, we classify the T-avoiding elements in Coxeter systems of type $I_2(m)$. Note that in Coxeter systems of type $I_2(m)$, the only products of commuting generators have length 1. Although the following is a quick result, we believe that it does not already appear in the literature.
\begin{theorem}\label{thm:i2m}
There are no $\tII$ elements in Coxeter systems of type $I_2(m)$.
\begin{proof}
	The graph for the Coxeter system of $I_2(m)$ appears in Figure~\ref{fig:I}. Note that the graph consists of two vertices, namely, $s_1$ and $s_2$, and a single edge with weight $m$. Also, recall that $W(I_2(m))$ is a star reducible Coxeter group. This implies that any $\tII$ elements in $W(I_2(m))$ must not be FC, as all of the FC elements have Property T or are $\tI$. The only non-FC element in $W(I_2(m))$ is the element of length $m$ that has exactly two reduced expressions consisting of alternating products of $s_1$ and $s_2$. Cleary, this element begins and ends with a product of non-commuting generators. Thus, this element has Property T. Hence $W(I_2(m))$ has no $\tII$ elements. 
\end{proof}	
\end{theorem}
 
%%%%%%%%%%%%%%%%%%%%%%%%%%%%%%%%

\section{T-Avoiding Elements in Coxeter groups of Types $\C_n$ and $B_n$}\label{chap:Cn}



%%%%%%%%%%%%%%%%%%%%%%%%%%
\subsection{T-Avoiding Elements in Coxeter Groups of Type $\C_n$}\label{sec:TAC}

In this section we will classify the T-avoiding elements in Coxeter systems of type $\C_n$, a new result. We will first show that there are no $\tII$ elements that are not FC in $W(\C_n)$.

Before we begin the proof we must first define the notion of a pushed-down representation of a heap. First recall that there are potentially many ways to draw the lattice point representation of a heap, each differing by the amount of vertical space between blocks. We wish to fix one such representation. Let $\w$ be a reduced expression for $w \in W(\C_n)$. We construct the \emph{pushed-down representation} of $H(\w)$ by first giving all blocks fully exposed to the bottom the same vertical position, and then all other blocks are placed as low as possible in the heap. That is, the heap has been constructed by placing all blocks in the lowest possible vertical position of the heap. Notice that we can now label the rows in the heap from bottom to top where the bottom-most row is row 1 and proceed naturally upward from there. 

We now define the height of a braid. Given the presence of a braid in the heap of a reduced expression $\w$, we define the \emph{height of the braid} to be the row number in which the uppermost block involved in the braid is located in the pushed-down representation. It is important to note that in the pushed-down representation, a braid may not appear in consecutive rows. That is, some of the blocks involved in a braid may be lower in the heap and the braid may not be immediately apparent. 

\begin{example}
Let $\w=s_0s_1s_3s_2s_1s_0s_1s_3$ be a reduced expression for $w \in W(C_3)$. The pushed-down representation of the heap for $\w$ is given in Figure~\ref{fig:pusheddownheap}. The height of the braid $s_1s_2s_1$, which is highlighted in \textcolor{teal}{teal} in Figure~\ref{fig:pusheddownheap}, is 5 since the upper block for $s_1$ is located in the fifth row of the pushed-down representation of the heap. Notice that the block for $s_3$ can slide up higher in the heap. If we were to slide the block for $s_3$ up until it hits the block for $s_2$ we would obtain the braid $s_3s_2s_3$ in the heap. In this case, the height for the braid $s_3s_2s_3$ is also 5.   

\begin{figure}[h!] \centering
\begin{tikzpicture}[scale=0.45]
	\heapblock{0}{10}{0}{purple}
	\heapblock{3}{8}{3}{purple}
	\heapblock{1}{8}{1}{teal}
	\heapblock{2}{6}{2}{teal}
	\heapblock{1}{4}{1}{teal}
	\heapblock{0}{2}{0}{purple}
	\heapblock{1}{0}	{1}{purple}
	\heapblock{3}{0}{3}{purple}
%	\heapblock{0}{0}{0}{purple}
%	\heapblock{2}{0}{2}{purple}
%	\heapblock{4}{0}{4}{purple}
\end{tikzpicture}	
\caption{Pushed-down representation of a heap.}\label{fig:pusheddownheap}
\end{figure}
\end{example}

We now show that there are no $\tII$ elements in $W(\C_n) \setminus \FC(\C_n)$. We first need the following lemma.

\begin{lemma}\label{lem:zigzag}
Let $w=uv$ (reduced) in a Coxeter system of type $\C_n$ such that $v$ is FC and has a reduced expression beginning with $s_0s_1\cdots s_{n-1}s_ns_{n-1}$. Then $w$ has Property T on the right.
\begin{proof}
	Let $w=uv$ (reduced) in a Coxeter system of type $\C_n$ such that $v$ is FC and has a reduced expression beginning with $s_0s_1 \cdots s_{n-1}s_ns_{n-1}$. The top of the heap for $v$ is shown in Figure~\ref{fig:lemzigzag1} which must be a convex subheap of the heap for some reduced expression for $w$. If $H(v)$ is equal to the heap in Figure~\ref{fig:lemzigzag1}, then $v$, and hence $w$, have Property T on the right. If the heap given in Figure~\ref{fig:lemzigzag1} is not the complete heap for $v$, then since $v$ is FC, it must be the case that the heap in Figure~\ref{fig:lemzigzag2} is a convex subheap of $H(v)$. Since we must avoid the impermissible heap configurations of Proposition~\ref{prop:impermiss} we see that $H(v)$ must have a zigzagging shape that changes direction every time it encounters $s_0$ or $s_n$. That is, 
	\[
		v=(s_0s_1 \cdots s_{n-1}s_ns_{n-1} \cdots s_1)^m 
		\begin{cases}
 			s_0s_1 \cdots s_{j-1}s_j, & \text{for } j \leq n\\
 			s_0s_1 \cdots s_{n-1}s_ns_{n-1} \cdots s_j, & \text{for } j> 1.
		\end{cases}
	\]  
	From this we see that $w$ ends with $s_{j-1}s_j$ or $s_{j+1}s_j$ which implies $w$ has Property T on the right.
\end{proof}
\end{lemma}

\begin{figure}[h!]\centering
\begin{tabular}{m{7cm} m{7cm}}
\begin{subfigure}{0.5\textwidth} \centering
\begin{tikzpicture}[scale=0.45]
	\sheapblock{0}{-18}{}{white}
	\dheapblock{0}{-4}{}{black}
	\dheapblock{1}{-6}{}{black}
	\dheapblock{2}{-8}{}{black}

	\heapblock{0}{0}{0}{purple}
	\heapblock{1}{-2}{1}{purple}
	\heapblock{2}{-4}{2}{purple}
	\heapblock{3}{-6}{3}{purple}
	\heapblock{4}{-8}{4}{purple}
	
	\node[] at (6,-10){$\ddots$}; 
	\node[] at (3,-10){$\ddots$};
	
	\dheapblock{5}{-12}{}{black}
	\dheapblock{6}{-14}{}{black}
	\dheapblock{5}{-16}{}{black}
	
	\heapblock{7}{-12}{n-1}{purple}
	\heapblock{8}{-14}{n}{purple}
	\heapblock{7}{-16}{n-1}{purple}
\end{tikzpicture}
\caption{}\label{fig:lemzigzag1}
\end{subfigure} &

\begin{subfigure}{0.5\textwidth} \centering
\begin{tikzpicture}[scale=0.45]
	\dheapblock{0}{-4}{}{black}
	\dheapblock{1}{-6}{}{black}
	\dheapblock{2}{-8}{}{black}

	\heapblock{0}{0}{0}{purple}
	\heapblock{1}{-2}{1}{purple}
	\heapblock{2}{-4}{2}{purple}
	\heapblock{3}{-6}{3}{purple}
	\heapblock{4}{-8}{4}{purple}
	
	\node[] at (6,-10){$\ddots$}; 
	\node[] at (3,-10){$\ddots$};
	
	\dheapblock{5}{-12}{}{black}
	\dheapblock{6}{-14}{}{black}
	\dheapblock{5}{-16}{}{black}
	\dheapblock{4}{-18}{}{black}
	
	\heapblock{7}{-12}{n-1}{purple}
	\heapblock{8}{-14}{n}{purple}
	\heapblock{7}{-16}{n-1}{purple}
	\heapblock{6}{-18}{n-2}{purple}
\end{tikzpicture}
\caption{}\label{fig:lemzigzag2}	
\end{subfigure}
\end{tabular}
\caption{The subheap of $w$ discussed in Lemma~\ref{lem:zigzag}.}\label{fig:lemzigzag}
\end{figure}


\begin{theorem}\label{thm:TavoidC}
There are no $\tII$ elements in $W(\C_n)\setminus \FC(\C_n)$. 	
\begin{proof}
	We proceed by contradiction. Let $w \in W(\C_n)\setminus\FC(\C_n)$ such that $w$ is $\tII$. Consider all possible pushed-down representations for heaps of $w$. Choose a representation  that has a minimal height braid among all braids appearing in all heaps for $w$ and let $k$ represent that minimum height. There may be a tie, in which case choose your favorite. Without loss of generality we assume the generators involved in the braid are $s_{i-1}$ and $s_{i},$ where the bond strength is case specific. In the following cases we give the height of a braid and unless we indicate otherwise whenever we refer to a block being in a specific row, we are considering the pushed-down representation of the heap. In order to consider the braids that we are looking for we need to allow some flexibility when referring to the vertical position of a given block. In the following cases, all subheaps are assumed to be convex. We proceed by considering 4 cases, where we handle $k=3$ when $m(s_{i-1},s_i)=3$ and $k=4$ when $m(s_{i-1},s_i)=4$ in Case (1). For Cases (2), (3), and (4) we assume that $k \geq 4$.
	
	Case (1): In this case, we are assuming that the braid is located in consecutive rows with the upper-most block in either row 3 or row 4 depending on the bond strength and the lowest block involved in the braid is located in the bottom-most row of the heap.
	
	Subcase (1.1): Suppose $k=3$. This implies that $m(s_{i-1}, s_i)=3$. Without loss of generality assume $s_{i-1}$ is in the bottom-most row of the heap. Clearly, the block for $s_{i+1}$ must be in the bottom-most row of the heap as well, otherwise $w$ has Property T, which is a contradiction to the original choice of $w$. Restricting our focus to the subheap of $w$ that contains the braid we are considering,  we see that this subheap of $w$ has the following form
	\begin{center}
	\begin{tikzpicture}[scale=0.45]
		%\sheapblock{2}{6}{i}{purple}
		\dheapblock{0}{2}{}{black}
		\heapblock{1}{0}{i-1}{purple}
		\heapblock{3}{0}{i+1}{purple}
		\heapblock{2}{2}{i}{purple}
		\heapblock{1}{4}{i-1}{purple}
	\end{tikzpicture}	
	\end{center}
 	where the blocks for $s_{i-1}$ and $s_{i+1}$ are in the bottom-most row of the heap and are thus fully exposed. Applying the braid move we get the subheap seen here
 	\begin{center}
 	\begin{tikzpicture}[scale=0.45]
 		\dheapblock{0}{0}{}{black}
 		%\dheapblock{2}{4}{}{black}
 		\heapblock{2}{0}{i+1}{purple}
 		\heapblock{1}{2}{i}{purple}
 		\heapblock{0}{4}{i-1}{purple}
 		\heapblock{1}{6}{i}{purple}
 		%\sheapblock{0}{8}{i-1}{purple}
 	\end{tikzpicture}
 	\end{center}
	which clearly has Property T since $s_{i-1}$ is now in the first row of the pushed-down representation. This is a contradiction to the way in which we chose $w$. 
	
	Subcase (1.2): Suppose $k=4$. This implies that $m(s_{i-1}, s_i)=4$. Without loss of generality assume $s_{i-1}=s_0$ and $s_{i}=s_1$. The other case being $s_{i-1}=s_{n-1}$ and $s_i=s_n$ which we could handle with a symmetric argument.
	
	Subcase (1.2.1): First we take $s_0$ to be the topmost block in the braid. Then we obtain the subheap seen here
	\begin{center}
	\begin{tikzpicture}[scale=0.45]
		\dheapblock{2}{2}{}{black}
		\heapblock{1}{0}{1}{purple}
		\heapblock{0}{2}{0}{purple}
		\heapblock{1}{4}{1}{purple}
		\heapblock{0}{6}{0}{purple}
	\end{tikzpicture}	
	\end{center}
	which clearly indicates that $w$ has Property T on the right. This is a contradiction to the way in which we chose $w$.
	
	Subcase (1.2.2): Now we take $s_1$ to be the topmost block involved in the braid. Clearly, the block for $s_2$ must be in the bottom-most row of the heap as well, otherwise $w$ has Property T, which is a contradiction to the orginal choice of $w$. Restricting our focus to the subheap of $w$ that contains the braid we are considering, we see that the subheap of $w$ has the following form
	\begin{center}
	\begin{tikzpicture}[scale=0.45]
		\heapblock{2}{6}{i}{purple}
		\dheapblock{0}{2}{}{black}
		\heapblock{1}{0}{i-1}{purple}
		\heapblock{3}{0}{i+1}{purple}
		\heapblock{2}{2}{i}{purple}
		\heapblock{1}{4}{i-1}{purple}
	\end{tikzpicture}	
	\end{center}
	where the blocks for $s_{i-1}$ and $s_{i+1}$ are in the bottom-most row of the heap and are thus fully exposed. Applying the braid move we get the subheap seen here
 	\begin{center}
 	\begin{tikzpicture}[scale=0.45]
 		\dheapblock{0}{0}{}{black}
 		\dheapblock{2}{4}{}{black}
 		\heapblock{2}{0}{2}{purple}
 		\heapblock{1}{2}{1}{purple}
 		\heapblock{0}{4}{0}{purple}
 		\heapblock{1}{6}{1}{purple}
 		\heapblock{0}{8}{0}{purple}
 	\end{tikzpicture}
 	\end{center}
	which shows that $w$ has Property T since $s_{i-1}$ is now in the first row of a pushed-down representation for $w$. This is a contradiction to the way in which we chose $w$. For the rest of the cases we will assume that $k \geq 4$.
	
	Case (2): Suppose the braid has height $k$ and assume the braid does not involve $s_0, s_1, s_{n-1}$ or $s_n$. This implies that $m(s_{i-1},s_i)=3$, $m(s_{i-2},s_{i-1})=3$ and $m(s_i,s_{i+1})=3$. Without loss of generality assume $s_i$ is in the $k$th row of the heap and if necessary we have brought the blocks for $s_{i-1}$ and $s_i$ up next to $s_i$ in row $k$. We now consider which blocks may occur in the $(k-3)$th row of the heap in two cases. 
	
	Subcase (2.1): Assume that the block for $s_{i-1}$ is in the heap in the $(k-3)$th row and we allow for the block for $s_{i+1}$ to be in the same row as well, but it does not necessarily have to be. In the following figures of subheaps, the block for $s_{i+1}$ will be represented in a \textcolor{purple}{purple}-striped block to indicate that it could be present but it does not have to be. The following is the subheap that we are considering
	\begin{center}
	\begin{tikzpicture}[scale=0.45]
		\dheapblock{4}{4}{}{black}
		\heapblock{2}{0}{i-1}{purple}
		\sheapblock{4}{0}{i+1}{purple}
		\heapblock{1}{2}{i-2}{purple}
		\heapblock{3}{2}{i}{teal}
		\heapblock{2}{4}{i-1}{teal}
		\heapblock{3}{6}{i}{teal}
	\end{tikzpicture}
	\end{center}
	where we have highlighted the braid in \textcolor{teal}{teal}. Notice that the block for $s_{i-2}$ is present in the $(k-2)$th row otherwise there is a braid with height less than $k$ and $w$ would not be reduced. Applying the braid move to the heap we get the following subheap
	\begin{center}
	\begin{tikzpicture}[scale=0.45]
		\dheapblock{3}{2}{}{black}
		\dheapblock{1}{6}{}{black}
		\heapblock{2}{0}{i-1}{rred}
		\sheapblock{4}{0}{i+1}{purple}
		\heapblock{1}{2}{i-2}{rred}
		\heapblock{2}{4}{i-1}{rred}
		\heapblock{3}{6}{i}{teal}
		\heapblock{2}{8}{i-1}{teal}
	\end{tikzpicture}
	\end{center}
	which has a new braid in it. This braid, which we have highlighted in \textcolor{rred}{red} for emphasis, has height $k-1$. In applying the braid move we have obtained a heap which has a braid with height less than $k$ our original choice. This is a contradiction to the way in which we chose our heap.
	
	Subcase (2.2): Assume that the block for $s_{i+1}$ is in the $(k-3)$th row of the heap and the block for $s_{i-1}$ does not appear in the $(k-3)$th row. The following is the subheap we are considering
	\begin{center}
	\begin{tikzpicture}[scale=0.45]
		\heapblock{3}{0}{i+1}{purple}
		\dheapblock{1}{0}{}{black}
		%\dheapblock{0}{2}{}{black}
		\heapblock{2}{2}{i}{teal}
		\heapblock{1}{4}{i-1}{teal}
		\heapblock{2}{6}{i}{teal}
	\end{tikzpicture}
	\end{center}
 	where the dotted square represents that $s_{i-1}$ may not occupy the indicated position and the braid is highlighted in \textcolor{teal}{teal}. Applying the braid move in the subheap we obtain the following heap
 	\begin{center}
 	\begin{tikzpicture}[scale=0.45]
 		\dheapblock{-1}{2}{}{black}
 		\heapblock{2}{0}{i+1}{purple}
 		\heapblock{0}{0}{i-1}{teal}
 		\heapblock{1}{2}{i}{teal}
 		\heapblock{0}{4}{i-1}{teal}
 	\end{tikzpicture}
 	\end{center}
	where the height of the new braid is $k-1$. This contradicts the way in which we chose the heap of $w$. From this we gather that the braid must contain $s_0,s_1, s_{n-1},$ or $s_n$.
	
	Case (3): Suppose the braid has height $k$ and assume the braid contains $s_2$ or $s_{n-2}$. Without loss of generality we assume that the braid contains $s_2$ as the other argument is symmetric to the one presented here. Notice that if the braid contains $s_2s_3s_2$, we are in Case (2), as a result we assume our braid is of the form $s_1s_2s_1$ or $s_2s_1s_2$. 
	
	Subcase (3.1): Assume the block for $s_1$ is in row $k$, then as we are not in Case (1) we know that at least one of the blocks for $s_0, s_2$ are in the $(k-3)$th row. 
	
	Subcase (3.1.1): Assume that the block for $s_0$ is in row $k-3$ but the block for $s_2$ is not in row $k-3$. The following is the subheap we are considering
	\begin{center}
	\begin{tikzpicture}[scale=0.45]
		\dheapblock{2}{0}{}{black}
		\dheapblock{0}{4}{}{black}
		\heapblock{0}{0}{0}{purple}
		\heapblock{1}{2}{1}{teal}
		\heapblock{2}{4}{2}{teal}
		\heapblock{1}{6}{1}{teal}
	\end{tikzpicture}	
	\end{center}
	where we have highlighted the braid in \textcolor{teal}{teal} and have indicated with dotted blocks positions which cannot be occupied. Applying the braid move to the heap we get the following subheap
	\begin{center}
	\begin{tikzpicture}[scale=0.45]
		\dheapblock{3}{2}{}{black}
		\heapblock{0}{0}{0}{purple}
		\heapblock{2}{0}{2}{teal}
		\heapblock{1}{2}{1}{teal}
		\heapblock{2}{4}{2}{teal}
	\end{tikzpicture}	
	\end{center}
	where the new braid now has height $k-1$. This contradicts the way in which we chose the heap of $w$. 
	
	Subcase (3.1.2): Assume that the block for $s_2$ is in row $k$. Potentially the block for $s_0$ is in row $k$, however it does not have to be, so we emphasize this in the following heap with a \textcolor{purple}{purple}-striped block.  Then the subheap we are considering is as follows
	\begin{center}
	\begin{tikzpicture}[scale=0.45]
		\dheapblock{0}{4}{}{black}
		\heapblock{2}{0}{2}{purple}
		\sheapblock{0}{0}{0}{purple}
		\heapblock{1}{2}{1}{teal}
		\heapblock{2}{4}{2}{teal}
		\heapblock{1}{6}{1}{teal}
	\end{tikzpicture}	
	\end{center}
	where the braid we are considering in the heap is highlighted in \textcolor{teal}{teal}. Notice that if this was the heap for $w$, the expression for $w$ would not be reduced. This forces the block for $s_3$ to be in the heap as follows:
	\begin{center}
	\begin{tikzpicture}[scale=0.45]
		\dheapblock{0}{4}{}{black}
		\heapblock{2}{0}{2}{purple}
		\sheapblock{0}{0}{0}{purple}
		\heapblock{1}{2}{1}{teal}
		\heapblock{2}{4}{2}{teal}
		\heapblock{1}{6}{1}{teal}
		\heapblock{3}{2}{3}{purple}
	\end{tikzpicture}	
	\end{center}
	Applying the braid move to the heap we get the following subheap
	\begin{center}
	\begin{tikzpicture}[scale=0.45]
		\dheapblock{1}{2}{}{black}
		\dheapblock{3}{6}{}{black}
		\sheapblock{0}{0}{0}{purple}
		\heapblock{2}{0}{2}{rred}
		\heapblock{3}{2}{3}{rred}
		\heapblock{2}{4}{2}{rred}
		\heapblock{1}{6}{1}{purple}
		\heapblock{2}{8}{2}{purple}
	\end{tikzpicture}	
	\end{center}
	where a new braid has appeared which we have highlighted in \textcolor{rred}{red}. Notice that the height of this new braid is $k-1$, which is a contradiction to the original choice of $w$.
	
	Subcase (3.2): Assume $s_2$ is in row $k$. This implies that at least one of the blocks for $s_1$ or $s_3$ is in the $(k-3)$th row. We proceed in cases.
	
	Subcase (3.2.1): Assume that only the block for $s_3$ is in row $k-3$. Then we have the subheap
	\begin{center}
	\begin{tikzpicture}[scale=0.45]
		\dheapblock{1}{0}{}{black}
		\dheapblock{3}{4}{}{black}
		\heapblock{3}{0}{3}{purple}
		\heapblock{2}{2}{2}{teal}
		\heapblock{1}{4}{1}{teal}
		\heapblock{2}{6}{2}{teal}
	\end{tikzpicture}	
	\end{center}
	where we have highlighted the heap in \textcolor{teal}{teal}. Applying the braid move to this subheap we obtain the following subheap 
	\begin{center}
	\begin{tikzpicture}[scale=0.45]
		\dheapblock{0}{2}{}{black}
		\heapblock{1}{0}{1}{teal}
		\heapblock{3}{0}{3}{purple}
		\heapblock{2}{2}{2}{teal}
		\heapblock{1}{4}{1}{teal}
	\end{tikzpicture}	
	\end{center}
	where the braid has height $k-1$, which is a contradiction to the way in which we chose the heap for $w$.
	
	Subcase (3.2.2): Assume that the block for $s_1$ is in row $k-3$ and the block for $s_3$ is not in row $k-3$. Then we have the subheap
	\begin{center}
	\begin{tikzpicture}[scale=0.45]
		\dheapblock{3}{0}{}{black}
		\dheapblock{3}{4}{}{black}
		\heapblock{1}{0}{1}{purple}
		%\sheapblock{3}{0}{3}{purple}
		\heapblock{2}{2}{2}{teal}
		\heapblock{1}{4}{1}{teal}
		\heapblock{2}{6}{2}{teal}
	\end{tikzpicture}
	\end{center}
	where we have highlighted the braid in teal. Notice that if this is the actual subheap, the corresponding expression for $w$ is not reduced, so we know that $s_0$ must appear in the heap which we illustrate here
	\begin{center}
	\begin{tikzpicture}[scale=0.45]
		\dheapblock{3}{0}{}{black}
		\dheapblock{3}{4}{}{black}
		\heapblock{0}{2}{0}{purple}
		\heapblock{1}{0}{1}{purple}
		%\sheapblock{3}{0}{3}{purple}
		\heapblock{2}{2}{2}{teal}
		\heapblock{1}{4}{1}{teal}
		\heapblock{2}{6}{2}{teal}
	\end{tikzpicture}
	\end{center}
	applying the braid move to the heap we obtain the following subheap
	\begin{center}
	\begin{tikzpicture}[scale=0.45]
		\dheapblock{0}{6}{}{black}
		\heapblock{1}{0}{1}{purple}
		%\sheapblock{3}{0}{3}{purple}
		\heapblock{0}{2}{0}{purple}
		\heapblock{1}{4}{1}{teal}
		\heapblock{2}{6}{2}{teal}
		\heapblock{1}{8}{1}{teal}
	\end{tikzpicture}	
	\end{center}
	Notice that there are no new braids. However, we know that this cannot be the bottom row of our subheap as then $w$ has Property T on the right. This implies that the block for $s_2$ is in the row beneath the blocks for $s_1$ and $s_3$. Notice that $s_0$ cannot be in this row since otherwise the heap would contain a lower braid. This leads to the heap 
	\begin{center}
	\begin{tikzpicture}[scale=0.45]
		\dheapblock{0}{-2}{}{black}
		\dheapblock{0}{6}{}{black}
		\heapblock{1}{0}{1}{purple}
		%\sheapblock{3}{0}{3}{purple}
		\heapblock{0}{2}{0}{purple}
		\heapblock{1}{4}{1}{teal}
		\heapblock{2}{6}{2}{teal}
		\heapblock{1}{8}{1}{teal}
		\heapblock{2}{-2}{2}{rred}
	\end{tikzpicture}	
	\end{center}
	where we have emphasized that $s_0$ cannot be in the position with a dotted block. Again this cannot be the bottom row of the subheap and utilizing the same argument we get the following heap
	\begin{center}
	\begin{tikzpicture}[scale=0.45]
		\dheapblock{0}{-2}{}{black}
		\dheapblock{1}{-4}{}{black}
		\dheapblock{0}{6}{}{black}
		\heapblock{1}{0}{1}{purple}
		%\sheapblock{3}{0}{3}{purple}
		\heapblock{0}{2}{0}{purple}
		\heapblock{1}{4}{1}{teal}
		\heapblock{2}{6}{2}{teal}
		\heapblock{1}{8}{1}{teal}
		\heapblock{2}{-2}{2}{rred}
		\heapblock{3}{-4}{3}{rred}
	\end{tikzpicture}	
	\end{center}
	Again this cannot be the bottom row of the heap. Iterating this argument, we get the following subheap:
	\begin{center}
	\begin{tikzpicture}[scale=0.45]
		\dheapblock{0}{12}{}{black}
		\dheapblock{0}{4}{}{black}
		\dheapblock{1}{2}{}{black}
		\heapblock{1}{6}{1}{purple}
		%\sheapblock{3}{0}{3}{purple}
		\heapblock{0}{8}{0}{purple}
		\heapblock{1}{10}{1}{teal}
		\heapblock{2}{12}{2}{teal}
		\heapblock{1}{14}{1}{teal}
		\heapblock{2}{4}{2}{rred}
		\heapblock{3}{2}{3}{rred}
		\heapblock{4}{0}{4}{rred}
		%\node[] at (7,-1.5){$\ddots$};
		\node[] at (5,-1.5){$\ddots$};
		\node[] at (3,-1.5){$\ddots$};
		
		%\dheapblock{10}{-3.5}{}{black}
		%\dheapblock{11}{-5.5}{}{black}
		%\dheapblock{10}{-7.5}{}{black}
		\dheapblock{6}{-7.5}{}{black}
		\dheapblock{7}{-5.5}{}{black}
		\dheapblock{6}{-3.5}{}{black}
		\heapblock{8}{-3.5}{n-1}{rred}
		\heapblock{9}{-5.5}{n}{rred}
		\heapblock{8}{-7.5}{n-1}{rred}
		
		
%		\node[] at (6,-9){$\iddots$};
%		\node[] at (4,-9){$\iddots$};
%		\node[] at (8,-9){$\iddots$};
	\end{tikzpicture}	
	\end{center}
	But then by Lemma~\ref{lem:zigzag} $w$ has Property T on the right. This is a contradiction to the way in which we chose $w$.
	
	Subcase (3.2.3): Assume that blocks for $s_1$ and $s_3$ are in row $k-3$. Then the subheap we are considering is
	\begin{center}
	\begin{tikzpicture}[scale=0.45]
		\dheapblock{3}{4}{}{black}
		\heapblock{1}{0}{1}{purple}
		\heapblock{3}{0}{3}{purple}
		\heapblock{2}{2}{2}{teal}
		\heapblock{0}{2}{0}{purple}
		\heapblock{1}{4}{1}{teal}
		\heapblock{2}{6}{2}{teal}
	\end{tikzpicture}	
	\end{center}
	where the braid is highlighted in \textcolor{teal}{teal}. Applying the braid move we get the following subheap
	\begin{center}
	\begin{tikzpicture}[scale=0.45]
		\dheapblock{0}{6}{}{black}
		\heapblock{1}{0}{1}{purple}
		\heapblock{3}{0}{3}{purple}
		\heapblock{0}{2}{0}{purple}
		\heapblock{1}{4}{1}{teal}
		\heapblock{2}{6}{2}{teal}
		\heapblock{1}{8}{1}{teal}
	\end{tikzpicture}
	\end{center}
	where no new braids have appeared, and in fact the original braid is higher now with height $k+1$. Notice however, that the $(k-3)$th row will not be the bottom row of our heap because $w$ would have Property T with respect to $s_1$ and $s_0$, a contradiction to our assumption. With this in mind we consider the $(k-4)$th row. Notice that $s_0$ will not appear in the $(k-4)$th row as we would have a lower braid. This implies that the $(k-4)$th row contains at least one of $s_2$ or $s_4$. We handle this scenario in three additional subcases. 
	
	Subcase (3.2.3.1): Suppose $s_2$ is in the $(k-4)$th row but $s_4$ is not. Then we have the subheap
	\begin{center}
	\begin{tikzpicture}[scale=0.45]
		\dheapblock{0}{8}{}{black}
		\dheapblock{0}{0}{}{black}
		\heapblock{2}{0}{2}{rred}
		\dheapblock{4}{0}{}{black}
		\heapblock{1}{2}{1}{purple}
		\heapblock{3}{2}{3}{purple}
		\heapblock{0}{4}{0}{purple}
		\heapblock{1}{6}{1}{teal}
		\heapblock{2}{8}{2}{teal}
		\heapblock{1}{10}{1}{teal}
	\end{tikzpicture}	
	\end{center}
	where we have highlighted the new block in \textcolor{rred}{red}. We have also placed dotted blocks in positions where no blocks may appear. Notice that if this new row is row 1, then $w$ would have Property T with respect to $s_2$ and $s_1$. As before, this cannot be the bottom row of our heap. However, the presences of $s_1$ or $s_3$ in the next row will create a braid with height less than $k$ as seen here 
	\begin{center}
	\begin{tikzpicture}[scale=0.45]
		\dheapblock{0}{8}{}{black}
		\dheapblock{0}{0}{}{black}
		\heapblock{2}{0}{2}{rred}
		\dheapblock{4}{0}{}{black}
		\heapblock{1}{2}{1}{rred}
		\heapblock{3}{2}{3}{rred}
		\heapblock{0}{4}{0}{purple}
		\heapblock{1}{6}{1}{teal}
		\heapblock{2}{8}{2}{teal}
		\heapblock{1}{10}{1}{teal}
		\sheapblock{3}{-2}{3}{rred}
		\sheapblock{1}{-2}{1}{rred}
	\end{tikzpicture}	
	\end{center}
	where we have highlighted the new blocks with \textcolor{rred}{red}-stripes. This is a contradiction to the way in which we chose $w$.
	
	Subcase (3.2.3.2): Suppose $s_4$ is in the $(k-4)$th row but $s_2$ is not. Then we have the subheap
	\begin{center}
	\begin{tikzpicture}[scale=0.45]
		\dheapblock{0}{8}{}{black}
		\dheapblock{0}{0}{}{black}
		\heapblock{4}{0}{4}{rred}
		\dheapblock{2}{0}{}{black}
		\heapblock{1}{2}{1}{purple}
		\heapblock{3}{2}{3}{purple}
		\heapblock{0}{4}{0}{purple}
		\heapblock{1}{6}{1}{teal}
		\heapblock{2}{8}{2}{teal}
		\heapblock{1}{10}{1}{teal}
	\end{tikzpicture}	
	\end{center}
	where we have highlighted the new block in \textcolor{rred}{red}.  Notice that if this new row is row 1, then $w$ would have Property T with respect to $s_3$ and $s_4$. Again, this implies that this is not the bottom row of the heap. Repeating this process again we obtain the subheap
	\begin{center}
	\begin{tikzpicture}[scale=0.45]
		\dheapblock{0}{8}{}{black}
		\sheapblock{3}{-2}{3}{rred}
		\sheapblock{5}{-2}{5}{rred}
		\dheapblock{0}{0}{}{black}
		\heapblock{4}{0}{4}{rred}
		\dheapblock{2}{0}{}{black}
		\heapblock{1}{2}{1}{purple}
		\heapblock{3}{2}{3}{purple}
		\heapblock{0}{4}{0}{purple}
		\heapblock{1}{6}{1}{teal}
		\heapblock{2}{8}{2}{teal}
		\heapblock{1}{10}{1}{teal}
	\end{tikzpicture}	
	\end{center}
	Then we have the subheap:
	\begin{center}
	\begin{tikzpicture}[scale=0.45]
		\dheapblock{0}{8}{}{black}
		\dheapblock{3}{-2}{}{black}
		\heapblock{5}{-2}{5}{rred}
		\dheapblock{0}{0}{}{black}
		\heapblock{4}{0}{4}{rred}
		\dheapblock{2}{0}{}{black}
		\heapblock{1}{2}{1}{purple}
		\heapblock{3}{2}{3}{purple}
		\heapblock{0}{4}{0}{purple}
		\heapblock{1}{6}{1}{teal}
		\heapblock{2}{8}{2}{teal}
		\heapblock{1}{10}{1}{teal}
	\end{tikzpicture}
	\end{center}
	 Iterating this process we end up with the subheap: 
	\begin{center}
	\begin{tikzpicture}[scale=0.45]
		\dheapblock{3}{-2}{}{black}
		\dheapblock{0}{8}{}{black}
%		\dheapblock{6}{0}{}{black}
		\heapblock{5}{-2}{5}{rred}
		\dheapblock{0}{0}{}{black}
		\heapblock{4}{0}{4}{rred}
		\dheapblock{2}{0}{}{black}
		\heapblock{1}{2}{1}{purple}
		\heapblock{3}{2}{3}{purple}
		\heapblock{0}{4}{0}{purple}
		\heapblock{1}{6}{1}{teal}
		\heapblock{2}{8}{2}{teal}
		\heapblock{1}{10}{1}{teal}
		
		%\node[] at (8,-4){$\ddots$};
		\node[] at (6.5,-4){$\ddots$};
		\node[] at (5, -4){$\ddots$};
		
		%\dheapblock{10}{-6.5}{}{black}
		\dheapblock{6}{-6.5}{}{black}
		\heapblock{8}{-6.5}{n-2}{rred}
		\dheapblock{7}{-12.5}{}{black}
		\dheapblock{8}{-10.5}{}{black}
		\dheapblock{7}{-8.5}{}{black}
		\heapblock{9}{-8.5}{n-1}{rred}
		\heapblock{10}{-10.5}{n}{rred}
		\heapblock{9}{-12.5}{n-1}{rred}
	\end{tikzpicture}
	\end{center}
	where we have highlighted the additions in \textcolor{rred}{red} and have placed dotted blocks in positions that blocks may not occupy. It is clear that  must continue the heap diagonally downward to the left, but if we end before reaching $s_0$ the heap would have Property T, which is a contradiction to the way in which we chose $w$. Suppose that the zig-zag continues on after reaching $s_0$. Then we are able to drop the block for $s_1$ down and create a lower braid. This is a contradiction to the way in which we chose $w$.
	
	Subcase (3.2.3.3): Suppose the blocks for $s_2$ and $s_4$ are in the $(k-4)$th row. Then we have the subheap
	\begin{center}
	\begin{tikzpicture}[scale=0.45]
		\dheapblock{0}{2}{}{black}
		\dheapblock{0}{10}{}{black}
		%\dheapblock{2}{-2}{}{black}
		%\sheapblock{3}{0}{3}{rred}
		%\sheapblock{5}{0}{5}{rred}
		\heapblock{2}{2}{2}{rred}
		\heapblock{4}{2}{4}{rred}
		\heapblock{1}{4}{1}{purple}
		\heapblock{3}{4}{3}{purple}
		\heapblock{0}{6}{0}{purple}
		\heapblock{1}{8}{1}{teal}
		\heapblock{2}{10}{2}{teal}
		\heapblock{1}{12}{1}{teal}
	\end{tikzpicture}	
	\end{center}
	This subheap has Property T in the bottom with respect to $s_2$ and $s_1$. Thus this cannot be the bottom row of our heap. Repeating this process, we see that $s_1$ will not be in row $k-5$ since this would create a lower braid. Thus we must have $s_3$ or $s_5$ in the $(k-5)$th row. We represent this with the following subheap 
	\begin{center}
	\begin{tikzpicture}[scale=0.45]
		\dheapblock{0}{10}{}{black}
		\dheapblock{0}{2}{}{black}
		\dheapblock{1}{0}{}{black}
		%\dheapblock{2}{-2}{}{black}
		\heapblock{3}{0}{3}{rred}
		\heapblock{5}{0}{5}{rred}
		\heapblock{2}{2}{2}{rred}
		\heapblock{4}{2}{4}{rred}
		\heapblock{1}{4}{1}{purple}
		\heapblock{3}{4}{3}{purple}
		\heapblock{0}{6}{0}{purple}
		\heapblock{1}{8}{1}{teal}
		\heapblock{2}{10}{2}{teal}
		\heapblock{1}{12}{1}{teal}
	\end{tikzpicture}	
	\end{center}
	where again we have highlighted the additions in \textcolor{rred}{red} and put dotted blocks where no block may appear. Notice that if we were to only place one of the blocks for $s_3$ or $s_5$ we would quickly be in Subcase (3.2.3.1) or (3.2.3.2) above. Thus we know we must place both $s_3$ and $s_5$ in the $(k-5)$th row. Again, if the $(k-5)$th row is the first row, then  $w$ would have Property T with respect to $s_3$ and $s_2$. This implies that the $(k-5)$th row is not the bottom-most row in our heap. Iterating this process we obtain the following subheap
	\begin{center}
	\begin{tikzpicture}[scale=0.45]
		\dheapblock{0}{10}{}{black} 
		\dheapblock{0}{2}{}{black}
		\dheapblock{1}{0}{}{black}
		\dheapblock{2}{-2}{}{black}
		\heapblock{1}{12}{1}{teal}
		\heapblock{2}{10}{2}{teal}
		\heapblock{1}{8}{1}{teal}
		\heapblock{0}{6}{0}{purple}
		\heapblock{1}{4}{1}{purple}
		\heapblock{3}{4}{3}{purple}
		\heapblock{2}{2}{2}{rred}
		\heapblock{3}{0}{3}{rred}
		\heapblock{4}{2}{4}{rred}
		\heapblock{5}{0}{5}{rred}
		\heapblock{4}{-2}{4}{rred}
		
		\node[] at (8,-4){$\ddots$};
		\node[] at (6.5,-4){$\ddots$};
		\node[] at (5, -4){$\ddots$};
		
		\dheapblock{7}{-7}{}{black}
		\dheapblock{8}{-9}{}{black}
		%\dheapblock{7}{-11}{}{black}
		\heapblock{9}{-7}{n-3}{rred}
		\heapblock{11}{-7}{n-1}{rred}
		\heapblock{10}{-9}{n-2}{rred}
		\heapblock{12}{-9}{n}{rred}
	\end{tikzpicture}
	\end{center}
	where again we see that if the row containing the blocks for $s_{n-2}$ and $s_n$ corresponds to row 1, then $w$ would have Property T with respect to $s_{n-2}$ and $s_{n-3}$. Continuing in this manner, we obtain the following subheap
	\begin{center}
	\begin{tikzpicture}[scale=0.450] 
		\dheapblock{0}{10}{}{black}
		\dheapblock{0}{2}{}{black}
		\dheapblock{1}{0}{}{black}
		\dheapblock{2}{-2}{}{black}
		\heapblock{1}{12}{1}{teal}
		\heapblock{2}{10}{2}{teal}
		\heapblock{1}{8}{1}{teal}
		\heapblock{0}{6}{0}{purple}
		\heapblock{1}{4}{1}{purple}
		\heapblock{3}{4}{3}{purple}
		\heapblock{2}{2}{2}{rred}
		\heapblock{3}{0}{3}{rred}
		\heapblock{4}{2}{4}{rred}
		\heapblock{5}{0}{5}{rred}
		\heapblock{4}{-2}{4}{rred}
		
		\node[] at (8,-4){$\ddots$};
		\node[] at (6.5,-4){$\ddots$};
		\node[] at (5, -4){$\ddots$};
		
		
		\dheapblock{6}{-7}{}{black}
		\dheapblock{7}{-9}{}{black}
		\dheapblock{8}{-11}{}{black}
		\dheapblock{7}{-13}{}{black}
		\heapblock{8}{-7}{n-4}{rred}
		\heapblock{10}{-7}{n-2}{rred}
		\heapblock{9}{-9}{n-3}{orange}
		\heapblock{11}{-9}{n-1}{rred}
		\heapblock{10}{-11}{n-2}{orange}
		\heapblock{12}{-11}{n}{rred}
		\heapblock{11}{-13}{n-1}{rred}
		\heapblock{9}{-13}{n-3}{orange}
%		\sheapblock{10}{-11}{n-2}{rred}
%		
%		\node[] at (7, -11){$\iddots$};
%		\node[] at (7,-13){$\iddots$};
%		
%		\sheapblock{3}{-14}{3}{rred}
%		\sheapblock{1}{-14}{1}{rred}
%		\sheapblock{2}{-16}{2}{rred}
%		\sheapblock{0}{-16}{0}{rred}
%		\sheapblock{4}{-16}{4}{rred}
	\end{tikzpicture}

	\end{center}
	where the \textcolor{rred}{red} blocks correspond to a portion of an FC element. Recall that as we created this subheap we placed blocks in a manner to prevent a braid from appearing with height less than $k$ we prevented blocks from being placed inside the double diagonal (signified by the dotted blocks). However the \textcolor{orange}{orange} blocks create a new braid with height less than $k$. This is a contradiction to the way in which we chose $w$.  
%	\begin{center}
%	\begin{tikzpicture}[scale=0.45] 
%		\heapblock{1}{12}{1}{teal}
%		\heapblock{2}{10}{2}{teal}
%		\heapblock{1}{8}{1}{orange}
%		\heapblock{0}{6}{0}{orange}
%		\heapblock{1}{4}{1}{orange}
%		\heapblock{3}{4}{3}{purple}
%		\heapblock{0}{2}{0}{orange}
%		\sheapblock{2}{2}{2}{rred}
%		\sheapblock{3}{0}{3}{rred}
%		\sheapblock{4}{2}{4}{rred}
%		\sheapblock{5}{0}{5}{rred}
%		\sheapblock{4}{-2}{4}{rred}
%		
%		\node[] at (7,-2){$\ddots$};
%		\node[] at (7,-4){$\ddots$};
%		
%		\sheapblock{9}{-3}{n-3}{rred}
%		\sheapblock{11}{-3}{n-1}{rred}
%		\sheapblock{10}{-5}{n-2}{rred}
%		\sheapblock{12}{-5}{n}{rred}
%		\sheapblock{11}{-7}{n-1}{rred}
%		\sheapblock{9}{-7}{n-3}{rred}
%		\sheapblock{0}{-5}{}{rred}
%		\sheapblock{2}{-5}{}{rred}
%	
%		\node[] at (7, -7){$\iddots$};
%		\node[] at (7,-9){$\iddots$};
%		\node[] at (5,-5){$\cdots$};
%		\node[] at (0,-1.5){$\vdots$};
%		\node[] at (0,-7){$\vdots$};
%		
%		\sheapblock{3}{-9}{3}{rred}
%		\sheapblock{1}{-9}{1}{rred}
%		\sheapblock{2}{-11}{2}{rred}
%		\sheapblock{0}{-11}{0}{rred}
%		\sheapblock{4}{-11}{4}{rred}
%	\end{tikzpicture}
%	\end{center}
%	where we see that this has led to $s_0$ being located in the $(k-4)$th row. We emphasize here with the not labeled \textcolor{rred}{red}-striped blocks that every possible brick is present in the area that was outlined in the previous heap. Notice that a new braid has now appeared in the heap. We have highlighted this in \textcolor{orange}{orange} above. This new braid has height $k-1$. This is a contradiction as the braid appears lower than the original braid we chose. 
		
	Case (4): Suppose the braid has height $k$ and assume the braid contains $s_1$ or $s_{n-1}$. Without loss of generality we assume that the braid contains $s_1$, as the the other argument is symmetric to the one presented here. Notice that if the braid is  $s_1s_2s_1$, then we are in Case (3), so assume the braid consists of $s_0$ and $s_1$. 
	
	Subcase (4.1): In this case, we take our braid to be $s_1s_0s_1s_0$. Assume that, if necessary, the blocks that complete the braid have been brought up next to $s_1$ in the $k$th row. We now consider which blocks can occur in the $(k-3)$th row and $(k-4)$th row in two cases. We know that row $k-3$ is not the bottom row of our heap which implies $s_1$ must be in row $k-4$. Notice that if the block for $s_2$ is not in the $(k-3)th$ row then the expression for $w$ was not reduced. Assume the block for $s_2$ is located in the $(k-3)$th row. Then the subheap we are considering is
	\begin{center}
	\begin{tikzpicture}[scale=0.45]
		\dheapblock{2}{4}{}{black}
		\heapblock{0}{0}{0}{orange}
		\heapblock{2}{0}{2}{purple}
		\heapblock{1}{2}{1}{orange}
		\heapblock{0}{4}{0}{orange}
		\heapblock{1}{6}{1}{orange}
	\end{tikzpicture}	
	\end{center}
	where the braid we mentioned is highlighted in \textcolor{orange}{orange}. Applying the braid move we get the following subheap
	\begin{center}
	\begin{tikzpicture}[scale=0.45]
		\dheapblock{2}{4}{}{black}
		\heapblock{2}{0}{2}{purple}
		\heapblock{1}{2}{1}{orange}
		\heapblock{0}{4}{0}{orange}
		\heapblock{1}{6}{1}{orange}
		\heapblock{0}{8}{0}{orange}
	\end{tikzpicture}
	\end{center}
	in which we see that the original braid is now located higher in the heap with height $k+1$. Since $k > 4$ (otherwise we are in Case (1)), we know that in the original heap, $s_0$ and $s_2$ are located above row 1. This implies that the heap for $w$ has more rows underneath, which we will now systematically fill in. 
	
	Subcase (4.1.1): We first consider if the block for $s_1$ is located in row $k-4$ in the heap immediately above and $s_3$ is allowed but not required to be there. This leads to the following heap:
	\begin{center}
	\begin{tikzpicture}[scale=0.45]
		\dheapblock{2}{6}{}{black}
		\sheapblock{3}{0}{3}{purple}
		\heapblock{1}{0}{1}{rred}
		\dheapblock{3}{0}{}{black}
		\heapblock{2}{2}{2}{rred}
		\heapblock{1}{4}{1}{rred}
		\heapblock{0}{6}{0}{orange}
		\heapblock{1}{8}{1}{orange}
		\heapblock{0}{10}{0}{orange}
	\end{tikzpicture}	
	\end{center}
	Notice that the block for $s_0$ cannot be in between the blocks for $s_1$ which are highlighted in \textcolor{rred}{red}, since otherwise the corresponding expression for $w$ is not reduced. This implies that we have
	\begin{center}
	\begin{tikzpicture}[scale=0.45]
		\dheapblock{0}{2}{}{black}
		\dheapblock{2}{6}{}{black}
		\sheapblock{3}{0}{3}{purple}
		\heapblock{1}{0}{1}{rred}
		\dheapblock{3}{0}{}{black}
		\heapblock{2}{2}{2}{rred}
		\heapblock{1}{4}{1}{rred}
		\heapblock{0}{6}{0}{orange}
		\heapblock{1}{8}{1}{orange}
		\heapblock{0}{10}{0}{orange}
	\end{tikzpicture}	
	\end{center}
	where we have indicated the absence of the block for $s_0$ as usual. In the above there is a new braid has height $k-2$ which is has height lower than $k$, the height of the original braid that we chose. This is a contradiction to the way in which we chose $w$. 
	
	Subcase (4.1.2): Now we consider the case where the block for $s_3$ is in the $(k-4)$th row and the block for $s_1$ is not. This leads to the following subheap:
	\begin{center}
	\begin{tikzpicture}[scale=0.45]
		\dheapblock{2}{6}{}{black}
		\dheapblock{1}{0}{}{black}
		\dheapblock{0}{2}{}{black}
		\heapblock{3}{0}{3}{rred}
		\heapblock{2}{2}{2}{purple}
		\heapblock{1}{4}{1}{orange}
		\heapblock{0}{6}{0}{orange}
		\heapblock{1}{8}{1}{orange}
		\heapblock{0}{10}{0}{orange}
	\end{tikzpicture}
	\end{center}
	Although there are no new braids present, the bottom row of the subheap is not the bottom row of the heap for $w$ since otherwise $w$ would have Property T. Repeating the above argument we extend our heap to look like
	\begin{center}
	\begin{tikzpicture}[scale=0.45]
		%\dheapblock{3}{6}{}{black}
		%\dheapblock{4}{4}{}{black}
		%\dheapblock{2}{0}{}{black}
		%\dheapblock{5}{2}{}{black}
		\dheapblock{2}{8}{}{black}
		\dheapblock{2}{0}{}{black}
		\dheapblock{1}{2}{}{black}
		\heapblock{4}{0}{4}{rred}
		\heapblock{3}{2}{3}{rred}
		\heapblock{2}{4}{2}{purple}
		\heapblock{1}{6}{1}{orange}
		\heapblock{0}{8}{0}{orange}
		\heapblock{1}{10}{1}{orange}
		\heapblock{0}{12}{0}{orange}
	\end{tikzpicture}
	\end{center} 
	where again we see no new braids. Again, we know that the bottom row of the subheap above is not the bottom row of the heap for $w$ since otherwise $w$ would have Property T. Iterating this process we obtain a heap that looks like:
	\begin{center}
	\begin{tikzpicture}[scale=0.45]
		%\dheapblock{3}{6}{}{black}
		%\dheapblock{4}{4}{}{black}
		%\dheapblock{5}{2}{}{black}
		\dheapblock{2}{8}{}{black}
		\heapblock{0}{12}{0}{orange}
		\heapblock{1}{10}{1}{orange}
		\heapblock{0}{8}{0}{orange}
		\heapblock{1}{6}{1}{orange}
		\heapblock{2}{4}{2}{purple}
		\dheapblock{1}{2}{}{black}
		\heapblock{3}{2}{3}{rred}
		\dheapblock{2}{0}{}{black}
		\heapblock{4}{0}{4}{rred}
		
		%\node[] at (7,-1.5){$\ddots$};
		\node[] at (5,-1.5){$\ddots$};
		\node[] at (3,-1.5){$\ddots$};
		
		%\dheapblock{10}{-3.5}{}{black}
		%\dheapblock{11}{-5.5}{}{black}
		%\dheapblock{10}{-7.5}{}{black}
		\dheapblock{5}{-7.5}{}{black}
		\dheapblock{6}{-5.5}{}{black}
		\dheapblock{5}{-3.5}{}{black}
		\heapblock{7}{-3.5}{n-1}{rred}
		\heapblock{8}{-5.5}{n}{rred}
		\heapblock{7}{-7.5}{n-1}{rred}
		
		
%		\node[] at (6,-9){$\iddots$};
%		\node[] at (4,-9){$\iddots$};
%		\node[] at (8,-9){$\iddots$};
	\end{tikzpicture}	
	\end{center}
	Then by Lemma~\ref{lem:zigzag} we know that $w$ has Property T on the right. This is a contradiction to the way in which we chose the heap for $w$. 
	
	Subcase (4.2): Assume the braid of height $k$ is $s_0s_1s_0s_1$. We now consider which blocks may occur in row $k-4$. Notice that $s_0$ cannot be in the $(k-4)$th row as the corresponding expression for $w$ would not be reduced. This implies that $s_2$ is in the $(k-4)$th row as the $(k-3)$th row cannot be row 1, otherwise we are in Case (1.2.1). From this we get the subheap
	\begin{center}
	\begin{tikzpicture}[scale=0.45]
		\heapblock{2}{0}{2}{purple}
		\dheapblock{0}{0}{}{black}
		\dheapblock{2}{4}{}{black}
		\heapblock{1}{2}{1}{orange}
		\heapblock{0}{4}{0}{orange}
		\heapblock{1}{6}{1}{orange}
		\heapblock{0}{8}{0}{orange}
	\end{tikzpicture}	
	\end{center}
	where we have highlighted the braid in \textcolor{orange}{orange}. Applying the braid move we get the following subheap
	
	\begin{center}
	\begin{tikzpicture}[scale=0.45]
		\dheapblock{2}{4}{}{black}
		\heapblock{0}{0}{0}{orange}
		\heapblock{2}{0}{2}{purple}
		\heapblock{1}{2}{1}{orange}
		\heapblock{0}{4}{0}{orange}
		\heapblock{1}{6}{1}{orange}
	\end{tikzpicture}	
	\end{center}
	where the height of the braid is now $k-1$. This is a contradiction to our original assumption that the heap we started with contains the lowest braid.
	
	Therefore, it follow that $W(\C_n)$ does not contain any not FC $\tII$ elements. 
\end{proof}
\end{theorem}

We will now classify the $\tII$ elements in $W(\C_n)$. We first classify $\tII$ elements in $W(\C_n)$ for $n$ odd and then proceed to the classification for $n$ even.

\begin{theorem}
	If $n$ is odd, then there are no $\tII$ elements in Coxeter systems of type $\C_n$.
	\begin{proof}
		Consider the Coxeter system of type $\C_n$. By Theorem~\ref{thm:TavoidC} we know that $W(\C_n)$ contains no $\tII$ elements that are not FC. Recall that $W(\C_n)$ is a star reducible Coxeter group, which implies that $W(\C_n)$ contains no $\tII$ elements that are FC. Thus, ,as $W(\C_n)$ has no $\tII$ elements that are FC and no $\tII$ elements that are not FC, $W(\C_n)$ has no $\tII$ elements.
	\end{proof}
\end{theorem}

We next classify the $\tII$ elements in Coxeter systems of type $\C_n$ for $n$ even. Recall that $W(\C_n)$ for $n$ even is not a star reducible Coxeter group. In Theorem~\ref{thm:TavoidC} we showed that $W(\C_n)$ does not have $\tII$ elements that are not FC. This leaves us with only the FC elements to check.

\begin{theorem}
	If $n$ is even, then the only $\tII$ elements in $W(\C_n)$ are sandwich stacks.
	\begin{proof}
		Let $w \in W(\C_n)$ such that $w$ is $\tII$. By Theorem~\ref{thm:TavoidC}, we know that $w$ is an FC element. Further, we can restrict our search to the subset of non-cancellable elements that are not star reducible. Specifically we can consider the non-cancellable elements that do not contain Property T. Recall that in Remark~\ref{rem:noncancel} we stated that the only $\tII$ elements with full support are sandwich stacks. Thus, the only $\tII$ elements in $W(\C_n)$ for $n$ odd are sandwich stacks.
	\end{proof}
\end{theorem}

%%%%%%%%%%%%%%%%%%%
\subsection{T-Avoiding Elements of Type $B_n$}\label{sec:typeB}

Unlike the type $\C_n$ case, classifying T-avoiding elements in Coxeter systems of type $B_n$ is straight forward. 

\begin{theorem}\label{thm:typeB}
There are no $\tII$ elements in Coxeter systems of type $B_n$.
\begin{proof}
	Each $W(B_n)$ is a parabolic subgroup of $W(\C_k)$ for $k \geq n$ and $k$ odd. Since $W(\C_k)$ for $k$ odd has no $\tII$ elements, then $W(B_n)$ will not have any $\tII$ elements.
\end{proof}
\end{theorem}
   
In the next Chapter, we characterize Property T and T-avoiding in Coxeter systems in terms of signed pattern avoidance.   

%%%%%%%%%%%%%%%%%%%%%%%%%%%%%%%%

\section{Characterization of Property T in Coxeter systems of Type $B_n$}\label{chap:BnandCn}

Mimicking the classification of T-avoiding elements of Coxeter systems of Type $D_n$ seen in~\cite{Gern2013a}, we characterize Property T and T-avoiding elements of Coxeter systems in type $B_n$. We start by introducing some combinatorial tools for Coxeter systems of type $B_n$. 

\subsection{Combinatorial Tools}\label{sec:Btools}

Recall from Example~\ref{ex:B} that $W(B_n) \cong \Sym_n^B$ (also called the hyperoctahedral group). We define $\Sym_n^B$ to be the group of all bijections $w$ of the set $\{-n, \ldots, -1, 0, 1, 2, \ldots, n\}$ such that 
\[w(-a)=-w(a)\] for all $a \in \{-n, \ldots, -1, 0, 1, 2, \ldots, n\}$. For $w \in \Sym_n^B$ we write $w=[a_1, a_2, \ldots, a_n]$, to mean that $w(i)=a_i$ for $i \in \{1,2, \ldots n\}$ and call this the signed permutation notation of $w$. That is, we can write $w \in W(B_n)$ using signed permutation notation 
\[ w=[w(1),w(2), \ldots, w(n-1), w(n)], \]
where we write a bar underneath a number in place of a negative sign in order to simplify notation. 

As a set of generators for $\Sym_n^B$ we take $S(B_n)=\{s_0,s_1,s_2, \ldots, s_{n-1}\}$, where for each $i \in \{1,2,\ldots n-1\}$, we have
\[s_i=[1,2, \ldots i-1, i+1,i,i+2, \ldots, n-1,n] \] and we identify $s_0$ with
\[s_0=[\underline{1}, 2 \ldots, n].\] Further $w(-i)=-w(i)$ for $|i| \in \{1,2, \ldots, n\}$. The following propositions provide insight into what happens to a given signed permutation when we multiply by $s_i$ on the right or the left.

\begin{proposition}
	Let $w \in W(B_n)$ with corresponding signed permutation 
	\[w=[w(1),w(2), \ldots ,w(n)].\] Suppose $s_i \in S(B_n)$. If $i \geq 1$, then multiplying $w$ on the right by $s_i$ has the effect of interchanging $w(i)$ and $w(i+1)$ in  the signed permutation notation. If $i=0$, then multiplying $w$ on the right by $s_i$ has the effect of switching the sign of $w(1)$. 	
	\begin{proof}
	This follows from~\cite[Section 8.1 and A3.1]{Bjorner2005}.	
	\end{proof}
\end{proposition}

\begin{proposition}
Let $w \in W(B_n)$ with corresponding signed permutation 
	\[w=[w(1),w(2), \ldots ,w(n)].\] Suppose $s_i \in S(B_n)$. If $i \geq 1$, then multiplying on the left by $s_i$ has the effect of interchanging the entries whose absolute values are $i$ and $i+1$ in the signed permutation notation. If $i=0$, then multiplying $w$ on the left by $s_i$ has the effect of switching the sign of the entry whose absolute value is $1$.
	\begin{proof}
	This follows from~\cite[Section 8.1 and A3.1]{Bjorner2005}.	
	\end{proof}
\end{proposition}

Suppose $w \in W(B_n)$ has reduced expression $\w=s_{x_1}s_{x_2}\cdots s_{x_n}$. We may construct the signed permutation of $w$ from left to right as it is the easier way to multiply based upon the above propositions. However, note that our convention is still composition from left to right. We provide an example of this construction below.

\begin{example}
Let $w \in W(B_6)$ with a given reduced expression $\w=s_0s_1s_3s_4s_5s_2$. Then we iteratively build the signed permutation as follows. First, $s_0=[\underbar{1},2,3,4,5,6]$ by definition. Next $s_0s_1=[2,\underbar{1},3,4,5,6]$ since multiplying by $s_1$ on the right hand side switches the values in position 1 and position 2. Repeating this we get $s_0s_1s_3=[2,\underbar{1},4,3,5,6]$ and ultimately we end with $w=[2,4,\underbar{1},5,6,3]$. 

%Notice that if we were to construct the signed permutation for $w$ from right to left, we would start with $s_2=[1,3,2,4,5,6]$. Next we would have $s_5s_2=[1,3,2,4,6,5]$. However, $s_4s_5s_2=[1,3,2,5,6,4]$. Notice this time we were not able to just switch $w(i)$ and $w(i+1)$ instead we found $4$ and $5$ and switched their relative positions, which is more difficult than constructing the signed permutation notation notation left to right, which is why we choose to construct left to right.
\end{example}

Given the signed permutation notation for an element $w \in W(B_n)$ we can easily calculate the left and right descent sets of $w$. The following proposition explains how.

\begin{proposition}\label{prop:descent}
Let $w \in W(B_n)$. Then 
\[ \mathcal{R}(w)=\{s_i \in S \mid w(i) > w(i+1)\} \]
where $w(0)=0$ by definition.
\begin{proof}
	This is~\cite[Proposition 8.1.2]{Bjorner2005}.
\end{proof}
\end{proposition}

We now will introduce the concept of signed pattern avoidance, which will help with the classification of the T-avoiding elements in Coxeter systems of type $B_n$. Our approach mimics the one found in~\cite{Gern2013a}. Let $w \in W(B_n)$, and let $a,b,c \in \mathbb{Z}$. We say that $w$ \emph{contains the signed consecutive pattern} $abc$ if there is some $i \in \{1,2, \ldots, n-2\}$ such that $(|w(i)|,|w(i+1)|,|w(i+2)|)$ is in the same relative order as $(|a|,|b|,|c|)$ and $\sgn(w(i))=\sgn(a), \sgn(w(i+1))=\sgn(b)$, and $\sgn(w(i+2))=\sgn(c)$, where typically one takes $a,b,c$ to be a subset of the set $\{\pm1,\pm2,\pm3\}$. We say that $w$ \emph{avoids the signed consecutive pattern} $abc$ if there is no $i \in \{1,2, \ldots, n-2\}$ such that $\left(|w(i)|, |w(i+1)|, |w(i+2)|\right)$ is in the same consecutive order as $\left(|a|, |b|, |c| \right)$ such that $\sgn(w(i))=\sgn(a)$, $\sgn(w(i+1))=\sgn(b)$, and $\sgn(w(i+2))=\sgn(c)$.

\begin{example}
Let $w \in W(B_4)$ with signed permutation \[w=[\underline{2},4, \underline{1}, 3].\] We see that $w$ has the signed consecutive pattern $\underline{2} 3 \underline{1}$, since $(|w(1)|, |w(2)|, |w(3)|)$ are in the same relative order as $(|-2|, |3|, |-1|)$, and $\sgn(w(1))=\sgn(-2)$, $\sgn(w(2))=\sgn(3)$, and $\sgn(w(3))=\sgn(-1)$. However, $w$ avoids the signed consecutive pattern $1\underline{2}3$.
\end{example}

Occasionally, we will need to factor $w \in W(B_n)$ in a specific manner. Let $I=\{s,t\}$ for $s, t \in S(B_n)$ such that $s$ and $t$ do not commute. Define $W^I$ as the set of all $w \in W(B_n)$ such that $\LD(w) \cap I= \emptyset$ and define $W_I=\langle s,t \rangle$. In~\cite{Humphreys1990}, it is shown that any element $w \in W(B_n)$ can be written as $w=w^Iw_I$ (reduced) where $w^I \in W^I$ and $w_I \in W_I$.
%%%%%%%%%%%%%%%%%%%%%%%

\subsection{Signed Consecutive Patterns and Property T}\label{sec:TAB}

%\textcolor{red}{Time permitting we will streamline the corollaries into the Propositions themselves or as a remark at the end of the Propositions}

%In this section we classify the T-avoiding elements in Coxeter systems of type $B_n$. Our main result in this section is Theorem~\ref{thm:classificationofB}. Notice that if $n=2$, $W(B_2) \cong W(I_2(4))$, which by Theorem~\ref{thm:i2m} we know has no $\tII$ elements. We proceed with $n \geq 3$. First we need some preparatory Propositions. 

By Theorem~\ref{thm:typeB} there are no $\tII$ elements in  Coxeter systems of type $B_n$. If $n=2$, there are 8 elements in $W(B_2)$. In this case, the $\tI$ elements are
\begin{align*}
[12] &= e\\
[\underbar{1}2]&= s_0\\
[21] &= s_1	
\end{align*}
and the elements with Property T are
\begin{align*}
[\underbar{2}1] &= s_1s_0\\
[2\underbar{1}]	&= s_0s_1\\
[1\underbar{2}] &= s_1s_0s_1\\
[\underbar{21}] &= s_0s_1s_0\\
[\underbar{12}] &= s_0s_1s_0s_1 =s_0s_1s_0s_1.
\end{align*}
For the remainder of this section we assume that $n \geq 3$. There are $2^3 \cdot 3!=48$ possible choices of signed consecutive triples as seen in the table below. 

\begin{center}
\begin{tabular}{|l|l|l|l|l|l|l|l|}
\hline
\cellcolor{white}$123$ & \cellcolor{white}$\underline{1}23$ & \cellcolor{white}$1\underline{2}3$ & \cellcolor{white}$12\underline{3}$ & \cellcolor{white}$\underline{12}3$ & \cellcolor{white}$\underline{1}2\underline{3}$ & \cellcolor{white}$1\underline{23}$ & \cellcolor{white}$\underline{123}$ \\
\hline
\cellcolor{white}$132$ & \cellcolor{white}$\underline{1}32$ & \cellcolor{white}$1\underline{3}2$ & \cellcolor{white}$13\underline{2}$ & \cellcolor{white}$\underline{13}2$ & \cellcolor{white}$\underline{1}3\underline{2}$ & \cellcolor{white}$1\underline{32}$ & \cellcolor{white}$\underline{132}$ \\
\hline
\cellcolor{white}$213$ & \cellcolor{white}$\underline{2}13$ & \cellcolor{white}$2\underline{1}3$ & \cellcolor{white}$21\underline{3}$ & \cellcolor{white}$\underline{21}3$ & \cellcolor{white}$\underline{2}1\underline{3}$ & \cellcolor{white}$2\underline{13}$ & \cellcolor{white}$\underline{213}$ \\
\hline
\cellcolor{white}$231$ & \cellcolor{white}$\underline{2}31$ & \cellcolor{white}$2\underline{3}1$ & \cellcolor{white}$23\underline{1}$ & \cellcolor{white}$\underline{23}1$ & \cellcolor{white}$\underline{2}3\underline{1}$ & \cellcolor{white}$2\underline{31}$ & \cellcolor{white}$\underline{231}$ \\
\hline
\cellcolor{white}$312$ & \cellcolor{white}$\underline{3}12$ & \cellcolor{white}$3\underline{1}2$ &\cellcolor{white}$31\underline{2}$ & \cellcolor{white}$\underline{31}2$ & \cellcolor{white}$\underline{3}1\underline{2}$ & \cellcolor{white}$3\underline{12}$ & \cellcolor{white}$\underline{312}$ \\
\hline
\cellcolor{white}$321$ & \cellcolor{white}$\underline{3}21$ & \cellcolor{white}$3\underline{2}1$ & \cellcolor{white}$32\underline{1}$ & \cellcolor{white}$\underline{32}1$ & \cellcolor{white}$\underline{3}2\underline{1}$ & \cellcolor{white}$3\underline{21}$ & \cellcolor{white}$\underline{321}$\\
\hline
\end{tabular}
\end{center}

The following propositions provide a characterization of elements with Property T in Coxeter systems of type $B_n$ for $n \geq 3$ in terms of pattern containment.

\begin{proposition}\label{lem:sts}
An element $w \in W(B_n)$ has a reduced expression ending in $s_is_{i+1}s_i$ where $m(s_i,s_{i+1})=3$ if and only if $w$ contains one of the following signed consecutive patterns beginning at position $i$:
%
\begin{center}
\begin{tabular}{llll}
$321$             & $32\underbar{1}$ & $1 \underbar{23}$ & $\underbar{123}$  \\
$31 \underbar{2}$ & $3 \underbar{12}$ & $21\underbar{3}$  & $2 \underbar{13}$ 

\end{tabular}	
\end{center}
%
\begin{proof}
	Suppose $w$ contains one of the signed consecutive patterns seen above. Then there is some $i$ such that $w(i) > w(i+1) > w(i+2)$. By Proposition~\ref{prop:descent}, $s_i,s_{i+1} \in \mathcal{R}(w)$. Since $m(s_i,s_{i+1})=3$ and $s_i, s_{i+1} \in \RD(w)$, $w$ ends in $s_is_{i+1}s_{i}$ or $s_{i+1}s_is_{i+1}$.
	
	Conversely, suppose $w$ has a reduced expression ending in $s_is_{i+1}s_i$ where $m(s_i,s_{i+1})=3$. This implies that $w_I=s_is_{i+1}s_i$ where $I=\{s_i, s_{i+1}\}$ which implies that $s_i,s_{i+1} \in \mathcal{R}(w)$. Since $s_i,s_{i+1} \in \mathcal{R}(w)$, we see that $w(i)>w(i+1)>w(i+2)$ by Proposition~\ref{prop:descent}. Thus, $w$ contains one of the signed consecutive patterns seen above.
\end{proof}	
\end{proposition}

%\begin{corollary}\label{lem:endswithsts}
%	An element $w \in W(B_n)$ has a reduced expression beginning in $s_is_{i+1}s_i$ where $m(s_i,s_{i+1})=3$ if and only if $w$ contains one of the following consecutive patterns beginning in position $i$:
%
%\begin{center}
%\begin{tabular}{llll}
%$321$             & $32\underbar{1}$ & $1 \underbar{23}$ & $\underbar{123}$  \\
%$31 \underbar{2}$ & $3 \underbar{12}$ & $21\underbar{3}$  & $2 \underbar{13}$ 
%
%\end{tabular}	
%\end{center}
%
%	\begin{proof}
%		 We know that $w$ has no reduced expressions beginning with $s_is_{i+1}s_i$ if and only if $w^{-1}$ has no reduced expression ending with $s_is_{i+1}s_i$ which by Lemma~\ref{lem:sts} happens only if $w^{-1}$ avoids the signed consecutive patterns seen above beginning in position $i$.
%	\end{proof}
%\end{corollary}

\begin{proposition}\label{lem:ts}
An element $w \in W(B_n)$ has a reduced expression ending in $s_is_{i+1}$ where $m(s_i,s_{i+1})=3$ if and only if $w$ contains one of the following consecutive patterns beginning at position $i$:
%
\begin{center}
\begin{tabular}{llll}
$231$             & $23\underbar{1}$ & $12 \underbar{3}$ & $\underbar{1}2\underbar{3}$  \\
$\underline{1}32$ & $13 \underbar{2}$ & $\underbar{2}1\underbar{3}$  & $\underbar{213}$ 

\end{tabular}	
\end{center}
%
\begin{proof}	
	Suppose $w$ contains one of the signed consecutive patterns seen above. Then there is some $i$ such that $w(i+1)>w(i)>w(i+2)$. By Proposition~\ref{prop:descent}, $s_{i+1} \in \mathcal{R}(w)$. Now multiplying on the right by $s_{i+1}$ we see that $ws_{i+1}(i+1)=w(i+2)$ and $ws_{i+1}(i)=w(i)$. We know that $w(i+2)<w(i)$, which implies that $s_i \in \mathcal{R}(ws_{i+1})$, and hence $w$ has a reduced expression that ends in $s_is_{i+1}$.
	
	 Conversely, suppose that $w$ has a reduced expression ending in $s_is_{i+1}$ where $m(s_i,s_{i+1})=3$. Then $w(i+2)<w(i+1)$ and $w(i)<w(i+1)$. Since $s_i \in \mathcal{R}(ws_{i+1})$ we have $w(i+2)=ws_{i+1}(i+1)<ws_{i+1}(i)=w(i)$. Thus, we have that $w(i+1) > w(i) > w(i+2)$. Hence $w$ contains one of the signed consecutive patterns from above.
\end{proof}	
\end{proposition}

%\begin{corollary}\label{lem:endswithst}
%	An element $w \in W(B_n)$ has a reduced expression beginning in $s_is_{i+1}$ where $m(s_i,s_{i+1})=3$ if and only if $w$ contains one of the following consecutive patterns beginning in position $i$:
%
%\begin{center}
%\begin{tabular}{llll}
%$231$             & $23\underbar{1}$ & $12 \underbar{3}$ & $\underbar{1}2\underbar{3}$  \\
%$\underline{1}32$ & $13 \underbar{2}$ & $\underbar{2}1\underbar{3}$  & $\underbar{213}$ 
%
%\end{tabular}	
%\end{center}.
%	\begin{proof}
%		We know that $w$ has no reduced expressions beginning with $s_is_{i+1}$ if and only if $w^{-1}$ has no reduced expression ending with $s_is_{i+1}$ which by Lemma~\ref{lem:ts} happens only if $w^{-1}$ avoids the signed consecutive patterns seen above in position $i$. 
%	\end{proof}
%\end{corollary}

\begin{proposition}\label{lem:st}
An element $w \in W(B_n)$ has a reduced expression ending in $s_{i+1}s_i$ where $m(s_i, s_{i+1})=3$ if and only if $w$ contains one of the following consecutive patterns beginning at position $i$:
%
\begin{center}
\begin{tabular}{llll}
$312$             & $3\underbar{1}2$ & $3 \underbar{2}1$ & $3\underbar{21}$  \\
$2\underbar{3}1$ & $2\underbar{31}$ & $1\underbar{32}$  & $\underbar{132}$ 

\end{tabular}	
\end{center}
%
\begin{proof}
	Suppose that $w$ contains one of the signed consecutive patterns seen above.  Then there is some $i$ such that $w(i)>w(i+2)>w(i+1)$. By Proposition~\ref{prop:descent} we see that $s_i \in \mathcal{R}(w)$. Multiplying on the right by $s_i$ we get $ws_i(i+1)=w(i)$ and $ws_i(i+2)=w(i+2)$. By above $w(i)>w(i+2)$, and by Proposition~\ref{prop:descent} $s_{i+1} \in \mathcal{R}(ws_i)$. This implies that $w$ has a reduced expression ending in $s_{i+1}s_i$. 
	
	Conversely suppose $w$ ends in a reduced expression with $s_{i+1}s_i$ where $m(s_i,s_{i+1})=3$. Then $w_I=s_{i+1}s_i$. We see that $w(i)>w(i+1)$ and $w(i+2)>w(i+1)$. Since $s_{i+1} \in \mathcal{R}(ws_i)$, we have $w(i+2)=ws_i(i+2)<ws_i(i+1)=w(i)$. From this we have $w(i)>w(i+2)$, so $w(i)>w(i+2)>w(i+1)$. Hence, $w$ contains one of the signed consecutive patterns seen above. 
\end{proof}
\end{proposition}

%\begin{corollary}\label{lem:endswithts}
%An element $w \in W(B_n)$ has a reduced expression beginning in $s_{i+1}s_i$ where $m(s_i, s_{i+1})=3$ if and only if $w$ contains one of the following consecutive patterns beginning in position $i$:
%
%\begin{center}
%\begin{tabular}{llll}
%$312$             & $3\underbar{1}2$ & $3 \underbar{2}1$ & $3\underbar{21}$  \\
%$2\underbar{3}1$ & $2\underbar{31}$ & $1\underbar{32}$  & $\underbar{132}$ 
%
%\end{tabular}	
%\end{center}
%	\begin{proof}
%		Let $s_i,s_{i+1} \in S(B_n)$ such that $m(s,t)=3$ and $s_0 \notin\{s_i,s_{i+1}\}$. We know that $w$ has no reduced expression beginning with $s_{i+1}s_i$ if and only if $w^{-1}$ has no reduced expression ending with $s_{i+1}s_i$ which by Lemma~\ref{lem:st} happens only if $w^{-1}$ avoids the signed consecutive patterns seen above beginning in position $i$.
%		\end{proof}
%\end{corollary}

\begin{proposition}\label{lem:endswiths0}
Let $w \in W(B_n)$. Then $w$ has a reduced expression ending in $s_1s_0$ if and only if $w$ contains one of the signed consecutive patterns:
\begin{center}
\begin{tabular}{llllllll}
$\underbar{12}3$ & $\underbar{123}$ & $\underbar{13}2$            & $\underbar{132}$ & $\underbar{2}13$ & $\underbar{2}1\underbar{3}$ & $\underbar{21}3$            & $\underbar{213}$ \\
$\underbar{3}12$ & $\underbar{31}2$ & $\underbar{3}1\underbar{2}$ & $\underbar{312}$ & $\underbar{3}21$ & $\underbar{32}1$            & $\underbar{3}2\underbar{1}$ & $\underbar{321}$
\end{tabular}	
\end{center}

\begin{proof}
	Suppose $w \in W(B_n)$ such that $w$ ends with $s_1s_0$. Then $s_0 \in \RD(w)$ and $s_1 \in \RD(ws_0)$. This implies that $ws_0(1)>ws_0(2)$ by~Proposition~\ref{prop:descent}. We see that $ws_0(1)=w(-1)=-w(1)$ and $ws_0(2)=2$. Hence $-w(1)=ws_0(1)>ws_0(2)=w(2)$. Further, since $s_0 \in \RD(w)$, we see that $w(0)>w(1)$. Notice that in all of the above patterns $w(0)>w(1)$ and $-w(1)>w(2)$.
	
	Conversely, suppose $w \in W(B_n)$ contains one of the above signed consecutive patterns. Notice that in all of the above patterns $w(0)>w(1)$ and $-w(1)>w(2)$. Since $w(0)>w(1)$, we know that $s_0 \in \RD(w)$. Multiplying on the right by $s_0$ we see that $ws_0(1)=-w(1)$ and $ws_0(2)=w(2)$. Note that since $ws_0(1)=-w(1)>w(2)=ws_0(2)$, $s_1 \in \RD(ws_0)$. Thus, $w$ ends with $s_1s_0$. 
\end{proof}
\end{proposition}

%\begin{corollary}\label{lem:beginswiths0}
%	Let $w \in W(B_n)$. Then $w$ has a reduced expression beginning in $s_0s_1$ if and only if $w^{-1}(0)>w^{-1}(1)$ and $-w^{-1}(1)>w^{-1}(2)$.
%	\begin{proof}
%		Let $w \in W(B_n)$. We know that $w$ has no reduced expressions beginning in $s_0s_1$ if and only if $w^{-1}$ has no reduced expressions ending in $s_0s_1$. By Lemma~\ref{lem:endswiths0} we know that this occurs if and only if $w^{-1}(0)>w^{-1}(1)$ and $-w^{-1}(1)>w^{-1}(2)$.
%	\end{proof}
%\end{corollary}

\begin{proposition}\label{lem:endswiths_1}
Let $w \in W(B_n)$. Then $w$ has a reduced expression ending in $s_0s_1$ if and only if $w$ contains one of the following signed consecutive patterns:
\begin{center}
\begin{tabular}{llllllll}
$2 \underbar{1} 3$ & $ 2 \underbar{13}$ & $ \underbar{23}1$  & $ \underbar{231}$  & $3\underbar{2}1 $ & $ 3 \underbar{21}$ & $ 3 \underbar{1}2$ & $ 3\underbar{12}$ \\
$ 1 \underbar{2}3$ & $ 1 \underbar{23}$ & $ 1 \underbar{3}2$ & $ 1 \underbar{32}$ & $ \underbar{12}3$ & $ \underbar{123}$  & $ \underbar{13}2$  & $ \underbar{132}$
\end{tabular}
\end{center}
\begin{proof}
	Suppose $w \in W(B_n)$ such that $w$ ends with $s_0s_1$. Then $s_1 \in \RD(w)$ and $s_0 \in \RD(ws_1)$. Then $ws_1(0)>ws_1(1)$. We see that $ws_1(0)=0$ and $ws_1(1)=w(2)$. This implies that $0=ws_1(0)>ws_1(1)=2$. Further, since $s_1 \in \RD(w)$ this implies that $w(1) > w(2)$. Thus, if $w$ ends with $s_0s_1$, then $w(1)>w(2)$ and $w(0)>w(2)$. Notice that in the above signed consecutive patterns, all of the patterns have $w(0)>w(2)$ and $w(1)>w(2)$.
	
	Conversely, suppose $w \in W(B_n)$ contains one of the above signed consecutive patterns. Notice that in all of the above patterns $w(1)>w(2)$ and $w(0)>w(2)$. This implies that $s_1 \in \RD(W)$. Multiplying $w$ on the right by $s_1$ we see that $ws_1(0)=w(0)$ and $ws_1(1)=w(2)$. Note that since $ws_1(0)=w(0)>w(2)=ws_1(1)$, $s_0 \in \RD(ws_1)$. Thus, $w$ ends with $s_0s_1$. 
\end{proof}	
\end{proposition}

\begin{remark}
In Lemmas~\ref{lem:sts}--\ref{lem:endswiths_1} we classified elements that end with a given structure. For each there is a begins with analog that can be quickly computed through evaluating the given signed consecutive patterns in $w^{-1}$. Notice that some patterns appear in more than one proposition. For example, the signed consecutive triple $3\underbar{1}2$ appears in Proposition~\ref{lem:ts} and Proposition~\ref{lem:endswiths_1}.
\end{remark}

%\begin{corollary}\label{lem:beginswiths1}
%	Let $w \in W(B_n)$. Then $w$ has a reduced expression beginning in $s_1s_0$ if and only if $w^{-1}(0)>w^{-1}(2)$ and $w^{-1}(1)>w^{-1}(2)$.
%	\begin{proof}
%		Let $w \in W(B_n)$. We know that $w$ has no reduced expressions beginning in $s_1s_0$ if and only if $w^{-1}$ has no reduced expressions ending in $s_1s_0$. By Lemma~\ref{lem:endswiths_1} we know that this occurs if and only if $w^{-1}(0)>w^{-1}(2)$ and $w^{-1}(1)>w^{-1}(2)$.
%	\end{proof}
%\end{corollary}

\begin{proposition}\label{lem:prodofcommA}
Let $w \in W(B_n)$ such that each entry for $w$ in the signed permutation notation is positive and both $w$ and $w^{-1}$ avoid the signed consecutive patterns $321$, $231$, and $312$, or $w(-1)=-1$ and every other entry is positive in the signed permutation notation. Then $w$ is a product of commuting generators.
\begin{proof}
	When all entries are positive this from an appropriate translation of~\cite[Lemma 2.2.9]{Gern2013a}. It quickly follows that if $w(1)=-1$, then $w=s_0u$ (reduced) where $u$ is a product of commuting generators coming from %$\{s_1, \s_2, \ldots, s_{n-1}\}$.
\end{proof}	
\end{proposition}

%\begin{proposition}\label{lem:prodofCommB}
%Let $w \in W(B_n)$ be $\tI$ and let $i \in \{1,2, \ldots, n\}$. Then $w$ satisfies all the following conditions:
%\begin{enumerate}[leftmargin=2cm]
%\item $w(j) > \min\{w(i-1), w(i)\}$ for all $j >i$;\label{it:trivT1}
%\item $w(k) < \max\{w(i-1), w(i)\}$ for all $k < i-1$;\label{it:trivT2}
%\item If $w(i), w(i+1) > 0$, then $w(j)>0$ for all $j \geq i$;\label{it:trivT3}
%\item If $w(i), w(i+1) < 0$, then $w(j)<0$ for all $j \leq i+1$.\label{it:trivT4}
%\end{enumerate}
%\begin{proof}
%	Suppose there is some least $j>i$ such that $w(j) \leq \min\{w(i-1), w(i)\}$. Note that $j>i$ so $w(j) \neq w(i)$, and $w(j) \neq w(i-1)$ so $w(j) < \min\{w(i-1), w(i)\}$. Then $w(j-2) \geq \min\{w(i+1), w(i)\}>w(j)$. This implies that either $w(j-1)>w(j-2)>w(j)$ or $w(j-2)>w(j-1)>w(j)$, which implies $w$ contains the signed consecutive pattern $231$ or $321$, which is a contradiction to $w$ being a $\tI$ element by Propositions~\ref{lem:sts} and~\ref{lem:st}. Thus, proving~\ref{it:trivT1}.
%	
%	Suppose there exists a maximal $k<i-1$ such that $w \geq \max\{w(i-1), w(i)\}$. Note that $k < i-1$ so $k \neq i$ and $k \neq i-1$. Then $w(k)> \max\{w(i-1), w(i)\}$. Since $k$ is maximal $w(k+1) \leq  \max\{w(i-1), w(i)\}$ and $w(k+2) \leq \max\{w(i-1), w(i)\}$. This implies that either $w(k+2)<w(k+1)<w(k)$ or $w(k+1)<w(k+2)<w(k)$, which implies $w$ contains the signed consecutive pattern $321$ or $312$, which is a contradiction to $w$ being a $\tI$ element by Propositions~\ref{lem:sts} and~\ref{lem:ts}. Thus, proving~\ref{it:trivT2}.
%	
%	It is easy to see that Assertion~\ref{it:trivT1} implies~\ref{it:trivT3} and Assertion~\ref{it:trivT2} implies~\ref{it:trivT4}.
%\end{proof}	
%\end{proposition}

\begin{proposition}\label{lem:231}
Let $w \in W(B_n)$ such that $w$ contains the signed consecutive pattern $\underline{2}31$ and does not contain the signed consecutive pattern $231$. Then $w$ has Property T.
\begin{proof}
	Let $w \in W(B_n)$ such that $w$ contains the signed consecutive pattern $\underline{2}31$ and does not contain the signed consecutive pattern $231$.
	
	Case (1): Suppose $w$ has the signed permutation notation $w=[\underline{2},3,1]$. This implies that $w=s_1s_0s_2$. Some reduced expression for $w$ begins with a product of non-commuting generators. Thus, $w$ has Property T.
	
	Case (2): Suppose that $w$ has the signed permutation notation $w=[\underline{a},b,c, \ast, \ldots, \ast]$ where $\underline{a}bc$ corresponds to the signed consecutive pattern $\underline{2}31$, and $\ast$ indicates unknown values for $w(i)$ for $i=4,5, \ldots, n$. We now consider the possible signed consecutive pattern $bc \ast$. The possibilities are: $312$, $31\underline{2}$, $321$, $32\underline{1}$, $213$, or $21\underline{3}$. We know that $b$ and $c$ must be positive since they are positive in $w$ and we also know that $b>c$ by the original signed consecutive pattern. Note that by Propositions~\ref{lem:sts},and~\ref{lem:st} all of these patterns imply that $w$ has a reduced expression that begins or ends with a product of non-commuting generators. Thus, $w$ has Property T.
	
	Case (3): Suppose that $w$ has the signed permutation notation 
	\[w=[\ast, \ldots, \ast, \underline{a},b,c, \ast, \ldots, \ast]\] where $\underline{a}bc$ corresponds to the signed consecutive pattern $\underline{2}31$, and $\ast$ indicates unknown values for $w(i)$ for $|w(i)|\neq a,b,c$. We now consider the possible signed consecutive pattern $\ast \underline{a}b$. The possibilities are: $1 \underline{2} 3$, $\underline{1} \underline{2}3$, $2 \underline{1} 3$, $\underline{2} \underline{1} 3$, $2 \underline{3}1$, $\underline{23}1$, $3 \underline{1} 2$, $\underline{3} \underline{1} 2$, $3 \underbar{1}2$, or $\underbar{31}2$. Note that by Propositions~\ref{lem:st},~\ref{lem:endswiths0}, and~\ref{lem:endswiths_1} all of these patterns imply that $w$ has a reduced expression that begins or ends with a product of non-commuting generators. Thus, $w$ has Property T. 
%	
%	Case (4): Suppose that $w$ has the signed permutation notation $w=[\ast, \ldots, \ast, \underline{a},b,c, \ast, \ldots, \ast]$ where $\underline{a}bc$ corresponds to the signed consecutive pattern $\underline{2}31$, and $\ast$ indicates unknown values for $w(i)$ for $|w(i)|\neq a,b,c$. In this case we can apply either Case (2) or Case (3) and we can conclude that $w$ has a reduced expression that begins or ends with a product of non-commuting generators. Thus, $w$ has Property T.
\end{proof}	
\end{proposition}

\begin{proposition}\label{lem:2312}
Let $w \in W(B_n)$ such that $w$ contains the signed consecutive pattern $\underline{2}3\underline{1}$. Then $w$ has Property T.
\begin{proof}
	Let $w \in W(B_n)$ such that $w$ contains the signed consecutive pattern $\underline{2}3\underline{1}$.
	
	Case (1): Suppose $w$ has the signed permutation notation $w=[\underline{2},3,\underline{1}]$. This implies that $w=s_0s_1s_0s_2$. Some reduced expression for $w$ begins with a product of non-commuting generators. Thus, $w$ has Property T.
	
	Case (2): Suppose that $w$ has the signed permutation notation $w=[\underline{a},b,\underline{c}, \ast, \ldots, \ast]$ where $\underline{a}b\underline{c}$ corresponds to the signed consecutive pattern $\underline{2}3\underline{1}$, and $\ast$ indicates unknown values for $w(i)$ for $i=4,5, \ldots n$. We now consider the possible signed consecutive pattern $b \underline{c} \ast$. The possibilities are: $3\underline{1}2$, $3\underline{1}\underline{2}$, $3\underline{2}1$, $3 \underline{2}\underline{1}$, $2\underline{1}3$, $2\underline{1}\underline{3}$, $2\underline{3}1$, or $2 \underline{31}$. We know that $b$ must be positive since it is positive in $w$, $c$ must be negative since it is negative in $w$, and we also know that $|b|>|c|$ by the original signed consecutive pattern. Note that by Proposition~\ref{lem:ts} all of these patterns imply that $w$ has a reduced expression that begins or ends with a product of non-commuting generators. Thus, $w$ has Property T.
	
	Case (3): Suppose that $w$ has the signed permutation notation \[w=[\ast, \ldots,\ast, \underline{a},b,\underline{c}, \ast, \ldots, \ast]\] where $\underline{a}b\underline{c}$ corresponds to the signed consecutive pattern $\underline{2}3\underline{1}$, and $\ast$ indicates unknown values for $w(i)$ for $|w(i)|\neq a,b,c$. We now consider the possible signed consecutive pattern $\ast ab$. The possibilities are: $1 \underline{2} 3$, $\underline{1} \underline{2} 3$, $1\underline{3}2$, $\underline{13}2$,  $2 \underline{1} 3$, $\underline{2} \underline{1} 3$, $2 \underline{3}1$, $\underline{23}1$, $3 \underline{1}2$, $\underline{31}2$, $3 \underline{2} 1$, or $\underline{3} \underline{2} 1$. We know that $a$ must be negative, $b$ must be positive and $|a|<|b|$ by the original signed permutation. Note that by Propositions~\ref{lem:st},~\ref{lem:endswiths0}, and~\ref{lem:endswiths_1} all of these patterns imply that $w$ has a reduced expression that begins or ends with a product of non-commuting generators. Thus, $w$ has Property T. 
%
%	Case (4): Suppose that $w$ has the signed permutation notation $w=[\ast, \ldots,\ast, \underline{a},b,\underline{c}, \ast, \ldots, \ast]$ where $\underline{a}b\underline{c}$ corresponds to the signed consecutive pattern $\underline{2}3\underline{1}$, and $\ast$ indicates unknown values for $w(i)$ for $|w(i)|\neq a,b,c$. In this case we can apply either Case (2) or Case (3) and we can conclude that $w$ has a reduced expression that begins or ends with a product of non-commuting generators. Thus, $w$ has Property T.
\end{proof}	
\end{proposition}

\begin{proposition}\label{lem:123}
Let $w \in W(B_n)$ such that $w$ contains the signed consecutive pattern $\underline{1}23$ and does not contain the signed consecutive patterns $132$ or $123$. Then $w$ has Property T unless $n=3$ in which case, $w$ is $\tI$.
\begin{proof}
	Let $w \in W(B_n)$ such that $w$ contains the signed consecutive pattern $\underline{1}23$ and does not contain the signed consecutive patterns $132$ or $123$.
	
	Case (1): Suppose $w$ has the signed permutation notation $w=[\underline{1}23]$. This implies that $w=s_0$. Clearly, $w$ is a $\tI$ element as it is a single generator.
	
	Case (2): Suppose that $w$ has the signed permutation notation $w=[\underline{a},b,c, \ast, \ldots, \ast]$ where $\underline{a}bc$ corresponds to the signed consecutive pattern $\underline{1}23$, and $\ast$ indicates unknown values for $w(i)$ for $i=4,5, \ldots n$. We now consider the possible signed consecutive patterns $bc \ast$. The possibilities are: $231$, $23 \underline{1}$, $132$, $13 \underline{2}$, $123$, $12 \underline{3}$. We know that $b$ and $c$ are positive, and we also know that $|b|<|c|$ by the original signed consecutive pattern. Note that by Propositions~\ref{lem:ts}, and~\ref{lem:endswiths_1} all of these patterns imply that $w$ has a reduced expression that begins or ends with a product of non-commuting generators. Thus, $w$ has Property T.
	
	Case (3): Suppose that $w$ has the signed permutation notation \[w=[\ast, \ldots, \ast, \underline{a},b,c, \ast, \ldots, \ast]\] where $\underline{a}bc$ corresponds to the signed consecutive pattern $\underline{1}23$, and $\ast$ indicates unknown values for $w(i)$ for $|w(i)|\neq a,b,c$. We now consider the possible signed consecutive patterns $\ast \underline{a} b$. The possibilities are: $1 \underline{2} 3$, $\underline{1} \underline{2} 3$, $1\underline{3}2$, $\underline{13}2$,  $2 \underline{1} 3$, $\underline{2} \underline{1} 3$, $2 \underline{3}1$, $\underline{23}1$, $3 \underline{1}2$, $\underline{31}2$, $3 \underline{2} 1$, or $\underline{3} \underline{2} 1$. We know that $a$ must be negative, $b$ must be positive and $|a|<|b|$ by the original signed permutation. Note that by Propositions~\ref{lem:st},~\ref{lem:endswiths0}, and~\ref{lem:endswiths_1} all of these patterns imply that $w$ has a reduced expression that begins or ends with a product of non-commuting generators. Thus, $w$ has Property T. 
%	
%	Case (4): Suppose that $w$ has the signed permutation notation $w=[\ast, \ldots, \ast, \underline{a},b,c, \ast, \ldots, \ast]$ where $\underline{a}bc$ corresponds to the signed consecutive pattern $\underline{1}23$, and $\ast$ indicates unknown values for $w(i)$ for $|w(i)|\neq a,b,c$. In this case we can apply either Case (2) or Case (3) and we can conclude that $w$ has a reduced expression that begins or ends with a product of non-commuting generators. Thus, $w$ has Property T.
\end{proof}	
\end{proposition}

\begin{proposition}\label{lem:132}
Let $w \in W(B_n)$ such that $w$ contains the signed consecutive pattern $\underline{1}32$ and does not contain the signed consecutive pattern $213$. Then $w$ has Property T unless $n=3$ in which case, $w$ is $\tI$..
\begin{proof}
	Let $w \in W(B_n)$ such that $w$ contains the signed consecutive pattern $\underline{1}32$ and does not contain the signed consecutive pattern $213$.
	
	Case (1): Suppose $w$ has the signed permutation notation $w=[\underline{1}32]$. This implies that $w=s_0s_2$. Clearly, $w$ is a $\tI$ element as it is a product of commuting generators.
	
	Case (2): Suppose that $w$ has the signed permutation notation $w=[\underline{a},b,c, \ast, \ldots, \ast]$ where $\underline{a}bc$ corresponds to the signed consecutive pattern $\underline{1}32$, and $\ast$ indicates unknown values for $w(i)$ for $i=4,5, \ldots n$. We now consider the possible signed consecutive pattern $bc \ast$. The possibilities are: $231$, $23 \underline{1}$, $13 2$, $13 \underline{2}$, $123$, or $12\underline{3}$. We know that $b$ and $c$ are positive, and we also know that $|b|<|c|$ by the original signed consecutive pattern. Note that by Propositions~\ref{lem:sts},~\ref{lem:st}, and~\ref{lem:prodofcommA} all of these patterns imply that $w$ has a reduced expression that begins or ends with a product of non-commuting generators. Thus, $w$ has Property T.
	
	Case (3): Suppose that $w$ has the signed permutation notation 
	\[w=[\ast, \ldots, \ast, \underline{a},b,c, \ast, \ldots, \ast]\] where $\underline{a}bc$ corresponds to the signed consecutive pattern $\underline{1}32$, and $\ast$ indicates unknown values for $w(i)$ for $w(i)\neq a,b,c$. We now consider the possible signed consecutive pattern $\ast \underline{a} b$. The possibilities are: $3 \underline{1} 2$, $\underline{3} \underline{1} 2$, $2 \underline{1} 3$, $\underline{2} \underline{1} 3$, $3 \underline{2} 1$, $\underline{3} \underline{2} 1$, $1 \underline{2} 3$, $\underline{12}3$, $1\underline{3}2$, $\underline{13}2$, $2\underline{3}1$, or $\underline{23}1$. We know that $a$ must be negative, $b$ must be positive and $|a|<|b|$ by the original signed permutation. Note that by Propositions~\ref{lem:st},~\ref{lem:endswiths0}, and~\ref{lem:endswiths_1} all of these patterns imply that $w$ has a reduced expression that begins or ends with a product of non-commuting generators. Thus, $w$ has Property T. 
%	
%	Case (4): Suppose that $w$ has the signed permutation notation $w=[\ast, \ldots, \ast, \underline{a},b,c, \ast, \ldots, \ast]$ where $\underline{a}bc$ corresponds to the signed consecutive pattern $\underline{1}32$, and $\ast$ indicates unknown values for $w(i)$ for $w(i)\neq a,b,c$. In this case we can apply either Case (2) or Case (3) and we can conclude that $w$ has a reduced expression that begins or ends with a product of non-commuting generators. Thus, $w$ has Property T.
\end{proof}	
\end{proposition}

\begin{remark}
Notice that in Propositions~\ref{lem:231},~\ref{lem:123} and~\ref{lem:132} Suppose that $w=[\underline{a},b,c,d, \ast, \ldots, \ast]$ where $\ast$ indicates unknown values for $w(i)$ for $i=5,6, \ldots, n$. When $\underline{a}bc$ is the signed consecutive pattern $\underline{2}31$ and $bcd$ is the signed consecutive pattern $213$, we know that $w$ does not have Property T on the right. However, by Theorem~\ref{thm:typeB}, we know that $w$ must then have Property T on the left. The same goes for when $\underline{a}bc$ is the signed consecutive pattern $\underline{1}23$ and $bcd$ is either the signed consecutive pattern $123$ or the signed consecutive pattern $132$. Also when $\underline{a}bc$ is the signed consecutive pattern $\underline{1}32$ and $bcd$ is the signed consecutive pattern $213$.	
\end{remark}


%We now are ready to tackle one of the main results of this thesis.

%\begin{theorem}\label{thm:classificationofB}
%There are no $\tII$ elements in $W(B_n)$.	


%\begin{proof}
%We proceed by contradiction. Suppose that $w \in W(B_n)$ is a $\tII$ element. There are $2^3 \cdot 3!=48$ possible choices of signed consecutive patterns for $w(1)w(2)w(3)$ where $w=[w(1), w(2), w(3), \ast, \ldots, \ast]$. These 48 signed consecutive patterns are seen in the table below. We only consider these signed consecutive patterns in the first three entries of the signed permutation representation, as if we can eliminate all possibilities we have a contradiction to $w$ being a $\tII$ element.
%
%\begin{center}
%\begin{tabular}{|l|l|l|l|l|l|l|l|}
%\hline
%\cellcolor{white}$123$ & \cellcolor{white}$\underline{1}23$ & \cellcolor{white}$1\underline{2}3$ & \cellcolor{white}$12\underline{3}$ & \cellcolor{white}$\underline{12}3$ & \cellcolor{white}$\underline{1}2\underline{3}$ & \cellcolor{white}$1\underline{23}$ & \cellcolor{white}$\underline{123}$ \\
%\hline
%\cellcolor{white}$132$ & \cellcolor{white}$\underline{1}32$ & \cellcolor{white}$1\underline{3}2$ & \cellcolor{white}$13\underline{2}$ & \cellcolor{white}$\underline{13}2$ & \cellcolor{white}$\underline{1}3\underline{2}$ & \cellcolor{white}$1\underline{32}$ & \cellcolor{white}$\underline{132}$ \\
%\hline
%\cellcolor{white}$213$ & \cellcolor{white}$\underline{2}13$ & \cellcolor{white}$2\underline{1}3$ & \cellcolor{white}$21\underline{3}$ & \cellcolor{white}$\underline{21}3$ & \cellcolor{white}$\underline{2}1\underline{3}$ & \cellcolor{white}$2\underline{13}$ & \cellcolor{white}$\underline{213}$ \\
%\hline
%\cellcolor{white}$231$ & \cellcolor{white}$\underline{2}31$ & \cellcolor{white}$2\underline{3}1$ & \cellcolor{white}$23\underline{1}$ & \cellcolor{white}$\underline{23}1$ & \cellcolor{white}$\underline{2}3\underline{1}$ & \cellcolor{white}$2\underline{31}$ & \cellcolor{white}$\underline{231}$ \\
%\hline
%\cellcolor{white}$312$ & \cellcolor{white}$\underline{3}12$ & \cellcolor{white}$3\underline{1}2$ &\cellcolor{white}$31\underline{2}$ & \cellcolor{white}$\underline{31}2$ & \cellcolor{white}$\underline{3}1\underline{2}$ & \cellcolor{white}$3\underline{12}$ & \cellcolor{white}$\underline{312}$ \\
%\hline
%\cellcolor{white}$321$ & \cellcolor{white}$\underline{3}21$ & \cellcolor{white}$3\underline{2}1$ & \cellcolor{white}$32\underline{1}$ & \cellcolor{white}$\underline{32}1$ & \cellcolor{white}$\underline{3}2\underline{1}$ & \cellcolor{white}$3\underline{21}$ & \cellcolor{white}$\underline{321}$\\
%\hline
%\end{tabular}
%\end{center}
%
%We can use Lemma~\ref{lem:sts} and Corollary~\ref{lem:endswithsts} to eliminate the signed consecutive patterns highlighted in \textcolor{turq}{turquoise}. In addition, using Lemma~\ref{lem:st} and Corollary~\ref{lem:endswithst} to eliminate the signed consecutive patterns highlighted in \textcolor{red}{red}. From Lemma~\ref{lem:ts} and Corollary~\ref{lem:endswithts} we eliminate the signed consecutive patterns highlighted in \textcolor{ggreen}{green}.  Using Lemma~\ref{lem:endswiths0} and Corollary~\ref{lem:beginswiths0} we are able to eliminate the signed consecutive patterns highlighted in \textcolor{yellow}{yellow}. Also Lemma~\ref{lem:endswiths_1} and Corollary~\ref{lem:beginswiths1} show that $w$ will not have the signed consecutive patterns highlighted in \textcolor{brown}{brown}. We also use Lemma~\ref{lem:prodofcommA} to eliminate the signed consecutive patterns highlighted in \textcolor{blue}{blue}. From Propositions~\ref{lem:231} and~\ref{lem:2312} we are able to eliminate signed consecutive patterns highlighted in \textcolor{purple}{purple}. Finally, we can use Propositions~\ref{lem:123} and~\ref{lem:132} to eliminate signed consecutive patterns highlighted in \textcolor{orange2}{orange}. Since all of the above patterns are eliminated as possibilities for $w(1)w(2)w(3)$ and there are no other signed consecutive patterns that are possible for these positions, and hence $w$ is not a $\tII$ element in the Coxeter group of type $B_n$.
%\end{proof}
%\end{theorem}
%
%
%The upshot of Theorem~\ref{thm:classificationofB} is that the only T-avoiding elements in Coxeter systems of type $B_n$ are products of commuting generators and the identity.

%%%%%%%%%%%%%%%%%%%%
\subsection{Future Work}\label{sec:open}
In Sections~\ref{sec:tavoidA}--\ref{sec:tavoidI}, we relayed the known results involving T-avoiding elements in types $\widetilde{A}_n, A_n, D_n, F_4,$ and $F_5$, and proved results involving T-avoiding elements in type $I_2(m)$. It remains to be shown that the conjecture in Section~\ref{sec:tavoidA} regarding the classification of the $\tII$ elements in type $\widetilde{A}_n$ holds. The classification of $\tII$ elements in Coxeter systems of type $F_n$ for $n \geq 6$ also remains open.

We also portrayed several other Coxeter systems in Figures~\ref{fig:fincoxgraphs} and~\ref{fig:infincoxgraphs}. The classification of $\tII$ elements in the Coxeter systems of type $E_n$ remains an open problem. However, we do know that these groups have $\tII$ elements as $W(D_n)$ (which has $\tII$ elements) is a parabolic subgroup of $W(E_n)$. The classification of $\tII$ elements in the Coxeter systems of type $H_n$ is also an open problem. 

A majority of the irreducible affine Coxeter systems  currently do not have a classification of the T-avoiding elements. Specifically, Coxeter systems of type $\widetilde{B}_n, \widetilde{D}_n, \widetilde{E}_6, \widetilde{E}_7, \widetilde{E}_8$, and $\widetilde{G}_4$ do not have a classification. Future work could include classifying the T-avoiding elements of the Coxeter systems mentioned above.

%%%%%%%%%%%%%%%%%%%%%%%%%%%%%%%%

\bibliographystyle{plain}
\bibliography{T-Avoiding}

\end{document}