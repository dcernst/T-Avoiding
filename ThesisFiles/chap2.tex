\chapter{T-Avoiding Elements in Type $B_n$}

In this section we classify the T-Avoiding elements in Coxeter groups of type $B_n$. We start by introducing necessary tools and finish with a proof of the classification. Note that this proof closely follows the classification of T-avoiding elements of Type $D_n$ seen in~\cite{Gern2013a}.

\section{Tools for the Classification}

Recall from Example~\ref{ex:B} that we can represent each element $w \in W(B_n)$ as a member of the signed permutation group. As a result we can write $w \in W(B_n)$ using one-line notation 
\[ w=[w(1),w(2), \ldots, w(n-1), w(n)] \]
where we write a bar underneath a number in place of a negative sign in order to simplify notation. This is obtained from the Coxeter group in the following way. We identify $s_i \in S(B_n)$ via 
\[s_i=[1,2, \ldots i-1, i+1,i,i+2, \ldots, n-1,n] \] and we identify $s_0 \in S(B_n)$ via
\[s_0=[\underbar{1}, 2 \ldots, n].\] Further $w(-i)=-w(i)$ for $i \in \{1,2, \ldots, n\}$.

\begin{example}
Let $w \in W(B_6)$ with a given reduced expression $\w=s_0s_1s_3s_4s_5s_2$. Then we can write $w=[2, 4, \underbar{1}, 5, 6, 3]$. 
\end{example}

It will be useful to easily determine what what happens to the window notation of a given element $w \in W(B_n)$ when we multiply on the right or left by $s_i \in S(B_n)$. The following Proposition allows us to do this.

\begin{proposition}
	Let $w \in W(B_n)$ with corresponding signed permutation 
	\[w=[w(1),w(2), \ldots w(n)].\]
	Suppose $s_i \in S(B_n)$. If $i \geq 1$, then multiplying $w$ on the right by $s_i$ has the effect of interchanging $w(i)$ and $w(i+1)$. Multiplying on the left by $s_i$ has the effect of interchanging the entries whose absolute values are $i$ and $i+1$.
	
	If $i=0$, then multiplying $w$ on the right by $s_i$ has the effect of switching the sign of $w(1)$. Multiplying $w$ on the left by $s_i$ has the effect of switching the sign of the entry whose absolute value is $1$.
	\begin{proof}
	This follows from~\cite[Section 8.1 and A3.1]{Bjorner2005}.	
	\end{proof}
\end{proposition}

Given the one line notation for an element $w \in W(B_n)$ we can easily calculate the left and right descent sets of $w$. The following proposition explains how.

\begin{proposition}[Bj\"{o}rner,~\cite{Bjorner2005}]\label{prop:descent}
Let $w \in W(B_n)$. Then 
\[ \mathcal{R}(w)=\{s_i \in S: w(i) > w(i+1)\} \]
where w(0)=0 by definition.
\begin{proof}
	This is,~\cite{Bjorner2005} Proposition 8.1.2.
\end{proof}
\end{proposition}

We now will introduce the concept of signed pattern avoidance which will help with the classification of the T-avoiding elements in the Coxeter group of type $B_n$. This notion was first introduced in~\cite{Gern2013a}. Let $w \in W(B_n)$. We say that $w$ \emph{avoids the consecutive pattern} $abc$ if there is no $i \in \{1,2, \ldots, n-2\}$ such that $(w(i),~w(i+1),~w(i+2))$ is in the same relative order as $(a,b,c).$ We say that $w$ \emph{avoids the signed consecutive pattern} $abc$ if there is no $i \in \{1,2, \ldots, n-2\}$ such that $\left(|w(i)|, |w(i+1)|, |w(i+2)|\right)$ is in the same consecutive order as $\left(|a|, |b|, |c| \right)$ and such that $\sgn(w(i))=\sgn(a)$, $\sgn(w(i+1))=\sgn(b)$, and $\sgn(w(i+2))=\sgn(c)$.

\begin{example}
Let $w \in W(B_4)$ with signed permutation \[w=[\underbar{2},4, \underbar{1}, 3].\] We see that $w$ has the signed consecutive pattern $\underbar{2} 3 \underbar{1}$, since $(|w(1)|, |w(2)|, |w(3)|)$ are in the same relative order as $(|-2|, |3|, |-1|)$, and $\sgn(w(1))=\sgn(-2)$, $\sgn(w(2))=\sgn(3)$, and $\sgn(w(3))=\sgn(-1)$. However, $w$ avoids the signed consecutive pattern $1\underbar{2}3$.
\end{example}

\section{Classification of T-Avoiding Elements in Type $B_n$}\label{sec:TAB}

In this section we will classify the T-avoiding elements in Coxeter groups of type $B_n$. As in the previous classifications seen in Chapter~\ref{chap:TandTavoid} $W(B_n)$ has trivial T-avoiding elements as all Coxeter groups contain elements that are products of commuting generators. This leaves us to classify any non-trivial T-avoiding elements in $W(B_n)$. The following is our classification.

\begin{theorem}\label{thm:classificationofB}
There are no non-trivial T-avoiding elements in $W(B_n)$.	
\end{theorem}
 
In order to prove this we will use the notion of signed pattern avoidance seen above. Before we prove this theorem we first need some preparatory lemmas. 

\begin{lemma}\label{lem:sts}
Let $s,t \in S(B_n)$ such that $m(s,t)=3$, and $s_0 \notin \{s,t\}.$ Then $w$ has a reduced expression ending in $sts$ if and only if $w$ has the consecutive pattern $321$.
\begin{proof}
	Let $i \geq 1$, let $I=\{s_i,s_{i+1}\}$ and write $w=w^Iw_I$ as in 2.2.4 in~\cite{Bjorner2005}. Observe that if $w$ has a reduced expression ending in two non-commuting generators $s_i, s_{i+1}$ in some order then we have $w_I \in \{s_is_{i+1}, s_{i+1}s_i\}$.
	
	Suppose $w$ has the consecutive pattern $321$. Then there is some $i$ such that $w(i) > w(i+1) > w(i+1)$. By \ref{prop:descent} $s_i,s_{i+1} \in \mathcal{R}(w)$. By \textcolor{red}{Tyson's reference to simply laced coxeter group stuff 1.2.1} $w$ ends in $s_is_{i+1}s_{i+2}$.
	
	Conversely, suppose $w$ ends in $s_is_i+1s_i$. This implies that either $w_I=s_is_{i+1}$ or $w_I=s_{i+1}s_i$ which implies that $s_i,s_{i+1} \in \mathcal{R}(w)$. Since $s_i,s_{i+1} \in \mathcal{R}(w)$, we see that $w(i)>w(i+1)>w(i+2)$ by \ref{prop:descent}. Thus $w$ has the consecutive pattern 321.
	Therefore, $w$ has a reduced expression ending in $sts$ if and only if $w$ has the consecutive pattern $321$. 
\end{proof}	
\end{lemma}

\begin{corollary}\label{lem:endswithsts}
	Let $s,t \in S(B_n)$ such that $m(s,t)=3$, and $s_0 \notin\{s,t\}$. Then $w$ has a reduced expression beginning with $sts$ if and only if $w^{-1}$ has the consecutive pattern $321$.
	\begin{proof}
		Let $s,t \in S(B_n)$ such that $m(s,t)=3$, and $s_0 \notin\{s,t\}$. We know that $w$ has no reduced expressions beginning with $sts$ if and only if $w^{-1}$ has no reduced expression ending with $sts$ which by Theorem~\ref{lem:endswithsts} happens only if $w^{-1}$ avoids the consecutive pattern $321$.
	\end{proof}
\end{corollary}

\begin{lemma}\label{lem:ts}
Let $s,t \in S(B_n)$ such that $m(s,t)=3$, and $s_0 \notin \{s,t\}.$ Then $w$ has a reduced expression ending in $st$ if and only if $w$ has the consecutive pattern $231$.
\begin{proof}
	Let $i \geq 1$, let $I=\{s_i,s_{i+1}\}$ and write $w=w^Iw_I$ as in 2.2.4 in~\cite{Bjorner2005}. Observe that if $w$ has a reduced expression ending in two non-commuting generators $s_i, s_{i+1}$ in some order then we have $w_I \in \{s_is_{i+1}, s_{i+1}s_i\}$.
	
	Suppose that $w$ has the consecutive pattern $231$. Then there is some $i$ such that $w(i+1)>w(i)>w(i+2)$. By \ref{prop:descent} $s_{i+1} \in \mathcal{R}(w)$. Now multiplying on the right by $s_{i+1}$ we see that $ws_{i+1}(i+1)=w(i+2)$ and $ws_{i+1}(i)=w(i)$. We know that $w(i+2)<w(i)$, this implies that $s_i \in \mathcal{R}(ws_{i+1})$. This implies $w$ has a reduced expression that ends in $s_is_{i+1}$.
	
	 Conversely, suppose that $w$ has a reduced expression ending in $s_is_{i+1}$. Then $w(i+2)<w(i+1)$ and $w(i)<w(i+1)$. Since $s_i \in \mathcal{R}(ws_{i+1})$ we have $w(i+2)=ws_{i+1}(i+1)<ws_{i+1}(i)=w(i)$. Thus we have that $w(i+1) > w(i) > w(i+2)$. Hence $w$ has the consecutive pattern 231.
	Therefore, $w$ has a reduced expression ending in $st$ if and only if $w$ has the consecutive pattern $231$.
\end{proof}	
\end{lemma}

\begin{corollary}\label{lem:endswithst}
	Let $s,t \in S(B_n)$ such that $m(s,t)=3$, and $s_0 \notin\{s,t\}$. Then $w$ has a reduced expression beginning with $st$ if and only if $w^{-1}$ has the consecutive pattern $231$.
	\begin{proof}
		Let $s,t \in S(B_n)$ such that $m(s,t)=3$, and $s_0 \notin\{s,t\}$. We know that $w$ has no reduced expressions beginning with $st$ if and only if $w^{-1}$ has no reduced expression ending with $st$ which by Theorem~\ref{lem:endswithsts} happens only if $w^{-1}$ avoids the consecutive pattern $231$.
	\end{proof}
\end{corollary}

\begin{lemma}\label{lem:st}
Let $s,t \in S(B_n)$ such that $m(s,t)=3$, and $s_0 \notin \{s,t\}.$ Then $w$ has a reduced expression ending in $ts$ if and only if $w$ has the consecutive pattern $312$.
\begin{proof}
	Let $i \geq 1$, let $I=\{s_i,s_{i+1}\}$ and write $w=w^Iw_I$ as in 2.2.4 in~\cite{Bjorner2005}. Observe that if $w$ has a reduced expression ending in two non-commuting generators $s_i, s_{i+1}$ in some order then we have $w_I \in \{s_is_{i+1}, s_{i+1}s_i\}$.
	
	Suppose that $w$ has the consecutive pattern $312$.  Then there is some $i$ such that $w(i)>w(i+2)>w(i+1)$. By \ref{prop:descent} we see that $s_i \in \mathcal{R}(w)$. Multiplying on the right by $s_i$ we get $ws_i(i+1)=w(i)$ and $ws_i(i+2)=w(i+2)$. By above $w(i)>w(i+2)$, and by \ref{prop:descent} $s_{i+1} \in \mathcal{R}(ws_i)$. This implies that $w$ has a reduced expression ending in $s_{i+1}s_i$. 
	
	Conversely suppose $w$ ends in a reduced expression with $s_{i+1}s_i$ Then $w_I=s_{i+1}s_i$. We see that $w(i)>w(i+1)$ and $w(i+2)>w(i+1)$. Since $s_{i+1} \in \mathcal{R}(ws_i)$, we have $w(i+2)=ws_i(i+2)<ws_i(i+1)=w(i)$. From this we have $w(i)>w(i+2)$, so $w(i)>w(i+2)>w(i+1)$. Hence, $w$ has the consecutive pattern $312$. Therefore, $w$ has a reduced expression ending in $ts$ if and only if $w$ has the consecutive pattern $312$.
\end{proof}
\end{lemma}

\begin{corollary}\label{lem:endswithts}
	Let $s,t \in S(B_n)$ such that $m(s,t)=3$, and $s_0 \notin\{s,t\}$. Then $w$ has a reduced expression beginning with $ts$ if and only if $w^{-1}$ has the consecutive pattern $312$.
	\begin{proof}
		Let $s,t \in S(B_n)$ such that $m(s,t)=3$, and $s_0 \notin\{s,t\}$. We know that $w$ has no reduced expressions beginning with $ts$ if and only if $w^{-1}$ has no reduced expression ending with $ts$ which by Theorem~\ref{lem:endswithsts} happens only if $w^{-1}$ avoids the consecutive pattern $312.$
	\end{proof}
\end{corollary}

\begin{lemma}\label{lem:endswiths0}
Let $w \in W(B_n)$. Then $w$ has a reduced expression ending in $s_1s_0$ if and only if $w(0)>w(1)$ and $-w(1)>w(2)$.
\begin{proof}
	Suppose $w \in W(B_n)$ such that $w$ ends with $s_1s_0$. Then $s_0 \in \RD(w)$ and $s_1 \in \RD(ws_0)$. This implies that $ws_0(1)>ws_0(2)$ by~\ref{prop:descent}. We see that $ws_0(1)=w(-1)=-w(1)$ and $ws_0(2)=2$. Hence $-w(1)=ws_0(1)>ws_0(2)=w(2)$. Further, since $s_0 \in \RD(w)$, we see that $w(0)>w(1)$.
	
	Conversely, suppose $w \in W(B_n)$ such that $w(0)>w(1)$ and $-w(1)>w(2)$. Since $w(0)>w(1)$ so $s_0 \in \RD(w)$. Multiplying on the right by $s_0$ we see that $ws_0(1)=-w(1)$ and $ws_0(2)=w(2)$. Note that since $ws_0(1)=-w(1)>w(2)=ws_0(2)$, $s_1 \in \RD(ws_0)$. Thus $w$ ends with $s_1s_0$. Therefore, $w$ has a reduced expression ending in $s_1s_0$ if and only if $w(0)>w(1)$ and $-w(1)>w(2)$.
\end{proof}
\end{lemma}

\begin{corollary}\label{lem:beginswiths0}
	Let $w \in W(B_n)$. Then $w$ has a reduced expression beginning in $s_0s_1$ if and only if $w^{-1}(0)>w^{-1}(1)$ and $-w^{-1}(1)>w^{-1}(2)$.
	\begin{proof}
		Let $w \in W(B_n)$. We know that $w$ has no reduced expressions beginning in $s_0s_1$ if and only if $w^{-1}$ has no reduced expressions ending in $s_0s_1$. By Lemma~\ref{lem:endswiths0} we know that this occurs if and only if $w^{-1}(0)>w^{-1}(1)$ and $-w^{-1}(1)>w^{-1}(2)$.
	\end{proof}
\end{corollary}

\begin{lemma}\label{lem:endswiths_1}
Let $w \in W(B_n)$. Then $w$ has a reduced expression ending in $s_0s_1$ if and only if $w(0)>w(2)$ and $w(1)>w(2)$.
\begin{proof}
	Suppose $w \in W(B_n)$ such that $w$ ends with $s_0s_1$. Then $s_1 \in \RD(w)$ and $s_0 \in \RD(ws_1)$. Then $ws_1(0)>ws_1(1)$. We see that $ws_1(0)=0$ and $ws_1(1)=w(2)$. This implies that $0=ws_1(0)>ws_1(1)=2$. Further, since $s_1 \in \RD(w)$ this implies that $w(1) > w(2)$. Thus if $w$ ends with $s_0s_1$, then $w(1)>w(2)$ and $w(0)>w(2)$.
	
	Conversely, suppose $w \in W(B_n)$ such that $w(1)>w(2)$ and $w(0)>w(2)$. This implies that $s_1 \in \RD(W)$. Multiplying $w$ on the right by $s_1$ we see that $ws_1(0)=w(0)$ and $ws_1(1)=w(2)$. Note that since $ws_1(0)=w(0)>w(2)=ws_1(1)$, $s_0 \in \RD(ws_1)$. Thus $w$ ends with $s_0s_1$. Therefore, $w$ has a reduced expression ending in $s_0s_1$ if and only if $w(1)>w(2)$ and $w(0)>w(2)$.
\end{proof}	
\end{lemma}

\begin{corollary}\label{lem:beginswiths1}
	Let $w \in W(B_n)$. Then $w$ has a reduced expression beginning in $s_1s_0$ if and only if $w^{-1}(0)>w^{-1}(2)$ and $w^{-1}(1)>w^{-1}(2)$.
	\begin{proof}
		Let $w \in W(B_n)$. We know that $w$ has no reduced expressions beginning in $s_1s_0$ if and only if $w^{-1}$ has no reduced expressions ending in $s_1s_0$. By Lemma~\ref{lem:endswiths0} we know that this occurs if and only if $w^{-1}(0)>w^{-1}(2)$ and $w^{-1}(1)>w^{-1}(2)$.
	\end{proof}
\end{corollary}

\begin{lemma}\label{lem:prodofcommA}
Let $w \in W(B_n)$ such that each entry for $w$ in the one-line notation is positive and both $w$ and $w^{-1}$ avoid the consecutive patterns $321$, $231$, and $312$, then $w$ is a product of commuting generators.
\begin{proof}
	This is~\cite[Lemma 2.2.9]{Gern2013a}.
\end{proof}	
\end{lemma}

\begin{lemma}\label{lem:prodofCommB}
Let $w \in W(B_n)$ be trivially T-avoiding and let $i \in \{1,2, \ldots, n\}$. Then $w$ satisfies the following conditions:
\begin{enumerate}
\item $w(j) > \min(\{w(i-1), w(i)\})$ for all $j >i$;\label{it:trivT1}
\item $w(k) < \max(\{w(i-1), w(i)\})$ for all $k < i-1$;\label{it:trivT2}
\item if $w(i), w(i+1) > 0$, then $w(j)>0$ for all $j \geq i$;\label{it:trivT3}
\item if $w(i), w(i+1) < 0$, then $w(j)<0$ for all $j \leq i+1$.\label{it:trivT4}
\end{enumerate}
\begin{proof}
	Suppose there is some least $j>i$ such that $w(j) \leq \min(\{w(i-1), w(i)\})$. Note that $j>i$ so $j \neq i$, and $j \neq i-1$ so $w(j) < \min(\{w(i-1), w(i)\})$. Note that $j$ is the least so $w(j-2) \geq \min(\{w(i+1), w(i)\})>w(j)$. This implies that either $w(j-1)>w(j-2)>w(j)$ or $w(j-2)>w(j-1)>w(j)$, which implies $w$ has the consecutive pattern $231$ or $321$ which is a contradiction to $w$ being a non-trivial T-avoiding element by Lemmas~\ref{lem:sts} and~\ref{lem:st}. Thus proving~\ref{it:trivT1}.
	
	Suppose there exists a maximal $k<i-1$ such that $w \geq \max(\{w(i-1), w(i)\})$. Note that $k < i-1$ so $k \neq i$ and $k \neq i-1$. Then $w(k)> \max(\{w(i-1), w(i)\})$. Since $k$ is maximal then $w(k+1) \leq  \max(\{w(i-1), w(i)\})$ and $w(k+2) \leq \max(\{w(i-1), w(i)\})$. This implies that either $w(k+2)<w(k+1)<w(k)$ or $w(k+1)<w(k+2)<w(k)$, which implies $w$ has the consecutive patter $321$ or $312$ which is a contradiction to $w$ being a non-trivial T-avoiding element by Lemmas~\ref{lem:sts} and~\ref{lem:ts}. Thus proving~\ref{it:trivT2}.
	
	It is easy to see that assertion~\ref{it:trivT1} implies~\ref{it:trivT3} and assertion~\ref{it:trivT2} implies~\ref{it:trivT4}.
\end{proof}	
\end{lemma}

\begin{lemma}\label{lem:231}
Let $w \in W(B_n)$ such that $w$ has the consecutive pattern $\underbar{2}31$. Then $w$ has Property T.
\begin{proof}
	Let $w \in W(B_n)$ such that $w$ has the consecutive pattern $\underbar{2}31$.
	
	Case 1: Suppose $w$ has the one-line notation $w=[\underbar{2},3,1]$. This implies that $w=s_1s_0s_2$. Clearly, $w$ begins with a product of non-commuting generators. Thus $w$ has Property T.
	
	Case 2: Suppose that $w$ has the one-line notation $w=[\underbar{a},b,c, \ast, \ldots, \ast]$ where $\underbar{a},b,c$ correspond to the signed consecutive pattern $\underbar{2},3,1$. We now consider the signed consecutive pattern that can arise involving $b,c, \ast$. The following are the possibilities for the signed consecutive pattern that can arise: $31\pm2$, $32 \pm1$, or $21\pm3$. We know that $b,c$ must be positive since they are positive in $w$ and we also know that $b>c$ by the original signed consecutive pattern. Note that by Lemmas~\ref{lem:sts},~\ref{lem:ts}, and~\ref{lem:endswiths0} all of these patterns imply that $w$ ends or begins with a product of noncommuting generators. Thus $w$ has Property T.
	
	Case 3: Suppose that $w$ has the one-line notation $w=[\ast, \ldots, \ast, \underbar{a},b,c]$ where $\underbar{a},b,c$ correspond to the signed consecutive pattern $\underbar{2},3,1$. We now consider the signed consecutive pattern that can arise involving $\ast, \underbar{a}, b$. The following are the possibilities for the signed consecutive pattern that can arise: $\pm1 \underbar{2} 3$, $\pm 2 \underbar{1} 3$, and $\pm 3 \underbar{1} 2$. Note that by Lemmas~\ref{lem:ts},~\ref{lem:endswiths0}, and~\ref{lem:endswiths_1} all of these patterns implies that $w$ ends or begins with a product of noncommuting generators. Thus $w$ has Property T. 

	Therefore, if $w \in W(B_n)$ contains the consecutive pattern $\underbar{2}31$, then $w$ has Property T.
\end{proof}	
\end{lemma}

\begin{lemma}\label{lem:2312}
Let $w \in W(B_n)$ such that $w$ has the consecutive pattern $\underbar{2}3\underbar{1}$. Then $w$ has Property T.
\begin{proof}
	Let $w \in W(B_n)$ such that $w$ has the consecutive pattern $\underbar{2}3\underbar{1}$.
	
	Case 1: Suppose $w$ has the one-line notation $w=[\underbar{2},3,\underbar{1}]$. This implies that $w=s_0s_1s_0s_2$. Clearly, $w$ begins with a product of non-commuting generators. Thus $w$ has Property T.
	
	Case 2: Suppose that $w$ has the one-line notation $w=[\underbar{a},b,\underbar{c}, \ast, \ldots, \ast]$ where $\underbar{a},b,\underbar{c}$ correspond to the signed consecutive pattern $\underbar{2},3,\underbar{1}$. We now consider the signed consecutive pattern that can arise involving $b,\underbar{c}, \ast$. The following are the possibilities for the signed consecutive pattern that can arise: $3\underbar{1}\pm2$, $3\underbar{2} \pm1$, or $2\underbar{1}\pm3$. We know that $b$ must be positive since it is positive in $w$, $c$ must be negative since it is negative in $w$, and we also know that $|b|>|c|$ by the original signed consecutive pattern. Note that by Lemmas~\ref{lem:sts},~\ref{lem:ts}, and~\ref{lem:endswiths0} all of these patterns imply that $w$ ends or begins with a product of noncommuting generators. Thus $w$ has Property T.
	
	Case 3: Suppose that $w$ has the one-line notation $w=[\ast, \ldots, \ast, \underbar{a},b,\underbar{c}]$ where $\underbar{a},b,\underbar{c}$ correspond to the signed consecutive pattern $\underbar{2},3,\underbar{1}$. We now consider the signed consecutive pattern that can arise involving $\ast, \underbar{a}, b$. The following are the possibilities for the signed consecutive pattern that can arise: $\pm1 \underbar{2} 3$, $\pm 2 \underbar{1} 3$, and $\pm 3 \underbar{1} 2$. We know that $a$ must be negative, $b$ must be positive and $|a|<|b|$ by the original signed permutation. Note that by Lemmas~\ref{lem:ts},~\ref{lem:endswiths0}, and~\ref{lem:endswiths_1} all of these patterns implies that $w$ ends or begins with a product of noncommuting generators. Thus $w$ has Property T. 

	Therefore, if $w \in W(B_n)$ contains the consecutive pattern $\underbar{2}3\underbar{1}$, then $w$ has Property T.
\end{proof}	
\end{lemma}

\begin{lemma}\label{lem:123}
Let $w \in W(B_n)$ such that $w$ has the consecutive pattern $\underbar{1}23$. Then $w$ has Property T or is a trivial T-avoiding element.
\begin{proof}
	Let $w \in W(B_n)$ such that $w$ has the consecutive pattern $\underbar{1}23$.
	
	Case 1: Suppose $w$ has the one-line notation $w=[\underbar{1}23]$. This implies that $w=s_0$. Clearly, $w$ is a trivial T-avoiding element as it is a single generator.
	
	Case 2: Suppose that $w$ has the one-line notation $w=[\underbar{a},b,c, \ast, \ldots, \ast]$ where $\underbar{a},b,c$ correspond to the signed consecutive pattern $\underbar{1},2,3$. We now consider the signed consecutive pattern that can arise involving $b,c, \ast$. The following are the possibilities for the signed consecutive pattern that can arise: $23\pm1$, $13 \pm2$, or $12\pm3$. We know that $b,c$, and we also know that $|b|<|c|$ by the original signed consecutive pattern. Note that by Lemmas~\ref{lem:sts},~\ref{lem:ts}, and~\ref{lem:endswiths0} all of these patterns imply that $w$ ends or begins with a product of noncommuting generators. Thus $w$ has Property T.
	
	Case 3: Suppose that $w$ has the one-line notation $w=[\ast, \ldots, \ast, \underbar{a},b,c]$ where $\underbar{a},b,c$ correspond to the signed consecutive pattern $\underbar{2},3,1$. We now consider the signed consecutive pattern that can arise involving $\ast, \underbar{a}, b$. The following are the possibilities for the signed consecutive pattern that can arise: $\pm3 \underbar{1} 2$, $\pm 2 \underbar{1} 3$, and $\pm 1 \underbar{2} 3$. We know that $a$ must be negative, $b$ must be positive and $|a|<|b|$ by the original signed permutation. Note that by Lemmas~\ref{lem:ts},~\ref{lem:endswiths0}, and~\ref{lem:endswiths_1} all of these patterns implies that $w$ ends or begins with a product of noncommuting generators. Thus $w$ has Property T. 

	Therefore, if $w \in W(B_n)$ contains the consecutive pattern $\underbar{1}23$, then $w$ has Property T or is a trivial T-avoiding element.
\end{proof}	
\end{lemma}

\begin{lemma}\label{lem:132}
Let $w \in W(B_n)$ such that $w$ has the consecutive pattern $\underbar{1}32$. Then $w$ has Property T or is a trivial T-avoiding element.
\begin{proof}
	Let $w \in W(B_n)$ such that $w$ has the consecutive pattern $\underbar{1}32$.
	
	Case 1: Suppose $w$ has the one-line notation $w=[\underbar{1}32]$. This implies that $w=s_0s_2$. Clearly, $w$ is a trivial T-avoiding element as it is a single generator.
	
	Case 2: Suppose that $w$ has the one-line notation $w=[\underbar{a},b,c, \ast, \ldots, \ast]$ where $\underbar{a},b,c$ correspond to the signed consecutive pattern $\underbar{1},2,3$. We now consider the signed consecutive pattern that can arise involving $b,c, \ast$. The following are the possibilities for the signed consecutive pattern that can arise: $23\pm1$, $13 \pm2$, or $12\pm3$. We know that $b,c$, and we also know that $|b|<|c|$ by the original signed consecutive pattern. Note that by Lemmas~\ref{lem:sts},~\ref{lem:ts}, and~\ref{lem:endswiths0} all of these patterns imply that $w$ ends or begins with a product of noncommuting generators. Thus $w$ has Property T.
	
	Case 3: Suppose that $w$ has the one-line notation $w=[\ast, \ldots, \ast, \underbar{a},b,c]$ where $\underbar{a},b,c$ correspond to the signed consecutive pattern $\underbar{2},3,1$. We now consider the signed consecutive pattern that can arise involving $\ast, \underbar{a}, b$. The following are the possibilities for the signed consecutive pattern that can arise: $\pm3 \underbar{1} 2$, $\pm 2 \underbar{1} 3$, and $\pm 3 \underbar{2} 1$. We know that $a$ must be negative, $b$ must be positive and $|a|<|b|$ by the original signed permutation. Note that by Lemmas~\ref{lem:ts},~\ref{lem:endswiths0}, and~\ref{lem:endswiths_1} all of these patterns implies that $w$ ends or begins with a product of noncommuting generators. Thus $w$ has Property T. 

	Therefore, if $w \in W(B_n)$ contains the consecutive pattern $\underbar{1}23$, then $w$ has Property T or is a trivial T-avoiding element.
\end{proof}	
\end{lemma}

We can prove~\ref{thm:classificationofB}.

\begin{proof}
Suppose that $w \in W(B_n)$ is a non-trivial T-avoiding element. There are $2^3 \cdot 3!$ possible choices of signed consecutive patterns for $w(1)w(2)w(3)$ where $w=[w(1), w(2), w(3), \ast, \ldots, \ast]$.

\begin{center}
\begin{tabular}{|l|l|l|l|l|l|l|l|}
\hline
\cellcolor{blue!30}$123$ & \cellcolor{orange2!40}$\underbar{1}23$ & \cellcolor{brown!50}$1\underbar{2}3$ & \cellcolor{red!25}$12\underbar{3}$ & \cellcolor{brown!50}$\underbar{12}3$ & \cellcolor{red!25}$\underbar{1}2\underbar{3}$ & \cellcolor{turq!40}$1\underbar{23}$ & \cellcolor{turq!40}$\underbar{123}$ \\
\hline
\cellcolor{blue!30}$132$ & \cellcolor{orange2!40}$\underbar{1}32$ & \cellcolor{brown!50}$1\underbar{3}2$ & \cellcolor{red!25}$13\underbar{2}$ & \cellcolor{brown!50}$\underbar{13}2$ & \cellcolor{red!25}$\underbar{1}3\underbar{2}$ & \cellcolor{ggreen!50}$1\underbar{32}$ & \cellcolor{ggreen!50}$\underbar{132}$ \\
\hline
\cellcolor{blue!30}$213$ & \cellcolor{yellow!50}$\underbar{2}13$ & \cellcolor{brown!50}$2\underbar{1}3$ & \cellcolor{turq!40}$21\underbar{3}$ & \cellcolor{yellow!50}$\underbar{21}3$ & \cellcolor{red!25}$\underbar{2}1\underbar{3}$ & \cellcolor{turq!40}$2\underbar{13}$ & \cellcolor{red!25}$\underbar{213}$ \\
\hline
\cellcolor{red!25}$231$ & \cellcolor{purple2!50}$\underbar{2}31$ & \cellcolor{ggreen!50}$2\underbar{3}1$ & \cellcolor{red!25}$23\underbar{1}$ & \cellcolor{brown!50}$\underbar{23}1$ & \cellcolor{purple2!50}$\underbar{2}3\underbar{1}$ & \cellcolor{ggreen!50}$2\underbar{31}$ & \cellcolor{brown!50}$\underbar{231}$ \\
\hline
\cellcolor{ggreen!50}$312$ & \cellcolor{yellow!50}$\underbar{3}12$ & \cellcolor{ggreen!50}$3\underbar{1}2$ &\cellcolor{turq!40}$31\underbar{2}$ & \cellcolor{yellow!50}$\underbar{31}2$ & \cellcolor{yellow!50}$\underbar{3}1\underbar{2}$ & \cellcolor{turq!40}$3\underbar{12}$ & \cellcolor{yellow!50}$\underbar{312}$ \\
\hline
\cellcolor{turq!40}$321$ & \cellcolor{yellow!50}$\underbar{3}21$ & \cellcolor{ggreen!50}$3\underbar{2}1$ & \cellcolor{turq!40}$32\underbar{1}$ & \cellcolor{yellow!50}$\underbar{32}1$ & \cellcolor{yellow!50}$\underbar{3}2\underbar{1}$ & \cellcolor{ggreen!50}$3\underbar{21}$ & \cellcolor{yellow!50}$\underbar{321}$\\
\hline
\end{tabular}
\end{center}

We can use Lemma~\ref{lem:sts} and Corollary~\ref{lem:endswithsts} to eliminate the signed consecutive patterns highlighted in \textcolor{turq}{turquoise}. We can use Lemma~\ref{lem:st} and Corollary~\ref{lem:endswithst} to eliminate the signed consecutive patterns highlighted in \textcolor{red}{red}. We can use Lemma~\ref{lem:ts} and Corollary~\ref{lem:endswithts} to eliminate the consecutive patterns highlighted in \textcolor{ggreen}{green}.  We can use Lemma~\ref{lem:endswiths0} and Corollary~\ref{lem:beginswiths0} to eliminate the signed consecutive patterns highlighted in \textcolor{yellow}{yellow}. We can use Lemma~\ref{lem:endswiths_1} and Corrollary~\ref{lem:beginswiths1} to eliminate signed consecutive patterns highlighted in \textcolor{brown}{brown}. We can use Lemma~\ref{lem:prodofcommA} to elminate the signed consecutive patterns highlighted in \textcolor{blue}{blue}. We can use Lemmas~\ref{lem:231} and~\ref{lem:2312} to eliminate signed consecutive patterns highlighted in \textcolor{purple}{purple}. Finally we can use Lemmas~\ref{lem:123} and~\ref{lem:132} to eliminate signed consecutive patterns highlighted in \textcolor{orange2}{orange}. Since all of the above patterns are eliminated as possibilities for $w(1)w(2)w(3)$ and there are no other signed consecutive patterns that are possible for these positions, $w$ can not be a non-trivial T-avoiding element in the Coxeter group of type B. Therefore, there are no non-trivial T-avoiding elements in $W(B_n)$.
\end{proof}
