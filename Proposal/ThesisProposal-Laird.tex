\documentclass[11pt]{article}

\usepackage{url}
\usepackage{tikz}
\usetikzlibrary{decorations.markings}
\usetikzlibrary{arrows,shapes,positioning}
\usepackage{calc}
\usepackage[compact]{titlesec}
\usepackage{caption}
\usepackage[labelformat=simple,labelfont={}]{subcaption}
\usepackage[margin=1in]{geometry} 
\usepackage{fancyhdr}
\usepackage{amsmath}
\usepackage{amsthm}
\usepackage{amssymb}
\usepackage{mathtools}
\usepackage{enumitem}
\usepackage{graphicx}
\usepackage{color}
\definecolor{darkblue}{rgb}{0, 0, .6}
\definecolor{grey}{rgb}{.7, .7, .7}
\usepackage[breaklinks]{hyperref}
\hypersetup{
	colorlinks=true,
	linkcolor=darkblue,
	anchorcolor=darkblue,
	citecolor=darkblue,
	pagecolor=darkblue,
	urlcolor=darkblue,
	pdftitle={},
	pdfauthor={}
}

\def\F{\mathbb{F}}
\def\N{\mathbb{N}}
\def\R{\mathbb{R}}
\def\Z{\mathbb{Z}}
\newcommand{\w}{\mathsf{w}}
\DeclareMathOperator{\FC}{FC}
\DeclareMathOperator{\CFC}{CFC}
\DeclareMathOperator{\supp}{supp}
\newcommand{\TL}{\mathrm{TL}}
%\renewcommand{\C}{\mathrm{C}}
\renewcommand{\H}{\mathcal{H}}
\newcommand{\J}{\mathcal{J}}
\renewcommand{\L}{\mathcal{L}}
\renewcommand{\R}{\mathcal{R}}

\theoremstyle{definition}
\newtheorem{theorem}{Theorem}%[subsection]
\newtheorem{definition}[theorem]{Definition}
\newtheorem{proposition}[theorem]{Proposition}
\newtheorem{lemma}[theorem]{Lemma}
\newtheorem{corollary}[theorem]{Corollary}
\newtheorem{goal}[theorem]{Goal}
\newtheorem{conjecture}[theorem]{Conjecture}
\newtheorem{claim}[theorem]{Claim}
\newtheorem*{unnumbered-corollary}{Corollary}
\newtheorem*{unnumbered-conjecture}{Conjecture}
\newtheorem*{unnumbered-claim}{Claim}

\newtheorem{problem}{Problem}

\theoremstyle{remark}
\newtheorem{example}{Example}[section]
\newtheorem*{remark}{Remark}
\newtheorem*{question}{Question}
\newtheorem*{observation}{Observation}

\setlength{\parindent}{0pt}

\titleformat*{\section}{\Large \bfseries \color{darkblue}}
\titleformat*{\subsection}{\large \bfseries \color{darkblue}}

\newcommand{\blankline}{\pagebreak[2]\vspace{.5\baselineskip}}
\newcommand{\alert}[1]{\textcolor{darkblue}{\textbf{#1}}}

\begin{document}

%TikZ code
\tikzstyle{double-bead} =[postaction={decorate,decoration={markings, mark=at position .4 with {\filldraw[purple] circle (1.75pt);}, mark=at position .6 with {\filldraw[purple] circle (1.75pt);} }}]  %this puts two beads on an edge at 40 percent and 60 percent of the way across the edge
\tikzstyle{single-bead} =[postaction={decorate,decoration={markings, mark=at position .5 with {\filldraw[purple] circle (1.75pt);}}}]  %this puts one bead on an edge at the half way point

\title{\alert{T-avoiding elements of Coxeter groups}}
\author{Thesis Proposal for Taryn Laird\\
Directed by Dana C.~Ernst}
 
\maketitle

\section*{Overview of Research Project}

In mathematics, one uses groups to study symmetry.  In particular, a reflection group can be used to study the reflection and rotational symmetry of an object.  A Coxeter group can be thought of as a generalized reflection group, where the group is generated by a set of elements of order two (i.e., reflections) and there are rules for how the generators interact with each other.  Every element of a Coxeter group can be written as an expression in the generators, and if the number of generators in an expression is minimal, we say that the expression is reduced.  An element $w$ of a Coxeter group is called \emph{T-avoiding} if $w$ does not have a reduced expression beginning or ending with a pair of non-commuting generators.

\blankline

During the 2010--2011 academic year, I mentored Joseph Cormier, Zachariah Goldenberg, Jessica Kelly, and Christopher Malbon at Plymouth State University on an original research project aimed at exploring the T-avoiding elements in Coxeter groups of types $A$ and $B$.  In particular, we classified the T-avoiding elements in both types of groups.  In the case of type $A$, our results are a reformulation of known results, but with a much simpler proof.  Unfortunately, the students never finished writing up their results for publication and I have since forgotten the proof in the case of type $B$.  Taryn's first goal will be to understand the classification in type $A$ and then re-discover the proof in type $B$.

\blankline

During the 2011--2012 academic year, I mentored Ryan Cross, Katie Hills-Kimball, and Christie Quaranta at PSU on a project aimed at exploring the T-avoiding elements in Coxeter groups of type $F$.  In particular, the students successfully classified the T-avoiding elements in the infinite Coxeter group of type $F_{5}$, as well as the finite Coxeter group of type $F_{4}$. At the time, the students conjectured that our classification holds more generally for arbitrary $F_{n}$.

\blankline

In the Spring of 2013, I worked with Selina Gilbertson at NAU on extending the results obtained by Ryan, Katie, and Christie the previous year.  The initial goal was to prove that there were no new T-avoiding elements (other than tacking on products of commuting generators) in type $F_n$ for $n\geq 6$.  However, Selina discovered that this is horribly wrong.  It appears that the classification of T-avoiding elements in higher ranks gets more and more complicated.  We believe that we have the correct classification of the T-avoiding elements in type $F_6$ and Selina was able to put most of the pieces of a proof together in one semester.  This is a hard problem!  The second goal will be for Taryn to understand the classification of the T-avoiding elements in Coxeter groups of types $F_4$ and $F_5$ and then put the finishing touches on the classification in type $F_6$.  

\blankline

In his PhD thesis, Tyson Gern classified the T-avoiding elements in Coxeter groups of type $D$. A third goal for Taryn will be to read the relevant chapters from Tyson's thesis and understand the proof of his classification of the T-avoiding elements.  

\blankline

As a result of the work I did for my PhD thesis, I ``know" what all of the T-avoiding elements in Coxeter groups of type $\widetilde{C}$ are.  However, I've never taken the time to write down a formal proof.  The fourth goal for Taryn will be to prove that the classification I believe to be true is actually correct.

\blankline

Once Taryn has completed the tasks above, I will set her the task of exploring the T-avoiding elements in some of the Coxeter groups that do not yet have classifications (e.g., $F_n$ for $n\geq 7$).

\section*{Outline of Activities and Goals}

The intention is for Taryn to work on a problem that is as close to the frontier of mathematical research as her background allows, with the publication of an article in a research journal a lofty but obtainable goal. Furthermore, Taryn will be expected to present the findings of her research at a conference and/or one of our department's seminars. The experience of conducting the proposed research will be invaluable, and in particular, it will strengthen her application to PhD programs.

\blankline
 
There are many open problems related to the proposed research, which will provide a wealth of problems for the student to explore.  During the time period of the project, I will meet weekly with Taryn. Initially, these meetings will be used to get the student up to speed on the relevant material. Outside of these meetings, Taryn will be provided with further reading and enriching activities. As she gains sufficient familiarity with the necessary mathematical content, the meetings will become more collaborative as we work to experiment, conjecture, discover patterns, and prove theorems. Taryn will be expected to spend several hours each week exploring independently. I will also assist and mentor her as she prepares material for presentations. The ultimate goal of the project is to obtain original results. In this case, I will help Taryn prepare a manuscript that summarizes our results with the intention of submitting the work to a peer-reviewed research journal.

\end{document}