\chapter{T-Avoiding Elements in Types $\widetilde{A}_n, A_n, D_n, F_n$, and $I_2(m)$}\label{chap:TandTavoid}

%%%%%%%%%%%%%%%%%%%%%%%%%%

\section{Types $\widetilde{A}_n$ and $A_n$}
We start with T-avoiding elements in Coxeter systems of type $\widetilde{A}_n$ and $A_n$. We first focus on non-trivial T-avoiding elements in $W(\widetilde{A}_n)$.

\begin{theorem}
 If $n \geq 2$ and $n$ is even, then there are no non-trivial T-avoiding elements in $W(\widetilde{A}_n)$. Otherwise, if $n \geq 2$ and $n$ is even then $W(\widetilde{A}_n)$ contains non-trivial T-avoiding elements.
\begin{proof}
	This is~\cite[Proposition~3.1.2.]{Fan1999} after an appropriate translation of terminology.
\end{proof}
\end{theorem}

%Now we will classify the non-trivial T-avoiding elements in $W(A_n)$. Recall that $W(A_n)$ is a star reducible Coxeter group which by our observations above implies that if $W(A_n)$ has non-trivial T-Avoiding elements, these elements will be not FC. In~\cite{Fan1999}, the non-trivial T-avoiding elements were classified. The classification is seen in the following theorem.

%We will now proceed into the classification of T-avoiding elements in $W(\widetilde{A}_n)$. Similar to $W(A_n)$, $W(\widetilde{A}_n)$ will also have trivial T-avoiding elements since $W(\widetilde{A}_n)$ contains elements that are products of commuting generators. We will now proceed to classifying non-trivial T-avoiding elements in $W(\widetilde{A_n})$. The following classification is a result from~\cite{Fan1999}.

The classification seen in~\cite{Fan1999} did not specifically classify the non-trivial T-avoiding elements for type $\widetilde{A}_n$ for $n$ odd. Since $W(\widetilde{A}_n)$ for $n$ odd is not star reducible we know that the non-trivial T-avoiding elements could be FC. The following is our conjecture regarding what the non-trivial T-avoiding elements are in $W(\widetilde{A}_n)$ for $n$ odd.
\begin{conjecture}
	The only non-trivial T-avoiding elements in $W(\widetilde{A}_n)$ for $n$ odd are of the form $w=(s_0s_2 \cdots s_{n-2}s_ns_1s_3 \cdots s_{n-3}s_{n-1})^k$  for $k \in \mathbb{Z}^+$. 
\end{conjecture} 

Recall that $W(A_n)$, $n$ even, is not a star reducible Coxeter group. Hence it makes sense that the T-avoiding elements in $W(\widetilde{A}_n)$, $n$ odd, can be FC. Further, as $W(A_n)$ is a parabolic subgroup of $W(\widetilde{A}_n)$ and $W(A_n)$ is a star reducible Coxeter group, the FC non-trivial T-avoiding elements must have full support. First notice, that $w=(s_0s_2 \cdots s_{n-2}s_ns_1s_3 \cdots s_{n-3}s_{n-1})^k$ is reduced, FC and has full support. In addition, $w$ is in fact T-avoiding. Notice that the above non-trivial T-avoiding elements are FC. As stated in the conjecture we believe that these are the only non-trivial T-avoiding elements. However, it is not immediately obvious that there are not any non FC non-trivial T-avoiding elements. Classifying these non-trivial T-avoiding elements remains an open problem. We now proceed with discussion of T-avoiding elements in Coxeter groups of type $A_n$. 

\begin{corollary}
There are no non-trivially T-avoiding elements in $W(A_n)$. 
\begin{proof}
Notice that the Coxeter graph of type $A_n$ can be obtained from the Coxeter graph of type $\widetilde{A}_k$, for $k > n$. This is done by removing the appropriate number of vertices and edges from the Coxeter graph of type $\widetilde{A}_k$. Since $W(\widetilde{A}_k)$, $k$ even, has no non-trivial T-avoiding elements this forces $W(A_n)$ to not have non-trivial T-avoiding elements. Thus $W(A_n)$ can not have bad elements.  
%Since $W(A_n)$ is a parabolic subgroup of $W(\widetilde{A}_n)$ this is a consequence of~\cite[Proposition 3.1.2.]{Fan1999}. Specifically, we can obtain the Coxeter graph of type $A_n$ from the Coxeter graph of type $\widetilde{A}_n$ for $n$ even by removing the appropriate number of vertices and edges. From this we can see that if $W(A_n)$ was to have non-trivial T-avoiding elements, this would imply that $W(\widetilde{A}_n)$ for $n$ even would also have non-trivial T-avoiding elements as well. Thus $W(A_n)$ can not have bad elements.
\end{proof}
\end{corollary}


%%%%%%%%%%%%%%%%

\section{T-Avoiding Elements in Type $D_n$}

In this section we will classify the T-avoiding elements in Coxeter systems of type $D_n$. Recall that $W(D_n)$ is a star reducible Coxeter group and as a result of this any potential non-trivial T-avoiding elements are not FC.

\begin{theorem}
 The only non-trivial T-avoiding elements in $W(D_n)$ for $n \geq 4$ are given by the heaps \textcolor{red}{Once we figure out the heap include it here.}.
\begin{proof}
	This is a consequence of~\cite[Section 2.2]{Gern2013a}. For the full details regarding this classification see~\cite{Gern2013a}. Note that in this classification, Gern  refers to non-trivially T-avoiding elements as ``bad."
\end{proof}
\end{theorem}


%%%%%%%%%%%%%%%

\section{T-Avoiding Elements in Type $F_n$}

In this section we classify what is known regarding the T-avoiding elements in the Coxeter groups of type $F_n$ for $n \geq 4$. Note that all of the following results are unpublished. %Recall that $W(F_n)$ is a star reducible Coxeter group so any non-trivial T-avoiding element in $W(F_n)$ will not be FC.

We start with the Coxeter system of type $F_5$.  Recall that $W(F_5)$ is a star reducible Coxeter group so any non-trivial T-avoiding elements will not be FC. Before we begin the classification we introduce the notion of a specific element in $W(F_5)$ called a single \emph{bowtie}, which is given by the heap in Figure~\ref{fig:singbowtie}. Note that in Figure~\ref{fig:singbowtie4}, the \textcolor{orange}{orange} blocks correspond to the elements that have bond strength 4. It turns out that the expression determined by the heap is in fact reduced, but it is not clear that it is not FC where we have highlighted the braid in \textcolor{teal}{teal} in Figure~\ref{fig:singbowtiebraid}. We can obtain a ``stack of bowties" by removing the top most layer of the given heap for the bowtie and adding a new single bowtie to the stack, as seen in Figure~\ref{fig:stackobowties}. Similar to a single bowtie, the expression that corresponds to a stack of bowties is reduced and not FC. These heaps are referenced in the following unpublished theorem by Cross, Ernst, Hills-Kimball, and Quaranta in 2012 which classifies the T-avoiding elements in the Coxeter group of type $F_5$.

\begin{figure*}[h!]
\begin{tabular}{m{7cm} m{7cm}}
\begin{subfigure}{0.5\textwidth} \centering
\begin{tikzpicture}[scale=0.5]
	\heapblock{1}{10}{1}{purple}
	\heapblock{3}{10}{3}{orange}
	\heapblock{5}{10}{5}{purple}
	\heapblock{2}{8}{2}{orange}
	\heapblock{4}{8}{4}{purple}
	\heapblock{3}{6}{3}{orange}
	\heapblock{2}{4}{2}{orange}
	\heapblock{4}{4}{4}{purple}
	\heapblock{1}{2}{1}{purple}
	\heapblock{3}{2}{3}{orange}
	\heapblock{5}{2}{5}{purple}
\end{tikzpicture}
\caption{}\label{fig:singbowtie4}
\end{subfigure}&

\begin{subfigure}{0.5\textwidth}\centering
\begin{tikzpicture}[scale=0.5]
	\heapblock{1}{10}{1}{purple}
	\heapblock{3}{10}{3}{purple}
	\heapblock{5}{10}{5}{purple}
	\heapblock{2}{8}{2}{purple}
	\heapblock{4}{8}{4}{teal}
	\heapblock{3}{6}{3}{teal}
	\heapblock{2}{4}{2}{purple}
	\heapblock{4}{4}{4}{teal}
	\heapblock{1}{2}{1}{purple}
	\heapblock{3}{2}{3}{purple}
	\heapblock{5}{2}{5}{purple}
\end{tikzpicture}
\caption{}\label{fig:singbowtiebraid}
\end{subfigure}
\end{tabular}
\caption{A single bowtie in $W(F_5)$.}\label{fig:singbowtie}	
\end{figure*}

\begin{figure*}[h!] \centering
\begin{tikzpicture}[scale=0.5]
\heapblock{1}{10}{1}{purple}
	\heapblock{3}{10}{3}{orange}
	\heapblock{5}{10}{5}{purple}
	\heapblock{2}{8}{2}{orange}
	\heapblock{4}{8}{4}{purple}
	\heapblock{3}{6}{3}{orange}
	\heapblock{2}{4}{2}{orange}
	\heapblock{4}{4}{4}{purple}
	\heapblock{1}{2}{1}{purple}
	\heapblock{3}{2}{3}{orange}
	\heapblock{5}{2}{5}{purple}
	\heapblock{2}{0}{2}{orange}
	\heapblock{4}{0}{4}{purple}
	\heapblock{3}{-2}{3}{orange}
	\heapblock{2}{-4}{2}{orange}
	\heapblock{4}{-4}{4}{purple}
	\heapblock{1}{-6}{1}{purple}
	\heapblock{3}{-6}{3}{orange}
	\heapblock{5}{-6}{5}{purple}
	
	\node[] at (3,-8){$\vdots$};
	
	\heapblock{1}{-10}{1}{purple}
	\heapblock{3}{-10}{3}{orange}
	\heapblock{5}{-10}{5}{purple}
	\heapblock{2}{-12}{2}{orange}
	\heapblock{4}{-12}{4}{purple}
	\heapblock{3}{-14}{3}{orange}
	\heapblock{2}{-16}{2}{orange}
	\heapblock{4}{-16}{4}{purple}
	\heapblock{1}{-18}{1}{purple}
	\heapblock{3}{-18}{3}{orange}
	\heapblock{5}{-18}{5}{purple}
\end{tikzpicture}
\caption{A stack of bowties in $W(F_5)$.}\label{fig:stackobowties}
\end{figure*}

\begin{theorem}
The only non-trivial T-avoiding elements in $W(F_5)$ are stacks of bowties. \qed	
\end{theorem}


As a result of the classification in type $F_5$, Cross et al. were also able to classify the T-avoiding elements in $W(F_4)$. 

\begin{corollary}
There are no non-trivial T-avoiding elements in the Coxeter system of type $F_4$. \qed	
\begin{proof}
	Since there are no non-trivial T-avoiding elements in $W(F_5)$ that do not have full support, we know that there can not be any non-trivial T-avoiding elements in $W(F_4)$. Because if there were non-trivial T-avoiding elements they would also be non-trivially T-avoiding in $W(F_5)$.
\end{proof}
\end{corollary}

Cross et al. conjectured that in Coxeter systems of type $F_n$ for $n \geq 5$, an element is non-trivially T-avoiding if and only if it is a stack of bowties multiplied by a product of commuting generators. In 2013, Gilbertson and Ernst worked with this conjecture and quickly found it to be false. The heap seen in Figure~\ref{fig:f6bat} corresponds to a non-trivial T-avoiding element in $F_6$ that is not a bowtie. It turns out that like the bowties discussed above these elements can also be stacked to create an infinite number of non-trivial T-avoiding elements. In addition, as $n$ gets large there are a number of things that can be altered that result in additional non-trivially T-avoiding elements. From this we conjecture that the classification of T-avoiding elements in Coxeter systems of type $F_n$ for $n \geq 6$ gets complicated very quickly. Classifying T-avoiding elements in $W(F_n)$ for $n \geq 6$ remains an open problem. 

\begin{figure}[h!]\centering
\begin{tikzpicture}[scale=0.5]
	\heapblock{2}{10}{2}{orange}
	\heapblock{4}{10}{4}{purple}
	\heapblock{6}{10}{6}{purple}
	\heapblock{3}{8}{3}{orange}
	\heapblock{5}{8}{5}{purple}
	\heapblock{2}{6}{2}{orange}
	\heapblock{4}{6}{4}{purple}
	\heapblock{1}{4}{1}{purple}
	\heapblock{3}{4}{3}{orange}
	\heapblock{2}{2}{2}{orange}
	\heapblock{1}{0}{1}{purple}
	\heapblock{3}{0}{3}{orange}
	\heapblock{2}{-2}{2}{orange}
	\heapblock{4}{-2}{4}{purple}
	\heapblock{3}{-4}{3}{orange}
	\heapblock{5}{-4}{5}{purple}
	\heapblock{2}{-6}{2}{orange}
	\heapblock{4}{-6}{4}{purple}
	\heapblock{6}{-6}{6}{purple}
\end{tikzpicture}
\caption{A non-trivial T-avoiding element in $W(F_6)$}\label{fig:f6bat}
\end{figure}

%%%%%%%%%%%%%%%%

\section{T-Avoiding Elements in Type $I_2(m)$}

We next will classify the T-avoiding elements in Coxeter groups of type $I_2(m)$. Note that in Coxeter groups of type $I_2(m)$, the only products of commuting generators have length 1. Although the following is a quick result, we believe that it does not already appear in the literature.
\begin{theorem}
There are no non-trivial T-avoiding elements in the Coxeter system of type $I_2(m)$.
\begin{proof}
	The graph for the Coxeter system of $I_2(m)$ appears in Figure~\ref{fig:I}. Note that the graph consists of two vertices, namely, $s_1$ and $s_2$, and a single edge with weight $m$. Also, recall that $W(I_2(m))$ is a star reducible Coxeter group. This implies that any non-trivial T-avoiding elements in $W(I_2(m))$ must not be FC, as all of the FC elements have Property T. The only non-FC element in $W(I_2(m))$ is the element of length $m$ that has exactly two reduced expressions consisting of alternating products of $s_1$ and $s_2$. Cleary, this element begins and ends with a product of noncommuting generators. Thus, this element has Property T. Hence $W(I_2(m))$ has no non-trivial T-avoiding elements. 
\end{proof}	
\end{theorem}
 



